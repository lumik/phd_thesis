\chapter{Objectives of the thesis}
\label{objectives}

As can be perceived from the above overview of the published works, UV RRS
method seems to promise a variety of advantages for nucleic acids study
compared to classical (non-resonance) Raman scattering.
In particular: the possibility of measurement in solutions of low
concentrations comparable to the concentrations used for measurement of
UV absorption in 1 cm cuvettes, low sample consumption, low or none fluorescent
background, reduction of intense Raman bands in the spectrum and thus
simplification of its shape, increased sensitivity of the band intensity to the
stacking interaction of adjacent nucleobase molecules, and the possibility to
distinguish the overlapping bands of different nucleobases by using various
excitation wavelengths.
Based on these expectations, the project to enrich the set of experimental
techniques available at the Division of Biomolecular Physics of the Institute
of Physics CU by UV RRS, was established. Since a suitable spectrometer is not
commercially offered, it was clear that it would be necessary to build-up the
spectrometer at the spot.
As the first step, a continuous argon laser with intracavity frequency
doubling, UV sensitive CCD detector, and some optical elements were purchased.

To set up on this basis a complete instrument and to establish a methodological
platform for UV RRS studies of nucleic acids was just the main task of the
dissertation.
This challenge in itself included the specification of remaining devices
(especially the spectrograph), optical and electronic elements to be acquired,
design of the sample compartment, the spectrometer assembly, testing its
parameters and prospects with respect to UV RRS investigation of nucleic acids.
Despite some theoretical considerations, only the parameters obtained by the
first experiments concerning the sensitivity, accuracy of the spectral scale,
and reproducibility were undoubtable base for further improvement of the
equipment.
Based on these, the original spectrometer design was modified to reach optimal
results.

An equally challenging task was to build the methodology of the experiment.
This included data collection and basic data processing using continuous
spectral calibration.
Above all, it was necessary to collect a sufficient set of UV RRS spectra on
simpler NA molecular models, mostly mononucleotides and polynucleotides, to
compile them with published data and then to create reliable interpretation
tables.

The solving of the above-mentioned tasks is described in the main parts of this
dissertation,
\cref{spectrometer_building}.
The next part of the thesis deals with particular applications of the UV RRS
methodology in solving specific issues related to the structural arrangements
of specific NA motives and their stability.
In these experiments, UV RRS results were completed by measurements of UV
absorption and classical (non-resonance) Raman scattering.
UV absorption was measured by using a commercial double-beam UV/VIS
Lambda 12 (Perkin Elmer) spectrometer adjusted for precise control of the
sample temperature.
The measurement of classical Raman scattering was carried out at a homemade
spectrometer using a Jobin-Yvon Spex 270M spectrograph, a liquid
nitrogen‐cooled CCD detector (Princeton Instruments) and a 90-degree
scattering geometry
\parencite{Palacky2011}.
A utility allowing automatic measurement of temperature spectral dependency
(with continuous spectral calibration) was created for this spectrometer within
the framework of this dissertation.
