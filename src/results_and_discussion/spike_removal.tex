\subsection{Cosmic ray signal removal}
\label{subsec:spike_removal}

Raman scattering is a weak phenomenon, and therefore its measurements need to
be performed using very sensitive detectors.
Hand in hand with sensitivity also comes susceptibility to artifacts
caused by signals originating from different sources than the sample under
investigation.
Among the major ones is signal caused by cosmic rays, which is characteristic
by sharp lines usually impacting only a few pixels of the CCD detector.
The spectroscopic software WinSpec, together with the Princeton Instruments
CCD detector, allowed to clear all the accumulated charge caused by cosmic rays
before each measurement and set up spike detection based on differences between
measured frames and automatic removal of sharp lines from the spectra.
However, these corrections wer not sufficient, especially first measured frame
was often corrupted by sharp lines caused by the interaction of cosmic ray
particles with the CCD detector
(see \figref{cosmic_spikes:spectrum}).

\begin{figure}
	\centering
	\input{results_and_discussion/assets/spike_removal/spikes}
	\vspace{3mm}
	\caption[%
		UVRR spectrum containing cosmic ray signal.
	]{%
		\captiontitle{%
			UVRR spectrum containing cosmic ray signal.
		}
		The spectrum was acquired with 5\,mW of 244\,nm excitation laser at the
		sample from 500\,\g{m}M (in phosphates) poly(dAdT) at 20\,\textdegree{}C as
		the	first frame with the 60\,s accumulation time.
		Two sharp lines originating from cosmic ray interaction with the CCD
		detector at 1178 and 1679\,\icm{} are clearly visible.
	}
	\label{\figlabel{cosmic_spikes:spectrum}}
\end{figure}

Two programs were used for additional correction of cosmic ray signal, both
developed in MATLAB programming environment
\parencite{Matlab}.
The first one, \emph{Spikie}
\parencite{Spikie2011},
was developed in the scope of
\textcite{Klener2011}
and was based on the detection of the sharp lines inside the measured spectrum.
This program was used mainly for more complicated scenarios where the cosmic
ray lines interfered strongly with Raman bands of the measured samples.

The second program, \emph{Spycor}
\parencite{Spycor2018},
was created as part of this thesis and is based on a comparison of consecutive
spectra where it counts on the fact that during macroscopic Raman measurement,
the spectra change only slowly with time and that the consecutive spectra are
similar.
It calculates the standard deviation of a specified number of spectra preceding
and following the analyzed spectrum, multiplies it by a user-defined constant,
and uses it as a threshold for spike detection.
It then replaces the spike by an average of the spectra used for the standard
deviation calculation. The spectrum from
\figref{cosmic_spikes:spectrum}
opened in the Spycor program can be seen in
\figref{cosmic_spikes:spycor}.

\begin{figure}
	\centering
	\ig{1}{results_and_discussion/assets/spike_removal/spycor}
	\caption[%
		Spycor -- program for spike removal.
	]{%
		\captiontitle{%
			Spycor -- program for spike removal.
		}
		The UVRR spectrum from \figref{cosmic_spikes:spectrum} opened in the
		Spycor program
		\parencite{Spycor2018}.
		The two spikes were identified and replaced by the average spectrum
		from the consecutive frames.
	}
	\label{\figlabel{cosmic_spikes:spycor}}
\end{figure}
