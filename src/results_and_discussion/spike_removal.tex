\subsection{Cosmic ray signal removal}

Raman scattering is a weak phenomenon and therefore its measurements need to be
performed using very sensitive detectors.
Hand in hand with a sensitivity is coming also a susceptibility to artifacts
caused by signals originating from different sources than the sample under
investigation.
One of the major one is signal caused by cosmic rays which is characteristic
by sharp lines usually impacting only few pixels of the CCD detector.
The spectroscopic software WinSpec together with the Princeton Instruments
CCD detector allowed to clear all the accumulated charge caused by cosmic rays
before each measurement and set up detection based on differences between
measured frames and automatic removal of sharp lines from the spectra.
But these correction weren't sufficient, especially first measured frame
was often corrupted by sharp lines caused by interaction of cosmic ray
particles with the CCD detector
(see \figref{cosmic_spikes:spectrum}).

\begin{figure}
	\centering
	\input{results_and_discussion/assets/spike_removal/spikes}
	\vspace{3mm}
	\caption[%
		UVRR pectrum containing cosmic ray signal.
	]{%
		\captiontitle{%
			UVRR pectrum containing cosmic ray signal.
		}
		The spectrum was aquired with 5\,mW of 244\,nm excitation laser at sample
		from 500\,\g{m}M (in phosphates) poly(dAdT) at 20\,\textdegree{}C as the
		first frame with the 60\,s accumulation time.
		Two sharp lines originating from cosmic ray interaction with the CCD
		detector at 1178 and 1679\,\icm{} are clearly visible.
	}
	\label{\figlabel{cosmic_spikes:spectrum}}
\end{figure}

Two programs were used to additional correction of cosmic Ray, both of them
developed in MatLab programming environment (The Mathworks, Natick, MA, USA).
The first one was \emph{Spikie}
\parencite{Spikie2011}
was developed in scope of
\textcite{Klener2011}
and was based on detection of the sharp lines inside measured spectrum.
This program was used mainly for more complicated scenarios where the cosmic
ray lines interfered strongly with Raman bands of the measured samples.

The second program, \emph{Spycor}
\parencite{Spycor2018},
was created as part of this thesis and is based on comparisson of consecutive
spectra where it counts on the fact that during macroscopic Raman measurement
the spectra change only slowly with time and that the consecutive spectra are
similar.
It calculates standard deviation of specified number of spectra preceeding and
following the analyzed spectrum, multiplies it by the user defined constant
and uses it as a threshold for detection of spikes.
It then replaces the spike by average of the spectra used for the standard
deviation calibration. The spectrum from
\figref{cosmic_spikes:spectrum}
opened in the Spycor program can be seen in
\figref{cosmic_spikes:spycor}.

\begin{figure}
	\centering
	\ig{1}{results_and_discussion/assets/spike_removal/spycor}
	\caption[%
		Spycor -- program for spike removal.
	]{%
		\captiontitle{%
			Spycor -- program for spike removal.
		}
		The UVRR spectrum from \figref{cosmic_spikes:spectrum} opened in the
		Spycor program
		\parencite{Spycor2018}.
		The two spikes were identified and replaced by the average spectrum
		from the consecutive frames.
	}
	\label{\figlabel{cosmic_spikes:spycor}}
\end{figure}
