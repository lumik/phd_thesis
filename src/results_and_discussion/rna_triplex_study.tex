\section[\texorpdfstring{%
    PolyA and PolyU Complexes, effect of Mg\textsuperscript{2+}
}{%
    PolyA and PolyU Complexes, effect of
		Mg\texttwosuperior\textplussuperior
}]{%
    PolyA and PolyU Complexes, effect of Mg\textsuperscript{2+}
}%

UV RRS was used to study formation of complexes in mixtures of complementary
polynucleotides, polyA and polyU.
Special attention was given to the role of magnesium ions.
It was known that even relatively low concentrations of \ch{Mg^{2+}}
ions can stabilize the polyU$\cdot$polyA$\cdot$polyU triple helical structure,
even when the concentrations of polyU and polyA are equal, i.e., their
stoichiometry is ideal for formation of duplexes.
This effect was observed by UV absorption
\parencite{%
	Kankia2003,%
	Sorokin2003%
},
isothermal titration calorimetry, ultrasound velocimetry and densimetry
\parencite{Kankia2003},
as well as off-resonance Raman spectroscopy
\parencite{Herrera2010}.
Our work widened the NA concentration range of the \ch{Mg^{2+}} titrated
polyA + polyU mixture and provided well-resolved UVRR spectra of individual
components, i.e., single stranded polyA, single stranded polyU,
polyA$\cdot$polyU duplex, and polyU$\cdot$polyA$\cdot$polyU triplex, mutually
aligned in absolute intensity.

The polynucleotides were dissolved in a cacodylate buffer; its pH was adjusted
to 6.4.
1.5 mM concentrations in nucleic bases were used for measurements of resonance
Raman spectra excited at 244 nm.
The samples were annealed by heating to 90\,\textdegree{}C and slowly cooled
for 12 h to room temperature.
Before and after each Raman measurement, UV absorption spectrum was checked to
monitor possible photoinduced changes in the samples caused by the exciting
radiation.
The signals of the cell wall, water and cacodylate buffer were subtracted; the
spectrum of cacodylate was used as an internal intensity standard.
The remaining background was adjusted with cubic splines.
Then, the consecutively recorded spectra were linearly extrapolated to a zero
exposure time to exclude any contribution of possible photoproducts from the
final spectrum.

The basic UVRR spectra of single-stranded polyA and polyU, polyA$\cdot$polyU
double helices, and polyU$\cdot$polyA$\cdot$polyU triple helices
(\figref{rna_triplex:spectra})
were acquired in solution without magnesium ions, the double and triple helix
spectra from equimolar and 1:2-ratios of polyA$\cdot$polyU, respectively, at
which exclusively duplexes or triplexes are formed at the used ionic strength
\parencite{Stevens1964}.
Due to the use of an internal standard (cacodylate), the spectra were
normalized to the same intensity scale, which enables the correct construction
of differential spectra and an assessment of intensity changes.

\begin{figure}
	\centering
	\input{results_and_discussion/assets/rna_triplex/%
		rna_triplex_spectra}
	\caption[%
		Resonance Raman spectra (244\,nm excitation) of polyA, polyU and
		their 1:1 and 1:2 mixtures in aqueous solution.
	]{%
		\captiontitle{%
			Resonance Raman spectra (244\,nm excitation) of a) polyA, b) polyU and
			their c) 1:1 and d) 1:2 mixtures in aqueous solution.
		}%
		The spectra are normalized to the same concentration of phosphate units.
	}
	\label{\figlabel{rna_triplex:spectra}}
\end{figure}


The effects of duplex and triplex formation on UVRR spectra are obvious from
the difference spectra in
\figref{rna_triplex:difference}.
The observed spectral changes related to the duplex formation were mostly
consistent with the ones reported in non-resonance Raman spectra
\parencite{%
	OConnor1984,%
	Hanus1999%
}
or resonance Raman spectra excited at different UV wavelengths
\parencite{%
	Gfrorer1993a,%
	Perno1989,%
	Grygon1990%
},
with several exceptions originating from the specificity of RR spectra excited
at 244\,nm and/or from the better precision and spectral resolution of our
spectra.
A detailed analysis of the spectral changes was performed with respect to known
modifications in the vicinities of vibrating nucleoside moieties on the one
hand and interpretations of particular vibrational modes proposed in the
literature on the other hand.
Due to the reliability of both the Raman band positions and intensities in our
spectra, some interpretations found in the literature were confirmed and others
excluded.

\begin{figure}
	\centering
	\input{results_and_discussion/assets/rna_triplex/%
		rna_triplex_difference}
	\caption[%
		Difference resonance Raman spectra (244-nm excitation) calculated from
		those in
		\figref{rna_triplex:spectra}
		that reflect spectral effects of duplex formation from two single
		strands and triplex formation from the duplex and single-strand polyU.
	]{%
		\captiontitle{%
			Difference resonance Raman spectra (244-nm excitation) calculated from
			those in
			\figref{rna_triplex:spectra}
			that reflect spectral effects of a) duplex formation from two single
			strands and b) triplex formation from the duplex and single-strand polyU.
		}%
	}
	\label{\figlabel{rna_triplex:difference}}
\end{figure}

For the triplex formation, it was necessary to distinguish between the meaning
of the difference spectrum features for the adenosine and uridine bands.
In the case of adenosine, the changes originate from the polyA strand in the
triplex or the duplex structure and thus directly reflect changes caused by the
incorporation of the third strand via Hoogsteen base-pairing.
The main changes of the adenosine bands are intensity decreases indicating
hypochromism due to stronger stacking interaction, caused mainly by the
interaction between the imidazole part of adenine and the uracils in the third
strand.
The triplex formation also leads to frequency shifts of some adenosine bands.
Upshifts observed for the 1375, 1338 and 1177\,\icm{} adenosine vibrations
correlate with their known upshifts with H-bonding at hydrogen bond acceptor
sites
\parencite{Fujimoto1998}.
Because the adenine acceptor site that participates in the new polyU strand
bonding by Hoogsteen pairing is the N7 atom of adenine, we concluded that these
changes are due to the interaction at this site.

In the case of uridine bands, the difference spectrum indicates superposition
of the spectral changes caused in the first polyU strand, which was originally
present in the duplex, upon triplex formation and the spectral changes in the
second polyU strand, which was originally a free polyU strand, i.e.,
practically disordered.
We expected that the latter types of changes, which might be similar to the
ones observed in the difference spectrum for duplex formation, should dominate.
This expectation was true for the intensity changes of the 1396\,\icm{}
hypochromic band and the 1627\,\icm{} hyperchromic band, but significant
differences were observed for various uracil bands.
Most of them were the vibrational modes comprising motion of the C2O carbonyl
group (see
\figref{interpretation:structure}
for the numbering of atoms), which indicates different geometry and/or
H-bonding in this molecular region between the Watson-Crick paired and the
Hoogsteen paired uracils.
The most pronounced intensity change in the uridine spectrum was observed for
the band of ring-breathing vibration
\parencite{Fodor1985}
at 783\,\icm{}, which did not display any intensity variations during double
helix formation, but showed a strong hyperchromism upon triple helix formation.
To our knowledge, the literature does not provide any explanation for this
phenomenon.

The effect of \ch{Mg^{2+}} ions to the triplex formation was studied by means of
UV RRS titration experiment.
Magnesium ions were gradually added to an equimolar mixture of polyA and polyU
that originally contained only polyA$\cdot$polyU duplexes.
Each titration series (four series in total) was analyzed by the SVD algorithm,
and the second spectral component always showed a continuous linear increase in
its respective coefficients, following the increase of \ch{Mg^{2+}}
concentration.
The spectral shape of this component was almost identical with a difference
spectrum calculated from the spectra in
\figref{rna_triplex:spectra}
for the expected transition:
\begin{equation*}
	\ch{
		2 (polyA$\cdot$polyU) -> polyU$\cdot$polyA$\cdot$polyU + polyA
		%2(polyApolyU) -> polyUpolyApolyU + polyA
	}.
\end{equation*}

The only differences seemed to reflect some interaction of \ch{Mg^{2+}} ions
with the polyA incorporated into the triple helix and/or with the free polyA
strand.
This very good agreement proved the correctness of the spectral normalization
used for the spectra shown in
\figref{rna_triplex:spectra},
as the SVD results reflected the changes in the spectral shape independently of
the intensity scaling.
It was then possible to analyze the \ch{Mg^{2+}} titration by a fit to a sum of
the spectra from the set in
\figref{rna_triplex:spectra},
i.e., the spectra of polyA$\cdot$polyU duplex, polyU$\cdot$polyA$\cdot$polyU
triplex, and single strands.
The resulting duplex and triplex proportions indicated that the reaction
equilibrium followed an approximately linear dependence on the concentration of
\ch{Mg^{2+}} ions.
The equilibrium between the duplex and triplex form was reached for 12\,mM
concentration of magnesium ions.

Our results extend the data already published in a few works devoted to the
phenomenon of \ch{Mg^{2+}} induced triplex formation in an equimolar solution
of polyA and polyU.
The published data were obtained, however, for very different polynucleotide
concentrations.
Considering these results, we concluded that the mechanism of the \ch{Mg^{2+}}
function in the triple helix formation is complicated and comprises both
specific and nonspecific interactions.
The complete study has been published in \textcite{Klener2015}.
