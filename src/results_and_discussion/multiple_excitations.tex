\subsection{Redesign for multiple excitation wavelengths}

In the beginning, the spectrometer allowed only 244\,nm as the excitation beam
wavelength.
The next step in spectrometer development was to enable more excitation
wavelengths provided by the laser, as described in
\cref{subsec:focus_optimization}.
We decided to use prism optics for that purpose for the same reasons applied in
calibration beam optics selection, i.e., prisms have a very weak
dependence of reflectance on wavelength and are not susceptible to aging
compared to broadband metallic mirrors.
On the other side, if we used laser mirrors, we would be constrained only to
wavelengths covered by these mirrors (moreover it is more costly to have set of
mirrors for each excitation wavelength) and all the mirrors would need to be
changed when the different excitation wavelength was used.

Some method of removal of the unwanted frequencies from the excitation beam
also needed to be introduced because the laser manufacturer did not provide a
fundamental line light removal equipment for 229-nm excitation.
Pellin-Broca prism can be used for that purpose.
It has the advantage that there always exists a rotation angle of the prism
that the incoming and outgoing light deviates by exactly 90\textdegree{}, and
if the prism is rotated along its axis, the position of the outgoing beam at
90\textdegree{} does not change.
Moreover, the angle of incidence of the incoming light is near Brewster's
angle, so the amount of reflected light for p-polarization (our situation) is
small.

Overall, the laser mirror M1 was replaced by Pellin-Broca prism (PB), which
separated the excitation beam from unwanted light (for example, from
fundamental laser lines).
The Pellin-Broca prism was placed on a precise rotation stage which enabled the
selection of excitation wavelength, which was outgoing in a right-angle
direction to the beam from the laser.
The unwanted light was guided to the beam blocker (BB).
Mirrors M2 and M3 were replaced then by right angle prisms.
All these changes are depicted in
\figref{multiple_excitations:apparatus_schema}.

\begin{figure}
	\centering
	\input{results_and_discussion/assets/multiex_schema}
	\caption[%
		Top-view schema of the apparatus for multiple excitation wavelengths
		and with side-view inset of the sample space.%
	]{%
		\captiontitle{%
			Top-view schema of the apparatus for multiple excitation wavelengths
			and with side-view inset of the sample space.%
		}
		Prisms in total reflection mode replaced the right-angle laser mirrors
		optimized for 244\,nm excitation.
		M1 was replaced by Pellin-Broca prism PB, which separates unwanted
		frequencies from the excitation beam and sends them to the beam blocker BB.
		Right angle prisms replaced M2 and M3.
		The prism MC2 is flipped to the position for measurement, and the
		calibration lamp is switched off.
		The explanation of the rest of the symbols is the same as in
		\figref{wavenumber_calibration:apparatus_schema}.
	}
	\label{\figlabel{multiple_excitations:apparatus_schema}}
\end{figure}

Each new excitation wavelength required a new filter for elastically scattered
light removal.
We chose to support excitations at 257 and 229\,nm on top of the 244\,nm
excitation.
These excitation wavelengths covered the available range from frequency-doubled
\ch{Ar^+} ion laser.
It allowed us to take advantage of changes in resonance enhancements with
different excitation laser frequencies.
229\,nm excitation has the stronger enhancement of pyrimidine bases and can be
used to measure aromatic amino acids.
On the other hand, guanine is enhanced the most with 257\,nm.
The edge filter for 257-nm light removal was bought from Semrock, but there was
no filter available in the market for 229\,nm.
Josef Kapitan (Palackeho University, Olomouc) kindly provided one (Materion
Edge Filter) to us.
