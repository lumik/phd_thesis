\subsection{Redesign for multiple excitation wavelengths}

At the beginning, the spectrometer allowed to use only 244\,nm as the
excitation beam wavelength. Next step in spectrometer development was to
enable more excitation wavelengths provided by the laser as described in
\cref{subsec:focus_optimization}.
We decided to use prism optics for that purpose from the same reasons as we
chose it for guiding the calibration beam, i.e. prisms have very weak
dependence of reflectance on wavelenght and are not susceptible to aging
compared to broadband metalic mirrors. On the other side, if we used laser
mirrors, we will be constrained only to wavelengths which are covered by
these mirrors (moreover it is more costly to have set of mirrors for each
excitation wavelength) and all the mirrors needs to be changed when different
excitation wavelength is used.

Some method of removal of the unwanted frequencies from the excitation beam
also needed to be introduced because the laser manufacturer didn't provide
fundamental line light removal equipment for 229-nm excitation. Pellin-Broca
prism can be used for that purpose. It has advantage that there always exist
rotation angle of the prism that the incoming and outgoing light deviates by
exactly 90\textdegree{} and if you rotate the prism along its axis the position
of outgoing beam at 90\textdegree{} doesn't change. Moreover, the angle of
incidence of incoming light is near to Brewster angle so the amount of
reflected light for p-polarization (our situation) is small.

Overall, the laser mirror M1 was replaced by Pellin-Broca prism (PB) which
separated the excitation beam from unwanted light (for example from fundamental
laser lines). The Pellin-Broca prism was placed on precise rotation stage
which enabled the selection of excitation wavelength which was outgoing in
right angle direction to the beam from laser. The unwanted light was guided to
the beam blocker (BB). Mirrors M2 and M3 were replaced then by right angle
prisms. All these changes are depicted in
\figref{multiple_excitations:apparatus_schema}.

\begin{figure}
	\centering
	\input{results_and_discussion/assets/multiex_schema}
	\caption{Top-view schema of the apparatus for multiple excitation wavelengths
		and with side-view inset of the sample space. The right-angle laser
		mirrors optimized for 244-nm excitation were replaced by prisms in total
		reflection mode. M1 was replaced by Pellin-Broca prism PB which separates
		unwanted frequencies from the excitation beam and sends them to the beam
		blocker BB. M2 and M3 were replaced by right angle prisms. The prism MC2 is
		flipped to the position for measurement and calibration lamp is switched
		off. The explanation of rest of the symbols is the same as in
		\figref{wavenumber_calibration:apparatus_schema}.}
	\label{\figlabel{multiple_excitations:apparatus_schema}}
\end{figure}

New filters for elastically scattered light removal needed to be used for each
new excitation wavelenght. We chose to support excitations at 257 and 229\,nm
on top of the 244-nm excitation.
\MISSING \{discussion about usability of different excitation wavelengths\}
The edge filter for 257-nm light removal was bought from Semrock but there was
no filter available in the market for 229\,nm but Josef Kapitan (Palackeho
University, Olomouc) kindly provided one (Materion Edge Filter) to us.
