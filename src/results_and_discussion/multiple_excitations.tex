\subsection{Redesign for multiple excitation wavelengths}

At the beginning, the spectrometer allowed to use only 244\,nm as the
excitation beam wavelength. Next step in spectrometer development was to
enable more excitation wavelengths provided by the laser as described in
\cref{focus_optimization}.
We decided to use prism optics for that purpose from the same reasons as we
chose it for guiding the calibration beam, i.e. prisms have very weak
dependence of reflectance on wavelenght and are not susceptible to aging
compared to broadband metalic mirrors. On the other side, if we used laser
mirrors, we will be constrained only to wavelengths which are covered by
these mirrors (moreover it is more costly to have set of mirrors for each
excitation wavelength) and all the mirrors needs to be changed when different
excitation wavelength is used.

Some method of removal of the unwanted frequencies from the excitation beam
also needed to be introduced because the laser manufacturer didn't provide
fundamental line light removal equipment for 229-nm excitation. Pellin-Broca
prism can be used for that purpose. It has advantage that there always exist
rotation angle of the prism that the incoming and outgoing light deviates by
exactly 90$^\circ$ and if you rotate the prism along its axis the position of
outgoing beam at 90$^\circ$ doesn't change. Moreover, the angle of incidence of
incoming light is near to Brewster angle so the amount of reflected light for
p-polarization (our situation) is small.

Overall, the laser mirror M1 was replaced by Pellin-Broca prism (PB) which
separated the excitation beam from unwanted light (for example from fundamental
laser lines). The Pellin-Broca prism was placed on precise rotation stage
which enabled the selection of excitation wavelength which was outgoing in
right angle direction to the beam from laser. The unwanted light was guided to
the beam blocker (BB). Mirrors M2 and M3 were replaced then by right angle
prisms. All these changes are depicted in
\figref{multiple_excitations:apparatus_schema}.

\begin{figure}
	\centering
	\begin{tikzpicture}[font=\sffamily]

% settings
\newcommand*{\cellBorderWidth}{3\pgflinewidth}
\newcommand*{\cellLineWidth}{1.5\pgflinewidth}
\definecolor{glassBorderColor}{RGB}{0,128,255}
\definecolor{glassFillColor}{RGB}{230,242,255}
\definecolor{waterFillColor}{RGB}{0,128,255}
\definecolor{hclColor}{RGB}{255,128,0}
\definecolor{unwantedLightColor}{RGB}{153,17,0}
\tikzset{
	clip
}
\tikzset{
	mirror element/.style = {color = black, line width = 2 * \pgflinewidth},
	real laser beam/.style = {color = cyan, line width = 2 * \pgflinewidth},
	laser beam/.style = {real laser beam},
	scattered ray/.style = {color = red!60, line width = 1.5 * \pgflinewidth},
	scattered fill/.style = {fill = red, draw = none, fill opacity = 0.2},
	glass/.style = {color = glassBorderColor, opacity = 0.5,%
		fill = glassFillColor, fill opacity = 0.5},
	sample cell/.style = {color = glassBorderColor, opacity = 0.5,%
		double = glassFillColor, double distance = \cellBorderWidth,
		line width = \cellLineWidth, line cap = rect},
	water fill/.style = {fill = waterFillColor, fill opacity = 0.1},
	mirror surface/.style = {color = black!20, fill = black!10},
	nd filter/.style = {color = black, opacity = 0.2, fill = black,%
		fill opacity = 0.1},
	nd carousel/.style = {color = black, opacity = 0.4, fill = black,%
		fill opacity = 0.2},
	notch/.style = {color = black, opacity = 0.4, fill = black,%
		fill opacity = 0.1},
	aperture/.style = {color = black, line width = 2 * \pgflinewidth},
	aperture filldraw/.style = {color = black!40, fill = black!20,%
		clip even odd rule},
	clip even odd rule/.code = {\pgfseteorule},
	shutter/.style = {nd carousel},
	shutter blade/.style = {dashed, line width = 1.5 * \pgflinewidth,%
		opacity = 0.4},
	hcl/.style = {nd carousel},
	hcl lamp/.style = {color = black!40, line width = 3 * \pgflinewidth,%
		line cap = round},
	hcl ray/.style = {color = black, opacity = 0.2,%
		line width = 1.5 * \pgflinewidth},
	hcl beam/.style = {draw = none, fill = black, opacity = 0.1},
	beam blocker/.style = {mirror element},
	unwanted ray/.style = {color = unwantedLightColor,%
		line width = 1.5 * \pgflinewidth}
};
\clip (-.1,-2.1) rectangle (14,6.6);
\coordinate (shutter) at (11,6);
\newcommand*{\samplePosWidth}{10}
\newcommand*{\samplePosHeight}{3}
\coordinate (M1) at (13,6);
\coordinate (M2) at (13,\samplePosHeight);
\coordinate (BeamBlocker) at ($ (M2) + (-0.35,1.2) $);
\newcommand*{\beamBlockerWidth}{0.5}
\coordinate (samplePos) at (\samplePosWidth,\samplePosHeight);
\newcommand*{\cellWidth}{1};
\coordinate (cassegrainM1Center) at (\samplePosWidth - 0.5,\samplePosHeight);
\newcommand*{\cassegrainMARadius}{1.5}
\coordinate (cassegrainM2Center) at (\samplePosWidth - 0.7,\samplePosHeight);
\newcommand*{\cassegrainMBRadius}{0.4}
\newcommand*{\sqrttwo}{1.414213}
\coordinate (HCL) at (6,0);
\coordinate (AC1) at (6,1.2);  % calibration aperture
\coordinate (LC1) at (6,0.6);  % calibration lens
\coordinate (MS1Edge1) at (4.5,\samplePosHeight + 0.5);
\coordinate (MS1Edge2) at (3.5,\samplePosHeight - 0.5);
\coordinate (MC1Edge1) at (5.5,2.5);
\coordinate (MC1Edge2) at (6.5,1.5);
\coordinate (MC1Edge3) at (5.5,1.5);
\coordinate (MC2FlippedEdge1) at (4.4,2.5);
\newcommand*{\MCBFlippedH}{1.0};  % MC2 height
\newcommand*{\MCBFlippedD}{1.0};  % MC2 depth
\coordinate (parabolaFocus) at (3,0.4);
\coordinate (MS2Edge1) at (4.5,1);
\coordinate (MS2EdgeControl1) at (245:0.5);
\coordinate (MS2Edge2) at (3.5,0);
\coordinate (MS2EdgeControl2) at (17:0.3);
\newcommand*{\apertureOuterRadius}{0.3}
\newcommand*{\apertureInnerRadius}{0.1}
\newcommand*{\pbrot}{305}  % Pellin broca prism rotation

% Pellin-Broca prism path def
% B -----
% |      ----C
% |           \
% |            \
% A ----------- D
\coordinate (PBB) at ($ (M1) + (0.06,0.2) $);
\coordinate (PBA) at ($ (PBB) + (270 + \pbrot:0.6) $);
\coordinate (PBD) at ($ (PBA) + (0 + \pbrot:0.9) $);
\path[name path=PBBtoC] (PBB) -- ++(345 + \pbrot:10);
\path[name path=PBDtoC] (PBD) -- ++(120 + \pbrot:10);
\path[name intersections={of=PBBtoC and PBDtoC, by=PBC}];
\path[name path=PBAB] (PBA) -- (PBB);
\path[name path=PBBC] (PBB) -- (PBC);
\path[name path=PBDA] (PBD) -- (PBA);

% laser
\draw (0,5.5) rectangle ++(3,1) node[pos=.5] {laser};

% laser beam Pellin-Broca coordinates
\path[name path=LToPBAB] (shutter) -- (M1);
\path[name intersections = {of = LToPBAB and PBAB, by = PBAB1}];
\draw[real laser beam] (3,6) -- (shutter);
\path[name path=LToPBBC] (PBAB1) -- ++(340:2);
\path[name intersections = {of = LToPBBC and PBBC, by = PBBC1}];
\path[name path=LToPBDA] (M2) -- (M1);
\path[name intersections = {of = LToPBDA and PBDA, by = PBDA1}];

% Unvanted laser beam
\path[name path=ULToPBBC] (PBAB1) -- ++(347:2);
\path[name intersections = {of = ULToPBBC and PBBC, by = UPBBC1}];
\path[name path=ULToPBDA] (M2) -- ($ (M1) + (-0.12,0) $);
\path[name intersections = {of = ULToPBDA and PBDA, by = UPBDA1}];

% Draw unwanted beam
\draw[unwanted ray] (PBAB1) -- (UPBBC1) -- (UPBDA1)
	-- ($ (BeamBlocker) + (0.08,0) $);

% Draw laser beam
\draw[laser beam] (shutter) -- (PBAB1) -- (PBBC1) -- (PBDA1) -- (M2)
	-- (samplePos);

% neutral density filters
\newcommand*{\ndfilterA}{(3.5,5.8) rectangle ++(0.2,0.4)}
\newcommand*{\ndfilterB}{(3.5,5) rectangle ++(0.2,0.4)}
\draw[nd filter] \ndfilterA;
\draw[nd filter] \ndfilterB;
\draw[nd carousel] (3.45,4.9) rectangle ++(0.3,0.1);
\draw[nd carousel] (3.45,5.4) rectangle ++(0.3,0.4);
\draw[nd carousel] (3.45,6.2) rectangle ++(0.3,0.1);
\node[below] at (3.6,4.9) {ND};

\draw[shutter blade] ($ (shutter) + (0,0.3) $) -- ++(0,-0.6);
\draw[shutter] ($ (shutter) + (-0.1,-0.3) $) -- ++(0.2,0) -- ++(0,-0.4)
	-- ++(-0.2,0) -- cycle;
\node[below] at ($ (shutter) + (0,-0.7) $) {shutter};

% Pellin-Broca prism
\draw[glass] (PBA) -- (PBB) -- (PBC) -- (PBD) -- cycle;
\node[above, shift = {(0.25cm,-0.05cm)}] at (PBB) {PB};
% Mirror 2
\draw[glass] ($ (M2) + (0.2,0.2) $)
	-- node[below, shift = {(0.35cm,0.05cm)}, color = black, opacity = 1.0]{M2}
	($ (M2) + (-0.2,-0.2) $) -- ++(0,0.4) -- cycle;
% Mirror 3 - it should be under the beam so we have to draw it first
%\draw[glass] ($ (samplePos) + (-0.2,0.2) $) rectangle ++(0.4,-0.4);
%\node[above] at ($ (samplePos) + (0,0.5) $) {M3};


% Unwanted frequencies beam blocker
\draw[beam blocker] ($ (BeamBlocker) + (0.5 * \beamBlockerWidth,0) $)
	-- ++(-\beamBlockerWidth,0) node[left] {BB};

% Aperture 1
\draw[aperture] ($ (M2) + (-0.5,\apertureOuterRadius) $)
	node[above]{A1}
	-- ++(0,-\apertureOuterRadius + \apertureInnerRadius);
\draw[aperture] ($ (M2) + (-0.5,-\apertureOuterRadius) $)
	-- ++(0,\apertureOuterRadius - \apertureInnerRadius);
% aperture 2
\draw[aperture filldraw]
	(samplePos) circle (\apertureOuterRadius)
	(samplePos) circle (\apertureInnerRadius);

% Draw laser focusing lens
\newcommand*{\LARadius}{0.7}
\coordinate (L1Center) at (\samplePosWidth+1.6-\LARadius,\samplePosHeight);
\path[name path=L1Arc, shift={(L1Center)}]
	(270:\LARadius) arc (-90:90:\LARadius);
\path[name path=toL1Arc1] ($ (samplePos)  + (0,0.4) $) -- ++(10,0);
\path[name path=toL1Arc2] ($ (samplePos)  + (0,-0.4) $) -- ++(10,0);
\path[name intersections={of=L1Arc and toL1Arc1, by=L11}];
\mypgfextractangle{\LAAAngle}{L1Center}{L11}
\path[name intersections={of=L1Arc and toL1Arc2, by=L12}];
\mypgfextractangle{\LABAngle}{L1Center}{L12}
\draw[glass] (L11) arc (\LAAAngle:\LABAngle-360:\LARadius) -- ++(-0.05,0)
	-- ($ (L11) + (-0.05,0) $) -- cycle;
\node[above] at (L11) {L1};


%%%%%%%%%%%%%%%%%%%%%
% draw the cassegrain

% calculate intersections with mirror 1 (the objective mirror)
% mirror1 arc
\path[name path=M1arc, shift={(cassegrainM1Center)}]
	(90:\cassegrainMARadius) arc (90:270:\cassegrainMARadius);
% upper top ray
\path[name path=toCassegrainM11] (samplePos) -- ++(135:5);
\path[name intersections={of=M1arc and toCassegrainM11, by=cassegrainM11}];
\mypgfextractangle{\cassegrainMAAAngle}{cassegrainM1Center}{cassegrainM11}
% upper bottom ray
\path[name path=toCassegrainM12] (samplePos) -- ++(165:5);
\path[name intersections={of=M1arc and toCassegrainM12, by=cassegrainM12}];
\mypgfextractangle{\cassegrainMABAngle}{cassegrainM1Center}{cassegrainM12}
% lower top ray
\path[name path=toCassegrainM13] (samplePos) -- ++(195:5);
\path[name intersections={of=M1arc and toCassegrainM13, by=cassegrainM13}];
\mypgfextractangle{\cassegrainMACAngle}{cassegrainM1Center}{cassegrainM13}
% lower bottom ray
\path[name path=toCassegrainM14] (samplePos) -- ++(225:5);
\path[name intersections={of=M1arc and toCassegrainM14, by=cassegrainM14}];
\mypgfextractangle{\cassegrainMADAngle}{cassegrainM1Center}{cassegrainM14}

\draw[scattered ray] (samplePos) -- (cassegrainM11);
\draw[scattered ray] (samplePos) -- (cassegrainM12);
\draw[scattered fill] (samplePos) -- (cassegrainM11)
	arc (\cassegrainMAAAngle:\cassegrainMABAngle:\cassegrainMARadius) -- cycle;
\draw[scattered ray] (samplePos) -- (cassegrainM13);
\draw[scattered ray] (samplePos) -- (cassegrainM14);
\draw[scattered fill] (samplePos) -- (cassegrainM13)
	arc (\cassegrainMACAngle:\cassegrainMADAngle:\cassegrainMARadius) -- cycle;

% draw the cell
\draw[sample cell, water fill]
	($ (samplePos)%
		+ (-0.5 * \cellBorderWidth - \cellLineWidth,-0.5 * \cellWidth) $)
		rectangle ++(\cellWidth,\cellWidth);
\node[below] at ($ (samplePos) + (0.5 * \cellWidth,-0.5 * \cellWidth)%
	+ (-0.5 * \cellBorderWidth,0) + (-0.5 * \cellLineWidth,0) $) {S};

% calculate intersections with mirror 2 (the objective mirror)
% mirror2 arc
\path[
	name path=M2arc, shift={(cassegrainM2Center)}] (90:\cassegrainMBRadius)
		arc (90:270:\cassegrainMBRadius);
% calculate cassegrain mirror2 edges
\path[
	name intersections={of=M2arc and toCassegrainM12, by=cassegrainM2Edge1}];
\mypgfextractangle{\cassegrainMBAAngle}{cassegrainM2Center}{cassegrainM2Edge1}
\path[
	name intersections={of=M2arc and toCassegrainM13, by=cassegrainM2Edge2}];
\mypgfextractangle{\cassegrainMBDAngle}{cassegrainM2Center}{cassegrainM2Edge2}
% upper bottom ray
\path[name path=toCassegrainM22] ($ (samplePos) + (0,0.1) $) -- ++(-10,0);
\path[name intersections={of=M2arc and toCassegrainM22, by=cassegrainM22}];
\mypgfextractangle{\cassegrainMBBAngle}{cassegrainM2Center}{cassegrainM22}
% lower top ray
\path[name path=toCassegrainM23] ($ (samplePos) + (0,-0.1) $) -- ++(-10,0);
\path[name intersections={of=M2arc and toCassegrainM23, by=cassegrainM23}];
\mypgfextractangle{\cassegrainMBCAngle}{cassegrainM2Center}{cassegrainM23}

\draw[scattered ray] (cassegrainM11) -- (cassegrainM2Edge1);
\draw[scattered ray] (cassegrainM12) -- (cassegrainM22);
\draw[scattered fill] (cassegrainM2Edge1)
	arc (\cassegrainMBAAngle:\cassegrainMBBAngle:\cassegrainMBRadius)
		-- (cassegrainM12)
	arc (\cassegrainMABAngle:\cassegrainMAAAngle:\cassegrainMARadius) -- cycle;
\draw[scattered ray] (cassegrainM13) -- (cassegrainM23);
\draw[scattered ray] (cassegrainM14) -- (cassegrainM2Edge2);
\draw[scattered fill] (cassegrainM23)
	arc (\cassegrainMBCAngle:\cassegrainMBDAngle:\cassegrainMBRadius)
		-- (cassegrainM14)
	arc (\cassegrainMADAngle:\cassegrainMACAngle:\cassegrainMARadius) -- cycle;

% to mirror MS1
% path representing the mirror
\path[name path=MS1Path] (MS1Edge1) -- (MS1Edge2);
% intersections with the mirror
% ray 1
\path[name path=toCassegrainM21] (cassegrainM2Edge1) -- ++(-10,0);
\path[name intersections={of=MS1Path and toCassegrainM21, by=MS11}];
% ray 2
\path[name intersections={of=MS1Path and toCassegrainM22, by=MS12}];
% ray 3
\path[name intersections={of=MS1Path and toCassegrainM23, by=MS13}];
% ray 4
\path[name path=toCassegrainM24] (cassegrainM2Edge2) -- ++(-10,0);
\path[name intersections={of=MS1Path and toCassegrainM24, by=MS14}];

\draw[scattered ray] (cassegrainM2Edge1) -- (MS11);
\draw[scattered ray] (cassegrainM22) -- (MS12);
\draw[scattered fill] (cassegrainM2Edge1)
	arc (\cassegrainMBAAngle:\cassegrainMBBAngle:\cassegrainMBRadius) -- (MS12)
		-- (MS11) -- cycle;
\draw[scattered ray] (cassegrainM23) -- (MS13);
\draw[scattered ray] (cassegrainM2Edge2) -- (MS14);
\draw[scattered fill] (cassegrainM23)
	arc (\cassegrainMBCAngle:\cassegrainMBDAngle:\cassegrainMBRadius) -- (MS14)
		-- (MS13) -- cycle;

% draw first mirror of cassegrain
\draw[mirror element]
	(cassegrainM11)
		arc (\cassegrainMAAAngle:\cassegrainMABAngle:\cassegrainMARadius)
			node[left,pos=0.5] {O};
\draw[mirror element]
	(cassegrainM13)
		arc (\cassegrainMACAngle:\cassegrainMADAngle:\cassegrainMARadius);
% mirror 2
\draw[mirror element]
	(cassegrainM2Edge1)
		arc (\cassegrainMBAAngle:\cassegrainMBDAngle:\cassegrainMBRadius);

% draw notch
\draw[notch] ($ (samplePos) + (-3,-0.5) $) rectangle ++(0.2,1);
\node[above] at ($ (samplePos) + (-2.9,0.5) $) {EF};

% Parabolic mirror MS2
\newcommand*{\parabolicMirrorDef}{%
		(MS2Edge2)
		.. controls ($ (MS2Edge2) + (MS2EdgeControl2) $)
			and ($ (MS2Edge1) + (MS2EdgeControl1) $)
		.. (MS2Edge1)
}
\path[name path=MS2Path] \parabolicMirrorDef;
% ray 1
\path[name path=toMS21] (MS11) -- ++(0,-10);
\path[name intersections={of=MS2Path and toMS21, by=MS21}];
% ray 2
\path[name path=toMS22] (MS12) -- ++(0,-10);
\path[name intersections={of=MS2Path and toMS22, by=MS22}];
% ray 3
\path[name path=toMS23] (MS13) -- ++(0,-10);
\path[name intersections={of=MS2Path and toMS23, by=MS23}];
% ray 4
\path[name path=toMS24] (MS14) -- ++(0,-10);
\path[name intersections={of=MS2Path and toMS24, by=MS24}];

\draw[scattered ray] (MS11) -- (MS21);
\draw[scattered ray] (MS12) -- (MS22);
\begin{scope}
	\clip (MS11) -- ($ (MS21) + (0,-1) $) -- ($ (MS22) + (0,-1) $) -- (MS12)
		-- cycle;
	\draw[scattered fill] (MS12) -- \parabolicMirrorDef -- (MS11) -- cycle;
\end{scope}
\draw[scattered ray] (MS13) -- (MS23);
\draw[scattered ray] (MS14) -- (MS24);
\begin{scope}
	\clip (MS13) -- ($ (MS23) + (0,-1) $) -- ($ (MS24) + (0,-1) $) -- (MS14)
		-- cycle;
	\draw[scattered fill] (MS14) -- \parabolicMirrorDef -- (MS13) -- cycle;
\end{scope}

% draw MS1 over all incident rays on that mirror
\draw[mirror element]
	(MS1Edge1) -- node[above,shift={(-0.5cm,-0.1cm)}]{MS1} (MS1Edge2);

% Draw calibration source
\newcommand*{\hclBeamHalfWidth}{0.25}
\draw [hcl] ($ (HCL) + (-0.3,0) $) rectangle ++(0.6,-1);
\node[right] at ($ (HCL) + (0.3,-0.5) $) {HCL};
\draw [hcl lamp] ($ (HCL) + (-0.1 + 0.02, 0) $) --
	++(0.2 - 0.04,0);

% path representing calibration lens surface
\newcommand*{\LCARadius}{0.8}
\coordinate (LC1Center) at ($ (LC1) + (0,\LCARadius) $);
\path[name path=LC1Arc, shift={(LC1Center)}]
	(180:\LCARadius) arc (-180:0:\LCARadius);
% ray 1
\path[name path = toMC11] ($ (HCL) + (-\hclBeamHalfWidth, 0) $) -- ++(0,5);
\path[name intersections = {of = LC1Arc and toMC11, by = LC11}];
\draw[hcl ray] (HCL) -- (LC11);
% ray 2
\path[name path = toMC12] ($ (HCL) + (\hclBeamHalfWidth, 0) $) -- ++(0,5);
\path[name intersections = {of = LC1Arc and toMC12, by = LC12}];
\draw[hcl ray] (HCL) -- (LC12);

% path representing the mirror MC1
\path[name path = MC1Path] (MC1Edge1) -- (MC1Edge2);
% ray 1
\path[name intersections = {of = MC1Path and toMC11, by = MC11}];
\draw[hcl ray] (LC11) -- (MC11);
% ray 2
\path[name intersections = {of = MC1Path and toMC12, by = MC12}];
\draw[hcl ray] (LC12) -- (MC12);

% fill the beam from source to prism MC1
\draw[hcl beam] (HCL) -- (LC11) -- (MC11) -- (MC12) -- (LC12) -- cycle;

% Draw calibration lens
\path[name path=toLC1Arc1] ($ (LC1) + (-0.5,0) $) -- ++(0,1);
\path[name path=toLC1Arc2] ($ (LC1)  + (0.5,0) $) -- ++(0,1);
\path[name intersections={of=LC1Arc and toLC1Arc1, by=LC11}];
\mypgfextractangle{\LCAAAngle}{LC1Center}{LC11}
\path[name intersections={of=LC1Arc and toLC1Arc2, by=LC12}];
\mypgfextractangle{\LCABAngle}{LC1Center}{LC12}
\draw[glass] (LC11) arc (\LCAAAngle:\LCABAngle:\LCARadius) -- ++(0,0.05)
	-- ($ (LC11) + (0,0.05) $) -- cycle;
\node[right] at ($ (LC1) + (0.5,0.1) $) {LC1};

% Calibration aperture 1
\draw[aperture] ($ (AC1) + (-0.5,0) $) -- ($ (AC1) + (-0.3,0) $);
\draw[aperture] ($ (AC1) + (0.3,0) $) -- ($ (AC1) + (0.5,0) $)
	node[right]{AC1};

% path representing the mirror MC2
\path[name path = MC2Path] (MC2FlippedEdge1) -- ++(0,-\MCBFlippedH);
% ray 1
\path[name path = toMC21] (MC11) -- ++(-4,0);
\path[name intersections = {of = MC2Path and toMC21, by = MC21}];
\draw[hcl ray] (MC11) -- (MC21);
% ray 2
\path[name path = toMC22] (MC12) -- ++(-4,0);
\path[name intersections = {of = MC2Path and toMC22, by = MC22}];
\draw[hcl ray] (MC12) -- (MC22);
% fill the beam to prism MC2
\draw[hcl beam] (MC11) -- (MC21) -- (MC22) -- (MC12) -- cycle;

% draw the calibration prism MC1
\draw[glass] (MC1Edge1) -- (MC1Edge2) -- (MC1Edge3) -- cycle;
\node[right] at ($ (MC1Edge1) + (0.5,-0.4) $) {MC1};

% draw the calibration prism MC2
\draw[glass] (MC2FlippedEdge1) -- ++(0,-\MCBFlippedH)
	-- node[below, color = black, opacity=1.0] {MC2}
	++(\MCBFlippedD,0) -- ++(0,\MCBFlippedH) -- cycle;

% draw scattered beam from MS1 to parabolic mirror MS2
\draw[scattered ray] (MS21) -- (parabolaFocus);
\draw[scattered ray] (MS22) -- (parabolaFocus);
\begin{scope}
	\clip (parabolaFocus) -- (MS21) -- ++(1,0) -- ($ (MS22) + (1,0) $) -- (MS22)
		-- cycle;
	\draw[scattered fill] (parabolaFocus) -- \parabolicMirrorDef -- cycle;
\end{scope}
\draw[scattered ray] (MS23) -- (parabolaFocus);
\draw[scattered ray] (MS24) -- (parabolaFocus);
\begin{scope}
	\clip (parabolaFocus) -- (MS23) -- ++(1,0) -- ($ (MS24) + (1,0) $) -- (MS24)
		-- cycle;
	\draw[scattered fill] (parabolaFocus) -- \parabolicMirrorDef -- cycle;
\end{scope}

% draw the parabolic mirror MS2
\draw[mirror element]
	(MS2Edge2)
	.. controls ($ (MS2Edge2) + (MS2EdgeControl2) $)
		and ($ (MS2Edge1) + (MS2EdgeControl1) $)
	.. node[below,shift={(0.5cm,0.1cm)}]{MS2} (MS2Edge1);

\draw (0,0) rectangle ++(3,2.5) node[pos=.5] {spectrograph};

% side view
\begin{scope}[shift={(\samplePosWidth - 2,-1.7)}]

\newcommand*{\samplePosSideWidth}{2}
\newcommand*{\samplePosSideHeight}{2.5}
\coordinate (samplePosSide) at (\samplePosSideWidth,\samplePosSideHeight);
\coordinate (cassegrainM1SideCenter)
	at (\samplePosSideWidth - 0.5,\samplePosSideHeight);
\coordinate (cassegrainM2SideCenter)
	at (\samplePosSideWidth - 0.7,\samplePosSideHeight);

% clip the view
\clip (-.05,-.35) rectangle ++(3.3,3.4);
\draw (-.05,-.35) rectangle ++(3.3,3.4);

% laser beam
\draw[laser beam] (4,0) -- (\samplePosSideWidth,0) coordinate (M3Side)
	-- (samplePosSide);

% mirror 3
%\draw[mirror element] ($ (M3Side) + (-0.3,0.3) $)
%	-- node[above,shift={(0.35cm,-0.05cm)}]{M3} ($ (M3Side) + (0.3,-0.3) $);
\draw[glass] ($ (M3Side) + (-0.2,0.2) $) -- ++(0.4,0)
	node[above right,shift = {(-0.1cm,-0.1cm)}, color = black, opacity = 1.0]{M3}
	-- ++(0,-0.4)
	-- cycle;

% aperture 2
\draw[aperture] ($ (M3Side) + (\apertureOuterRadius,0.8) $)
	node[above, shift={(0.05,0)}]{A2}
	-- ++(-\apertureOuterRadius + \apertureInnerRadius,0);
\draw[aperture] ($ (M3Side) + (-\apertureOuterRadius,0.8) $)
	-- ++(\apertureOuterRadius - \apertureInnerRadius,0);

% draw the cassegrain
% calculate intersections with mirror 1 (the objective mirror)
% mirror1 arc
\path[name path=M1SideArc, shift={(cassegrainM1SideCenter)}]
	(90:\cassegrainMARadius) arc (90:270:\cassegrainMARadius);
% upper top ray
\path[name path=toCassegrainM11Side] (samplePosSide) -- ++(135:5);
\path[name intersections={of=M1SideArc and toCassegrainM11Side,
	by=cassegrainM11Side}];
% upper bottom ray
\path[name path=toCassegrainM12Side] (samplePosSide) -- ++(165:5);
\path[name intersections={of=M1SideArc and toCassegrainM12Side,
	by=cassegrainM12Side}];
% lower top ray
\path[name path=toCassegrainM13Side] (samplePosSide) -- ++(195:5);
\path[name intersections={of=M1SideArc and toCassegrainM13Side,
	by=cassegrainM13Side}];
% lower bottom ray
\path[name path=toCassegrainM14Side] (samplePosSide) -- ++(225:5);
\path[name intersections={of=M1SideArc and toCassegrainM14Side,
	by=cassegrainM14Side}];

\draw[scattered ray] (samplePosSide) -- (cassegrainM11Side);
\draw[scattered ray] (samplePosSide) -- (cassegrainM12Side);
\draw[scattered fill] (samplePosSide) -- (cassegrainM11Side)
	arc (\cassegrainMAAAngle:\cassegrainMABAngle:\cassegrainMARadius) -- cycle;
\draw[scattered ray] (samplePosSide) -- (cassegrainM13Side);
\draw[scattered ray] (samplePosSide) -- (cassegrainM14Side);
\draw[scattered fill] (samplePosSide) -- (cassegrainM13Side)
	arc (\cassegrainMACAngle:\cassegrainMADAngle:\cassegrainMARadius) -- cycle;

% draw the cell
\draw[sample cell, water fill]
	($ (samplePosSide) + (-0.5 * \cellBorderWidth - \cellLineWidth,%
		-0.5 * \cellBorderWidth - \cellLineWidth) $)
		rectangle ++(\cellWidth,10);

% calculate intersections with mirror 2 (the objective mirror)
% mirror2 arc
\path[
	name path=M2SideArc, shift={(cassegrainM2SideCenter)}] (90:\cassegrainMBRadius)
		arc (90:270:\cassegrainMBRadius);
% calculate cassegrain mirror2 edges
\path[
	name intersections={%
		of=M2SideArc and toCassegrainM12Side, by=cassegrainM2SideEdge1}];
\path[
	name intersections={%
		of=M2SideArc and toCassegrainM13Side, by=cassegrainM2SideEdge2}];
% upper bottom ray
\path[name path=toCassegrainM22Side]%
	($ (samplePosSide) + (0,0.1) $) -- ++(-10,0);
\path[name intersections={%
	of=M2SideArc and toCassegrainM22Side, by=cassegrainM22Side}];
% lower top ray
\path[name path=toCassegrainM23Side]
	($ (samplePosSide) + (0,-0.1) $) -- ++(-10,0);
\path[name intersections={%
	of=M2SideArc and toCassegrainM23Side, by=cassegrainM23Side}];

\draw[scattered ray] (cassegrainM11Side) -- (cassegrainM2SideEdge1);
\draw[scattered ray] (cassegrainM12Side) -- (cassegrainM22Side);
\draw[scattered fill] (cassegrainM2SideEdge1)
	arc (\cassegrainMBAAngle:\cassegrainMBBAngle:\cassegrainMBRadius)
		-- (cassegrainM12Side)
	arc (\cassegrainMABAngle:\cassegrainMAAAngle:\cassegrainMARadius) -- cycle;
\draw[scattered ray] (cassegrainM13Side) -- (cassegrainM23Side);
\draw[scattered ray] (cassegrainM14Side) -- (cassegrainM2SideEdge2);
\draw[scattered fill] (cassegrainM23Side)
	arc (\cassegrainMBCAngle:\cassegrainMBDAngle:\cassegrainMBRadius)
		-- (cassegrainM14Side)
	arc (\cassegrainMADAngle:\cassegrainMACAngle:\cassegrainMARadius) -- cycle;

% to mirror MS1
% ray 1
\draw[scattered ray] (cassegrainM2SideEdge1) -- ++(-10,0);
\draw[scattered ray] (cassegrainM22Side) -- ++(-10,0);
\draw[scattered fill] (cassegrainM2SideEdge1)
	arc (\cassegrainMBAAngle:\cassegrainMBBAngle:\cassegrainMBRadius)
		-- ++(-10,0) -- ($ (cassegrainM2SideEdge1) + (-10,0) $) -- cycle;
\draw[scattered ray] (cassegrainM23Side) -- ++(-10,0);
\draw[scattered ray] (cassegrainM2SideEdge2) -- ++(-10,0);
\draw[scattered fill] (cassegrainM23Side)
	arc (\cassegrainMBCAngle:\cassegrainMBDAngle:\cassegrainMBRadius) --
		++(-10,0) -- ($ (cassegrainM2SideEdge1) + (-10,0) $) -- cycle;

% draw first mirror of cassegrain
\draw[mirror element]
	(cassegrainM11Side)
		arc (\cassegrainMAAAngle:\cassegrainMABAngle:\cassegrainMARadius)
			node[left,pos=0.5] {O};
\draw[mirror element]
	(cassegrainM13Side)
		arc (\cassegrainMACAngle:\cassegrainMADAngle:\cassegrainMARadius);
% mirror 2
\draw[mirror element]
	(cassegrainM2SideEdge1)
		arc (\cassegrainMBAAngle:\cassegrainMBDAngle:\cassegrainMBRadius);

\end{scope}

\end{tikzpicture}

	\caption{Top-view schema of the apparatus for multiple excitation wavelengths
		and with side-view inset of the sample space. The right-angle laser
		mirrors optimized for 244-nm excitation were replaced by prisms in total
		reflection mode. M1 was replaced by Pellin-Broca prism PB which separates
		unwanted frequencies from the excitation beam and sends them to the beam
		blocker BB. M2 and M3 were replaced by right angle prisms. The prism MC2 is
		flipped to the position for measurement and calibration lamp is switched
		off. The explanation of rest of the symbols is the same as in
		\figref{wavenumber_calibration:apparatus_schema}.}
	\label{\figlabel{multiple_excitations:apparatus_schema}}
\end{figure}
