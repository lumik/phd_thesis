\begin{tikzpicture}[font=\sffamily, >=Latex]

\tikzset{
	mirror element/.style={color=black, line width=2*\pgflinewidth},
	axis/.style={color=black, dash pattern=on 3pt off 3pt},
	ray arrow/.tip={Latex[length=3mm]},
	angle arrow/.tip={Latex[length=2mm]},
	incident ray/.style={->,color=black, line width=2*\pgflinewidth,%
		>=ray arrow},
	difracted ray/.style={->,color=cyan, line width=2*\pgflinewidth,%
		>=angle arrow},
	directed angle/.style={->,color=black, >=angle arrow},
};

\clip (1,-.1) rectangle (11,4.1);

\pgfdeclarelayer{bg}    % declare background layer
\pgfsetlayers{bg,main}  % set the order of the layers (main is the standard layer)

\newcommand*{\gratingStepA}{0.1}
\newcommand*{\gratingStepB}{0.3}
\newcommand*{\gratingStepY}{0.1}
\newcommand*{\halfGrooveRepetition}{15}
\newcommand*{\canvasHeight}{4}
\newcommand*{\angleRadiusA}{1.0}
\newcommand*{\angleRadiusB}{1.7}
\newcommand*{\angleRadiusC}{1.8}
\newcommand*{\angleRadiusD}{2.5}

% draw the grating
\draw[mirror element] (0,0)
	% first half of the grating
	\foreach \x in {1,...,\halfGrooveRepetition} {
		-- ++(\gratingStepA,\gratingStepY) -- ++(\gratingStepB,-\gratingStepY)}
	coordinate (gratingMiddle)
	% second half of the grating
	\foreach \x in {1,...,\halfGrooveRepetition} {
		-- ++(\gratingStepA,\gratingStepY) -- ++(\gratingStepB,-\gratingStepY)};

% draw axis perpendicular to the grating offseted from the bottom of the groove
\draw[axis] ($ (gratingMiddle) + (-\gratingStepB/2,\gratingStepY/2) $)
	coordinate (zero)
	-- ++(0,10);

% draw the rays in background for them not to overlay the grating
\begin{pgfonlayer}{bg}    % select the background layer
	\draw[difracted ray] (zero) -- ($ (zero) + (5,\canvasHeight) $)
		coordinate (difractedEnd);
	\draw[incident ray] ($ (zero) + (-1.7,\canvasHeight) $)
		coordinate (incidentStart) -- (zero);
\end{pgfonlayer}

% calculate incident and difracted light angles
\mypgfextractangle{\incidentAngle}{zero}{incidentStart}
\mypgfextractangle{\diffractedAngle}{zero}{difractedEnd}

% calculate middle of Ebert angle \varphi
\pgfmathparse{\incidentAngle/2 + \diffractedAngle/2}
\let\diffractionAxisAngle\pgfmathresult

% draw axis in the middle of Ebert angle \varphi
\draw[axis] (zero) -- ++(\diffractionAxisAngle:10);

% draw angles
\draw[directed angle] ($ (zero) + (\incidentAngle:\angleRadiusA) $)
	arc (\incidentAngle:\diffractedAngle:\angleRadiusA)
	node[above,pos=0.85] {$\varphi$};
\draw[directed angle] ($ (zero) + (90:\angleRadiusB) $)
	arc (90:\incidentAngle:\angleRadiusB)
	node[above,pos=0.5] {$\alpha_i$};
\draw[directed angle] ($ (zero) + (90:\angleRadiusC) $)
	arc (90:\diffractedAngle:\angleRadiusC)
	node[above,pos=0.7] {$\alpha_m$};
\draw[directed angle] ($ (zero) + (90:\angleRadiusD) $)
	arc (90:\diffractionAxisAngle:\angleRadiusD)
	node[above,pos=0.5] {$\vartheta_m$};
\draw[directed angle] ($ (zero) + (\diffractionAxisAngle:\angleRadiusD) $)
	arc (\diffractionAxisAngle:\diffractedAngle:\angleRadiusD)
	node[above,pos=0.5] {$\frac{\varphi}{2}$};

\end{tikzpicture}
