\subsection{Spinning cell}
\label{spinning_cell}

The next problem which needed to be solved was a photodecomposition of samples
by incoming laser light.
Resonance Raman spectroscopy uses excitation light with frequency inside the
electronic absorption band.
It means that the investigated molecules accept a significant part of the power
from the incoming laser beam, and this excess energy can destroy the samples.
It also locally
increases temperature and causes the thermal lens effect, which distorts laser
focus.
We had to decide between several so far developed and successfully employed
approaches, as described in
\cref{introduction_sample_placement},
to minimize these effects.

As we wanted to study RNAs on our apparatus, which are highly sensitive to
contamination by ubiquitous RNase H and therefore cleaning of the sample cells
was a factor of major importance for us, we decided to utilize the idea of a
spinning cell to minimize photodecomposition of samples.
We designed the cell to be also usable with small volumes of the samples, the
inner radius of 4\,mm and height of 5\,mm.
It means that the maximal volume could be $\sim250$\,\g{m}L, but due to
centrifugal force, the smaller amount of samples could be measured;
for example, 50\,\g{m}L of sample results in an $\sim 4$\g{m}m thick layer
attached to the cell wall if the cell is rotated sufficiently fast.

To estimate the required rotation speed, we modeled the behavior of the water
inside the cell.
We neglected all the complicated effects like surface tension and currents
inside the liquid.
We can use energy conservation, where the sum of kinetic energy $E_\text{k}$
and gravitation potential energy $U_\text{g}$ is constant total energy $E$.
If we consider a closed system, the change in potential energy can go only to
kinetic energy and vice versa, so we can write
\begin{equation*}
	E_\text{k} = \frac{mr^2\omega^2}{2} = -U_\text{g} = mg(z - z_0),
\end{equation*}
where $m$ stands for mass, $\omega$ for angular speed, $z$ for the height of
the equipotential surface, $z_0$ for reference height for potential energy and
$g = 9.80665$\, Nkg$^{-1}$ as gravitation acceleration constant.
Solving the equation results into
\begin{equation}
	z = z_0 + \frac{r^2\omega^2}{2g}.
	\label{\eqnlabel{spinning_cell:liquid_surface}}
\end{equation}

The value for $z_0$ can be calculated from the sample volume
\begin{equation*}
	V = 2\text{\g{p}}h(R^2 - R_2^2)
		+ 2\text{\g{p}}\int_{R_1}^{R_2}{rz(r)\text{d}r},
\end{equation*}
where $R$ is the internal radius of the cell, $h$ is the height of the cell,
$R_1$ is the radius where the surface of the liquid touches the floor of the
cell, and $R_2$ is the radius where the surface of the liquid meets the ceiling
of the cell, so it means that $R \geq R2 > R1 \geq 0$.
The values of $R_1$ and $R_2$ can be estimated from
\eqnref{spinning_cell:liquid_surface}.

We can estimate the sufficient rotation speed of about 4000\,rpm from sample
model results for 50\,\g{m}L of a sample displayed in
\figref{spinning_cell:model}.

\begin{figure}
	\centering
	\begin{subfigure}[b]{0.49\columnwidth}
		\centering
		\includegraphics[width=\columnwidth]%
			{results_and_discussion/assets/spinning_cell_model/model_500rpm}
		\caption{500 rpm}
		\label{\figlabel{spinning_cell:model_500rpm}}
	\end{subfigure}
	\begin{subfigure}[b]{0.49\columnwidth}
		\centering
		\includegraphics[width=\columnwidth]%
			{results_and_discussion/assets/spinning_cell_model/model_1000rpm}
		\caption{1000 rpm}
		\label{\figlabel{spinning_cell:model_1000rpm}}
	\end{subfigure}
	\begin{subfigure}[b]{0.49\columnwidth}
		\centering
		\includegraphics[width=\columnwidth]%
			{results_and_discussion/assets/spinning_cell_model/model_2000rpm}
		\caption{2000 rpm}
		\label{\figlabel{spinning_cell:model_2000rpm}}
	\end{subfigure}
	\begin{subfigure}[b]{0.49\columnwidth}
		\centering
		\includegraphics[width=\columnwidth]%
			{results_and_discussion/assets/spinning_cell_model/model_4000rpm}
		\caption{4000 rpm}
		\label{\figlabel{spinning_cell:model_4000rpm}}
	\end{subfigure}
	\caption[%
		Model of the spinning cell.%
	]{%
		\captiontitle{%
			Model of the spinning cell following
			\eqnref{spinning_cell:liquid_surface}
			with different rotation speeds using 4\,mm as internal diameter, 5\,mm
			as height, and 50\,\g{m}L of liquid.%
		}
	}
	\label{\figlabel{spinning_cell:model}}
\end{figure}

The sample holder was inspired by the work of
\textcite{Shriver1974},
but we used FPM o-rings with an inner diameter of 11\,mm and 1\,mm diameter of
the rubber as displayed in
\figref{spinning_cell:drawing}
instead of the nylon cone.
The knurled aluminum nut compressed the o-ring, effectively centering and
securing the spinning cell to the holder chuck.
The cell holder was secured to the driving motor shaft by M2 screws.

\begin{figure}
	\centering
	\input{results_and_discussion/assets/spinning_cell_drawing}
	\caption[%
		Spinning cell holder.%
	]{%
		\captiontitle{%
			Spinning cell holder.%
		}
		A Teflon plug seals the cell (in blue), the knurled nut (light grey)
		secures the cell to the holder chuck (dark grey) by compressing the o-ring
		(black).
		The cell is attached to the driving motor shaft by M2 screws.
	}
	\label{\figlabel{spinning_cell:drawing}}
\end{figure}

As a driving motor, we used a DC motor (Maxon A-max 110119) controlled by a
homemade power source that could produce from 0\,V to 9\,V at the output.
The power supply was equipped with a digital voltmeter for user convenience.
The capabilities of the motor were measured with an attached fully filled cell.
A tiny dot was stuck to the cell, and a photodiode with an attached
oscillometer was then used for rotation frequency measurement.
The results of the measurement are shown in
\figref{spinning_cell:rotation}.
The values of rotation speed $\omega$ as a function of input voltage $U$ were
fitted by linear dependence
\begin{gather*}
	\omega = a_1U + a_0,\\
	a_1 = (1071 \pm 4) \text{V}^{-1}, a_0 = 2 \pm 20.
\end{gather*}

\begin{figure}
	\centering
	\input{results_and_discussion/assets/spinning_cell_rotation/rotation}
	\caption[%
		Spinning cell rotation evaluation.%
	]{%
		\captiontitle{%
			Spinning cell rotation evaluation.%
		}
		The dependence of rotation speed $\omega$ on the input voltage for the
		driving motor.
	}
	\label{\figlabel{spinning_cell:rotation}}
\end{figure}

The dependence has not got any significant constant factor $a_0$, and with the
maximal power of 9\,V accepted by the motor, about 9600\,rpm was achieved,
which was satisfactory for the spinning cell to centrifuge the samples to its
walls completely.
