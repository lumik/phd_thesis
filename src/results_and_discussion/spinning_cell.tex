\subsection{Spinning cell}

Next problem which needed to be solved was photodecomposition of samples by
incoming laser light. Resonance Raman spectroscopy uses excitation light
with frequency inside electronic absorption band. It means that the
investigated molecules accept significant part of the power from incoming
laser beam and this excesive energy can destroy the samples. It also locally
increases temperature and causes thermal lens effect which distorts laser
focus. Over the time, resonance Raman spectroscopists invented methods to
minimize these effects.

\textcite{Kiefer1971} reported spinning cell for resonance Raman spectroscopy
which elimited the above mentioned difficulties and increased the observed
Raman intensity by a factor of about 10. A schematic diagram of the cell is
given in
\figref{spinning_cell:1971Kiefer_rotating_cell}.
They used cylindrical quartz cell with outer diameter of 60\,mm and a height
of 25\,mm glued to circular piece of brass. The rotation speed could vary from
0 to 3000 rpm. The cell could contain maximally $\sim 65$\,mL of liquid, but
only $\sim 15$ mL were necessary for measurement, because the liquid is driven
to the walls of the cell by centrifugal force.

\insertfigure{%
	Cell for resonance Raman measurements in liquids proposed by
	\textcite(Kiefer1971). The picture was taken from the original paper.%
}{results_and_discussion/assets/1971Kiefer_rotating_cell}%
{.3}%
{spinning_cell:1971Kiefer_rotating_cell}%
{t}

\textcite{Shriver1974} improved the design of the spinning cell holder. They
used pyrex tubes which could be directly inserted into the holder. The tube
was secured inside holder by split nylon cone compressed by knurled aluminium
nut attached to aluminium body of chuck (see
\figref{spinning_cell:1974Shriver_spinning_cell}).
The cell could contain 1mL samples or less for measurement in which case the
spectra were obtained from thin film of the liquid on the sides of the cell.

\insertfigure{%
	Detail of sample spinner, for 180$^\circ$ or oblique illumination. A:
	complete system with evacuated Pyrex jacket, J, surrounding the sample tube,
	S. Cold (or warm) gas such as \ch{N2} is passed through J to control the
	sample temperature. L$_1$ is the focusing cylindrical lens, L$_2$ is the
	collecting lens, M is the small front surface mirror. C is the sample chuck.
	B: detail of the sample chuck: O, split nylon cone; P, knurled aluminium nut
	attached to aluminum body of chuck; Q, spinner shaft. The spinner should be
	constructed to minimize wobble of the sample tube, which decreases the Raman
	signal at high absorber concentrations. Adapted from \textcite{Shriver1974}.%
}{results_and_discussion/assets/1974Shriver_spinning_cell}%
{.35}%
{spinning_cell:1974Shriver_spinning_cell}
{t}

The other approach to solve these problems was proposed by
\textcite{Ziegler1981} who used bubble-free flowing jet stream of sample with
about 1\,mm in diameter recirculated by a micropump. \textcite{Asher1983}
created improved sample jet stream apparatus for continuous circulation of the
sample. The apparatus used jet nozzle from a dye laser with 0.2-mm wide
optically uniform stream and teflon stainless-steel centrifugal pump which
maintained the continuous flow of the sample from reservoir. The schema can be
seen in \figref{spinning_cell:1983Asher_free_jet}. The advantage of this
approach is that the sample potentially damaged by the excitation beam don't
need to be reused because it can be continually removed from the measurement.
The problems connected with absorption of excitation beam and selfabsorption
of the scattered light can be also minimized by using sufficiently thin stream.
The major disadvantages of jest stream devices are that cleaning of the
apparatus is much more difficult (which can be dangerous for samples sensitive
to contamination like RNAs), usually backscattering geometries, where it is
more difficult to fill input slit of spectrograph by the scattered light and
that large volumes of the samples are usually needed. The needed volumes of
samples are not usually reported in the resonance Raman studies, but it
is usually tens of mL -- 50\,mL \parencite{Ziegler1981},
2 -- 10\,mL \parencite{Fodor1985}.

\insertfigure{%
	Sample jet stream assembly. Adopted from \textcite{Asher1983}.
}{results_and_discussion/assets/1983Asher_free_jet}%
{.3}%
{spinning_cell:1983Asher_free_jet}
{t}

As we also wanted to study RNAs on our apparatus, which are highly sensitive
to contamination by ubiquitous RNase H and therefore cleaning of the sample
cells was factor of major importance for us, we decided to utilize the idea
of spinning cell to minimize photodecomposition of samples. We designed the
cell to be also usable with small volumes of the samples, inner radius of 4\,mm
and hight of 5\,mm. It means that the maximal volume could be $\sim250$\,\g{m}L
but due to centrifugal force the smaller amount of samples could be measured,
for example 50\,\g{m}L of sample results in $\sim 4$\g{m}m-thick layer attached
to the cell walls if the cell is rotated sufficiently fast.

To estimate required rotation speed we modeled the behavior of the water
inside the cell. We neglected all the complicated effects like surface tension
and currents inside liquid. We can use conservation of energy, where the sum
of kinetic energy $E_\text{k}$ and gravitation potential energy $U_\text{g}$ is
constant total energy $E$. If you consider closed system, the change in
potential energy can go only to kinetic energy and vice versa so we can write
\begin{equation*}
	E_\text{k} = \frac{mr^2\omega^2}{2} = -U_\text{g} = mg(z - z_0),
\end{equation*}
where $m$ stands for mass, $\omega$ for angular speed, $z$ for height of
equipotential surface, $z_0$ for reference height for potential energy and
$g = 9.80665$\, Nkg$^{-1}$ as gravitation acceleration constant. Solving the
equation results into
\begin{equation}
	z = z_0 + \frac{r^2\omega^2}{2g}.
	\label{\eqnlabel{spinning_cell:liquid_surface}}
\end{equation}

The value for $z_0$ can be calculated from the sample volume
\begin{equation*}
	V = 2\text{\g{p}}h(R^2 - R_2^2)
		+ 2\text{\g{p}}\int_{R_1}^{R_2}{rz(r)\text{d}r},
\end{equation*}
where $R$ is the internal radius of the cell, $h$ is height of the cell, $R_1$
is radius where the surface of the liquid touches the floor of the cell and
$R_2$ is the radius where the surface of the liquid meets the ceiling of the
cell, so it means that $R \geq R2 > R1 \geq 0$. The values of $R_1$ and $R_2$
can be estimated from the \eqnref{spinning_cell:liquid_surface}.

We can estimate the sufficient rotation speed about 4000\,rpm from sample model
results for 50\,\g{m}L of sample displayed in
\figref{spinning_cell:model}.

\begin{figure}
	\centering
	\begin{subfigure}[b]{0.49\columnwidth}
		\centering
		\includegraphics[width=\columnwidth]%
			{results_and_discussion/assets/spinning_cell_model/model_500rpm}
		\caption{500 rpm}
		\label{\figlabel{spinning_cell:model_500rpm}}
	\end{subfigure}
	\begin{subfigure}[b]{0.49\columnwidth}
		\centering
		\includegraphics[width=\columnwidth]%
			{results_and_discussion/assets/spinning_cell_model/model_1000rpm}
		\caption{1000 rpm}
		\label{\figlabel{spinning_cell:model_1000rpm}}
	\end{subfigure}
	\begin{subfigure}[b]{0.49\columnwidth}
		\centering
		\includegraphics[width=\columnwidth]%
			{results_and_discussion/assets/spinning_cell_model/model_2000rpm}
		\caption{2000 rpm}
		\label{\figlabel{spinning_cell:model_2000rpm}}
	\end{subfigure}
	\begin{subfigure}[b]{0.49\columnwidth}
		\centering
		\includegraphics[width=\columnwidth]%
			{results_and_discussion/assets/spinning_cell_model/model_4000rpm}
		\caption{4000 rpm}
		\label{\figlabel{spinning_cell:model_4000rpm}}
	\end{subfigure}
	\caption{Model of the spinning cell following
		\eqnref{spinning_cell:liquid_surface}
		with different rotation speeds using 4\,mm as internal diameter, 5\,mm
		as height and 50\,\g{m}L of liquid.}
	\label{\figlabel{spinning_cell:model}}
\end{figure}

The sample holder was inspired by the work of \textcite{Shriver1974} but we
used FPM o-rings with inner diameter 11\,mm and 1-mm diameter of the rubber as
displayed in the \figref{spinning_cell:drawing} instead of the nylon cone. The
o-ring was compressed by the knurled aluminium nut so it effectively centered
and secured the spinning cell to the holder chuck. The whole cell was secured
to the driving motor shaft by M2 screws.

\begin{figure}
	\centering
	\input{results_and_discussion/assets/spinning_cell_drawing}
	\caption{Spinning cell holder. The cell (in blue) is sealed by teflon plug,
		The knurled nut (light grey) secures the cell to the holder chuck (dark
		grey) by compressing the o-ring (black). The cell can be secured to the
		driving motor shaft by M2 screws.}
	\label{\figlabel{spinning_cell:drawing}}
\end{figure}

As a driving motor, we used DC motor (Maxon A-max 110119) controlled by
home-made power source which could produce from 0V to 9V at output. The
power supply was equipped with digital voltmeter for the user convenience. The
capabilities of the motor were measured with attached full cell. Small dot
was sticked to the cell and photodiod with attached oscillometer was then used
for rotation frequency measurement. The results of the measurement are shown
in \figref{spinning_cell:rotation}. The values of rotation speed $\omega$
as a function of input voltage $U$ were fitted by linear dependence
\begin{gather*}
	\omega = a_1U + a_0,\\
	a_1 = (1071 \pm 4) \text{V}^{-1}, a_0 = 2 \pm 20.
\end{gather*}

\begin{figure}
	\centering
	\input{results_and_discussion/assets/spinning_cell_rotation/rotation}
	\caption{Spinning cell rotation evaluation. The dependence of rotation
		speed $\omega$ on the input voltage for the driving motor.}
	\label{\figlabel{spinning_cell:rotation}}
\end{figure}

The dependence hasn't got any significant constant factor $a_0$ and with the
maximal power of 9V accepted by the motor, about 9600 rpm is achieved which
is satisfactory for spinning cell to completely centrifuge the samples to its
walls.
