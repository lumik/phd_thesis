\subsection{Spinning cell}
\label{spinning_cell}

Next problem which needed to be solved was photodecomposition of samples by
incoming laser light. Resonance Raman spectroscopy uses excitation light
with frequency inside electronic absorption band. It means that the
investigated molecules accept significant part of the power from incoming
laser beam and this excesive energy can destroy the samples. It also locally
increases temperature and causes thermal lens effect which distorts laser
focus. We had to decide between several so far developed and successfully
employed approaches, as described in \cref{sample_cell}, to minimize
these effects.

As we wanted to study also RNAs on our apparatus, which are highly sensitive
to contamination by ubiquitous RNase H and therefore cleaning of the sample
cells was factor of major importance for us, we decided to utilize the idea
of spinning cell to minimize photodecomposition of samples. We designed the
cell to be also usable with small volumes of the samples, inner radius of 4\,mm
and hight of 5\,mm. It means that the maximal volume could be $\sim250$\,\g{m}L
but due to centrifugal force the smaller amount of samples could be measured,
for example 50\,\g{m}L of sample results in $\sim 4$\g{m}m-thick layer attached
to the cell walls if the cell is rotated sufficiently fast.

To estimate required rotation speed we modeled the behavior of the water
inside the cell. We neglected all the complicated effects like surface tension
and currents inside liquid. We can use conservation of energy, where the sum
of kinetic energy $E_\text{k}$ and gravitation potential energy $U_\text{g}$ is
constant total energy $E$. If you consider closed system, the change in
potential energy can go only to kinetic energy and vice versa so we can write
\begin{equation*}
	E_\text{k} = \frac{mr^2\omega^2}{2} = -U_\text{g} = mg(z - z_0),
\end{equation*}
where $m$ stands for mass, $\omega$ for angular speed, $z$ for height of
equipotential surface, $z_0$ for reference height for potential energy and
$g = 9.80665$\, Nkg$^{-1}$ as gravitation acceleration constant. Solving the
equation results into
\begin{equation}
	z = z_0 + \frac{r^2\omega^2}{2g}.
	\label{\eqnlabel{spinning_cell:liquid_surface}}
\end{equation}

The value for $z_0$ can be calculated from the sample volume
\begin{equation*}
	V = 2\text{\g{p}}h(R^2 - R_2^2)
		+ 2\text{\g{p}}\int_{R_1}^{R_2}{rz(r)\text{d}r},
\end{equation*}
where $R$ is the internal radius of the cell, $h$ is height of the cell, $R_1$
is radius where the surface of the liquid touches the floor of the cell and
$R_2$ is the radius where the surface of the liquid meets the ceiling of the
cell, so it means that $R \geq R2 > R1 \geq 0$. The values of $R_1$ and $R_2$
can be estimated from the \eqnref{spinning_cell:liquid_surface}.

We can estimate the sufficient rotation speed about 4000\,rpm from sample model
results for 50\,\g{m}L of sample displayed in
\figref{spinning_cell:model}.

\begin{figure}
	\centering
	\begin{subfigure}[b]{0.49\columnwidth}
		\centering
		\includegraphics[width=\columnwidth]%
			{results_and_discussion/assets/spinning_cell_model/model_500rpm}
		\caption{500 rpm}
		\label{\figlabel{spinning_cell:model_500rpm}}
	\end{subfigure}
	\begin{subfigure}[b]{0.49\columnwidth}
		\centering
		\includegraphics[width=\columnwidth]%
			{results_and_discussion/assets/spinning_cell_model/model_1000rpm}
		\caption{1000 rpm}
		\label{\figlabel{spinning_cell:model_1000rpm}}
	\end{subfigure}
	\begin{subfigure}[b]{0.49\columnwidth}
		\centering
		\includegraphics[width=\columnwidth]%
			{results_and_discussion/assets/spinning_cell_model/model_2000rpm}
		\caption{2000 rpm}
		\label{\figlabel{spinning_cell:model_2000rpm}}
	\end{subfigure}
	\begin{subfigure}[b]{0.49\columnwidth}
		\centering
		\includegraphics[width=\columnwidth]%
			{results_and_discussion/assets/spinning_cell_model/model_4000rpm}
		\caption{4000 rpm}
		\label{\figlabel{spinning_cell:model_4000rpm}}
	\end{subfigure}
	\caption{Model of the spinning cell following
		\eqnref{spinning_cell:liquid_surface}
		with different rotation speeds using 4\,mm as internal diameter, 5\,mm
		as height and 50\,\g{m}L of liquid.}
	\label{\figlabel{spinning_cell:model}}
\end{figure}

The sample holder was inspired by the work of \textcite{Shriver1974} but we
used FPM o-rings with inner diameter 11\,mm and 1-mm diameter of the rubber as
displayed in the \figref{spinning_cell:drawing} instead of the nylon cone. The
o-ring was compressed by the knurled aluminium nut so it effectively centered
and secured the spinning cell to the holder chuck. The whole cell was secured
to the driving motor shaft by M2 screws.

\begin{figure}
	\centering
	\input{results_and_discussion/assets/spinning_cell_drawing}
	\caption{Spinning cell holder. The cell (in blue) is sealed by teflon plug,
		The knurled nut (light grey) secures the cell to the holder chuck (dark
		grey) by compressing the o-ring (black). The cell can be secured to the
		driving motor shaft by M2 screws.}
	\label{\figlabel{spinning_cell:drawing}}
\end{figure}

As a driving motor, we used DC motor (Maxon A-max 110119) controlled by
home-made power source which could produce from 0V to 9V at output. The
power supply was equipped with digital voltmeter for the user convenience. The
capabilities of the motor were measured with attached full cell. Small dot
was sticked to the cell and photodiod with attached oscillometer was then used
for rotation frequency measurement. The results of the measurement are shown
in \figref{spinning_cell:rotation}. The values of rotation speed $\omega$
as a function of input voltage $U$ were fitted by linear dependence
\begin{gather*}
	\omega = a_1U + a_0,\\
	a_1 = (1071 \pm 4) \text{V}^{-1}, a_0 = 2 \pm 20.
\end{gather*}

\begin{figure}
	\centering
	\input{results_and_discussion/assets/spinning_cell_rotation/rotation}
	\caption{Spinning cell rotation evaluation. The dependence of rotation
		speed $\omega$ on the input voltage for the driving motor.}
	\label{\figlabel{spinning_cell:rotation}}
\end{figure}

The dependence hasn't got any significant constant factor $a_0$ and with the
maximal power of 9V accepted by the motor, about 9600 rpm is achieved which
is satisfactory for spinning cell to completely centrifuge the samples to its
walls.
