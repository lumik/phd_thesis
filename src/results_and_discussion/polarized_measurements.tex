\subsection{Polarized measurements}

The next possible improvement of the measurement capabilities was to enhance
the spectrometer for polarized measurements. There were two categories of
polarizers commercially available for deep UV light, crystalline polarizers and
wire grid polarizers, so we tried to evaluate both of them. We inserted them
into the gathered signal beam path right after the edge filter in continous
rotation mounts with engraved scale marked in $2^\circ$ increments placed
on flip mount for easy removal. We rotad the scale so that $0^\circ$ meant
vertical polarization and $90^\circ$ horizontal.

The most of the crystalline polarizers are not created from materials
transparent to UV light so we needed to chose from the limited selection,
where the most appropriate with cost/value considerations at that time were
\g{a}-BBO Glan-Laser polarizers (Thorlabs). The disadvantage of this solution
was small clear aperture (1\,cm) and small angular field of view. The
advantages are good transparency for deep UV light (greater than 60\% from
210\,nm) and great extinction ratio ($\sim 100000:1$).

The selection of wire-grid polarizers usable for UV was also limited because
they have much narrower range of accepted wavelengths. We chose the one using
dielectric nanowire arrays on silica glass substrate (Meadowlark). It had
polarized transmission from 60 to 70\% at 245 to 285\,nm respectively and
extinction ratio greater than 60 in the same range. The advantage of this
polarizer was larger clear aperture (2.54\,cm) and slightly larger angular
field of view.

Polarization dependence of the light gathering optics and especially of
spectrograph grating was shielded by using quartz-wedge depolarizer (Thorlabs).
The two polarizers were evaluated using Hg lamp as a source of light from
from sample space. At first, we measured ratio of throughput between vertical
and horizontal polarization which were mainly different due to polarization
dependence of spectrograph grating reflectance. Vertical ratio was $~1.65$
and $~1.41$ times higher ther horizontal for wire-grid and Glan-Laser
polarizers, respectively. The difference can be caused by the smaller clear
aperture of the later one.

Next, the ratio between horizontal and vertical polarization throughput with
inserted depolarizer was measured. The spectra were accumulated for 0.1\,s
and intensity was measured at the maximum of peak at 253.6-nm Hg line in
detector counts. The results are displayed in
\tabref{polarized_measurements:polarizer_evaluation}.
There is significant difference between intenzity with horizontal and vertical
polarization for wire-grid polarizer which indicates that there is strong
dependence of its characteristics on polarizer tilt. There is also higher
variation in signal for Glan-laser polarizer (especially visible for
horizontal polarization), which is probably effect of combination of slight
optics movement during manipulation with polarizer and its limited aperture.
But the most significant observation is that Glan-laser polarizer has
$\sim 2.7$ times higher throughput then the wire-grid polarizer at the
253.6\,nm nevertheless its limited aperture. Therefore we chose Glan-laser
polarizer for any further polarization measurements.

\begin{table}
	\centering
	\begin{tabular}{lccc}
\toprule
polarizer  & $I_\text{v} (10^6)$
                               & $I_\text{h} (10^6)$
															                     & $I_\text{v} / I_\text{h}$
																									                     \\
\midrule
wire-grid  & $0.906 \pm 0.022$ & $0.832 \pm 0.018$ & $1.089 \pm 0.003$ \\
Glan-laser & $2.337 \pm 0.012$ & $2.349 \pm 0.123$ & $0.997 \pm 0.053$ \\
\bottomrule
\end{tabular}

	\caption{Performace comparison of wire-grid and Glan-laser polarizer. $I$
	stands for intenzity of 253.6-nm Hg line at maximum in detector counts and
	subscripts v and h stand for vertical and horizontal polarizer orientation.}
	\label{\tablabel{polarized_measurements:polarizer_evaluation}}
\end{table}
