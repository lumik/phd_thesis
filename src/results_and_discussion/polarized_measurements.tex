\subsection{Polarized measurements}

The next possible improvement of the measurement capabilities was to enhance
the spectrometer for polarized measurements.
There were two categories of polarizers commercially available for deep UV
light, crystalline polarizers and wire grid polarizers, so we tried to evaluate
both of them.
We inserted them into the gathered signal beam path right behind the edge
filter in continous rotation mounts with an engraved scale marked in
2\textdegree{} increments placed on a flip mount for easy removal.
We rotated the scale so that 0\textdegree{} meant vertical polarization and
90\textdegree{} horizontal.

Most crystalline polarizers are not created from materials transparent to UV
light so we needed to choose from the limited selection, where the most
appropriate with cost/value considerations at that time were \g{a}-BBO
Glan-Laser polarizers (Thorlabs).
The disadvantage of this solution was a small clear aperture (1\,cm) and a
small angular field of view.
The advantages are good transparency for deep UV light (greater than 60\% from
210\,nm) and an excellent extinction ratio ($\sim 100000:1$).

The selection of wire-grid polarizers usable for UV was also limited because
they have a much narrower range of accepted wavelengths.
We chose the one using dielectric nanowire arrays on a silica glass substrate
(Meadowlark).
It had polarized transmission from 60 to 70\% at 245 to 285\,nm, respectively,
and an extinction ratio greater than 60 in the same range.
The advantage of this polarizer was a larger clear aperture (2.54\,cm) and a
slightly larger angular field of view.

Polarization dependence of the light-gathering optics and especially of
spectrograph grating was shielded by using a quartz-wedge depolarizer
(Thorlabs).
The two polarizers were evaluated using a Hg lamp as a source of light from
sample space.
At first, we measured the ratio of throughput between vertical and horizontal
polarization, which were mainly different due to polarization dependence of
spectrograph grating reflectance.
The vertical ratio was $~1.65$ and $~1.41$ times higher than horizontal for
wire-grid and Glan-Laser polarizers, respectively.
The difference can be caused by the smaller clear aperture of the later one.

Next, the ratio between horizontal and vertical polarization throughput with
inserted depolarizer was measured.
The spectra were accumulated for 0.1\,s,
and intensity was measured at the maximum of the Hg line at 253.6\,nm in
detector counts.
The results are displayed in
\tabref{polarized_measurements:polarizer_evaluation}.
There is a significant difference between intensity with horizontal and vertical
polarization for a wire-grid polarizer, which indicates a strong dependence of
its characteristics on polarizer tilt.
There is also a higher variation in signal for the Glan-laser polarizer
(especially visible for horizontal polarization), which is probably the effect
of a combination of slight optics movement during manipulation with the
polarizer and its limited aperture.
However, the most significant observation is that the Glan-laser polarizer has
$\sim 2.7$ times higher throughput than the wire-grid polarizer at 253.6\,nm,
nevertheless its limited aperture.
Therefore we chose Glan-laser polarizer for any further polarization
measurements.

\begin{table}
	\centering
	\begin{tabular}{lccc}
\toprule
polarizer  & $I_\text{v} (10^6)$
                               & $I_\text{h} (10^6)$
															                     & $I_\text{v} / I_\text{h}$
																									                     \\
\midrule
wire-grid  & $0.906 \pm 0.022$ & $0.832 \pm 0.018$ & $1.089 \pm 0.003$ \\
Glan-laser & $2.337 \pm 0.012$ & $2.349 \pm 0.123$ & $0.997 \pm 0.053$ \\
\bottomrule
\end{tabular}

	\caption[%
		Performace comparison of wire-grid and Glan-laser polarizer.%
	]{%
		\captiontitle{%
			Performace comparison of wire-grid and Glan-laser polarizer.%
		}
		$I$ stands for intensity of 253.6\,nm Hg line at maximum detector counts,
		and subscripts v and h stand for vertical and horizontal polarizer
		orientation.
	}
	\label{\tablabel{polarized_measurements:polarizer_evaluation}}
\end{table}
