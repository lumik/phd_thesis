\section[%
	Temperature and concentration effects on transition from antiparallel	to
	parallel quadruplex in tel22
]{%
	Temperature and concentration effects on transition from antiparallel to
	parallel\\quadruplex in tel22
}

One class of the unusual nucleic acid conformations is
\emph{guanine quadruplexes} (GQ)
formed by stacked planar guanine tetrads.
The structure is stabilized by Hoogsteen base pairing and cation coordination.
GQ are abundant within human DNA, and their presence in telomeric guanine-rich
regions of human core telomeric repeats \ch{d[TTAG3]_n} is well known.
It was also observed that the GQ exhibit remarkable polymorphism, especially
those stabilized by \ch{K^+} ions
\parencite{Chaires2013}.

It is interesting to investigate GQ in the molecular crowding environment,
physiological temperatures, and in the presence of \ch{K^+} ions which mimicks
their actual \emph{in vivo} conditions in cells.
Recent \emph{in vitro} studies revealed that DNA concentration and thermal
treatment
of the sample might play an important role in GQ folding and interquadruplex
transitions
\parencite{Palacky2013}.

Circular dichroism and conventional Raman spectroscopy revealed that
thermodynamically equilibrated (heated and slowly cooled) samples of Tel22
change from antiparallel to hybrid “3 + 1” and parallel form and that
the fraction of the parallel form increases with the concentration of the
oligonucleotide or \ch{K^+} ions
\parencite{Palacky2013}.

The influence of slightly higher, but safely below the denaturation,
temperature of 37\,\textdegree{}C on highly concentrated (310\,mM
in phosphates as a model for highly crowded conditions) \ch{d[AG3(TTAG3)3]}
(Tel22) was studied.
The samples were proved by Raman spectroscopy to accommodate the antiparallel
conformation in an environment without \ch{K^+} ions (75\,mM \ch{Na^+-PBS}
buffer with overall concentration 150\,mM of \ch{Na^+} ions) ,and their
external
stability was confirmed, i.e., the antiparallel conformation was found to be
preserved after annealing, long term storage, slight temperature increase, high
\ch{Na^+} ion concentration (up to 1120\,mM).

When \ch{K^+} ions were added to 450\,mM concentration, a very swift spectral
change occurred ($< 1$\,min). It indicated the replacement of \ch{Na^+} by
\ch{K^+} ions and \ch{K^+}-antiparallel quadruplex formation, which after
some time changed its conformation to “3 + 1”.

Transition to the parallel quadruplex form was observed with progressing time
(72\,hours in total) if such sample was incubated at a physiological
temperature of 37\,\textdegree{}C.
This temperature is far below the melting point under given
conditions ($> 75$\,\textdegree{}C) as indicated the intensity increase of
1337\,\icm{} (marker of \ch{"C2'"}-endo/anti-dG) and decrease of
1326\,\icm{} (marker of \ch{"C2'"}-endo/syn-dG) bands
\parencite{%
	Nishimura1986b,%
	Benevides1988a,%
	Miura1994,%
	Miura1995,%
	Miura1995a,%
	Krafft2002%
}.
Furthermore, the samples incubated at 37\,\textdegree{}C remained fluid even
after the transition to the parallel form.

A similar spectral change can be seen for the same sample after annealing
(heating to 95\,\textdegree{}C and slowly cooling down to 20\,\textdegree{}C),
but the resulting sample forms a stiff gel.

NMR spectroscopy of the resulting parallel GQ samples suggested that the
incubation of the Tel22 sample at physiological temperature formed
intramolecular quadruplexes because the NMR signal was much lower, which
suggests
small randomly oriented complexes stabilized by the van der Waals interactions.
In contrast, the annealed samples form more stable higher-order intermolecular
structures indicated by stronger NMR signal.
This observation also explained why incubation results in fluid samples,
whereas annealed samples form a gel (Lindnerová-Mudroňová, private
communication).

This process suggests that the parallel GQ is energetically the most favorable
form.
To confirm the results from visible Raman spectroscopy and overcome problems
with fluorescence, the samples were diluted to 200\,\g{m}M concentration,
stored for 10 days at 5\,\textdegree{}C, and measured on the UVRR spectrometer
with
2\,mW of 244\,nm laser power at a sample in 30 frames of 120\,s scans.
A 3\,mL of sample was stirred in silica glass spectroscopic cells during
the measurement to prevent photodecomposition.
All measurements were performed at 10°C.
Background signals of PBS buffer and quartz cell were subtracted.

\Figref{telXXII:spectra} shows the results for thermally pristine samples,
which were stored at 5\,\textdegree{}C, 37\,\textdegree{}C incubated samples
and samples after annealing.
During the 72\,h incubation, the conversion from “3 + 1” conformation, which
has guanine in C2'-endo/syn conformation (guanine band at 1325\,\icm{}), to
parallel conformation with guanine in C2'-endo/anti conformation (guanine
band at 1338\,\icm{}) is not finished (see \cref{interpretation}).
The spectra also confirm that the conformation is stable even after dilution
and storage for 10 days.

\begin{figure}
	\centering
	\input{results_and_discussion/assets/tel22/tel22_spectra}
	\vspace{3mm}
	\caption[%
		Effect of standard anealing and incubation at 37\,\textdegree{}C on UVRRS
		of \ch{K^+}-Tel22.
	]{%
		\captiontitle{%
			Effect of standard anealing and incubation at 37\,\textdegree{}C on UVRR
			spectra of \ch{K^+}-Tel22.
		}
		The “3 + 1” conformation, which is formed quickly after the addition of
		\ch{K^+} ions to the solution and which has C2'-endo/syn conformation with
		increased intensity of guanine band at 1325\,\icm{}, switches to parallel
		conformation with C2'-endo/anti conformation with increased intensity of
		the guanine band at 1338\,\icm{}.
	}
	\label{\figlabel{telXXII:spectra}}
\end{figure}

These results confirm that UVRRS can be used for studying
different unusual NA conformations such as GQ. It also extends available
concentration ranges and can be used to investigate of the stability of
nucleic acid complexes while keeping structural information contained in the
vibrational spectra.
