\subsection{Thermostated sample holder}

One of the essential characteristics of complexes of biomolecules is their
behavior with changing temperature.
We wanted to enable these types of measurements also on the new UV Raman
spectrometer.
The spinning cell described in
\cref{spinning_cell}
showed to be impractical for the design of a temperature=controlled experiment.
A different approach needed to be devised, but it was still necessary to
minimize the photodecomposition of the samples during measurement.
We decided that we utilize a magnetic stirrer and traditional spectroscopic
1\,cm Raman cells.
We had much experience with recirculating liquid temperature controllers in our
laboratory, but we decided to design a less expensive temperature-controlled
cell holder based on a thermoelectric Peltier device.

The new cell holder consisted of a copper block surrounding the cell
(\figref{thermostated_holder:core_drawing})
with a front hole suitable for backscattering measurements.
There were also two hollows at the top of the block for convenient removal of
the cells.
We also designed an adapter for 5\,mm cells.
The block contained thermocouple Pt100 for measurement of temperature sealed by
thermal grease to ensure good heat transfer.

The copper block was placed inside a Teflon block which served as insulation
(\figref{thermostated_holder:insulation_drawing}).
The insulation layer also contained a magnetic stirrer (Variomag).
The stirrer could be used up to 95\,\textcelsius{}, stir volumes up to 5\,mL on
130 -- 1000\,rpm, and was controlled by an external controller.
The sample cell could be covered inside the holder by a Teflon cover
(\figref{thermostated_holder:stand_drawing}).

% inserting more pdf figures created by Inkscape on one page produces warning
% about multiple pdfs with page group
% I found suggestion for solution here for using qpdf for that but did not try
% to apply it:
% https://tex.stackexchange.com/questions/76273/multiple-pdfs-with-page-group-included-in-a-single-page-warning
\begin{figure}
	\centering
	\ig{1}{results_and_discussion/assets/thermostated_holder_drawing/core}
	\caption[%
		Thermostated cell holder copper core.%
	]{%
		\captiontitle{%
			Thermostated cell holder copper core.%
		}
	}
	\label{\figlabel{thermostated_holder:core_drawing}}
\end{figure}

\begin{figure}
	\centering
	\ig{1}{results_and_discussion/assets/thermostated_holder_drawing/insulation}
	\caption[%
		Teflon insulation of thermostated cell holder.%
	]{%
		\captiontitle{%
			Teflon insulation of thermostated cell holder.%
		}
	}
	\label{\figlabel{thermostated_holder:insulation_drawing}}
\end{figure}

The temperature was maintained by two thermoelectric Peltier modules
(TEC1-12710) chained in series for better performance, where the second module
could be replaced by copper shim if the performance of one is satisfactory.
The operation of the Peltier modules was controlled by a two-channel PID
controller (Eurotherm 3216) connected to the thermocouple placed inside the
copper block as described above.
The signals of the two channels (heating and cooling) of the PID controller
were converted to a bipolar signal by constructed H bridge so the direction
and intensity of heat transfer through the Peltier modules could be controlled.
The electrical overview schema can be seen in
\figref{thermostated_holder:electrical_schema}.

\begin{figure}
	\centering
	\input{%
		results_and_discussion/assets/thermostated_holder_drawing/%
		electrical_schema}
	\caption[%
		Electrical schema of the thermostated cell holder.%
	]{%
		\captiontitle{%
			Electrical schema of the thermostated cell holder.%
		}
	}
	\label{\figlabel{thermostated_holder:electrical_schema}}
\end{figure}

An aluminum profile heat sink
(\figref{thermostated_holder:heat_sink_drawing})
with two ventilators (Sukon, 2500\,rpm, 17\,W) was attached to the
thermoelectric modules to increase efficiency.
The aluminum profile was insulated from the copper core block by the Teflon
layer.
A universal aluminum right-angle adapter
(\figref{thermostated_holder:stand_drawing})
mounted the whole sample holder through the heat sink to a three-axis manual
stage (Thorlabs).
The adapter supported securing of different sample holders in various
positions. Overall 3D image of the holder can be seen in
\figref{thermostated_holder:3d}.

\begin{figure}
	\centering
	\ig{1}{results_and_discussion/assets/thermostated_holder_drawing/stand}
	\caption[%
		Teflon cover of sample cell and universal aluminum right-angle
		adapter for thermostated sample holder.%
	]{%
		\captiontitle{%
			Teflon cover of sample cell (on the right) and universal aluminum
			right-angle adapter for thermostated sample holder (on the right).%
		}
	}
	\label{\figlabel{thermostated_holder:stand_drawing}}
\end{figure}

\begin{figure}
	\centering
	\ig{1}{results_and_discussion/assets/thermostated_holder_drawing/heat_sink}
	\caption[%
		Aluminum profile heat sink for the thermostated cell holder.%
	]{%
		\captiontitle{%
			Aluminum profile heat sink for the thermostated cell holder.%
		}
	}
	\label{\figlabel{thermostated_holder:heat_sink_drawing}}
\end{figure}

\begin{figure}
	\centering
	\LOWQUALITY
	\includegraphics[width=.49\columnwidth]%
			{results_and_discussion/assets/thermostated_holder_drawing/3D01}
	\LOWQUALITY
	\includegraphics[width=.49\columnwidth]%
		{results_and_discussion/assets/thermostated_holder_drawing/3D02}
	\caption[%
		The 3D image of thermostated sample holder without thermoelectric
		Peltier modules, thermocouple, and stirrer.%
	]{%
		\captiontitle{%
			The 3D image of thermostated sample holder without thermoelectric
			Peltier modules, thermocouple, and stirrer.%
		}
	}
	\label{\figlabel{thermostated_holder:3d}}
\end{figure}

This design of thermostated sample holder enabled fast control of the
temperature in the range between 5 and 95\,\textcelsius{} of samples with
volumes in the range between 250 (with the adapter to 0.5\,mm sample cells) and
3000\,\g{m}L.
The heating from 5 to 95\,\textcelsius{} took $\sim$15\,min and cooling from 95
to 5\,\textcelsius{} was as fast as 25\,min.
The worst performance of the temperature stabilization was at room temperature
because of switching between cooling and heating, so the process of
stabilizing the temperature at 20\,\textcelsius{} from 15 or
25\,\textcelsius{} took $\sim 4$\,min 30\,s.
The stabilization of the rest of the temperatures took less than 3\,min with
the 5\,\textcelsius{} step in the range from 5 to 90\,\textcelsius{}.
Stabilization from 90 to 95\,\textcelsius{} took $\sim 4$\,min because
95\,\textcelsius{} is out of the recommended operating range of the used
Peltier modules.
All the temperatures from above were regarded as stabilized with the precision
of $\pm 0.3$\,\textcelsius{}.
The temperature then oscillated and continuously approached the setpoint
temperature and reached the setpoint temperature with $\pm 0.1$\,\textcelsius{}
precision with another $\sim 1$\,min of waiting.
The temperature of the liquid inside the cell followed the temperature measured
by thermocouple from the holder with approximately 10\,s delay (as measured by
independent thermocouple-based thermometer), so we used 5\,min delay for
temperature step of at most 5\,\textcelsius{} as a safe interval for
temperature stabilization between Raman measurements except from stabilization
at 20 and 95\,\textcelsius{}, where the safe delay needed to be prolonged to
7\,min.

During five years of usage of this holder, the performance of heating/cooling
of the thermostated holder slightly degraded, and the holder needed minor
maintenance.
The performance was mainly impaired by degradation of thermal grease, ensuring
reliable temperature measurement results by the thermocouple from inside the
holder.
The performance of Peltier modules decreases with time too.
We used one Peltier module initially but after three years we added the second
one to mitigate the performance loss of the first one.

In the beginning, we were slightly concerned if the fans used to increase the
efficiency of the heat sink would not affect the stability of the optical
elements (especially spectrograph and alignment of the excitation laser beam)
by vibrations.
However, the effect was unobservable.
A high amount of heat is also produced at high temperatures, but it could
freely dissipate from the sample space, and the room temperature was maintained
by air conditioning.
The air conditioning also kept low humidity inside the laboratory, but for
temperatures less then 15\,\textcelsius{} the continuous flow of dry air or
nitrogen was directed to the front surface of the sample cell.
