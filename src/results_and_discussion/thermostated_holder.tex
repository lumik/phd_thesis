\subsection{Thermostated sample holder}

One of the important characteristics of complexes of biomolecules is their
behavior with changing temperature. We wanted to enable these type of
measurements also on the new UV Raman spectrometer. The spinning cell described
in \cref{spinning_cell} showed to be impractical for the design of temperature
controlled experiment. Other approach needed to be devised but it was still
important to minimize the photodecomposition of the samples during measurement.
We decided that we utilize magnetic stirrer and traditional spectroscopic
1-cm Raman cells. We had a lot of experience with recirculating liquid
temperature controllers in our laboratory but we decided to design less
expensive temperature controlled cell holder based on thermoelectric Peltier
device.

The cell holder consisted of copper block surrounding the cell
(\figref{thermostated_holder:core_drawing})
with front hole suitable for backsacttering measurements. There were also two
hollows at the top of the block for convenient removal of the cells. We also
designed adapter for 5-mm cells. The block contained thermocouple Pt100 for
measurement of temperature sealed by thermal grease to ensure good heat
transfer.

The copper block was placed inside Teflon block which served as an insulation
(\figref{thermostated_holder:insulation_drawing}).
Insulation layer also contained magnetic stirer (Variomag). The stirrer could
be used up to 95\,\textcelsius{}, stir the wolumes up to 5\,mL on
130 -- 1000\,rpm, and was controlled by telemodul external controller. The
sample cell could be covered inside holder by teflon cover
(\figref{thermostated_holder:stand_drawing}).

% inserting more pdf figures created by Inkscape on one page produces warning
% about multiple pdfs with page group
% I found suggestion for solution here for using qpdf for that but did not try
% to apply it:
% https://tex.stackexchange.com/questions/76273/multiple-pdfs-with-page-group-included-in-a-single-page-warning
\insertfigure{%
	Thermostated cell holder copper core.%
}{results_and_discussion/assets/thermostated_holder_drawing/core}%
{1}%
{thermostated_holder:core_drawing}
{p}

\insertfigure{%
	Teflon insultaion of thermostated cell holder.%
}{results_and_discussion/assets/thermostated_holder_drawing/insulation}%
{1}%
{thermostated_holder:insulation_drawing}
{p}

The temperature was maintained by 2 thermoelectric Peltier modules (TEC1-12710)
chained in series for better performance where the second module could be
replaced by copper shim if the performance of one is satisfactory. The
operation of the Peltier modules was controlled by two channel PID controller
(Eurotherm 3216) connected to thermocouple placed inside the copper block as
temperature sensor as described above. The signal of the two channels (heating
and cooling) of the PID controller were converted to bipolar signal by
constructed H bridge so the direction and intensity of heat transfer through
the Peltier modules could be controlled. The electrical overview schema can be
seen in \figref{thermostated_holder:electrical_schema}.

\begin{figure}
	\centering
	\input{%
		results_and_discussion/assets/thermostated_holder_drawing/%
		electrical_schema}
	\caption{Electrical schema of the thermostated cell holder.}
	\label{\figlabel{thermostated_holder:electrical_schema}}
\end{figure}

An aluminum profile heat sink
(\figref{thermostated_holder:heat_sink_drawing})
with two ventilators (Sukon, 2500\,rpm, 17\,W) was attached to the
thermoelectric modules to increase their efficiency. The aluminum profile was
insulated from the copper core block by the teflon layer. The whole sample
holder was attached through the heat sing to a three axis manual stage
(Thorlabs) by an universal aluminum right-angle adapter
(\figref{thermostated_holder:stand_drawing}).
The adapter supported securing of different sample holders in various
positions. Overall 3D image of the holder can be seen in the
\figref{thermostated_holder:3d}.

\insertfigure{%
	Teflon cover of sample cell (on the right) and universla aluminum right-angle
	adapter for thermostated sample holder (on the right).%
}{results_and_discussion/assets/thermostated_holder_drawing/stand}%
{1}%
{thermostated_holder:stand_drawing}
{p}
\insertfigure{%
	Thermostated cell holder aluminum profile heat sink.%
}{results_and_discussion/assets/thermostated_holder_drawing/heat_sink}%
{1}%
{thermostated_holder:heat_sink_drawing}
{p}

\begin{figure}
	\centering
	\LOWQUALITY
	\includegraphics[width=.49\columnwidth]%
			{results_and_discussion/assets/thermostated_holder_drawing/3D01}
	\LOWQUALITY
	\includegraphics[width=.49\columnwidth]%
		{results_and_discussion/assets/thermostated_holder_drawing/3D02}
	\caption{3D image of thermostated sample holder without thermoelectric
		Peltier modules, thermocouple and stirrer.}
	\label{\figlabel{thermostated_holder:3d}}
\end{figure}

This design of thermostated sample holder enabled fast control of the
temperature in range between 5 and 95\,\textcelsius{} of samples with volumes
in the range between 250 (with the adapter to 0.5-mm sample cells) and
3000\,\g{m}L. The heating from 5 to 95\,\textcelsius{} took $\sim$15\,min and
cooling from 95 to 5\,\textcelsius{} was as fast as 25\,min. The worst
performance of the temperature stabilization was on room temperature because of
switching between cooling and heating, so the process of stabilization the
temperature at 20\,\textcelsius{} from 15\,\textcelsius{} or from
25\,\textcelsius{} took $\sim 4$\,min 30\,s, the other other temperatures
stabilization took less than 3\,min with the 5\,\textcelsius{} step in the
range from 5 to 90\,\textcelsius{}. Stabilization from 90 to 95\,\textcelsius{}
took $\sim 4$\,min because 95\,\textcelsius{} is out of the recommended
operation range of the used Peltier modules. All the temperatures from above
was regarded as stabilized with precision of $\pm 0.3$\,\textcelsius{}. The
temperature then oscillated and continuously approached the setpoint
temperature and reached the setpoint temperature with $\pm 0.1$\,\textcelsius{}
precision with another $\sim 1$\,min of waiting. The temperature of liquid
inside cell followed the temperature measured by thermocouple from the holder
with approximately 10\,s delay (as measured by independendet thermocouple-based
thermometer) so we used 5\,min delay for temperature step of at most
5\,\textcelsius{} as safe interval for temperature stabilization between
Raman measurements with the exception of stabilization at 20 and
95\,\textcelsius{} where the safe delay needed to be prolonged to 7\,min.

During 5 years of usage of this holder the performance of heating/cooling of
the thermostated holder slightly degraded and the holder needed minor
maintanance. The performance was mainly impaired by degradation of thermal
grease ensuring reliable results of temperature measurement by the thermocouple
from inside the holder. The performace of Peltier modules decreases with the
time. We used one Peltier module at the beginning but after 3 years we added
the second one to mitigate the decrease of performance loss of the first one.

At the beginning, we were slightly concerned, if the fans, used for increase of
the efficiency of the heat sink, wouldn't affect the stability of the optical
elements (especially spectrograph and alignment of excitation laser beam) by
vibrations. But the effect was unobservable. High amount of heat is also
produced at high temperatures but it could freely dissipate from the sample
space and the room temperature was maintained by air conditioning. The air
conditioning also kept low humidity inside the laboratory but for temperatures
less then 15\,\textcelsius{} the continuous flow of dry air or nitrogen was
directed to the front surface of the sample cell.
