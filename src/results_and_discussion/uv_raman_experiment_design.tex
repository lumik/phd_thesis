\section{Design of UV Raman experiment}

Work on the construction of a new UV Raman spectrometer was an incremental
process. It was necessary to consider the equipment already present in the lab,
mainly the laser system, CCD camera and optical table and specify parameters of
chosen commercial spectrograph. Then the first functional prototype was
constructed and improved for measurement of polarized UV Raman spectra. The
first wavenumber calibration technique of measured Raman spectra
was adopted based on the literature. Then we started to deal with
photo-decomposition of samples by designing a spinning cell which could contain
as small as cca. 20 \g{m}l volumes of samples. After that we widened the range
of usable excitation wavelengths of the instrument and decided to seek for a
new means of wavenumber calibration and after some research and trials and
errors we settled with calibration to the spectra of Pt lamp. And then, the
apparatus was enhanced to enable temperature measurements in backscattering
geometry. Next sections describe each step of the development of the UV Raman
spectrometer in more details.


\subsection{Initial equipment}
\label{initial_equipment}

As an excitation source, we used the Coherent Innova 300C
MotoFreD\texttrademark{} Ion Laser which enables intracavity frequency
doubling~\parencite{Asher1993b} of fundamental Ar ion emission lines by
nonlinear crystal. As a frequency doubling crystal we used BBO crystal (also
from Coherent) designed for doubling 488\,nm. Later we extended our
experimental options with additional two crystals designed for doubling 514 and
457\,nm. The BBO crystals are hygroscopic so they needed to be purged by steady
flow of 0.25--0.5 l/min of dry nitrogen of at least grade 5 purity (99.999\%)
\parencite{Innova300MotoFreDManual}.
As a source of this nitrogen we decided to use
nitrogen generator NG1/1 from Gas Generation LTD which uses the pressure
swing absorption technology.

The above mentioned crystals could be used for
frequency doubling also the adjacent Ar ion lines. All the frequency doubled
wavelengths and expected and measured output laser beam powers which we could
tune with these three crystals are listed in
\tabref{initial_equipment:laser_power_spec}. The output wavelengths were
measured with HighFinesse High Precision Wavelength Meter WS5 UV-II and output
powers were measured with Power Meter from Thorlabs. The output powers in the
table were somewhat maximal which we could achieve, usually the power was
lower depending on laser condition. The 264.342\,nm wavelength was not possible
with our laser configuration because we couldn't tune the fundamental
528.693\,nm laser line for visible operation.

\begin{table}
	\centering
	\newcommand*{\nodotcell}[1]{\multicolumn{1}{r@{\hphantom{.}}}{#1}}

\begin{tabular}{c@{\hspace{5mm}}r@{\hspace{7mm}}>{\hspace{6mm}}r@{.}l}
\toprule
$\lambda$ (nm) & \multicolumn{1}{c}{$P_\text{e}$ (mW)}
                     & \multicolumn{2}{c}{$P_\text{m}$ (mW)} \\
\midrule

228.962        &  10 &              2&37 \\
238.238        &  30 &             33&5  \\
243.989        & 100 & \nodotcell{110}   \\
248.250        &  60 &             33&2  \\
250.854        &  15 &             11&4  \\
257.261        & 100 & \nodotcell{170}   \\

\bottomrule
\end{tabular}

	\caption{Specifications of laser power of frequency doubled lines.
	$P_\text{e}$ denotes expected laser power and $P_\text{m}$ measured
	laser powers. The wavelengths are measured in air at 20\,\textcelsius{}.}
	\label{\tablabel{initial_equipment:laser_power_spec}}
\end{table}

As a detector, we used liquid nitrogen cooled Princeton Instruments
SPEC-10:2KBUV/LN backiluminated CCD camera enhanced for UV light detection.
The camera has $2048\times512$ pixels of $13.5\times13.5$\,\g{m}m and can
be controlled with WinSpec software through ST133B/U camera controller unit.

\subsection{Choice of spectrograph parameters}
\begin{docitemize}
	\item Introduce dispersion according to focal length and groove density of
	grating.
	\begin{docitemize}
		\item How the theoretical values were computed using theoretical dispersion
			from Horiba manual.
		\item How the experimental values were measured and how we extrapolated
			for possible new grating selection
	\end{docitemize}
	\item Produce table of possible scenarios:
	\begin{docitemize}
		\item Theoretical values \tabref{spectrograph_selection:dispersion_est}
		\item Experimental values
	\end{docitemize}
	\item Present decision criteria.
\end{docitemize}


As a spectrograph, we selected Horiba iHR550 Imaging Spectrometer with aperture
f/6.4, focal length 550 mm, magnification 1.1 and triple-grating turret so we
needed to select three gratings with the best parameters for our purpose. One
grating position we reserved for fluorescence measurements with 300 gr/mm, but
gratings for the other two positions needed to be selected based on the
required spectral range and dispersion. For the spectral range estimation
we neglected all the nonlinearities of dispersion in dependence on wavelength
and used dispersion from spectrograph specifications which are shown in the
\tabref{spectrograph_selection:dispersion_spec} for available gratings.


\begin{table}
	\centering
	\begin{tabular}{l|cccc}
\toprule
grating (gr/mm)    & 1200 & 1800 & 2400 & 3600 \\
$\frac{\text{d}\lambda}{\text{d}x}$ (nm/mm) & 1.34 & 0.81 & 0.53 & 0.16 \\
\bottomrule
\end{tabular}
	\caption{Grating dispersion specifications taken from Horiba iHR550
		specification document. The linear dispersion
		$\frac{\text{d}\lambda}{\text{d}x}$ defines the extent to which a spectral
		interval is spread out across the focal field.}
	\label{\tablabel{spectrograph_selection:dispersion_spec}}
\end{table}


Then we calculated the maximal measured wavenumber $\tilde{\nu}_2$ from the
$\tilde{\nu}_1 = 200$\,\icm{} for all the available excitation wavelengths
$\lambda_\text{e}$ from our Ar ion frequency double laser in Stokes Raman
according to formula
\begin{equation}
	\tilde{\nu}_2 = \lambda_\text{e}^{-1}
		- \left(\frac{1}{\lambda_\text{e}^{-1} - \tilde{\nu}_1}
			+ w\frac{\text{d}\lambda}{\text{d}x}\right)^{-1},
	\label{\eqnlabel{spectrograph_selection:spectral_range_max_est}}
\end{equation}
where $w$ denotes width of a CCD camera and $\frac{\text{d}\lambda}{\text{d}x}$
grating linear diespersion which defines the extent to which a spectral
interval is spread out across the focal field. We also calculated minimal
measured wavenumber $\tilde{\nu}_1$ from the maximal
$\tilde{\nu}_2 = 4000$\,\icm{} using the formula
\begin{equation}
	\tilde{\nu}_1 = \lambda_\text{e}^{-1}
		- \left(\frac{1}{\lambda_\text{e}^{-1} - \tilde{\nu}_2}
			- w\frac{\text{d}\lambda}{\text{d}x}\right)^{-1}.
	\label{\eqnlabel{spectrograph_selection:spectral_range_min_est}}
\end{equation}

From these values, the average wavenumber dispersion per pixel
\begin{equation}
	\bar{d} = \frac{\tilde{\nu}_2 - \tilde{\nu}_1}{N_w},
	\label{\eqnlabel{spectrograph_selection:dispersion_est}}
\end{equation}
where $N_w = 2048$ is number of camera width pixels. The results of these
calculations can be seen in \tabref{spectrograph_selection:dispersion_est}

\begin{table}
	\centering
	\begin{tabular}{crrrrrrrr}
\toprule
$\lambda$ (nm)
	& grating & 1200 & \multicolumn{2}{c}{1800}
		& \multicolumn{2}{c}{2400} & \multicolumn{2}{c}{3600} \\
\midrule
\multirow{3}{*}{228.962}
	& $\tilde{\nu}_1$ &   200 &   200 &   131 &   200 &  1551 &   200 &  3291 \\
	& $\tilde{\nu}_2$ &  6231 &  4057 &  4000 &  2804 &  4000 &  1020 &  4000 \\
	& $\bar{d}$       & 2.945 & 1.883 & 1.889 & 1.271 & 1.196 & 0.401 & 0.346 \\
\midrule
\multirow{3}{*}{238.238}
	& $\tilde{\nu}_1$ &   200 &   200 &   470 &   200 &  1762 &   200 &  3351 \\
	& $\tilde{\nu}_2$ &  5799 &  3774 &  4000 &  2610 &  4000 &   958 &  4000 \\
	& $\bar{d}$       & 2.734 & 1.745 & 1.723 & 1.177 & 1.093 & 0.370 & 0.317 \\
\midrule
\multirow{3}{*}{243.989}
	& $\tilde{\nu}_1$ &   200 &   200 &   660 &   200 &  1881 &   200 &  3385 \\
	& $\tilde{\nu}_2$ &  5554 &  3613 &  4000 &  2500 &  4000 &   923 &  4000 \\
	& $\bar{d}$       & 2.614 & 1.667 & 1.631 & 1.123 & 1.035 & 0.353 & 0.300 \\
\midrule
\multirow{3}{*}{248.250}
	& $\tilde{\nu}_1$ &   200 &   200 &   791 &   200 &  1963 &   200 &  3408 \\
	& $\tilde{\nu}_2$ &  5382 &  3502 &  4000 &  2424 &  4000 &   898 &  4000 \\
	& $\bar{d}$       & 2.531 & 1.612 & 1.567 & 1.086 & 0.995 & 0.341 & 0.289 \\
\midrule
\multirow{3}{*}{250.854}
	& $\tilde{\nu}_1$ &   200 &   200 &   868 &   200 &  2011 &   200 &  3422 \\
	& $\tilde{\nu}_2$ &  5282 &  3436 &  4000 &  2379 &  4000 &   884 &  4000 \\
	& $\bar{d}$       & 2.481 & 1.580 & 1.529 & 1.064 & 0.971 & 0.334 & 0.282 \\
\midrule
\multirow{3}{*}{257.261}
	& $\tilde{\nu}_1$ &   200 &   200 &  1046 &   200 &  2122 &   200 &  3454 \\
	& $\tilde{\nu}_2$ &  5046 &  3282 &  4000 &  2274 &  4000 &   850 &  4000 \\
	& $\bar{d}$       & 2.366 & 1.505 & 1.442 & 1.013 & 0.917 & 0.318 & 0.267 \\
\bottomrule
\end{tabular}

	\caption{Spectrograph dispersion estimation. Gratins are denoted by number
		of grooves per mm, $\tilde{\nu}_1$ and $\tilde{\nu}_2$ are lowest and
		highest detected frequencies in \icm calculated according to
		\cref{%
			\eqnlabel{spectrograph_selection:spectral_range_max_est},%
			\eqnlabel{spectrograph_selection:spectral_range_min_est}%
		},
		respectively. The $\bar{d}$ denotes average dispersion in \icm/px
		calculated from \eqnref{spectrograph_selection:dispersion_est}.}
	\label{\tablabel{spectrograph_selection:dispersion_est}}
\end{table}

We can see, that if we want to see the full Raman vibration range from the
low frequency fibrations from 200\,\icm to valence hydrogen strething
vibrations to 4000\,\icm at all possible excitation wavelengths we need to
select grating with 1200 gr/mm. For the Raman fingerprint region below
1800\,\icm we need to chose the grating with 2400 gr/mm.

So, finally we chose grating with 300 gr/mm for possible fluorescence
measurements 1200 gr/mm for the measurement of full range including valence
hydrogen stretching vibrations at higher wavelengths (e.g. 257\,nm excitation)
and 2400 gr/mm for the Raman fingerprint region measurements.

\subsection{First layout of the instrument}
\label{first_layout}

At first, we built a proof of concept apparatus to test if we selected the
correct optical elements.
The optical schema is depicted in
\figref{initial_layout:apparatus_schema}.
The laser beam is emitted with horizontal polarization, and we decided that we
want to guide it through a sample in the vertical direction and gather the
scattered light in right-angle geometry.
Because of the steric restrictions in our laboratory, we needed to place the
laser in such a way that it produced a laser beam in the opposite direction
then we gathered the scattered light.
Because of that, we needed to guide the beam in the loop using laser mirrors
enhanced for 244\,nm \emph{M1} and \emph{M2} (Thorlabs) to maintain the right
polarization direction because we wanted to measure not only depolarized Raman
spectra.
The beam was then turned up by another laser mirror enhanced for 244\,nm
\emph{M3} and hit the bottom of the sample cell, having polarization
perpendicular to the direction of scattered signal acquisition.
The excitation beam was focused onto the sample by the plano-convex lens
\emph{L1} with a focal length of 100\,mm.
The most practical place for this lens was between M2 and M3, so we could not
select a shorter focal length as is proposed in
\cref{subsec:focus_optimization}.
The excitation laser beam could be attenuated by
the neutral density filters \emph{ND} with a selection of 0.5, 1, 2, 3, and 4
optical densities (Thorlabs), which were placed inside the carousel (Thorlabs)
for easy switching.
We also used an old single-blade shutter that we found in our laboratory.
The shutter was controlled from the CCD port for external shutter by the
WinSpec software provided with the detector.
We needed to insert a condenser into the path because the leading edge of the
output signal from CCD was too sharp.
The shutter was open during measurement and closed the rest of the time
to prevent photodecomposition of samples by the UV laser beam.

\begin{figure}
	\centering
	\begin{tikzpicture}[font=\sffamily]

% settings
\newcommand*{\cellBorderWidth}{3\pgflinewidth}
\newcommand*{\cellLineWidth}{1.5\pgflinewidth}
\definecolor{glassBorderColor}{RGB}{0,128,255}
\definecolor{glassFillColor}{RGB}{230,242,255}
\definecolor{waterFillColor}{RGB}{0,128,255}
\tikzset{
	clip
}
\tikzset{
	mirror element/.style={color=black, line width=2*\pgflinewidth},
	laser beam/.style={color=cyan, line width=2*\pgflinewidth},
	scattered ray/.style={color=red!60, line width=1.5*\pgflinewidth},
	scattered fill/.style={fill=red, draw=none, fill opacity=0.2},
	glass/.style={color=glassBorderColor, opacity=0.5,%
		fill=glassFillColor, fill opacity=0.5},
	sample cell/.style={color=glassBorderColor, opacity=0.5,%
		double=glassFillColor, double distance=\cellBorderWidth,
		line width=\cellLineWidth, line cap=rect},
	water fill/.style={fill=waterFillColor, fill opacity=0.1},
	mirror surface/.style={color=black!20, fill=black!10},
	nd filter/.style={color=black, opacity=0.2, fill=black, fill opacity=0.1},
	nd carousel/.style={color=black, opacity=0.4, fill=black, fill opacity=0.2},
	notch/.style={color=black, opacity=0.4, fill=black, fill opacity=0.1},
	aperture/.style={color=black, line width=2*\pgflinewidth},
	aperture filldraw/.style={color=black!40, fill=black!20, clip even odd rule},
	clip even odd rule/.code={\pgfseteorule},
	shutter/.style={nd carousel},
	shutter blade/.style={dashed, line width = 1.5*\pgflinewidth, opacity = 0.4}
};
\clip (-.1,-2.1) rectangle (14,6.6);
\coordinate (shutter) at (11,6);
\newcommand*{\samplePosWidth}{10}
\newcommand*{\samplePosHeight}{3}
\coordinate (samplePos) at (\samplePosWidth,\samplePosHeight);
\newcommand*{\cellWidth}{1};
\coordinate (cassegrainM1Center) at (\samplePosWidth - 0.5,\samplePosHeight);
\newcommand*{\cassegrainMARadius}{1.5}
\coordinate (cassegrainM2Center) at (\samplePosWidth - 0.7,\samplePosHeight);
\newcommand*{\cassegrainMBRadius}{0.4}
\newcommand*{\sqrttwo}{1.414213}
\coordinate (MS1Edge1) at (4.5,\samplePosHeight + 0.5);
\coordinate (MS1Edge2) at (3.5,\samplePosHeight - 0.5);
\coordinate (parabolaFocus) at (3,0.4);
\coordinate (MS2Edge1) at (4.5,1);
\coordinate (MS2EdgeControl1) at (245:0.5);
\coordinate (MS2Edge2) at (3.5,0);
\coordinate (MS2EdgeControl2) at (17:0.3);
\newcommand*{\apertureOuterRadius}{0.3}
\newcommand*{\apertureInnerRadius}{0.1}

% laser
\draw (0,5.5) rectangle ++(3,1) node[pos=.5] {laser};

% Mirror 3 - it should be under the beam so we have to draw it first
\draw[mirror surface] ($ (samplePos) + (-0.3,-\sqrttwo * 0.3) $)
	rectangle ++(0.6,\sqrttwo*0.6);
\node[above] at ($ (samplePos) + (0,0.5) $) {M3};

% laser beam
\draw[laser beam] (3,6) -- (13,6) coordinate (M1)
	-- (13,\samplePosHeight) coordinate (M2) -- (samplePos);

% aperture 2
\draw[aperture filldraw]
	(samplePos) circle (\apertureOuterRadius)
	(samplePos) circle (\apertureInnerRadius);

% neutral density filters
\newcommand*{\ndfilterA}{(3.5,5.8) rectangle ++(0.2,0.4)}
\newcommand*{\ndfilterB}{(3.5,5) rectangle ++(0.2,0.4)}
\draw[nd filter] \ndfilterA;
\draw[nd filter] \ndfilterB;
\draw[nd carousel] (3.45,4.9) rectangle ++(0.3,0.1);
\draw[nd carousel] (3.45,5.4) rectangle ++(0.3,0.4);
\draw[nd carousel] (3.45,6.2) rectangle ++(0.3,0.1);
\node[below] at (3.6,4.9) {ND};

\draw[shutter blade] ($ (shutter) + (0,0.3) $) -- ++(0,-0.6);
\draw[shutter] ($ (shutter) + (-0.1,-0.3) $) -- ++(0.2,0) -- ++(0,-0.4)
	-- ++(-0.2,0) -- cycle;
\node[below] at ($ (shutter) + (0,-0.7) $) {shutter};

% Mirror 1 and 2
\draw[mirror element] ($ (M1) + (-0.3,0.3) $)
	-- node[above,shift={(0.35cm,-0.05cm)}]{M1} ($ (M1) + (0.3,-0.3) $);
\draw[mirror element] ($ (M2) + (0.3,0.3) $)
	-- node[below,shift={(0.35cm,0.05cm)}]{M2} ($ (M2) + (-0.3,-0.3) $);

% Aperture 1
\draw[aperture] ($ (M2) + (-0.5,\apertureOuterRadius) $)
	node[above]{A1}
	-- ++(0,-\apertureOuterRadius + \apertureInnerRadius);
\draw[aperture] ($ (M2) + (-0.5,-\apertureOuterRadius) $)
	-- ++(0,\apertureOuterRadius - \apertureInnerRadius);

% Draw laser focusing lens
\newcommand*{\LARadius}{0.7}
\coordinate (L1Center) at (\samplePosWidth+1.6-\LARadius,\samplePosHeight);
\path[name path=L1Arc, shift={(L1Center)}]
	(270:\LARadius) arc (-90:90:\LARadius);
\path[name path=toL1Arc1] ($ (samplePos)  + (0,0.4) $) -- ++(10,0);
\path[name path=toL1Arc2] ($ (samplePos)  + (0,-0.4) $) -- ++(10,0);
\path[name intersections={of=L1Arc and toL1Arc1, by=L11}];
\mypgfextractangle{\LAAAngle}{L1Center}{L11}
\path[name intersections={of=L1Arc and toL1Arc2, by=L12}];
\mypgfextractangle{\LABAngle}{L1Center}{L12}
\draw[glass] (L11) arc (\LAAAngle:\LABAngle-360:\LARadius) -- ++(-0.05,0)
	-- ($ (L11) + (-0.05,0) $) -- cycle;
\node[above] at (L11) {L1};


%%%%%%%%%%%%%%%%%%%%%
% draw the cassegrain

% calculate intersections with mirror 1 (the objective mirror)
% mirror1 arc
\path[name path=M1arc, shift={(cassegrainM1Center)}]
	(90:\cassegrainMARadius) arc (90:270:\cassegrainMARadius);
% upper top ray
\path[name path=toCassegrainM11] (samplePos) -- ++(135:5);
\path[name intersections={of=M1arc and toCassegrainM11, by=cassegrainM11}];
\mypgfextractangle{\cassegrainMAAAngle}{cassegrainM1Center}{cassegrainM11}
% upper bottom ray
\path[name path=toCassegrainM12] (samplePos) -- ++(165:5);
\path[name intersections={of=M1arc and toCassegrainM12, by=cassegrainM12}];
\mypgfextractangle{\cassegrainMABAngle}{cassegrainM1Center}{cassegrainM12}
% lower top ray
\path[name path=toCassegrainM13] (samplePos) -- ++(195:5);
\path[name intersections={of=M1arc and toCassegrainM13, by=cassegrainM13}];
\mypgfextractangle{\cassegrainMACAngle}{cassegrainM1Center}{cassegrainM13}
% lower bottom ray
\path[name path=toCassegrainM14] (samplePos) -- ++(225:5);
\path[name intersections={of=M1arc and toCassegrainM14, by=cassegrainM14}];
\mypgfextractangle{\cassegrainMADAngle}{cassegrainM1Center}{cassegrainM14}

\draw[scattered ray] (samplePos) -- (cassegrainM11);
\draw[scattered ray] (samplePos) -- (cassegrainM12);
\draw[scattered fill] (samplePos) -- (cassegrainM11)
	arc (\cassegrainMAAAngle:\cassegrainMABAngle:\cassegrainMARadius) -- cycle;
\draw[scattered ray] (samplePos) -- (cassegrainM13);
\draw[scattered ray] (samplePos) -- (cassegrainM14);
\draw[scattered fill] (samplePos) -- (cassegrainM13)
	arc (\cassegrainMACAngle:\cassegrainMADAngle:\cassegrainMARadius) -- cycle;

% draw the cell
\draw[sample cell, water fill]
	($ (samplePos)%
		+ (-0.5 * \cellBorderWidth - \cellLineWidth,-0.5 * \cellWidth) $)
		rectangle ++(\cellWidth,\cellWidth);
\node[below] at ($ (samplePos) + (0.5 * \cellWidth,-0.5 * \cellWidth)%
	+ (-0.5 * \cellBorderWidth,0) + (-0.5 * \cellLineWidth,0) $) {S};

% calculate intersections with mirror 2 (the objective mirror)
% mirror2 arc
\path[
	name path=M2arc, shift={(cassegrainM2Center)}] (90:\cassegrainMBRadius)
		arc (90:270:\cassegrainMBRadius);
% calculate cassegrain mirror2 edges
\path[
	name intersections={of=M2arc and toCassegrainM12, by=cassegrainM2Edge1}];
\mypgfextractangle{\cassegrainMBAAngle}{cassegrainM2Center}{cassegrainM2Edge1}
\path[
	name intersections={of=M2arc and toCassegrainM13, by=cassegrainM2Edge2}];
\mypgfextractangle{\cassegrainMBDAngle}{cassegrainM2Center}{cassegrainM2Edge2}
% upper bottom ray
\path[name path=toCassegrainM22] ($ (samplePos) + (0,0.1) $) -- ++(-10,0);
\path[name intersections={of=M2arc and toCassegrainM22, by=cassegrainM22}];
\mypgfextractangle{\cassegrainMBBAngle}{cassegrainM2Center}{cassegrainM22}
% lower top ray
\path[name path=toCassegrainM23] ($ (samplePos) + (0,-0.1) $) -- ++(-10,0);
\path[name intersections={of=M2arc and toCassegrainM23, by=cassegrainM23}];
\mypgfextractangle{\cassegrainMBCAngle}{cassegrainM2Center}{cassegrainM23}

\draw[scattered ray] (cassegrainM11) -- (cassegrainM2Edge1);
\draw[scattered ray] (cassegrainM12) -- (cassegrainM22);
\draw[scattered fill] (cassegrainM2Edge1)
	arc (\cassegrainMBAAngle:\cassegrainMBBAngle:\cassegrainMBRadius)
		-- (cassegrainM12)
	arc (\cassegrainMABAngle:\cassegrainMAAAngle:\cassegrainMARadius) -- cycle;
\draw[scattered ray] (cassegrainM13) -- (cassegrainM23);
\draw[scattered ray] (cassegrainM14) -- (cassegrainM2Edge2);
\draw[scattered fill] (cassegrainM23)
	arc (\cassegrainMBCAngle:\cassegrainMBDAngle:\cassegrainMBRadius)
		-- (cassegrainM14)
	arc (\cassegrainMADAngle:\cassegrainMACAngle:\cassegrainMARadius) -- cycle;

% to mirror MS1
% path representing the mirror
\path[name path=MS1Path] (MS1Edge1) -- (MS1Edge2);
% intersections with the mirror
% ray 1
\path[name path=toCassegrainM21] (cassegrainM2Edge1) -- ++(-10,0);
\path[name intersections={of=MS1Path and toCassegrainM21, by=MS11}];
% ray 2
\path[name intersections={of=MS1Path and toCassegrainM22, by=MS12}];
% ray 3
\path[name intersections={of=MS1Path and toCassegrainM23, by=MS13}];
% ray 4
\path[name path=toCassegrainM24] (cassegrainM2Edge2) -- ++(-10,0);
\path[name intersections={of=MS1Path and toCassegrainM24, by=MS14}];

\draw[scattered ray] (cassegrainM2Edge1) -- (MS11);
\draw[scattered ray] (cassegrainM22) -- (MS12);
\draw[scattered fill] (cassegrainM2Edge1)
	arc (\cassegrainMBAAngle:\cassegrainMBBAngle:\cassegrainMBRadius) -- (MS12)
		-- (MS11) -- cycle;
\draw[scattered ray] (cassegrainM23) -- (MS13);
\draw[scattered ray] (cassegrainM2Edge2) -- (MS14);
\draw[scattered fill] (cassegrainM23)
	arc (\cassegrainMBCAngle:\cassegrainMBDAngle:\cassegrainMBRadius) -- (MS14)
		-- (MS13) -- cycle;

% draw first mirror of cassegrain
\draw[mirror element]
	(cassegrainM11)
		arc (\cassegrainMAAAngle:\cassegrainMABAngle:\cassegrainMARadius)
			node[left,pos=0.5] {O};
\draw[mirror element]
	(cassegrainM13)
		arc (\cassegrainMACAngle:\cassegrainMADAngle:\cassegrainMARadius);
% mirror 2
\draw[mirror element]
	(cassegrainM2Edge1)
		arc (\cassegrainMBAAngle:\cassegrainMBDAngle:\cassegrainMBRadius);

% draw notch
\draw[notch] ($ (samplePos) + (-3,-0.5) $) rectangle ++(0.2,1);
\node[above] at ($ (samplePos) + (-2.9,0.5) $) {EF};

% Parabolic mirror MS2
\newcommand*{\parabolicMirrorDef}{%
		(MS2Edge2)
		.. controls ($ (MS2Edge2) + (MS2EdgeControl2) $)
			and ($ (MS2Edge1) + (MS2EdgeControl1) $)
		.. (MS2Edge1)
}
\path[name path=MS2Path] \parabolicMirrorDef;
% ray 1
\path[name path=toMS21] (MS11) -- ++(0,-10);
\path[name intersections={of=MS2Path and toMS21, by=MS21}];
% ray 2
\path[name path=toMS22] (MS12) -- ++(0,-10);
\path[name intersections={of=MS2Path and toMS22, by=MS22}];
% ray 3
\path[name path=toMS23] (MS13) -- ++(0,-10);
\path[name intersections={of=MS2Path and toMS23, by=MS23}];
% ray 4
\path[name path=toMS24] (MS14) -- ++(0,-10);
\path[name intersections={of=MS2Path and toMS24, by=MS24}];

\draw[scattered ray] (MS11) -- (MS21);
\draw[scattered ray] (MS12) -- (MS22);
\begin{scope}
	\clip (MS11) -- ($ (MS21) + (0,-1) $) -- ($ (MS22) + (0,-1) $) -- (MS12)
		-- cycle;
	\draw[scattered fill] (MS12) -- \parabolicMirrorDef -- (MS11) -- cycle;
\end{scope}
\draw[scattered ray] (MS13) -- (MS23);
\draw[scattered ray] (MS14) -- (MS24);
\begin{scope}
	\clip (MS13) -- ($ (MS23) + (0,-1) $) -- ($ (MS24) + (0,-1) $) -- (MS14)
		-- cycle;
	\draw[scattered fill] (MS14) -- \parabolicMirrorDef -- (MS13) -- cycle;
\end{scope}

% draw MS1 over all incident rays on that mirror
\draw[mirror element]
	(MS1Edge1) -- node[above,shift={(-0.5cm,-0.1cm)}]{MS1} (MS1Edge2);

\draw[scattered ray] (MS21) -- (parabolaFocus);
\draw[scattered ray] (MS22) -- (parabolaFocus);
\begin{scope}
	\clip (parabolaFocus) -- (MS21) -- ++(1,0) -- ($ (MS22) + (1,0) $) -- (MS22)
		-- cycle;
	\draw[scattered fill] (parabolaFocus) -- \parabolicMirrorDef -- cycle;
\end{scope}
\draw[scattered ray] (MS23) -- (parabolaFocus);
\draw[scattered ray] (MS24) -- (parabolaFocus);
\begin{scope}
	\clip (parabolaFocus) -- (MS23) -- ++(1,0) -- ($ (MS24) + (1,0) $) -- (MS24)
		-- cycle;
	\draw[scattered fill] (parabolaFocus) -- \parabolicMirrorDef -- cycle;
\end{scope}

% draw the parabolic mirror MS2
\draw[mirror element]
	(MS2Edge2)
	.. controls ($ (MS2Edge2) + (MS2EdgeControl2) $)
		and ($ (MS2Edge1) + (MS2EdgeControl1) $)
	.. node[below,shift={(0.5cm,0.1cm)}]{MS2} (MS2Edge1);

\draw (0,0) rectangle ++(3,2.5) node[pos=.5] {spectrograph};

% side view
\begin{scope}[shift={(\samplePosWidth - 2,-1.7)}]

\newcommand*{\samplePosSideWidth}{2}
\newcommand*{\samplePosSideHeight}{2.5}
\coordinate (samplePosSide) at (\samplePosSideWidth,\samplePosSideHeight);
\coordinate (cassegrainM1SideCenter)
	at (\samplePosSideWidth - 0.5,\samplePosSideHeight);
\coordinate (cassegrainM2SideCenter)
	at (\samplePosSideWidth - 0.7,\samplePosSideHeight);

% clip the view
\clip (-.05,-.35) rectangle ++(3.3,3.4);
\draw (-.05,-.35) rectangle ++(3.3,3.4);

% laser beam
\draw[laser beam] (4,0) -- (\samplePosSideWidth,0) coordinate (M3Side)
	-- (samplePosSide);

% mirror 3
\draw[mirror element] ($ (M3Side) + (-0.3,0.3) $)
	-- node[above,shift={(0.35cm,-0.05cm)}]{M3} ($ (M3Side) + (0.3,-0.3) $);

% aperture 2
\draw[aperture] ($ (M3Side) + (\apertureOuterRadius,0.8) $)
	node[above, shift={(0.05,0)}]{A2}
	-- ++(-\apertureOuterRadius + \apertureInnerRadius,0);
\draw[aperture] ($ (M3Side) + (-\apertureOuterRadius,0.8) $)
	-- ++(\apertureOuterRadius - \apertureInnerRadius,0);

% draw the cassegrain
% calculate intersections with mirror 1 (the objective mirror)
% mirror1 arc
\path[name path=M1SideArc, shift={(cassegrainM1SideCenter)}]
	(90:\cassegrainMARadius) arc (90:270:\cassegrainMARadius);
% upper top ray
\path[name path=toCassegrainM11Side] (samplePosSide) -- ++(135:5);
\path[name intersections={of=M1SideArc and toCassegrainM11Side,
	by=cassegrainM11Side}];
% upper bottom ray
\path[name path=toCassegrainM12Side] (samplePosSide) -- ++(165:5);
\path[name intersections={of=M1SideArc and toCassegrainM12Side,
	by=cassegrainM12Side}];
% lower top ray
\path[name path=toCassegrainM13Side] (samplePosSide) -- ++(195:5);
\path[name intersections={of=M1SideArc and toCassegrainM13Side,
	by=cassegrainM13Side}];
% lower bottom ray
\path[name path=toCassegrainM14Side] (samplePosSide) -- ++(225:5);
\path[name intersections={of=M1SideArc and toCassegrainM14Side,
	by=cassegrainM14Side}];

\draw[scattered ray] (samplePosSide) -- (cassegrainM11Side);
\draw[scattered ray] (samplePosSide) -- (cassegrainM12Side);
\draw[scattered fill] (samplePosSide) -- (cassegrainM11Side)
	arc (\cassegrainMAAAngle:\cassegrainMABAngle:\cassegrainMARadius) -- cycle;
\draw[scattered ray] (samplePosSide) -- (cassegrainM13Side);
\draw[scattered ray] (samplePosSide) -- (cassegrainM14Side);
\draw[scattered fill] (samplePosSide) -- (cassegrainM13Side)
	arc (\cassegrainMACAngle:\cassegrainMADAngle:\cassegrainMARadius) -- cycle;

% draw the cell
\draw[sample cell, water fill]
	($ (samplePosSide) + (-0.5 * \cellBorderWidth - \cellLineWidth,%
		-0.5 * \cellBorderWidth - \cellLineWidth) $)
		rectangle ++(\cellWidth,10);

% calculate intersections with mirror 2 (the objective mirror)
% mirror2 arc
\path[
	name path=M2SideArc, shift={(cassegrainM2SideCenter)}] (90:\cassegrainMBRadius)
		arc (90:270:\cassegrainMBRadius);
% calculate cassegrain mirror2 edges
\path[
	name intersections={%
		of=M2SideArc and toCassegrainM12Side, by=cassegrainM2SideEdge1}];
\path[
	name intersections={%
		of=M2SideArc and toCassegrainM13Side, by=cassegrainM2SideEdge2}];
% upper bottom ray
\path[name path=toCassegrainM22Side]%
	($ (samplePosSide) + (0,0.1) $) -- ++(-10,0);
\path[name intersections={%
	of=M2SideArc and toCassegrainM22Side, by=cassegrainM22Side}];
% lower top ray
\path[name path=toCassegrainM23Side]
	($ (samplePosSide) + (0,-0.1) $) -- ++(-10,0);
\path[name intersections={%
	of=M2SideArc and toCassegrainM23Side, by=cassegrainM23Side}];

\draw[scattered ray] (cassegrainM11Side) -- (cassegrainM2SideEdge1);
\draw[scattered ray] (cassegrainM12Side) -- (cassegrainM22Side);
\draw[scattered fill] (cassegrainM2SideEdge1)
	arc (\cassegrainMBAAngle:\cassegrainMBBAngle:\cassegrainMBRadius)
		-- (cassegrainM12Side)
	arc (\cassegrainMABAngle:\cassegrainMAAAngle:\cassegrainMARadius) -- cycle;
\draw[scattered ray] (cassegrainM13Side) -- (cassegrainM23Side);
\draw[scattered ray] (cassegrainM14Side) -- (cassegrainM2SideEdge2);
\draw[scattered fill] (cassegrainM23Side)
	arc (\cassegrainMBCAngle:\cassegrainMBDAngle:\cassegrainMBRadius)
		-- (cassegrainM14Side)
	arc (\cassegrainMADAngle:\cassegrainMACAngle:\cassegrainMARadius) -- cycle;

% to mirror MS1
% ray 1
\draw[scattered ray] (cassegrainM2SideEdge1) -- ++(-10,0);
\draw[scattered ray] (cassegrainM22Side) -- ++(-10,0);
\draw[scattered fill] (cassegrainM2SideEdge1)
	arc (\cassegrainMBAAngle:\cassegrainMBBAngle:\cassegrainMBRadius)
		-- ++(-10,0) -- ($ (cassegrainM2SideEdge1) + (-10,0) $) -- cycle;
\draw[scattered ray] (cassegrainM23Side) -- ++(-10,0);
\draw[scattered ray] (cassegrainM2SideEdge2) -- ++(-10,0);
\draw[scattered fill] (cassegrainM23Side)
	arc (\cassegrainMBCAngle:\cassegrainMBDAngle:\cassegrainMBRadius) --
		++(-10,0) -- ($ (cassegrainM2SideEdge1) + (-10,0) $) -- cycle;

% draw first mirror of cassegrain
\draw[mirror element]
	(cassegrainM11Side)
		arc (\cassegrainMAAAngle:\cassegrainMABAngle:\cassegrainMARadius)
			node[left,pos=0.5] {O};
\draw[mirror element]
	(cassegrainM13Side)
		arc (\cassegrainMACAngle:\cassegrainMADAngle:\cassegrainMARadius);
% mirror 2
\draw[mirror element]
	(cassegrainM2SideEdge1)
		arc (\cassegrainMBAAngle:\cassegrainMBDAngle:\cassegrainMBRadius);

\end{scope}

\end{tikzpicture}
	\caption[%
		Top-view schema of the apparatus with side-view inset of the sample
		space.%
	]{%
		\captiontitle{%
			Top-view schema of the apparatus with side-view inset of the sample
			space.%
		}
		A laser beam is emitted with horizontal polarization.
		It is guided through
			a carousel with neutral density (ND) filters
			and shutter
			and by laser mirrors M1 -- M3
			and laser focusing lens (L1)
			to the sample cell (S)
			through apertures A1 and A2.
		The scattered light is
			gathered by Cassegrain objective (O),
			reflected by the mirror MS1
			and focused on the entrance slit of spectrograph by parabolic mirror MS2.
			Edge filter (EF) suppresses elastically scattered light.
	}
	\label{\figlabel{initial_layout:apparatus_schema}}
\end{figure}

We needed to visualize the beam for proper laser beam path setup because the
wavelengths in use are outside of the visible light range.
Business cards showed to be the best equipment for beam visualization because
the paper used for them usually has strong fluorescence.
Excitation beam adjustment procedure uses two apertures \emph{A1}, \emph{A2}
and crosshair reticle printed on the paper fixed to the ceiling of the room.
At first, the kinematic holder of mirror M1 is adjusted, so the beam passes
through aperture A1.
If the beam is blocked by some of the other optical elements on its way to the
ceiling reticle, we can use a business card to detect if it passes through the
center of the aperture A1.
Then the mirror M2 is adjusted so that the beam passes through A2, and finally,
the mirror M3 is adjusted to hit the center of the reticle.
The procedure can be iteratively repeated if the result is not satisfactory but
the good alignment was usually achieved after the first iteration.

Raman scattered light is no longer monochromatic, so we decided to use
reflective optics in the light gathering part of the apparatus to avoid
chromatic aberration.
We selected Cassegrain objective \emph{O} (Newport) with $f_o = 13$\,mm
effective front focal length, back focal length in infinity, 0.4 numerical
aperture, and 9.72\,mm diameter of the output beam.
The elastically scattered light is removed by edge filter \emph{EF} (Semrock).
The rest of the collimated signal is then guided by UV enhanced broadband
aluminum mirror \emph{MS1} (Newport) to focusing off-axis parabolic mirror
\emph{MS2} (Newport) with $f_c = 101.6$\,mm effective focal length to the
spectrograph slit.
The spectrograph has f/6.4 aperture, whereas the focusing lens with the
beam diameter determined by objective has aperture $101.6 / 9.72 = f/10.5$,
which means that we slightly underfill the spectrometer's angle of acceptance.

\subsection{Focal length of exciting laser focusing lens estimation}

For the design of the spectrometer, it is necessary to optimize the focal
length of a laser focusing lens to deliver the most signal through the
scattered-light-imaging system onto the detector. Suppose that the focused
beam is cylindrically symmetric about the $z$ axis. Let consider focusing
angle of the beam $\vartheta$ as the full angle between the $1/e^2$ intensity
points in the far-field limit. In practice, the small angle approximation can
be used:
\begin{equation}
	\vartheta = \frac{d}{f},
	\label{\eqnlabel{focus_optimization:theta}}
\end{equation}
where $d$ is the diameter (between the $1/e^2$ intensity points) of laser
beam at the focusing lens and $f$ is the focal length of this lens.

According to theory~\parencite{Boyd1961,Boyd1962}, the radius of the minimum
spot $\omega_0$ may be expressed in terms of $\vartheta$ and focused laser
beam wavelength $\lambda$ as
\begin{equation*}
	\omega_0 = \frac{2\lambda}{\text{\g{p}}\vartheta}
	\label{\eqnlabel{focus_optimization:min_spot_diameter}}
\end{equation*}
and the confocal parameter $b$ (the distance between perpendicular sections
of beam with diameter $\sqrt{2}\omega_0$, see \figref{GaussianBeamWaist_wiki})
may be written as
\begin{equation*}
	b = \frac{2\pi\omega_0^2}{\lambda} =
		\frac{8\lambda}{\text{\g{p}}\vartheta^2}.
	\label{\eqnlabel{focus_optimization:confocal_parameter}}
\end{equation*}

\begin{figure}
	\centering
	%% Creator: Inkscape 0.48.3.1, www.inkscape.org
%% PDF/EPS/PS + LaTeX output extension by Johan Engelen, 2010
%% Accompanies image file 'GaussianBeamWaist_wiki.pdf' (pdf, eps, ps)
%%
%% To include the image in your LaTeX document, write
%%   \input{<filename>.pdf_tex}
%%  instead of
%%   \includegraphics{<filename>.pdf}
%% To scale the image, write
%%   \def\svgwidth{<desired width>}
%%   \input{<filename>.pdf_tex}
%%  instead of
%%   \includegraphics[width=<desired width>]{<filename>.pdf}
%%
%% Images with a different path to the parent latex file can
%% be accessed with the `import' package (which may need to be
%% installed) using
%%   \usepackage{import}
%% in the preamble, and then including the image with
%%   \import{<path to file>}{<filename>.pdf_tex}
%% Alternatively, one can specify
%%   \graphicspath{{<path to file>/}}
%%
%% For more information, please see info/svg-inkscape on CTAN:
%%   http://tug.ctan.org/tex-archive/info/svg-inkscape
%%
\begingroup%
  \makeatletter%
  \providecommand\color[2][]{%
    \errmessage{(Inkscape) Color is used for the text in Inkscape, but the package 'color.sty' is not loaded}%
    \renewcommand\color[2][]{}%
  }%
  \providecommand\transparent[1]{%
    \errmessage{(Inkscape) Transparency is used (non-zero) for the text in Inkscape, but the package 'transparent.sty' is not loaded}%
    \renewcommand\transparent[1]{}%
  }%
  \providecommand\rotatebox[2]{#2}%
  \ifx\svgwidth\undefined%
    \setlength{\unitlength}{333.15039062bp}%
    \ifx\svgscale\undefined%
      \relax%
    \else%
      \setlength{\unitlength}{\unitlength * \real{\svgscale}}%
    \fi%
  \else%
    \setlength{\unitlength}{\svgwidth}%
  \fi%
  \global\let\svgwidth\undefined%
  \global\let\svgscale\undefined%
  \makeatother%
  \begin{picture}(1,0.48482369)%
    \put(0,0){\includegraphics[width=\unitlength]{results_and_discussion/assets/GaussianBeamWaist_wiki.pdf}}%
    \put(0.20711366,0.258742){\color[rgb]{0,0,0}\makebox(0,0)[lb]{\smash{$\sqrt{2}\omega_0$}}}%
    \put(0.46945764,0.3786023){\color[rgb]{0,0,0}\makebox(0,0)[lb]{\smash{$b$}}}%
    \put(0.40642304,0.25814167){\color[rgb]{0,0,0}\makebox(0,0)[lb]{\smash{$\omega_0$}}}%
    \put(0.88008299,0.33378319){\color[rgb]{0,0,0}\makebox(0,0)[lb]{\smash{$\omega(z)$}}}%
    \put(0.97412911,0.23552655){\color[rgb]{0,0,0}\makebox(0,0)[lb]{\smash{$z$}}}%
    \put(0.7029858,0.28052652){\color[rgb]{0,0,0}\makebox(0,0)[lb]{\smash{$\vartheta$}}}%
    \put(0.38180954,0.10125008){\color[rgb]{0,0,0}\makebox(0,0)[lb]{\smash{$z_\mathrm{R}$}}}%
  \end{picture}%
\endgroup%

	\caption{The geometry of a focal region of a laser beam. The beam is
	considered to be cylindrically symmetric about the $z$ axis, the effective
	volume of the Raman sample was taken to be the volume of a cylinder of
	diameter $2\omega_0$ and length $2b$. Adopted from
	\textcite{GaussianBeamWaist}.}
	\label{\figlabel{GaussianBeamWaist_wiki}}
\end{figure}

The beam radius $\omega(z)$ can be described as the function of $z$
coordinate by equation
\begin{equation}
	\omega(z) = \omega_0\sqrt{1+\left(\frac{2z}{b}\right)^2}.
	\label{\eqnlabel{focus_optimization:beam_radius}}
\end{equation}

As a first estimation, consider non-resonance Raman scattering produced from
homogenous infinite medium and arising only from the region with the highest
illumination intensity. Furthermore, assume the loss of the illumination beam
intensity due to the Raman scattered light as negligible and that the Raman
medium is fully transparent to excitation and scattered light, it means that
there is negligible absorption and stimulated emission. The amount of Raman
scattered light from the thin transverse slice of the excitation beam is
independent of the distribution of the energy in the slice and therefore is
proportional to the total illuminating laser-beam power under these
assumptions. Then, view the Raman emission from the particular direction in
the plane of the slice. The emitted flux would be the integrated flux from
all elements on the line intersecting the slice so there would be a linear
increase of the number of elements with the increasing diameter, but each
element illumination power decreases with the square of the diameter of the
illuminating beam.

Considering the \eqnref{focus_optimization:beam_radius} derive the
approximate dependence of irradiance of that point by the Raman-scattered
light on the $z$ coordinate
\begin{equation*}
	I \propto \frac{1}{\omega(z)} \propto \frac{1}{\sqrt{1 + (2z/b)^2}}.
	\label{\eqnlabel{focus_optimization:slice_irradiance}}
\end{equation*}

So, the irradiance at $z = b$ will be reduced from the $z = 0$ by a factor of
$1/\sqrt{5} \doteq 0.447 \doteq 1/2$. For the next calculation, the brightest
region of excitation beam was approximated by the “source cylinder” of length
$2b$ and diameter $2\omega_0$ and therefore included only the brightest
region of Raman emission and neglect the emission from less-bright regions.
The source parameters are then \parencite{Barrett1968}:
\begin{align}
	\text{length:}&
		& L_\text{E}& = 16\lambda/(\text{\g{p}}\vartheta^2)
	\label{\eqnlabel{focus_optimization:L_E}}\\
	\text{diameter (width of scattering area):}&
		& W_\text{E}& = 4\lambda/(\text{\g{p}}\vartheta)
	\label{\eqnlabel{focus_optimization:W_E}}\\
	\text{volume:}&
		& V_\text{E}& = 64 \lambda^3/(\text{\g{p}}^2\vartheta^4)
	\label{\eqnlabel{focus_optimization:V_E}}\\
	\text{length-to-diameter ratio:}&
		& L_\text{E}/W_\text{E}& = 4/\vartheta.
	\label{\eqnlabel{focus_optimization:LW_E}}
\end{align}

These approximations apply to diffraction-limited Gaussian beams (
TEM\textsubscript{00}), for low-order transverse modes or combination of such
modes, the calculations based on these equations can be considered accurate
within an order of magnitude.

Many simple arguments \parencite{Atwood1963,Barrett1968} were used to propose
that the Raman spectrometer using right angle geometry scattering geometry
will gain higher signal if the laser beam is more focused. None of these
arguments, however, do not establish if there is an optimum degree of
focusing less then the largest practical value. \textcite{Barrett1968} showed
that such an optimum exists theoretically; they calculated its approximate
value, and they compared the Raman power that can be usefully collected at
that optimum degree of focusing with the total emitted Raman power.

It is known that the light is scattered to all directions by the Raman effect
and within the assumptions stated above, the Raman intensity is linearly
proportional to the excitation laser beam power. Therefore, if the amount of
Raman scattered light gathered by the spectrometer should be maximized, the
light needs to be collected from the largest solid angle and the largest
possible illuminated volume. Typical Raman spectrometer light collecting path
has two optical parts, spectrograph, and light collecting optics. Let
consider their parameters as given even though similar considerations as
bellow can also be applied to the light collecting optics optimization. Lets
further consider that the spectrograph consists of an entrance slit, imaging
optics, dispersing element, and CCD camera.

For simplicity approximate the source as flat ribbon with the same length and
width as specified above for the “source cylinder” which radiates uniformly
to all collected angles from all parts of its surface. Then, consider that
the light collecting optic displays the entrance slit of the spectrograph to
the center of the focal region of the illuminating beam. The width of the
slit $W_\mathrm{s}$ is determined by the spectrograph settings, particularly
by the desired spectral resolution, and its length $L_\mathrm{s}$ could be
estimated as an image of the CCD chip through the spectrograph. Moreover,
suppose that the aperture of light collecting optic is large enough further
not to constrain the collected Raman light over the spectrograph aperture.

Then there are two competing processes. The assumptions above that the amount
of the Raman scattered light from all thin transverse slices of the
excitation beam is the same implies that the total amount of Raman scattered
light from the source ribbon is dependent only on its length. As the focal
length of the excitation beam focusing lens is decreased, the source ribbon
is getting narrower, which results in more intensive emission of the
scattered light per unit area. So if we start with the source ribbon width
larger than the image of the spectrograph entrance slit width, the amount of
collected light increases with the narrower source ribbon. In the opposite,
the length (and therefore the total Raman scattered light intensity) of the
source ribbon decreases with the decreasing focal length of the excitation
beam focusing lens and so when the source ribbon gets shorter than the image
of the spectrograph entrance slit length, the amount of collected light
decreases with the shorter source ribbon.

So, lets split further analysis of the dependence of the Raman flux
transmitted through the spectrometer on the focusing angle of the excitation
beam $\vartheta$ (\eqnref{focus_optimization:theta}) to regions divided by
two focusing angles $\vartheta_\mathrm{W}$ and $\vartheta_\mathrm{L}$. The
$\vartheta_\mathrm{W}$ defines focusing angle at which the width of the
focusing ribbon $W_\text{E}$ is the same as the width $W$ of image of the
spectrograph entrance slit width $W_\text{S}$, and $\vartheta_\mathrm{L}$ is
angle which causes equality of the source ribbon length $L_\text{E}$ and
length $L$ of the spectrograph entrance slit image length $L_\text{S}$ on the
source. Further, define magnification of the scattered light collecting
optics as $M$. Then
\begin{align}
	L& = \frac{L_\mathrm{S}}{M}
	\label{\eqnlabel{focus_optimization:slit_length_magnification}}\\
	W& = \frac{W_\mathrm{S}}{M}
	\label{\eqnlabel{focus_optimization:slit_width_magnification}}
\end{align}
and using \eqnref{focus_optimization:L_E} and
\labelcref{\eqnlabel{focus_optimization:W_E}} and putting $L = L_\mathrm{E}$
and $W = W_\mathrm{E}$ respectively we get
\begin{align}
	\vartheta_\mathrm{W} = \frac{4\lambda}{\text{\g{p}}W}
		= \frac{4\lambda M}{\text{\g{p}}W_\mathrm{S}}
	\label{\eqnlabel{focus_optimization:theta_W}}\\
	\vartheta_\mathrm{L} = \sqrt{\frac{16\lambda}{\text{\g{p}}L}}
		= \sqrt{\frac{16\lambda M}{\text{\g{p}}L_\mathrm{S}}}.
	\label{\eqnlabel{focus_optimization:theta_L}}
\end{align}

For our spectrometer, we have $W_\mathrm{S} = 50$\,\g{m}m and beam diameter
$d = 0.9$\,mm. The wavelength can be calculated from the wavelength in vacuum
$\lambda_0 = 257.2$\,nm divided by the refractive index of water
$n_{257} = 1.3598$ \parencite{Hale1973} $\lambda = 189.1$\.nm The
$L_\mathrm{S}$ can be calculated from the height of the CCD chip
$L_\text{CCD} = 6.9$\,mm and magnification of spectrograph
$M_\text{spc} = 1.1$ as
$L_\text{S} = L_\text{CCD}/M_\text{spc} \doteq 6.3$\,mm, the magnification $M$
may be computed from focal length of objective $f_\text{o} = 13$\,mm and
focusing lens before monochromator $f \text{c} = 101.6$\,mm as
$M = f_\text{c}/f_\text{o} \doteq 7.82$. Then, from the above equations, we
calculate
\begin{align}
	\vartheta_\mathrm{W} \doteq 0.0376
	\label{\eqnlabel{focus_optimization:theta_W_calc}}\\
	\vartheta_\mathrm{L} \doteq 0.0346,
	\label{\eqnlabel{focus_optimization:theta_L_calc}}
\end{align}
which correspond regarding \eqnref{focus_optimization:theta} to
\begin{align}
	f_\mathrm{W} \doteq 24\,\text{mm}
	\label{\eqnlabel{focus_optimization:f_W_calc}}\\
	f_\mathrm{L} \doteq 26\,\text{mm}
	\label{\eqnlabel{focus_optimization:f_L_calc}}
\end{align}
respectively.

We can note, that $\vartheta_\mathrm{W}$ is greater then
$\vartheta_\mathrm{L}$, which is according to \textcite{Barrett1968} common
for the tabletop spectrometer with moderate resolution and with fairly fast
focal ration. So we can divide further analysis into three regions according
to excitation beam focusing angle
$\vartheta$: $0 \leq \vartheta < \vartheta_\text{L}$;
$\vartheta_\text{L} < \vartheta < \vartheta_\text{W}$;
$\vartheta_\text{W} < \vartheta$.

Under the approximations above the Raman flux fraction $F$, collected by the
spectrometer of the total Raman flux emitted by the source ribbon into all
angles (i.e. full solid angle of $4\text{\g{p}}$), can be estimated now. For
the simplicity, polarization effects of input and output are further ignored
and the Raman emission is assumed isotropic over all angles of collection.
Define the solid angle subtended by the spectrometer pupil as
$\Omega_\mathrm{S}$ and corresponding scattered light collecting solid angle
at the source ribbon as $\Omega$. Then
\begin{equation}
	\Omega = \Omega_\mathrm{S}M^2.
	\label{\eqnlabel{focus_optimization:omega_magnification}}
\end{equation}
For our spectrometer the solid angle of acceptance of monochromator may be
calculated from monochromator $f/\# = f/6.4$ as
$\Omega_\text{S} =
	2\text{\g{p}}\left[1 - \cos\left(\arctan\frac{1}{6.4}\right)\right]
	\doteq 0.075$\,rad.

The Raman flux fraction $F$ is then calculated from the solid angle fraction
$\Omega / 4\text{\g{p}}$ and the fraction of the light which is “masked out”
by the width of the slit image and fraction of the ribbon length as compared
to the height of the slit image using \cref{%
\eqnlabel{focus_optimization:L_E},%
\eqnlabel{focus_optimization:W_E},%
\eqnlabel{focus_optimization:slit_length_magnification},%
\eqnlabel{focus_optimization:slit_width_magnification},%
\eqnlabel{focus_optimization:omega_magnification}}.

For the first $0 \leq \vartheta < \vartheta_\text{L}$ region, the whole

slit-image width and height are illuminated, but the focused beam is wider
than slit width, and therefore the radiant flux density in the sample is
smaller (recall that we assumed that radiant losses in the sample are
negligible)
\begin{equation}
	F_\text{I}(\vartheta)
		= \frac{W}{W_\text{E}}\cdot\frac{\Omega}{4\text{\g{p}}}
		= \frac{\frac{W_\text{S}}{M}}{\frac{4\lambda}{\text{\g{p}}\vartheta}}
			\cdot\frac{\Omega_\text{S}M^2}{4\text{\g{p}}}
		=	\frac{W_\text{S}\Omega_\text{S}M}{16\lambda}\vartheta.
	\label{\eqnlabel{focus_optimization:F_I}}
\end{equation}

In the next second region
$\vartheta_\text{L} < \vartheta < \vartheta_\text{W}$,
only the whole slit-image width is illuminated, but the exciting beam is
focused to an area with smaller length than is the slit length and so the
radiant flux density is diminished by the ratio between scattering area
length and slit-image length
\begin{equation}
	F_\text{II}(\vartheta)
		= \frac{W}{W_\text{E}}\cdot\frac{\Omega}{4\text{\g{p}}}
			\cdot\frac{L_\text{E}}{L}
		=	\frac{\frac{W_\text{S}}{M}}{\frac{4\lambda}{\text{\g{p}}\vartheta}}
			\cdot\frac{\Omega_\text{S}M^2}{4\text{\g{p}}}
			\cdot\frac{\frac{16\lambda}{\text{\g{p}}\vartheta^2}}
				{\frac{L_\text{S}}{M}}
		= \frac{W_\text{S}\Omega_\text{S}M^2}{\text{\g{p}}L_\text{S}}
			\cdot\frac{1}{\vartheta}.
	\label{\eqnlabel{focus_optimization:F_II}}
\end{equation}

The Raman scattering area in the third region $\vartheta_\text{L} < \vartheta$
is so small that the Raman scattered light is collected from the whole beam,
and therefore whole radiant flux is used
\begin{equation}
	F_\text{III}(\vartheta)
		= \frac{\Omega}{4\text{\g{p}}}\cdot\frac{L_\text{E}}{L}
		= \frac{\Omega_\text{S}M^2}{4\text{\g{p}}}
			\cdot\frac{\frac{16\lambda}{\text{\g{p}}\vartheta^2}}
				{\frac{L_\text{S}}{M}}
		= \frac{4\lambda\Omega_\text{S}M^3}{\text{\g{p}}^2L_\text{S}}
			\cdot\frac{1}{\vartheta^2}.
	\label{\eqnlabel{focus_optimization:F_III}}
\end{equation}

\begin{figure}
	\centering
	\input{results_and_discussion/assets/F}
	\caption{A plot of the values of the fractional Raman flux $F$ as calculated
	for our experimental conditions using \cref{%
		\eqnlabel{focus_optimization:F_I},\eqnlabel{focus_optimization:F_II},%
		\eqnlabel{focus_optimization:F_III}}
	versus the focusing angle $\vartheta$ of the
	illuminating laser beam.}
	\label{\figlabel{focus_optimization:F}}
\end{figure}

The results of \cref{%
		\eqnlabel{focus_optimization:F_I},%
		\eqnlabel{focus_optimization:F_II},%
		\eqnlabel{focus_optimization:F_III}}

for our experimental setup can be seen in \figref{focus_optimization:F}. The
optimum value of $\vartheta$ is $\vartheta_\text{L}$. \textcite{Barrett1968}
proposed the optimum value to lie above the $\vartheta_\text{L}$ because the
signal from outside the cylinder along the excitation beam axis is also
transmitted through the spectrometer in the second and third region, but they
do not expect the optimum as high as $\vartheta_\text{W}$. The light
collecting optics magnification can be optimized according to the equations
from above too, but each equation must be multiplied by the length of the
slit image $L$. The Raman signal then does not decrease with the
magnification, which means that the magnification should be used as high as
practical.

From these considerations follows conclusion for our experimental setup, that
the optimal focal length of the laser beam focusing lens for the off-
resonance Raman spectroscopy should be somewhere between 18 and 22\,mm which
is impractically small.

All the above calculations, however, neglected loss of the excitation beam
power as it goes through the sample, but we want to use the instrument for
resonance Raman spectroscopy where significant absorption inside the sample
occurs. It can decrease the length of the source cylinder $L_E$ significantly
and can push the $\vartheta_\text{L}$ to much lower values. For example, we
typically use samples with absorbance between 5 and 50 at 257\,nm in a 1cm
cuvette. So, for example, the intensity of incident light is halved in
300\,\g{m}m path through the sample with absorbance 10 but the
$L_E = 802$\,\g{m}m at $\vartheta_\text{L}$. If we correct the
\eqnref{focus_optimization:L_E} to for example one fourth, then the
corrected $L_E,corr$ would be approximately 170\,\g{m}m and corresponding
focal length of the laser beam focusing lens about 52\,mm.

The absorbing medium brings additional complications. For example, the
anomalous refractive index near the absorption band can significantly
influence the focused beam waist, and laser beam absorption can cause local
heating which also influences refractive index and moreover produces currents
in the sample. There is also a dependence of the optimal focal length on the
excitation wavelength, but exchanges of the focusing lens in dependence on
the excitation wavelength was impractical.

However, in conclusion, the focusing lens should have the focal length as
small as possible following the experimental geometric constraints.

\subsection{Wavenumber calibration}
\label{wavenumber_calibration}

For correct Raman spectra interpretation, we needed precise wavenumber
calibration.
It can be done by measuring spectrum with known line positions in the same
spectrometer configuration as for the spectrum of the sample and then
interpolating and extrapolating the wavenumbers from these known line positions
to all detector pixels.
One group of calibration spectra is Raman spectra of organic solvents, but one
cannot use pure indene, which is widely used in visible Raman because it
strongly absorbs UV light, and its lines can be shifted if it is diluted.
So other solvents need to be found.
Organic solvents were used as wavenumber calibration for the first UV Raman
spectra measurements.
For example, \textcite{Harada1975} calibrated to Raman line of \ch{CCl4}.
However, early, it was clear that a single organic solvent was not
satisfactory, and therefore some teams started to use mixtures of solvents.
However, it is unclear how the components influence each other if a mixture of
solvents is used.
They can evaporate at different rates, so their ratio can be changing which can
shift their bands.
This problem can be solved by measuring the solvents separately, but it
prolongs the time spent by measuring the calibration spectra.
Some selected solvents are listed in
\tabref{wavenumber_calibration:solvents}.

\begin{table}
	\centering
	\begin{tabular}{m{8.7cm}l}
\toprule
solvent & citation \\
\midrule
\ch{CCl4}                                   & \textcite{Harada1975} \\
\addlinespace[.3em]
cyclohexane                                 & \textcite{Myers1988} \\
\addlinespace[.3em]
acetonitrile                                & \textcite{Gustafson1988} \\
\addlinespace[.3em]
ethanol                                     & \textcite{Su1990} \\
\addlinespace[.3em]
ethyl acetate/dioxane (2:1 v/v)             & \textcite{Toyama1991} \\
\addlinespace[.3em]
cyclohexane, acetone                        & \textcite{Kaminaka1992} \\
\addlinespace[.3em]
acetonitrile, acetone                       & \textcite{Benson1992} \\
\addlinespace[.3em]
cyclohexanone/acetonitrile (1:1 v/v)        & \textcite{Hashimoto1993} \\
\addlinespace[.3em]
ethanol, \textit{n}-penthane                & \textcite{Mukerji1995} \\
\addlinespace[.3em]
dioxane, \ch{CCl4}, acetonitrile            & \textcite{Russell1995} \\
\addlinespace[.3em]
ethyl acetate/dioxane (2:1 v/v) (257\,nm)
                                            & \textcite{Fujimoto1998} \\
\addlinespace[.3em]
cyclohexanone/acetonitrile (1:1 v/v) (213\,nm)
                                            & \textcite{Fujimoto1998} \\
\addlinespace[.3em]
acetone, ethanol, pentane                   & \textcite{Mukerji1998} \\
\addlinespace[.3em]
\textit{n}-pentane, cyclehexane, dimethylformamide, ethanol, acetonitrile,
acetic acid                                 & \textcite{Billinghurst2006} \\
\addlinespace[.3em]
dimethylsulfoxide                           & \textcite{Srivastava2008} \\
\addlinespace[.3em]
dimethyl formamide, acetonitrile, trichloroethylene, isopropanol, indene,
cyclohexane                                 & \textcite{Jayanth2009} \\
\addlinespace[.3em]
cyclohexane, \textit{N,N}-dimethylformamide, methanol, acetonitrile,
dimethyl sylfoxide, acetic acid             & \textcite{Oladepo2011} \\
\addlinespace[.3em]
acetonitrile, isopropanol, cyclohexane and dimethyl formamide
                                            & \textcite{Mondal2016} \\
\bottomrule
\end{tabular}

	\caption[%
		Selection of organic solvents which were used for UV Raman spectra
		wavenumber calibration in literature.%
	]{%
		\captiontitle{%
			Selection of organic solvents which were used for UV Raman spectra
			wavenumber calibration in literature.%
		}
	}
	\label{\tablabel{wavenumber_calibration:solvents}}
\end{table}

Some laboratories were not satisfied with the separate calibration and used
internal standards to improve spectra' calibration.
For example,
\textcite{Wen1998}
used \ch{Na2SO4} as internal frequency (981\,\icm) standard, and Raman
frequencies were calibrated to $\pm1 $\,\icm{} using \ch{CCl4}/\ch{CH3CN}
mixture.

The other approach was also to use the elastically scattered laser line.
\textcite{Kumamoto2012}
calibrated grating dispersion with the laser line (0\,\icm) and Raman bands of
boron nitride (1370\,\icm) and acetonitrile (2249\,\icm).

%\textcite{Myers1988} calibrated wavenumbers using cyclohexane in
combination with various \ch{H2} Raman-shifted laser lines and mercury
emission lines. \textcite{Su1990} calibrated Raman spectra in frequency with
ethanol. \textcite{Toyama1991, Toyama1996} used for wavenumber calibration
the Raman bands of ethyl acetate/dioxane (2:1 v/v), which had been measured
with visible-laser excitation on a spectrometer calibrated with indene Raman
bands and neon emission lines. \textcite{Kaminaka1992} calibrated Raman
shifts with cyclohexane and acetone as the secondary standard, while their
Raman bands were calibrated with indene, with the use of visible excitation.
\textcite{Benson1992} calibrated the wavenumber shift with the use of the
Raman spectra of acetonitrile and acetone. \textcite{Benson1993} used carbon
tetrachloride for wavenumber shift calibration standard.
\textcite{Hashimoto1993} made wavenumber calibration with the Raman bands of
cyclohexanone/acetonitrile (1:1 v/v). \textcite{Mukerji1995} calibrated spectra
against neat solution of ethanol and \textit{n}-penthane.
\textcite{Russell1995} performed wavenumber calibration in UV-laser excited
Raman spectra using the well known Raman lines of dioxane, carbon
tetrachloride, and acetonitrile. \textcite{Gustafson1988} calibrated
wavenumber shift using the Raman spectrum of acetonitrile.
\textcite{Leonard1994} calibrated wavenumber shift with the use of the Raman
spectrum of cyclohexane. \textcite{Takeuchi1990} made wavenumber calibration
by using the Raman bands of cyclohexanone/acetonitrile (1:1, v/v), the
wavenumber of which had been established with visible-laser excitation on a
spectrometer calibrated with indene Raman bands and neon emission lines.
\textcite{Takeuchi1995} calibrated wavenumber axis of the spectra by using
the spectrum of a mixture solution of ethyl acetate/dioxane (2:1 v/v) as a
wavenumber standard. \textcite{Tuma1995} performed wavenumber
calibration in UVRR spectra using well-known Raman lines of dioxane, carbon
tetrachloride, and acetonitrile. \textcite{Wen1997} calibrated Raman
frequencies using a standard liquid mixture of carbon tetrachloride and
acetonitrile. \textcite{Fujimoto1998} made wavenumber calibration of the
spectra by use of the Raman bands of a 2:1 (v/v) mixture of ethyl
acetate/1,4-dioxane (257\,nm excitation) or of a 1:1 (v/v) mixture of
cyclohexanone/acetonitrile (213\,nm). \textcite{Mukerji1998} calibrated data
with acetone, ethanol, and pentane. \textcite{Wen1998} used \ch{Na2SO4} as
internal frequency (981\,\icm) standard, and Raman frequencies were
calibrated to $\pm1 $\,\icm{} using \ch{ccl4}/\ch{CH3CN} mixture.
\textcite{Suen1999} calibrated against neat solutions of ethanol and acetone,
for which the Raman band frequencies were known. \textcite{Toyama1999}
performed wavenumber calibraion by use of the Raman bands of
ethylacetate--dioxane (2:1, v/v). \textcite{Wen1999} calibrated Raman
frequencies by using a standarad liquid mixture of carbon tetrachloride and
acetonitrile. \textcite{Sokolov2000} calibrated data with acetone, ethanol and
pentane. \textcite{Toyama2001} made wavenumber calibration of Raman spectra
by using the Raman spectrum of a 2:1 (v/v) mixture of ethylacetate/1,4-dioxane.
\textcite{Fujimoto2002} calibrated Raman spectrometer by using the Raman bands
of a 2:1 (v/v) mixture of ethylacetate and dioxane for 257-nm excitation or a
1:1 (v/v) mixture of cyclohexanone and acetonitrile for 229-nm excitation.
\textcite{Toyama2002} performed wavenumber calibration of Raman spectra by use
of the Raman bands of a 2:1 (v/v) mixture of ethyl acetate and 1,4-dioxane.
\textcite{Nelson2004} calibrated the Raman shift axis with pure ethanol.
\textcite{Toyama2005,Toyama2005a} made wavenumber calibration of Raman spectra
by use of Raman bands of a 2:1 (v/v) mixture of ethyl acetate and 1,4-dioxane.
\textcite{Billinghurst2006,Billinghurst2006a} perforemd frequency calibration
by measuring the Raman scattering of organic solvents for whom the peak
positions are known (\textit{n}-pentane, cyclehexane, dimethylformamide,
ethanol, acetonitrile, and acetic acid). \textcite{Jirasek2006} wavenumber
calibrated abscissa of each acquired spectrum using an ethanol spectrum as the
calibration standard. \textcite{Kundu2007} performed frequency calibration by
measuring the resonance Raman spectra of organic solvents whose peak positions
were known (acetonitrile, dimethylformamide, \textit{n}-hexane,
\textit{n}-pentane, cyclohexane and acetic acid). \textcite{Yarasi2007}
performed frequency calibration by measuring Raman scattering of solvents of
known frequencies (acetonitrile, dimethylformamide, carbon tetrachloride,
ethanol, and pentane). \textcite{Knee2008} calibrated spectra against ethanol,
acetone and pentane. \textcite{Billinghurst2009} performed calibration by
measuring the Raman scattering of solvents for which the peak positions are
known (\textit{n}-pentane, cyclohexane, \textit{N,N}-dimethilformamide,
ethanol, acetonitrile, and acetic acid). \textcite{Jayanth2009} calibrated
spectra using known solvent bands (dimethyl formamide, acetonitrile,
trichloroethylene, isopropanol, indene, and cyclohexane). \textcite{Shaw2009}
used a spectrum of ethanol to calibrate the abscissa of each spectrum.
\textcite{Jayanth2011} carried out calibration using spectra of standard
solvents: dimethyl formamide, acetonitrile, trichloroethylene, isopropanol,
indene, and cyclohexane. \textcite{Oladepo2011} performed frequency calibration
by measuring the Raman scattering of solvents for which the peak positions were
known (cyclohexane, \textit{N,N}-dimethylformamide, methanol, acetonitrile,
dimethyl sylfoxide and acetic acid). \textcite{Billinghurst2012} performed
frequency calibration by measuring the Raman scattering of organic solvents for
which the peak positions are known (\textit{n}-pentane, cyclohexane,
dimethylformamide, ethanol, acetonitrile and acetic acid).
\textcite{Kumamoto2012} calibrated grating dispersion with the laser line
(0\,\icm), and Raman bands of boron nitride (1370\,\icm) and acetonitrile
(2249\,\icm). \textcite{Srivastava2008,Muntean2013} used Raman spectra of
dimethylsulfoxide as a standard for wavenumber calibration.
\textcite{Mondal2016} calibrated the recorded spectra using known bands of
standard solvents, e.g. acetonitrile, isopropanol, cyclohexane and dimethyl
formamide.


Calibration to the Raman spectrum of organic solvent has the advantage that the
exact wavelength of excitation beam typically to precision in the order of
$10^{-3}$\,nm does not need to be known.
The wavelength of the source also does not need to be stable to this precision
between measurements.
The calibration spectrum can also be measured from the same place and in the
same configuration as the sample, so there is no danger of some offsets caused
by changes in the geometrical alignment of the spectrometer, for example, when
the objective is moved while focusing the light gathering part of the
spectrometer on the sample or filters are changed in the gathered light path.
The significant disadvantage of calibration to the Raman spectra of organic
solvents is that their Raman bands are broad and overlapping, and therefore
their positions cannot be determined with sufficiently high precision.
We can achieve the precision of $\pm1$ \,\icm{} for well-resolved dominant and
a few \icm{} for shoulders or weak bands.
Organic solvents also do not usually have many bands in the region between
1800 and 2800\,\icm{} and do not have a sufficiently large number of usable
Raman lines, especially if solvents usable in deep UV are wanted.

We had a good experience with the other approach to calibration, which uses
atomic emission spectra of standard lamps.
Usually, as is also the case in our laboratory, the neon lamp is used.
However, this lamp does not cover the deep UV range of light, so we decided to
try some organic solvents for UV Raman calibration as shown in
\tabref{wavenumber_calibration:solvents}.

We selected cyclohexane because it has a pretty rich Raman spectrum.
We also wanted to solve the problem that most standard organic solvents do not
have many bands between 1800 and 2800\,\icm{}, so we decided to use deuterated
cyclohexane-d12, which nicely fills that region.
To our knowledge, the cyclohexane-d12 was not used for calibration of UV Raman
spectra of nucleic acids before.

The spectra of cyclohexane and cyclohexane-d12 were measured on a homemade
Raman spectrometer using visible excitation of 532\,nm
\parencite{Palacky2011}
and calibrated to the spectrum of neon calibration lamp. At least three and
typically eight spectra for each Raman line with different grating positions
were taken.
The top halves of the well-resolved peaks were fitted to the gaussian
curve, and their precise positions were estimated.
All the measured positions were then averaged.
The results can be seen in
\tabref{wavenumber_calibration:cyclohexane_wavenumbers}.

\begin{table}
	\centering
	\begin{tabular}{lr@{\,$\pm$\,}lr@{.}l}
\toprule
\multicolumn{1}{c}{name} & \multicolumn{2}{c}{$\bar{\nu}$ (\icm)}
	& \multicolumn{2}{c}{x (px)} \\
\midrule

D1   &  298.55 & 0.08 & 2002&68 \\
D2   &  373.77 & 0.06 & 1952&43 \\
H1   &  384.19 & 0.15 & 1943&28 \\
H2   &  426.75 & 0.13 & 1914&58 \\
D3a  &  636.15 & 0.09 & 1775&71 \\
D3   &  724.48 & 0.10 & 1715&56 \\
D5   &  795.93 & 0.09 & 1666&70 \\
H3   &  802.41 & 0.13 & 1660&07 \\
D6   &  939.10 & 0.12 & 1568&22 \\
D7   & 1014.07 & 0.12 & 1516&56 \\
H5   & 1028.91 & 0.14 & 1503&78 \\
D8   & 1073.70 & 0.11 & 1474&89 \\
D9   & 1120.05 & 0.10 & 1442&68 \\
H6   & 1158.77 & 0.13 & 1413&54 \\
D10  & 1214.12 & 0.08 & 1376&89 \\
H7   & 1267.25 & 0.11 & 1337&14 \\
H8   & 1444.81 & 0.11 & 1211&48 \\
D11  & 1706.70 & 0.04 & 1026&44 \\
H13  & 1720.30 & 0.06 & 1014&25 \\
D12  & 1753.42 & 0.04 &  992&68 \\
H13a & 1765.45 & 0.03 &  981&85 \\
D16  & 1979.06 & 0.06 &  828&31 \\
D17  & 2004.03 & 0.06 &  810&09 \\
H14  & 2052.09 & 0.03 &  772&50 \\
D18  & 2083.28 & 0.05 &  751&69 \\
D19  & 2106.41 & 0.06 &  734&77 \\
H14a & 2117.79 & 0.05 &  723&54 \\
D21  & 2154.45 & 0.05 &  699&13 \\
D23  & 2198.41 & 0.05 &  666&48 \\
D25  & 2322.52 & 0.04 &  574&26 \\
H15  & 2350.57 & 0.03 &  550&63 \\
H16  & 2634.46 & 0.03 &  336&50 \\
H20  & 2853.91 & 0.05 &  168&13 \\
D28  & 2885.28 & 0.05 &  146&18 \\
D29  & 2914.56 & 0.06 &  123&83 \\
H21  & 2924.74 & 0.04 &  113&34 \\
H22  & 2939.47 & 0.05 &  102&30 \\

\bottomrule
\end{tabular}

	\caption[%
		Estimated wavenumbers $\tilde{\nu}$ of cyclohexane and cyclohexane-d12.%
	]{%
		\captiontitle{%
			Estimated wavenumbers $\tilde{\nu}$ of cyclohexane (H) and
			cyclohexane-d12 (D) with the corresponding pixel positions $x$ from UV
			Raman measurement.%
		}
	}
	\label{\tablabel{wavenumber_calibration:cyclohexane_wavenumbers}}
\end{table}

The spectrum of cyclohexane and cyclohexane-d12 can then be measured after each
UV Raman measurement in such a way that the measured sample is replaced by the
calibration sample so that the calibration samples are measured from the same
place as the measured sample (with the same grating position of spectrograph).
The precise pixel positions of their Raman lines were estimated from the
spectrum in the same way as in the visible Raman measurements.
These pixel positions can then be matched to known approximate pixel positions
of cyclohexane and cyclohexane-d12 using an appropriate weight function which
favors matches with more bands by moving the measured spectrum in the table
with calibration spectrum with, for example, 0.5 px step.
The outcome of this procedure is calibration line wavenumbers and their pixel
positions.
They can be fitted by a 3rd-degree polynomial to assign the wavenumbers to all
the 2048 detector pixels.
The spectrum is then linearly interpolated to have 1\,\icm{} steps.
Example UV Raman spectra of cyclohexane and cyclohexane-d12 calibrated by our
homemade calibration program with highlighted band positions are shown in
\figref{wavenumber_calibration:cyclohexane_spc}.

\begin{figure}
	\centering
	\input{results_and_discussion/assets/cyclohexane_calibration/%
cyclohexane_calibration}
	\caption[%
		UV Raman spectrum of cyclohexane and cyclohexane-d12 with 244\,nm
		excitation.%
	]{%
		\captiontitle{%
			UV Raman spectrum of cyclohexane and cyclohexane-d12 with 244\,nm
			excitation.%
		}
		Grey lines denote the positions of Raman lines used for calibration (see
		\tabref{wavenumber_calibration:cyclohexane_wavenumbers}).
	}
	\label{\figlabel{wavenumber_calibration:cyclohexane_spc}}
\end{figure}

As was written above, a different method for sub-\icm{} calibration precision
than calibration to organic solvents needs to be used.
Some authors proposed calibration to atomic lines of a mercury lamp.
For example,
\textcite{%
	Manoharan1990,%
	Efremov1991%
}
calibrated the spectrograph with the 253.6, 312.6, and 365 nm lines of a
low-pressure Hg lamp.
The disadvantage of low-pressure Hg-lamp usage is that its lines are relatively
sparse so other approaches, which combined the Hg spectrum with other methods
were proposed.
For example, \textcite{Myers1988} calibrated wavenumbers using cyclohexane
combined with various \ch{H2} Raman-shifted laser lines and mercury emission
lines.

We tried to use a mercury lamp too, but it did not bring any new precision to
the calibrated spectra, so we searched for the different calibration sources.
There was a similar effort in
\textcite{Wert2014},
who proposed calibration with zinc (202.5, 206.2, and 213.9\,nm) and cadmium
(214.4, 226.5, and 228.8\,nm) standard lamps and Positive Light
Indigo-S 210 -- 240\,nm tunable Ti:Sa laser if necessary.
However, we pursued a different approach and took inspiration from the
calibration of space instruments which broadly use hollow cathode platinum
lamps
\parencite{%
	Mount1977,%
	Reader1990,%
	Sansonetti1992%
}
for calibration in UV spectral range.

\emph{Hollow cathode lamps} (HCL) are routinely used in \emph{atomic absorption
spectroscopy} (AAS), and therefore they are commercially available on the
market together with power supplies because HCL requires a precision current
source.
We used P209 HCL Power Supply (Photron) with P840 Hollow Cathode Lamp -- Pt
(Photron) which provided a stable platinum atomic spectrum right after start.
The new spectrometer schema with calibration source is in
\figref{wavenumber_calibration:apparatus_schema}.
Right-angle prisms (MC1 and MC2) guided the calibration beam in total internal
reflection mode.
We chose prisms instead of mirrors because they are much more stable in time,
and we do not need to have any concerns about their aging.
Prism MC2 was placed on a motorized flip mount to easily switch between
measurement of calibration spectra and Raman spectra from samples.
The calibration beam was collimated by the lens LC1 with focal length of
10\,cm (Thorlabs) and could be attenuated by the aperture AC1 represented by
the iris diaphragm (Thorlabs).

\begin{figure}
	\centering
	\input{results_and_discussion/assets/calibration_schema}
	\caption[%
		Top-view schema of the apparatus with wavenumber calibration lamp
		and with side-view inset of the sample space.%
	]{%
		\captiontitle{%
			Top-view schema of the apparatus with wavenumber calibration lamp
			and with side-view inset of the sample space.%
		}
		The calibration beam from Pt \emph{hollow cathode lamp} (HCL) is collimated
		by calibration beam lens LC1 and guided by the calibration right-angle
		prisms (MC1 and MC2) in total internal reflection mode and and focused on
		the entrance slit of spectrograph by the parabolic mirror MS2 the same way
		as the scattered signal from samples.
		The calibration lamp signal can be attenuated by modifying the size of the
		calibration beam aperture AC1.
		The mirror MC2 was placed on a motorized flip mount to enable easy
		insertion for calibration spectra measurement.
		The explanation of the rest of the symbols is the same as in
		\figref{initial_layout:apparatus_schema}.
	}
	\label{\figlabel{wavenumber_calibration:apparatus_schema}}
\end{figure}

\begin{figure}
	\centering
	\input{results_and_discussion/assets/pt_calibration/pt_calibration}
	\caption[%
		UV Raman spectrum of platinum hollow cathode lamp.%
	]{%
		\captiontitle{%
			UV Raman spectrum of platinum hollow cathode lamp with the same grating
			position as in
			\figref{wavenumber_calibration:cyclohexane_spc}.%
		}
		Grey lines denote the positions of Raman lines used for calibration.
	}
	\label{\figlabel{wavenumber_calibration:pt_spc}}
\end{figure}

As can be seen in
\figref{wavenumber_calibration:pt_spc},
the Pt lamp has many sharp well-resolved lines with reasonable intensity.
The spectrum acquisition is also relatively fast; the spectrum was usually
acquired as an average of 100 of 0.1s scans.
Because of these parameters, the measured Raman spectra can be calibrated with
a precision as high as $\pm0.1$\,\icm.

\subsection{Polarized measurements}

The next possible improvement of the measurement capabilities was to enhance
the spectrometer for polarized measurements. There were two categories of
polarizers commercially available for deep UV light, crystalline polarizers and
wire grid polarizers, so we tried to evaluate both of them. We inserted them
into the gathered signal beam path right after the edge filter in continous
rotation mounts with engraved scale marked in $2^\circ$ increments placed
on flip mount for easy removal. We rotad the scale so that $0^\circ$ meant
vertical polarization and $90^\circ$ horizontal.

The most of the crystalline polarizers are not created from materials
transparent to UV light so we needed to chose from the limited selection,
where the most appropriate with cost/value considerations at that time were
\g{a}-BBO Glan-Laser polarizers (Thorlabs). The disadvantage of this solution
was small clear aperture (1\,cm) and small angular field of view. The
advantages are good transparency for deep UV light (greater than 60\% from
210\,nm) and great extinction ratio ($\sim 100000:1$).

The selection of wire-grid polarizers usable for UV was also limited because
they have much narrower range of accepted wavelengths. We chose the one using
dielectric nanowire arrays on silica glass substrate (Meadowlark). It had
polarized transmission from 60 to 70\% at 245 to 285\,nm respectively and
extinction ratio greater than 60 in the same range. The advantage of this
polarizer was larger clear aperture (2.54\,cm) and slightly larger angular
field of view.

Polarization dependence of the light gathering optics and especially of
spectrograph grating was shielded by using quartz-wedge depolarizer (Thorlabs).
The two polarizers were evaluated using Hg lamp as a source of light from
from sample space. At first, we measured ratio of throughput between vertical
and horizontal polarization which were mainly different due to polarization
dependence of spectrograph grating reflectance. Vertical ratio was $~1.65$
and $~1.41$ times higher ther horizontal for wire-grid and Glan-Laser
polarizers, respectively. The difference can be caused by the smaller clear
aperture of the later one.

Next, the ratio between horizontal and vertical polarization throughput with
inserted depolarizer was measured. The spectra were accumulated for 0.1\,s
and intensity was measured at the maximum of peak at 253.6-nm Hg line in
detector counts. The results are displayed in
\tabref{polarized_measurements:polarizer_evaluation}.
There is significant difference between intenzity with horizontal and vertical
polarization for wire-grid polarizer which indicates that there is strong
dependence of its characteristics on polarizer tilt. There is also higher
variation in signal for Glan-laser polarizer (especially visible for
horizontal polarization), which is probably effect of combination of slight
optics movement during manipulation with polarizer and its limited aperture.
But the most significant observation is that Glan-laser polarizer has
$\sim 2.7$ times higher throughput then the wire-grid polarizer at the
253.6\,nm nevertheless its limited aperture. Therefore we chose Glan-laser
polarizer for any further polarization measurements.

\begin{table}
	\centering
	\begin{tabular}{lccc}
\toprule
polarizer  & $I_\text{v} (10^6)$
                               & $I_\text{h} (10^6)$
															                     & $I_\text{v} / I_\text{h}$
																									                     \\
\midrule
wire-grid  & $0.906 \pm 0.022$ & $0.832 \pm 0.018$ & $1.089 \pm 0.003$ \\
Glan-laser & $2.337 \pm 0.012$ & $2.349 \pm 0.123$ & $0.997 \pm 0.053$ \\
\bottomrule
\end{tabular}

	\caption{Performace comparison of wire-grid and Glan-laser polarizer. $I$
	stands for intenzity of 253.6-nm Hg line at maximum in detector counts and
	subscripts v and h stand for vertical and horizontal polarizer orientation.}
	\label{\tablabel{polarized_measurements:polarizer_evaluation}}
\end{table}

\subsection{Spinning cell}
\begin{docitemize}
	\item Overview possible solutions
	\item Draw a schema
	\item Describe the used cell
	\item Describe the method of calibration of the dependence of the rotation
	speed on the input voltage
\end{docitemize}

\begin{figure}
	\centering
	\input{results_and_discussion/assets/spinning_cell_drawing}
	\caption{Spinning cell holder.}
	\label{\figlabel{spinning_cell:drawing}}
\end{figure}

\subsection{Redesign for multiple excitation wavelengths}

At the beginning, the spectrometer allowed to use only 244\,nm as the
excitation beam wavelength. Next step in spectrometer development was to
enable more excitation wavelengths provided by the laser as described in
\cref{subsec:focus_optimization}.
We decided to use prism optics for that purpose from the same reasons as we
chose it for guiding the calibration beam, i.e. prisms have very weak
dependence of reflectance on wavelenght and are not susceptible to aging
compared to broadband metalic mirrors. On the other side, if we used laser
mirrors, we will be constrained only to wavelengths which are covered by
these mirrors (moreover it is more costly to have set of mirrors for each
excitation wavelength) and all the mirrors needs to be changed when different
excitation wavelength is used.

Some method of removal of the unwanted frequencies from the excitation beam
also needed to be introduced because the laser manufacturer didn't provide
fundamental line light removal equipment for 229-nm excitation. Pellin-Broca
prism can be used for that purpose. It has advantage that there always exist
rotation angle of the prism that the incoming and outgoing light deviates by
exactly 90\textdegree{} and if you rotate the prism along its axis the position
of outgoing beam at 90\textdegree{} doesn't change. Moreover, the angle of
incidence of incoming light is near to Brewster angle so the amount of
reflected light for p-polarization (our situation) is small.

Overall, the laser mirror M1 was replaced by Pellin-Broca prism (PB) which
separated the excitation beam from unwanted light (for example from fundamental
laser lines). The Pellin-Broca prism was placed on precise rotation stage
which enabled the selection of excitation wavelength which was outgoing in
right angle direction to the beam from laser. The unwanted light was guided to
the beam blocker (BB). Mirrors M2 and M3 were replaced then by right angle
prisms. All these changes are depicted in
\figref{multiple_excitations:apparatus_schema}.

\begin{figure}
	\centering
	\input{results_and_discussion/assets/multiex_schema}
	\caption{Top-view schema of the apparatus for multiple excitation wavelengths
		and with side-view inset of the sample space. The right-angle laser
		mirrors optimized for 244-nm excitation were replaced by prisms in total
		reflection mode. M1 was replaced by Pellin-Broca prism PB which separates
		unwanted frequencies from the excitation beam and sends them to the beam
		blocker BB. M2 and M3 were replaced by right angle prisms. The prism MC2 is
		flipped to the position for measurement and calibration lamp is switched
		off. The explanation of rest of the symbols is the same as in
		\figref{wavenumber_calibration:apparatus_schema}.}
	\label{\figlabel{multiple_excitations:apparatus_schema}}
\end{figure}

New filters for elastically scattered light removal needed to be used for each
new excitation wavelenght. We chose to support excitations at 257 and 229\,nm
on top of the 244-nm excitation.
\MISSING \{discussion about usability of different excitation wavelengths\}
The edge filter for 257-nm light removal was bought from Semrock but there was
no filter available in the market for 229\,nm but Josef Kapitan (Palackeho
University, Olomouc) kindly provided one (Materion Edge Filter) to us.

\subsection{Redesign for backscattering}

Highly absorbing samples can be hard to measure in right-angle geometry
because of absorption of excitation light and selfabsorption of the scattered
light \parencite{Shriver1974}.
These problems can be solved by using backscattering geometry.
So we decided to improve the apparatus with an easily switchable
backscattering modality.

We utilized the fact that the Cassegrain objective has a blind spot in
collimation mirror area and placed there a small $0.5 \times 0.5$ right-angle
prism (M4) glued to a homemade holder attached to kinematic stand.
The Cassegrain objective was placed on a long travel manual transition stage so
it could be moved to the position for backscattering, where the laser beam
going up was almost touching the front side of the Cassegrain.
The prism M4 reflected the beam in the optical axis of the Cassegrain into the
sample, see
\figref{backscattering:apparatus_schema}.

\begin{figure}
	\centering
	\input{results_and_discussion/assets/backscattering_schema}
	\caption[%
		Top-view schema of the apparatus in backscattering configuration
		and with side-view inset of the sample space.%
	]{%
		\captiontitle{%
			Top-view schema of the apparatus in backscattering configuration
			and with side-view inset of the sample space.%
		}
		A small right-angle prism in total reflection configuration M4 is placed to
		the same position where previously was sample and Cassegrain objective O is
		moved forward so that M4 is right before its blind spot.
		The explanation of the rest of the symbols is the same as in
		\figref{multiple_excitations:apparatus_schema}.
	}
	\label{\figlabel{backscattering:apparatus_schema}}
\end{figure}

The holder for prism M4 is displayed in
\figref{backscattering_holder:drawing}.
It was designed so that if it is secured to the kinematic holder, the axis of
the prism is the same as the axis of the kinematic holder.
The width of the thin part of the holder at the end is the same as the width of
the ribs, which are holding the collimating mirror of the Cassegrain so that
they can be aligned and no scattered light is blocked by the holder.

For the right-angle geometry, the M4 was simply removed, and Cassegrain was
moved backward to focus inside the excitation laser light going up.

\begin{figure}
	\centering
	\input{results_and_discussion/assets/backscattering_holder_drawing}
	\caption[%
		Backscattering holder.%
	]{%
		\captiontitle{%
			Backscattering holder.%
		}
		The top view is on top; a side view is on the bottom.
	}
	\label{\figlabel{backscattering_holder:drawing}}
\end{figure}



\subsection{Thermostated sample holder}
\begin{docitemize}
	\item Draw the technical drawing
	\item Present the used elements
	\item Overview the performance
\end{docitemize}
