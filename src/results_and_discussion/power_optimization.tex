\subsection{Excitation laser power optimization}

Firstly, we tried to establish the optimal laser power for excitation in
right-angle geometry using spinning cell. We used commercial samples of polyU
(Sigma) dissolved in 80 mM cacodylate buffer with ph 6.4 to the concentration
of 500\,\g{m}M per nucleobase. We chose 244 nm excitation laser wavelength with
10, 2, and 1 mW at samples. The 150\,\g{m}L of samples were rotated at
9600\,rpm during the measurement. The spectral series were measured with 30\,s
(10 and 2\,mW excitation) and 60\,s (1\,mW excitation) accumulation time per
spectrum. Beginning of each measurement was slightly delayed to the start of
the exposure to the laser because the samples needed to be adjusted for the
optimal signal. The adjustment took 26\,s for 10\,mW excitation, 90\,s for
2\,mW, and 69\,s for 1\,mW excitation.

Further analysis was performed on the integral intensity of measured Raman
bands. The band shapes were modeled as described in
\cref{band_intensities}
with the experimentally estimated Lorentzian curve fraction coefficient
$c_\text{L} = 0.5$, which visually gave the best band shape fit. The
linear function $a_0 + a_1\wn$ was added to the shape function
\eqnref{band_intensities:shape}
as the model of the spectral background.

The analysis was performed on the polyU integral intensity represented by
it's band at 1231\,\icm{} compared to the intensity of cacodylate band at
607\,\icm{} to account for the excitation laser power and sample adjustment
variations during the measurement.
\Figref{power_optim:triplexes_pU}
clearly shows that the polyU band contains also shoulder at 1248\,\icm{} and
is overlapping with cacodylate band at 1276\,\icm. The band was therefore
modeled by the shape function
(\eqnref{band_intensities:shape})
with three components but it's intensity was calculated as a sum of the pulyU
bands without the cacodylate one. The cacodylate band was modeled as having
two components, at 607 and 640\,\icm, but only the first one was used for the
integral intensity calculation.

\begin{figure}
	\centering
	\input{results_and_discussion/assets/power_optimization_triplexes/%
		power_optimization_triplexes_pU}
	\caption{Example UV RR spectrum of polyU dissolved in cacodylate buffer. The
		fits of shape functions (\eqnref{band_intensities:shape}) to the bands of
		polyU and cacodylate used in intensity estimation fit are marked by green
		and red line respectively and the areas used for intensity calculation are
		filled with the corresponding colors.}
	\label{\figlabel{power_optim:triplexes_pU}}
\end{figure}

The time dependence of the intensity of polyU band at 1231\,\icm{} (normalized
to the intensity of cacodylate band at 607\,\icm) was modeled by the
exponential decay curve
\begin{equation}
	I = I_0 \text{e}^{-\lambda t} + b,
	\label{\eqnlabel{power_optim:decay}}
\end{equation}
where $I_0$ is initial intensity, $\lambda$ is decay constant and $b$ is
baseline constant. The normalized results (subtracted by $b$) can be seen in
\figref{power_optim:triplexes}

\begin{figure}
	\centering
	\input{results_and_discussion/assets/power_optimization_triplexes/%
		power_optimization_triplexes}
	\caption{Decrease of integral intensity of polyU band at 1231\,\icm{}
		normalized to the integral intensity of cacodylate band at 607\,\icm{}
		which was used as the internal intensity standard. The values were fitted
		by exponential decay curves \eqnref{power_optim:decay} and subtracted
		by the baseline constant $b$ from the fit.}
	\label{\figlabel{power_optim:triplexes}}
\end{figure}

We can see, that the lines intersect almost at the same point which is
expected outcome but that this point is not at 0\,s. This is caused by the fact
that the time axis denotes the ends of the spectrum accumulation whereas the
intensity comes from the whole time period of the spectrum acqusition which
is even more enhanced by the fact, that the intensity of the band lowers
during the measurement. The lifetimes $\tau$ can be calculated from decay
constants $\lambda$ from the equation
\eqnref{power_optim:decay}
\begin{equation}
	\tau = \frac{1}{\lambda}
	\label{\eqnlabel{power_optim:lifetime}}
\end{equation}
We can also calculate total accumulated relative energy from the decay curve
by
\begin{equation*}
	E_{0,\text{r}}
		= \int_0^{\infty}{I_0 \text{e}^{-\lambda t}\text{d}t}
		= \frac{I_0}{\lambda}
\end{equation*}
and energy accumulated from time $T_0$ by
\begin{equation*}
	E_r	= \int_{T_0}^{\infty}{I_0 \text{e}^{-\lambda t}\text{d}t}
		= \frac{I_0}{\lambda} \text{e}^{-\lambda T_0}.
\end{equation*}
Energy fractions can be then calculated by dividing the relative energies by
the maximal total relative energy. The resulting lifetimes, total accumulated
energy fractions $E_0$ and energy fractions $E$ accumulated from time
$T = 60\pm20$\,s, which was regarded as reasonable approximate for the time
needed for adjustment of the sample before measurement could start, can be
seen in
\tabref{power_optim:lifetimes_triplexes}.

\begin{table}
	\centering
	\begin{tabular}{cr@{$\,\pm\,$}lr@{$\,\pm\,$}lr@{$\,\pm\,$}lr@{$\,\pm\,$}l}
\toprule
P (mW)
   & \multicolumn{2}{c}{$\tau$\,(min)}
                 & \multicolumn{2}{c}{$E_0$}
                                & \multicolumn{2}{c}{$E$}
                                               & \multicolumn{2}{c}{$r$} \\
\midrule

10 &   2.3 & 0.1 &  1.00 & 0.11 &  0.64 & 0.12 &  0.64 & 0.10 \\
 2 &   7.6 & 0.6 &  0.67 & 0.08 &  0.59 & 0.08 &  0.88 & 0.04 \\
 1 &  16.8 & 4.5 &  0.74 & 0.21 &  0.70 & 0.21 &  0.94 & 0.02 \\
\bottomrule
\end{tabular}

	\caption{Lifetimes $\tau$ of the polyU in dependence on excitation power
		$P$. $E_0$ are total energies accumulated by detector divided by maximal
		value accross all the excitation powers $P$ and $E$ are energies
		accumulated from the time $T = 60\pm20$\,s which was needed for the
		adjustment of the samples before the acquisition can even start but
		the sample needs to be irradiated by the excitation laser. The last column
		contains fractions of the samples $r$ which were not destroyed by
		photodecomposition after the time $T$.
	}
	\label{\tablabel{power_optim:lifetimes_triplexes}}
\end{table}

The results show, that the 10\,mW power gave the best results if we don't take
into account adjustment of the samples but as you can see in the
\figref{power_optim:triplexes}
the signal intensity quickly lowers with the time. The lifetimes can also
vary significantly in dependence on samples which can lower the reliability of
the band intensity estimations. It can be seen from the table, that after
1-minute adjustment of the samples the 10\,mW excitation photodecomposes
about 35\% of the sample whereas with the 1\,mW excitation only around 6\% of
the samples are destroyed. The energies accumulated are calculated as integral
to infinity but with lower powers we are effectively able to measure only
smaller parts of the decay which would clearly favor measurements with higher
powers. But, the spectra in more complex systems can be also influenced by the
photoproducts and therefore we decided to use rather lower excitation powers
like 1\,mW or 0.5\,mW even thought the higher excitation power probably gives
higher quality spectra because we wanted to be sure that we are measuring only
desired molecules and that the spectra are not obscured by the interaction of
the samples with the photoproducts.

\begin{itemize}
	\item from DNA hairpins
\end{itemize}
