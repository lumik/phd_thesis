\subsection{Excitation laser power optimization}
\label{subsec:power_optim}

Firstly, we tried to establish the optimal laser power for excitation in
right-angle geometry using a spinning cell.
We used commercial samples of polyU (Sigma) dissolved in 80\,mM cacodylate
buffer with ph 6.4 to the concentration of 500\,\g{m}M per nucleobase.
We chose 244\,nm excitation laser wavelength with 10, 2, and 1 mW at samples.
The 150\,\g{m}L of the sample was rotated at 9600\,rpm during the measurement.
The spectral series were measured with 30\,s (10 and 2\,mW excitation) and
60\,s (1\,mW excitation) accumulation time per spectrum.
The beginning of each measurement was slightly delayed from the start of
the exposure to the laser because the samples needed to be adjusted for
optimal signal.
The adjustment took 26\,s for 10\,mW excitation, 90\,s for 2\,mW, and 69\,s
for 1\,mW excitation.

Further analysis was performed on the integral intensity of measured Raman
bands.
The band shapes were modeled as described in
\cref{band_intensities}
with the experimentally estimated Lorentzian curve fraction coefficient
$c_\text{L} = 0.5$, which visually gave the best band shape fit.
The linear function $a_0 + a_1\wn$ was added to the shape function
\eqnref{band_intensities:shape}
as the model of the spectral background.

The analysis was performed on the polyU integral intensity represented by
its band at 1231\,\icm{} compared to the intensity of the cacodylate band at
607\,\icm{} to account for the excitation laser power and sample adjustment
variations during the measurement.
\Figref{power_optim:triplexes_pU}
clearly shows that the polyU band also contains a shoulder at 1248\,\icm{}, and
is overlapping with the cacodylate band at 1276\,\icm.
Therefore, the band was modeled by the shape function
(\eqnref{band_intensities:shape})
with three components, but its intensity was calculated as a sum of the pulyU
bands without the cacodylate one.
The cacodylate band was modeled as having two components, at 607 and 640\,\icm,
but only the first one was used for the integral intensity calculation.

\begin{figure}
	\centering
	\input{results_and_discussion/assets/power_optimization_triplexes/%
		power_optimization_triplexes_pU}
	\caption[%
		Example UV RR spectrum of polyU dissolved in cacodylate buffer.%
	]{%
		\captiontitle{%
			Example UV RR spectrum of polyU dissolved in cacodylate buffer.%
		}
		The fits of shape functions
		(\eqnref{band_intensities:shape})
		to the bands of polyU and cacodylate used in intensity estimation fit are
		marked by green, and red lines, respectively, and the areas used for
		intensity calculation are filled with the corresponding colors.
	}
	\label{\figlabel{power_optim:triplexes_pU}}
\end{figure}

The time dependence of the intensity of polyU band at 1231\,\icm{} (normalized
to the intensity of cacodylate band at 607\,\icm) was modeled by the
exponential decay curve
\begin{equation}
	I = I_0 \text{e}^{-\lambda t} + b,
	\label{\eqnlabel{power_optim:decay}}
\end{equation}
where $I_0$ is initial intensity, $\lambda$ is the decay constant, and $b$ is
the baseline constant.
The normalized results (subtracted by $b$) can be seen in
\figref{power_optim:triplexes}.

\begin{figure}
	\centering
	\input{results_and_discussion/assets/power_optimization_triplexes/%
		power_optimization_triplexes}
	\caption[%
		Decrease of the integral intensity of polyU band at 1231\,\icm{} for
		different excitation laser powers in raw spectra.%
	]{%
		\captiontitle{%
			Decrease of the integral intensity of polyU band at 1231\,\icm{} for
			different excitation laser powers in raw spectra.%
		}
		It was normalized to the integral intensity of the cacodylate band at
		607\,\icm{}, which was used as the internal intensity standard.
		The values were fitted by exponential decay curves
		\eqnref{power_optim:decay}
		and the baseline constant $b$ from the fit was subtracted from the plots.
	}
	\label{\figlabel{power_optim:triplexes}}
\end{figure}

We can see, that the lines intersect almost at the same point, which is
expected outcome, but this point is not at 0\,s.
This is caused by the fact that the time axis denotes the ends of the spectrum
accumulation, whereas the intensity comes from the whole time period of the
spectrum acquisition, which is even more enhanced by the fact that band's
intensity lowers during the measurement.
The lifetimes $\tau$ can be calculated from decay constants $\lambda$ from
\eqnref{power_optim:decay}
\begin{equation}
	\tau = \frac{1}{\lambda}.
	\label{\eqnlabel{power_optim:lifetime}}
\end{equation}
We can also calculate total accumulated relative energy from the decay curve
by
\begin{equation*}
	E_{0,\text{r}}
		= \int_0^{\infty}{I_0 \text{e}^{-\lambda t}\text{d}t}
		= \frac{I_0}{\lambda}
\end{equation*}
and energy accumulated from time $T_0$ by
\begin{equation*}
	E_r	= \int_{T_0}^{\infty}{I_0 \text{e}^{-\lambda t}\text{d}t}
		= \frac{I_0}{\lambda} \text{e}^{-\lambda T_0}.
\end{equation*}
Energy fractions can then be calculated by dividing the relative energies by
the maximal total relative energy. The resulting lifetimes, total accumulated
energy fractions $E_0$, and energy fractions $E$ accumulated from time
$T = 60\pm20$\,s, which was regarded as reasonable approximate for the time
needed for adjustment of the sample before measurement could start, can be
seen in
\tabref{power_optim:lifetimes_triplexes}.

\begin{table}
	\centering
	\begin{tabular}{cr@{$\,\pm\,$}lr@{$\,\pm\,$}lr@{$\,\pm\,$}lr@{$\,\pm\,$}l}
\toprule
P (mW)
   & \multicolumn{2}{c}{$\tau$\,(min)}
                 & \multicolumn{2}{c}{$E_0$}
                                & \multicolumn{2}{c}{$E$}
                                               & \multicolumn{2}{c}{$r$} \\
\midrule

10 &   2.3 & 0.1 &  1.00 & 0.11 &  0.64 & 0.12 &  0.64 & 0.10 \\
 2 &   7.6 & 0.6 &  0.67 & 0.08 &  0.59 & 0.08 &  0.88 & 0.04 \\
 1 &  16.8 & 4.5 &  0.74 & 0.21 &  0.70 & 0.21 &  0.94 & 0.02 \\
\bottomrule
\end{tabular}

	\caption[%
		Lifetimes of polyU in dependence on excitation power estimated from raw
		spectra.%
	]{%
		\captiontitle{%
			Lifetimes $\tau$ of the polyU in dependence on excitation power $P$
			estimated from raw spectra.%
		}
		$E_0$ are total energies accumulated by detector divided by maximal value
		across all the excitation powers $P$, and $E$ are energies accumulated from
		time $T = 60\pm20$\,s, which was needed for the adjustment of the
		samples before the acquisition can even start, but the sample needs to be
		irradiated by the excitation laser.
		The last column contains fractions of the samples $r$ which were not
		destroyed by photodecomposition after the time $T$.
	}
	\label{\tablabel{power_optim:lifetimes_triplexes}}
\end{table}

The results show that the 10\,mW power gave the best results if we do not take
into account adjustment of the samples, but as you can see in
\figref{power_optim:triplexes},
the signal intensity quickly lowers with time.
The lifetimes can also vary significantly in dependence on a sample which can
lower the reliability of the band intensity estimations.
The table shows that, after 1-minute adjustment of the sample, the 10\,mW
excitation photodecomposes about 35\% of the sample, whereas, with the 1\,mW
excitation, only around 6\% of the sample is destroyed.
The accumulated energies are calculated as integral to infinity, but we can
effectively measure only smaller portions of the decay with lower powers,
favoring measurements with higher powers.

Further investigation of the spectral decay behavior shows that this simple
approach does not reveal that a single decay curve cannot model such a complex
system.
This effect is clearly visible if we subtract the cacodylate buffer background
from the spectra.
The resulting spectrum can be seen in
\figref{power_optim:triplexes2_pU}.

\begin{figure}
	\centering
	\input{results_and_discussion/assets/power_optimization_triplexes2/%
		power_optimization_triplexes2_pU}
	\caption[%
		Example UV RR spectrum of polyU dissolved in cacodylate buffer with
	  the cacodylate buffer background subtracted.%
	]{%
		\captiontitle{%
			Example UV RR spectrum of polyU dissolved in cacodylate buffer with
			the cacodylate buffer background subtracted.%
		}
		The fit of shape function
		(\eqnref{band_intensities:shape})
		to the band of polyU used in intensity estimation fit is marked by a green
		line together with the area used for the intensity calculation.
	}
	\label{\figlabel{power_optim:triplexes2_pU}}
\end{figure}

The polyU band at 1230\,\icm{} can then be modeled directly without the
interference of the cacodylate band at 1276\,\icm{}.
It can be seen from the polyU band intensity dependence plot in
\figref{power_optim:triplexes2}
that the background subtraction process slightly obscures the intensity
normalization so that the fast decay curves cannot be modeled here.
However, we can also see a slower decay trend which is not evident from the fit
to the intensities of the polyU band calculated from the spectra without
subtracted background.
This is caused by the fact that it is harder to reliably decouple
the contribution of the overlapping polyU and cacodylate bands with the
lowering polyU band intensity.

\begin{figure}
	\centering
	\input{results_and_discussion/assets/power_optimization_triplexes2/%
		power_optimization_triplexes2}
	\caption[%
		Decrease of the integral intensity of the polyU band at 1231\,\icm{}
		for different excitation powers in background-corrected spectra.%
	]{%
		\captiontitle{%
			Decrease of the integral intensity of the polyU band at 1231\,\icm{}
			for different excitation powers in background-corrected spectra.%
		}
		The intensity was normalized to the subtracted spectrum of cacodylate
		buffer, which was used as the internal intensity standard.
		The values were fitted by exponential decay curves
		\eqnref{power_optim:decay}.
		The baseline constant $b$ from the fit was subtracted from the plots.
	}
	\label{\figlabel{power_optim:triplexes2}}
\end{figure}

For analysis of the optimal measurement conditions, we ignored the first 3, 3,
and 10 spectra for excitations at 1, 2, and 10\,mW, respectively, which visibly
deviated from the slow decay curves.
We used
\eqnref{power_optim:decay}
to estimate the lifetime $\tau$, total accumulated energy fractions $E_0$, and
energy fractions $E$ accumulated from time $T = 60\pm20$\,s the same way as
for the samples without subtracted background (see
\tabref{power_optim:lifetimes_triplexes2}).
The baseline constant $b$ was used only for the power of 10 mW, where it
decayed quickly to constant noise caused by the fit for an almost undetectable
band.
The table clearly shows that the slow decay process depends linearly on
excitation power.
In other words, the total accumulated energy is the same for all the
excitation powers, and that the lifetime is inversely related to the excitation
power.

\begin{table}
	\centering
	\begin{tabular}{cr@{$\,\pm\,$}lr@{$\,\pm\,$}lr@{$\,\pm\,$}lr@{$\,\pm\,$}l}
\toprule
P (mW)
   & \multicolumn{2}{c}{$\tau$\,(min)}
                 & \multicolumn{2}{c}{$E_0$}
                                & \multicolumn{2}{c}{$E$}
                                               & \multicolumn{2}{c}{$r$} \\
\midrule

10 &   7.1 & 0.8 &  0.90 & 0.14 &  0.79 & 0.14 &  0.87 & 0.04 \\
 2 &  39.5 & 2.4 &  1.00 & 0.12 &  0.98 & 0.12 &  0.98 & 0.01 \\
 1 &  70.3 & 7.3 &  0.89 & 0.13 &  0.88 & 0.13 &  0.99 & 0.01 \\
\bottomrule
\end{tabular}

	\caption[%
		Lifetimes of the polyU in dependence on excitation power estimated from
		the background-corrected spectra.%
	]{%
		\captiontitle{%
			Lifetimes $\tau$ of the polyU in dependence on excitation power $P$
			estimated from the background-corrected spectra.%
		}
		$E_0$ are total energies accumulated by detector divided by maximal value
		across all the excitation powers $P$, and $E$ are energies accumulated from
		the time $T = 60\pm20$\,s, which was needed for the adjustment of the
		samples before the acquisition can even start, but the sample needs to
		be irradiated by the excitation laser.
		The last column contains fractions of the samples $r$ that were not
		destroyed by photodecomposition after time $T$.
	}
	\label{\tablabel{power_optim:lifetimes_triplexes2}}
\end{table}

We can see that the system is complex, and our current methods cannot correctly
analyze the underlying processes.
Photoproducts can influence the resulting spectra, and therefore we
decided to use relatively lower excitation powers like 1\,mW or 0.5\,mW even
though the higher excitation power could probably give higher quality spectra.

We also tried to optimize the excitation laser power for the backscattering
measurements in thermostated cell holder.
We used 3\,mL of 125\,\g{m}M polyU samples dissolved in 40\,mM cacodylate
buffer with pH 6.8.
257\,nm was used as an excitation laser wavelength and a magnetic stirred
stirred the samples.
The spectral series was measured in 20\,s accumulation time per
spectrum.
The measurements in the thermostated cell holder were easier to set up
because the backscattering geometry and robust construction of the holder
were less sensitive to displacements which meant that the adjustments could be
performed on helper samples and, the real samples weren't exposed to the
laser power before measurement.

The analysis of the results was performed with the same steps as the analysis
of the measurements with a spinning cell using the integral intensity of the
polyU band at 1231\,\icm{} and the cacodylate band at 607\,\icm{}.
As we used 20 times
larger volumes of samples and 4 times lower concentrations, we should roughly
expect 5 times longer lifetimes, which means that we could observe only the
fast decay components, (see
\figref{power_optim:hairpins}
and
\tabref{power_optim:lifetimes_hairpins}).

\begin{figure}
	\centering
	\input{results_and_discussion/assets/power_optimization_hairpins/%
		power_optimization_hairpins}
	\caption[%
		Decrease of the integral intensity of the polyU band at 1231\,\icm{}
		for different excitation laser powers in raw spectra using
		backscattering geometry.%
	]{%
		\captiontitle{%
			Decrease of the integral intensity of the polyU band at 1231\,\icm{}
			for different excitation laser powers in raw spectra using
			backscattering geometry.%
		}
		It was normalized to the integral intensity of the cacodylate band at
		607\,\icm{}, which was used as the internal intensity standard.
		The values were fitted by exponential decay curves
		\eqnref{power_optim:decay}.
		The baseline constant $b$ from the fit was subtracted from the plots.
	}
	\label{\figlabel{power_optim:hairpins}}
\end{figure}

\begin{table}
	\centering
	\begin{tabular}{cr@{$\,\pm\,$}lr@{$\,\pm\,$}l}
\toprule
P (mW)
   & \multicolumn{2}{c}{$\tau$\,(min)}
                 & \multicolumn{2}{c}{$E_0$} \\
\midrule

 4 &  15.8 & 0.1 &  1.00 & 0.10 \\
 2 &  30.5 & 0.8 &  0.95 & 0.10 \\
 1 &  61.0 & 5.5 &  0.94 & 0.13 \\
\bottomrule
\end{tabular}

	\caption[%
		Lifetimes of the polyU in dependence on excitation power
		estimated from raw spectra in backscattering geometry.%
	]{%
		\captiontitle{%
			Lifetimes $\tau$ of the polyU in dependence on excitation power $P$
			estimated from raw spectra in backscattering geometry.%
		}
		$E_0$ are total energies accumulated by detector divided by maximal value
		across all the excitation powers $P$.
	}
	\label{\tablabel{power_optim:lifetimes_hairpins}}
\end{table}

It can be seen that the lifetimes are proportionally longer with lower
excitation laser power and no further special effects are observable.
The lifetimes are slightly lower than the expected 5 times longer lifetimes as
compared to the
\tabref{power_optim:lifetimes_triplexes}
but almost in the range of estimated errors.
It can be caused by many effects like underestimated errors, different
excitation laser wavelength (257\,nm instead of 244\,nm), less effective
stirring in larger volumes, or longer laser path in the samples.
We can also see that higher intensity gives more reliable results with
lower errors because of a better SNR.
We decided to perform further measurements with 4\,mW of excitation power at
257\,nm and 5\,mW at 244\,nm.
