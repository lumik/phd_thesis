\subsection{Redesign for backscattering}

Highly absorbing samples can be hard to measure in right angle geometry
because of absorption of excitation light and selfabsorption of the scattered
light \parencite{Shriver1974}. These problems can be solved by using the
backscattering geometry. So we decided to improve the apparatus with easily
switchable backscattering modality.

We utilized the fact, that cassegrain objective have blind spot in the area of
collimation mirror and placed there small $0.5 \times 0.5$ right-angle prism
(M4) glued to home made holder attached to kinematic stand. The Cassegrain
objective was placed on long travel manual transition stage so it can be moved
to position for backscattering, where the laser beam going up was almost
touching the front side of the Cassegrain. The prism M4 the reflected the beam
in the optical axis of the Cassegrain into the sample, see
\figref{backscattering:apparatus_schema}.

\begin{figure}
	\centering
	\input{results_and_discussion/assets/backscattering_schema}
	\caption{Top-view schema of the apparatus in backscattering configuration
		and with side-view inset of the sample space. Small right-angle prism in
		total reflection configuration M4 is placed to the same position where
		previously was sample and Cassegrain objective O is moved forward in such
		a way that M4 is right before its blind spot. The explanation of rest of
		the symbols is the same as in
		\figref{multiple_excitations:apparatus_schema}.}
	\label{\figlabel{backscattering:apparatus_schema}}
\end{figure}

The holder for prism M4 is displayed in
\figref{backscattering_holder:drawing}.
It was designed so that if it is secured to the kinematic holder the axis
of the prism the same as the axis of the kinematic holder. The width of the
thin part of the holder at the end is the same as the width of ribs which
are holding collimating mirror of the Cassegrain so they can be aligned and
no scattered light is blocked by the holder.

For the right-angle geometry, the M4 was simply removed and Cassegrain was
moved backward so that its focus was placed inside the excitation laser light
going up.

\begin{figure}
	\centering
	\input{results_and_discussion/assets/backscattering_holder_drawing}
	\caption{Backscattering holder. Top view is on top, side view on bottom.}
	\label{\figlabel{backscattering_holder:drawing}}
\end{figure}
