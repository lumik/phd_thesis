\subsection{Redesign for backscattering}

Highly absorbing samples can be hard to measure in right-angle geometry
because of absorption of excitation light and selfabsorption of the scattered
light \parencite{Shriver1974}.
These problems can be solved by using backscattering geometry.
So we decided to improve the apparatus with an easily switchable
backscattering modality.

We utilized the fact that the Cassegrain objective has a blind spot in
collimation mirror area and placed there a small $0.5 \times 0.5$ right-angle
prism (M4) glued to a homemade holder attached to kinematic stand.
The Cassegrain objective was placed on a long travel manual transition stage so
it could be moved to the position for backscattering, where the laser beam
going up was almost touching the front side of the Cassegrain.
The prism M4 reflected the beam in the optical axis of the Cassegrain into the
sample, see
\figref{backscattering:apparatus_schema}.

\begin{figure}
	\centering
	\input{results_and_discussion/assets/backscattering_schema}
	\caption[%
		Top-view schema of the apparatus in backscattering configuration
		and with side-view inset of the sample space.%
	]{%
		\captiontitle{%
			Top-view schema of the apparatus in backscattering configuration
			and with side-view inset of the sample space.%
		}
		A small right-angle prism in total reflection configuration M4 is placed to
		the same position where previously was sample and Cassegrain objective O is
		moved forward so that M4 is right before its blind spot.
		The explanation of the rest of the symbols is the same as in
		\figref{multiple_excitations:apparatus_schema}.
	}
	\label{\figlabel{backscattering:apparatus_schema}}
\end{figure}

The holder for prism M4 is displayed in
\figref{backscattering_holder:drawing}.
It was designed so that if it is secured to the kinematic holder, the axis of
the prism is the same as the axis of the kinematic holder.
The width of the thin part of the holder at the end is the same as the width of
the ribs, which are holding the collimating mirror of the Cassegrain so that
they can be aligned and no scattered light is blocked by the holder.

For the right-angle geometry, the M4 was simply removed, and Cassegrain was
moved backward to focus inside the excitation laser light going up.

\begin{figure}
	\centering
	\input{results_and_discussion/assets/backscattering_holder_drawing}
	\caption[%
		Backscattering holder.%
	]{%
		\captiontitle{%
			Backscattering holder.%
		}
		The top view is on top; a side view is on the bottom.
	}
	\label{\figlabel{backscattering_holder:drawing}}
\end{figure}
