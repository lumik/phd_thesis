\subsection{Determination of the band integral intensity}
\label{band_intensities}

One of the important spectroscopic information is the integral intensity of
particular Raman bands from the spectrum. The band's shape is usually modeled
by Gaussian or Lorentzian curve, or as their combination. The combination of
the Lorentzian $\func{L}$ and Gaussian $\func{G}$ curve can be expressed as
\begin{equation}
	\func{S}(\wn; I_\text{m}, \mu, \sigma) =
		c_\text{L} \cdot \func{L}(\wn; I_\text{m}, \mu, \sigma)
		+ (1 - c_\text{L}) \cdot \func{G}(\wn; I_\text{m}, \mu, \sigma),
	\label{\eqnlabel{band_intensities:single_shape}}
\end{equation}
where $c_\text{L}$ is the Lorentzian curve fraction coefficient, $I_m$ is the
height, $\mu$ is the band position, $\sigma$ is the Gaussian root mean square
width, and $\wn$ is wavenumber. The Gaussian function is taken in the form of
\begin{equation*}
	\func{G}(\wn; I_\text{m}, \mu, \sigma) =
		I_\text{m} e^{-\frac{(\wn - \mu)^2}{2\sigma^2}}.
\end{equation*}

It is beneficial to select the parameters of the Lorentzian curve to match the
parameters of the Gaussian curve. This matching is straightforward for the
height $I_\text{m}$ and position $\mu$ parameters but less obvious for the
width $\sigma$ parameter. To give the proper meaning to the Lorentzian curve
fraction coefficient $c_\text{L}$ we decided to make the full width at half
maximum (FWHM) between the Gaussian and Lorentzian curve equal for the same
$\sigma$. Therefore we needed to calculate FWHM of the Gaussian curve. Let's
assume, without loss of generality,
$I_\text{m} = 1$
and
$\mu = 0$,
and only the positive solution of the quadratic equation
\begin{align}
	\frac{1}{2} &= \func{G}(\wn; 1, 0, \sigma)
		= e^{-\frac{\wn^2}{2\sigma^2}}, \nonumber \\
	\func{ln}2  &= \frac{\wn^2}{2\sigma^2}, \nonumber \\
	\wn         &= \sigma\sqrt{2\ln2},
		\label{\eqnlabel{band_intensities:HWHM}}\\
	\func{FWHD}_\text{G}(\sigma)
	            &= 2\wn = 2\sigma\sqrt{2\ln2}. \nonumber
\end{align}

The matching width parameter of the Lorentzian curve can be calculated as
\begin{align*}
	\frac{1}{2} &= \frac{1}{1 + p\wn^2}, \\
	1 + p\wn^2  &= 2, \\
	\wn         &= \frac{1}{\sqrt{p}}.
\end{align*}
Then with the result from \eqnref{band_intensities:HWHM} we get
\begin{align*}
	\sigma\sqrt{2\ln2}
	            &= \frac{1}{\sqrt{p}}, \\
  p           &= \frac{1}{2\sigma^2\ln2}.
\end{align*}
With this result the final form of the Lorentzian curve is
\begin{equation*}
	\func{L}(\wn; I_\text{m}, \mu, \sigma) =
		\frac{I_m}{1 + \frac{(\wn - \mu)^2}{2\sigma^2\ln2}}.
\end{equation*}
This definition of the Lorentzian curve means that with the Lorentzian curve
fraction coefficient $c_\text{L} = 0.5$ the contributions to FWHM of the
Lorentzian and the Gaussian curve to the combined band shape are equal.

The integral intensity of the band can be calculated as the sum of the
contributions of both bands. It obviously doesn't depend on the position of
the band $\mu$ so we can, without loss of generality, set it to zero during
further calculations
\begin{equation}
\func{I}(I_\text{m}, \sigma) =
	c_\text{L} \func{I}_\text{L}(I_\text{m}, \sigma)
	+ (1 - c_\text{L}) \func{I}_\text{G}(I_\text{m}, \sigma),
\end{equation}
where
\begin{align*}
	\func{I}_\text{G}(I_\text{m}; \sigma)
		&= I_\text{m}\int_{-\infty}^{\infty}
			{e^{-\frac{\wn^2}{2\sigma^2}}\text{d}\wn}
		= \begin{vmatrix}
			\wn = \sigma\sqrt{2}x \\
			\text{d}\wn = \sigma\sqrt{2}\text{d}x
		\end{vmatrix}
		= I_\text{m}\sigma\sqrt{2}\int_{-\infty}^{\infty}{e^{-x^2}\text{d}x} \\
		&= I_\text{m}\sigma\sqrt{2\text{\g{p}}}
\end{align*}
and
\begin{align*}
	\func{I}_\text{L}(I_\text{m}, \sigma)
		&= \int_{-\infty}^{\infty}
			{\frac{I_\text{m}}{1 + \frac{\wn^2}{2\sigma^2\ln2}}\text{d}\wn}
		= \begin{vmatrix}
			\wn = \sigma\sqrt{2ln2}x \\
			\text{d}\wn = \sigma\sqrt{2\ln2}\text{d}x
		\end{vmatrix}
		= I_\text{m}\sigma\sqrt{2\ln2}\int_{-\infty}^{\infty}
			{\frac{1}{1 + x^2}\text{d}x} \\
		&= I_\text{m}\sigma\text{\g{p}}\sqrt{2\ln2}.
\end{align*}

Raman bands in a typical Raman spectrum of complex samples are overlapped. We
solved this problem by modeling the band as a combination of more band-shape
functions from the
\eqnref{band_intensities:single_shape}
\begin{equation}
	\func{S}(\wn; I_{\text{m},1..n}, \mu_{1..n}, \sigma_{1..n}) =
		\sum_{i = 1}^n = 	\func{S}_i(\wn; I_{\text{m},i}, \mu_i, \sigma_i),
	\label{\eqnlabel{band_intensities:shape}}
\end{equation}
where $n$ is the number of the overlapping bands.

This band shape was fitted to the measured spectra. The slightly enhanced
Marquardt-Levenberg method
(see \cref{minimization})
was used for the nonlinear regression and
analytical derivatives was used for gradient computation.
