\subsection{Analysis of spectral series containing more components}

Generally, the standard approach used in optical spectroscopic studies of NA
for analysis of a system with more components is based on variation of some
parameter which influences the ratio between components and observe the
impact of the change on the spectra.
For example, concentration of ions or pH of the buffer can be modified in
titration experiment and and their influence on the prefered conformation or
complex formation can be observed
\parencite{Klener2015}.
Or increased temperature causes structural transition of a folded into and
unfolded state
\parencite{Klener2021}.

Absorbance at one wavelength (260\,nm as a rule for UV absorption of NA) is
usually used as the monitored parameter.
Temperature dependent measurements then models the temperature profile of the
absorbance by signmoidal curve (typically derived from the Van't Hoff equation)
with linear asymptotes to take into account the temperature effects outside
the region of the temperature transition
\parencite{%
	Owczarzy1997,%
	Owczarzy2005,%
	Mergny2009%
}.

Changes in RS are richer and influence intensity and positions of numerous
Raman bands.
Similar approach can be aplied to individual spectral parameters of Raman
bands
\parencite{%
	Duguid1996,%
	Mukerji1996,%
	Mercier1999,%
	Baumruk2001,%
	Movileanu2002a,%
	Knee2008%
}
but the results are influenced by uncertainties in the subtraction of the
background signal, the effects of overlapping bands and other experimental
errors.
It means that each individual spectral parameter gives different results which
makes it difficult to obtain a general picture of the underlying process.

Another alternative are plots of differential spectra for spectral changes
during subsequent parameter steps
\parencite{%
	Duguid1996,%
	Baumruk2001,%
	Chan1997,%
	Movileanu1999,%
	Movileanu2002%
}.
They provide excellent overview of thermally induced changes in individual
Raman bands but they do not allow to determine parameters of the underlying
chemical process neither separation of different simultaneous processes.

The advantages of both of the above methods can be provided by a multivariate
analysis of a series of Raman spectra.
One of the suitable multivariate analysis mehtod is
\emph{Principal compotent analysis} (PCA,
\cite{%
	Wold1987,%
	Malinowski2002%
})
which reduces the measured spectra to several spectral profiles (loadings) and
scores which indicate the portion of each profile in the measured spectra.
The scores can be then fitted by a function based on a underlying chemical
model and for example thermodynamic parameters for structural transitions
can be estimated in temperature dependent measurement
\parencite{Nemecek2013}.

PCA converts sets of experimental spectra $Y_i(\nu)$
into novel sets of mutually independent spectral profiles (loadings)
$U_j(\nu)$ and scores $P_{ij}$ representing their portions in the original
spectra
\begin{equation*}
	Y_i(\nu) = \sum_{j=1}^M P_{ij}U_j(\nu),
\end{equation*}
where $\nu$ is wavenumber, $i \in \{1,\dots,N\}$ enumerates spectra from the
original spectral series of size $N$ and $M$ is estimated number of
significant components.

Underlying chemical model for relationships between concentrations of the
chemical components $c_k(t_i)$ can be fitted to the loading $P_{ij}$
\begin{equation*}
	P_{ij}^\text{model} = \sum_{k=1}^{M_\text{c}} c_k(t_i) Q_{ki}(t_i)
\end{equation*}
where $t_i$ is the parameter value for spectrum number $i$ (for example
temperature, concentration of ions or pH),
$c_k(t_i)$ is concentration of chemical component $k$, which is dependent on
the parameter $t_i$,
$k$ enumerates on the number of chemical components $M_\text{c}$ and
$Q_{ki}(t_i)$ is polynomial in parameter $t_i$
\begin{equation*}
	Q_{ki}(t_i) = \sum_{l_k=0}^{L_k} a_{k,l_k} t_i^{l_k},
\end{equation*}
where $l$ enumerates on the degree of polynomial $L_k$ for the chemical
component $k$.
The underlying chemical model $c_k(t_i)$ explains the spectral changes
completely only when the polynomial $Q_{ki}(t_i)$ is of degree
$L_k = 0$, i.e. $Q_{ki}(t_i) = a_k$.
Higher degrees of the polynomial enhances the chemical model so that it can
better fit also other processes which are not sufficiently reflected in the
chemical model, for example
\textcite{Klener2021}
applied polynomials of first degree to accomodate for temperature effects on
folded/unfolded components even outside of the region of temperature transition
between the folded and unfoled state.

The relationships between $c_k(t_i)$ concentrations of chemical components from
the chemical model are usually not linear in internal parameters $b_{k,m}$ of
the model because the chemical model is usually dependent on equilibrium
constants $K$ in nonlinear manner.
Moreover, Van't Hoff equation is usually used in the case of temperature
dependent measurement which introduces exponential dependency of the
equilibrium constants on entropy and enthalpy.

The least squares regression with the internal model parameters $a_{k,l_k}$ and
$b_{k,m}$ results in minimization of sum of squares
\begin{equation*}
	S(a_{k,l_k},b_{k,m}) = \sum_{i=1}^N{\sum_{j=1}^N{
		\left[P_{ij} - P_{ij}^\text{model}(a_{k,l_k},b_{k,m})\right]^2
	}},
\end{equation*}
which is as discussed above linear in polynomial parameters $a_{k,l_k}$ and
nonlinear in parameters $b_{k,m}$.
Non-linear minimisation is usually much more expensive because it utilizes
iterative methods of searching for the minimum.
They are also susceptible to finding only local minima.
On the other hand, linear least squares regression is “just” solution of set
of normal equations
\parencite[p.~671]{NumericalRecipes}.
It means that the non-linear iterative minimization algorithm can be applied
to $b_{k,m}$ values, the $a_{k,l_k}$ values are estimated by linear least
squares for the given $c_k(t_i)$ which are determined by the values of
$b_{k,m}$.
