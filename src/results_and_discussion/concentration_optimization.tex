\subsection{Concentration optimization}

The next step was to determine the optimal concentration of the measured
sample. First, we tried to estimate dependence of lifetime on concentration. We
used the same sample preparation as for
\cref{subsec:power_optim}
but with variable concentration of samples of 500\,\g{m}M and 1000\,\g{m}M per
nucleobase and 1\,mW of excitation laser power. We also used the same data
treatment as in the previous section. We weren't able to detect fast decay
component for 500\,\g{m}M samples in this measurement so we tried to compare
at least the slow decay components the same way as in the previous section.
The results can be seen in
\figref{conc_optim:triplexes}
and
\tabref{conc_optim:lifetimes_triplexes}.

\begin{figure}
	\centering
	\input{results_and_discussion/assets/concentration_optimization_triplexes/%
		concentration_optimization_triplexes}
	\caption{Decrease of integral intensity of polyU band at 1231\,\icm{}
		normalized to the subtracted spectrum of cacodylate buffer which was used
		as the internal intensity standard. The values were fitted by exponential
		decay curves \eqnref{power_optim:decay} and subtracted by the baseline
		constant $b$ from the fit.}
	\label{\figlabel{conc_optim:triplexes}}
\end{figure}

\begin{table}
	\centering
	\begin{tabular}{cr@{$\,\pm\,$}lr@{$\,\pm\,$}lr@{$\,\pm\,$}lr@{$\,\pm\,$}l}
\toprule
c (\g{m}M)
   & \multicolumn{2}{c}{$\tau$\,(min)}
                & \multicolumn{2}{c}{$E_0$}
                               & \multicolumn{2}{c}{$E$}
                                              & \multicolumn{2}{c}{$r$} \\
\midrule

 500 &   53 & 3 &  0.55 & 0.06 &  0.54 & 0.06 &  0.981 & 0.006 \\
1000 &   96 & 6 &  1.00 & 0.12 &  0.99 & 0.12 &  0.990 & 0.003 \\
\bottomrule
\end{tabular}

	\caption{Lifetimes $\tau$ of the polyU in dependence on concentration
		$c$. $E_0$ are total energies accumulated by detector divided by maximal
		value accross all the excitation powers $P$ and $E$ are energies
		accumulated from the time $T = 60\pm20$\,s which was needed for the
		adjustment of the samples before the acquisition can even start but
		the sample needs to be irradiated by the excitation laser. The last column
		contains fractions of the samples $r$ which were not destroyed by
		photodecomposition after the time $T$.
	}
	\label{\tablabel{conc_optim:lifetimes_triplexes}}
\end{table}

First of all, it is important to notice that there is variation in the
estimated lifetime between this measurement and results from previous section
\tabref{power_optim:lifetimes_triplexes2}. The difference between these two
measurements was that we used larger slit width (70 \g{m}M instead of 50 \g{m}m
used in the previous section) which could have impact on the decay curve
because as we showed in the previous section the photodecomposition process
is very complex and involves changes in the sample absorbance which can have
large impact on the focus because of the anomalous dispersion.

The second observation is, that the lifetime seems to depend almost linearly on
the concentration which is proportional to the number of nucleotides in the
sample. This means that in our right angle experimental configuration all the
excitation laser energy is absorbed in the sample for both concentrations and
that the number of photodecomposed molecules is proportional to the absorbed
energy in this concentration range and the excitation laser power density
doesn't significantly influence the photodecomposition process.

This analysis means that samples can endure longer measurements on higher
concentrations. On the other hand, higher concentrations are also less cost
effective because they require more samples and it is harder to adjust focus
for them, the spectra are also more influenced by the signal of the sample cell
because you need to focus closer to the cell wall and floor. 1000\,\g{m}M
concentration was the highest practical value, higher concentrations had
excesive absorbance and were hard to set up for the measurement.

\begin{itemize}
	\item Concentration optimization for DNA hairpins
\end{itemize}
