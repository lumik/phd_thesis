\\
\section{Interpretation of UV resonance Raman spectra of nucleic acids}
\label{interpretation}

% setting global parameters for tables
\newlength{\assignwnl}
\settowidth{\assignwnl}{0000}
\newlength{\assignwnil}
\settowidth{\assignwnil}{(000)}
\newlength{\assignwnspl}
\setlength{\assignwnspl}{0.2cm}
\newlength{\assigntabrowindent}
\setlength{\assigntabrowindent}{.7em}

%%%

One of the major goals of this thesis was to summarize the current knowledge
about resonance Raman measurements of the nucleic acids and apply to a new
experiments.
Resonance Raman spectra of building blocks of nucleic acids AMP (250\,\g{m}M),
TMP (500\,\g{m}M), GMP (375\,\g{m}M) and polyC (500\,\g{m}M) and melting their
simple complexes poly(dAdT) (500\,\g{m}M), polyG (375\,\g{m}M) and
polyG$\cdot$polyC (500\,\g{m}M) were measured for that purpose.
The concentrations of the samples are in nucleic bases.

The spectroscopilal grade samples were purchased from Sigma and dissolved in
40\,mM cacodylate buffer which was adjusted to pH 6.8 by NaOH.

The Raman spectra were excited at 244\,nm and the samples were measured at
20\,\textdegree{}C or in the range from 5\,\textdegree{}C to 95\,\textdegree{}C
with the 5\,\textdegree{}C step. The samples were placed in 1 cm quartz cuvette
with constant stirring and the spectra were collected in backscattering
geometry.

The spectra were collected in 60 or 5 subsequent 60\,s frames for measurement
at one temperature or measurement of melting, respectively.
The data were then processed using the procedures described in
\cref{data_processing}
which included also subtraction of the cacodylate buffer background.

Melting curves of the spectra were analyzed by means of PCA and the spectrum
at 5\,\textdegree{}C and 95\,\textdegree{}C was constructed from first 4
principal components.

The results of the measurements normalized to the same concentrations of
nucleobases are displayed in
\cref{%
	\figlabel{interpretation:cac},%
	\figlabel{interpretation:at},%
	\figlabel{interpretation:gc}%
}.
It means that the spectra of complexes which contain two types of nucleobases
were doubled in intensity to make them directly comparable.
The spectra were decomposed to the combination of Gaussian and Lorentzian curve
according to the
\eqnref{band_intensities:shape}
with experimentally determined
$c_\text{L} = 0.39$.
The positions of the bands which result from the fit are shown in the figures
and summarized together with the estimated intensities in the
\cref{%
	\tablabel{interpretation:at},%
	\tablabel{interpretation:gc}%
}.
The precision of the well resolved bands is better that 2\,\icm{} in position
and 10\,\% in intensity but significantly lowers on overlapped bands.
Finally the available literature was searched for the band assignments and
added to the tables.
This information was then used in the next scientific work in this thesis and
the table was enhanced also for the information from these works.

Vibration modes were labeled according the pertinent nucleobase.
Adenosine vibrations are numbered according to
\textcite{Toyama1994},
vibrations of other nucleobases are numbered analogously.
Standard numbering of atoms was used
\parencite{Bloomfield1999},
see
\figref{interpretation:structure}.

\begin{figure}
	\centering
	\begin{tikzpicture}[font=\sffamily]
	\node (Thymine) {
		\chemfig{
			N1*6([:30](-sugar)-C2(=O)-N3(-@{TH3}H)-C4(=@{TO4}O)-C5(-CH_3)=C6(-H)-)
		}
	};
	\node [below = 0 of Thymine] {Thymine};
	\node [right = of Thymine, shift = {(0,0.0cm)}] (Adenine) {
		\chemfig{
			[:90]N3*6(-C4*5(-N9(-sugar)-C8(-H)=N7-#(,2.5mm))=C5-C6(-@{AH6}H_2 N)=@{AN1}N1-C2(-H)=)
		}
	};
	\node [below = -0.7cm of Adenine] {Adenine};
\end{tikzpicture}

\chemmove{%  needs to be outside the tikzpicture so that it doesn't introduce
% infinite loop because the original tikzpicture is resized by the links
	\draw[-,dash pattern=on 2pt off 2pt] (AH6)--(TO4);
	\draw[-,dash pattern=on 2pt off 2pt] (AN1)--(TH3);
}

\begin{tikzpicture}[font=\sffamily]
	\node (Cytosine) {
		\chemfig{
			N1*6([:30](-sugar)-C2(=@{CO2}O)-@{CN3}N3=C4(-N@{CH4}H_2)-C5(-H)=C6(-H)-)
		}
	};
	\node [below = 0 of Cytosine] {Cytosine};
	\node [right = of Cytosine, shift = {(0,0.0cm)}] (Guanine) {
		\chemfig{
			[:90]N3*6(-C4*5(-N9(-sugar)-C8(-H)=N7-#(,2.5mm))=C5-C6(=@{GO6}O)-N1(-@{GH1}H)-C2(-@{GH2}H_2 N)=)
		}
	};
	\node [below = -0.7cm of Adenine] {Guanine};
\end{tikzpicture}

\chemmove{%  needs to be outside the tikzpicture so that it doesn't introduce
% infinite loop because the original tikzpicture is resized by the links
	\draw[-,dash pattern=on 2pt off 2pt] (GO6)--(CH4);
	\draw[-,dash pattern=on 2pt off 2pt] (GH1)--(CN3);
	\draw[-,dash pattern=on 2pt off 2pt] (GH2)--(CO2);
}

\begin{tikzpicture}[font=\sffamily]
	\node (Deoxyribose) {
		\chemfig[cram width=2pt]{
			C1'?(-[:90]nucleobase)%
				<[:-135,1]C2'%
					([:90,0.7]-H)%
					([:-90,0.7]-H)%
				-[:180,,,,line width = 2pt]C3'%
					([:90,0.7]-H)%
					([:-90,0.7]-O-[@{op,.5},1]\hphantom{P})%
				>[:135,1]C4'%
					([:90,1]-C5'%
						(-[:0]H)%
						(-[:90]H)%
						-[:180]O-[:180]P%
							(-[:-90]O^{-})%
							(=[:180]O)%
						-[@{cl,.5}]\hphantom{O}%
					)%
					([:-90,0.7]-H)%
				-[:25,1.35]O4'?
		}
	};
	\node [below = 0 of Deoxyribose] {2'-deoxyribose};
\end{tikzpicture}

\polymerdelim[delimiters = {[]}, h align = false, rotate = -90]{cl}{op}
	\caption[%
		Nucleoside standard atom numbering and Watson-Crick hydrogen bonding.
	]{%
		\captiontitle{%
			Nucleoside standard atom numbering \parencite{Bloomfield1999} and
			Watson-Crick hydrogen bonding \parencite{Crick1954}.
		}}
	\label{\figlabel{interpretation:structure}}
\end{figure}

\begin{figure}
	\centering
	\input{results_and_discussion/assets/interpretation/cac/%
		interpretation_cac}
	\caption[%
		Spectrum of cacodylate solvent used for background subtraction with
		244\,nm excitation.
	]{%
		\captiontitle{%
			Spectrum of cacodylate solvent used for background subtraction with
			244\,nm excitation.
		}}
	\label{\figlabel{interpretation:cac}}
\end{figure}

\begin{figure}
	\centering
	\input{results_and_discussion/assets/interpretation/at/%
		interpretation_at}
	\caption[%
		Spectra of AMP, TMP and poly(dAdT) obtained at 5\,\textdegree{}C and
		95\,\textdegree{}C.
	]{%
		\captiontitle{%
			Spectra of AMP (a), TMP (b), poly(dAdT) (c) and (d) obtained at
			5\,\textdegree{}C and 95\,\textdegree{}C, repsectively.
		}%
		The spectra were excited at 244\,nm and spectral bands were decomposed
		according to the
		\eqnref{band_intensities:shape}
		with experimentally determined
		$c_\text{L} = 0.39$.
		Each spectrum is marked by scaling factor in the right side.
		The intensities were multiplied by this factor for a better readability.}
	\label{\figlabel{interpretation:at}}
\end{figure}

\begin{table}
	\centering
	\input{results_and_discussion/assets/interpretation/at/%
		assignment_table_at1}
	\caption[%
		Assignments of the resonance Raman bands observable in measurements
	  of nucelic acids containing adenine (A) and thymine (T) bases.
	]{
		\captiontitle{%
			Assignments of the resonance Raman bands observable in measurements
			of nucelic acids containing adenine (A) and thymine (T) bases.
		}%
		Positions of vibrations are in \icm{}: ss stands for AMP or TMP and ds for
		poly(dAdT) at 5 and 95\,\textdegree{}C, respectively. Relative intensities
		in per cents of strongest band from AMP at 1338\,\icm{} are in
		parenthesees. Intensities in poly(dAdT) are doubled to compensate half
		concentration.
		The vibrational motions use these abbreviations:
			\g{n} = stretching,
			\g{d} = bending,
			\g{d}\textsubscript{s} = scissoring,
			\g{r} = rocking,
			\g{w} = wagging,
			\g{t} = twisting,
			Pyr = pyrimidine,
			Pur = purine,
			Im = imidazole.
		Citations are marked by the samples which were used for the frequency
		determination:
			Ade -- adenine,
			Ado -- adenosine,
			MeAde -- 9-methyladenine,
			AcAdo -- 2',3',5'-tri-O-acetyladenosin,
			DAcAdo -- [8-D]AcAdo,
			Thy -- thymine,
			Thd -- thymidine.
		Findings from this thesis and works resulting from this thesis are in bold.
		(Continued, 1 of 4.)
	}
	\label{\tablabel{interpretation:at}}
\end{table}

\begin{table}
	\centering
	\input{results_and_discussion/assets/interpretation/at/%
		assignment_table_at2}
	\caption*{
		\captiontitle{%
			\tablename{}~\ref*{\tablabel{interpretation:at}}
			Assignments of the resonance Raman bands observable in measurements
			of nucelic acids containing adenine (A) and thymine (T) bases.
		}%
		(Continued, 2 of 4.)}
\end{table}

\begin{table}
	\centering
	\input{results_and_discussion/assets/interpretation/at/%
		assignment_table_at3}
	\caption*{
		\captiontitle{%
			\tablename{}~\ref*{\tablabel{interpretation:at}}
			Assignments of the resonance Raman bands observable in measurements
			of nucelic acids containing adenine (A) and thymine (T) bases.
		}%
		(Continued, 3 of 4.)}
\end{table}

\begin{table}
	\centering
	\input{results_and_discussion/assets/interpretation/at/%
		assignment_table_at4}
	\caption*{
		\captiontitle{%
			\tablename{}~\ref*{\tablabel{interpretation:at}}
			Assignments of the resonance Raman bands observable in measurements
			of nucelic acids containing adenine (A) and thymine (T) bases.
		}%
		(Continued, 4 of 4.)}
\end{table}

\begin{figure}
	\centering
	\input{results_and_discussion/assets/interpretation/gc/%
		interpretation_gc}
	\caption[%
			Spectra of GMP, polyC, poly(G) and polyG$\cdot$polyC obtained at
			5\,\textdegree{}C and 95\,\textdegree{}C.
	]{%
		\captiontitle{%
			Spectra of GMP (a), polyC (b), poly(G) (c) and (d) and
			polyG$\cdot$polyC (e) and (f) obtained at 5\,\textdegree{}C and
			95\,\textdegree{}C, repsectively.
		}%
		The spectra were excited at 244\,nm and spectral bands were decomposed
		according to the
		\eqnref{band_intensities:shape}
		with experimentally determined
		$c_\text{L} = 0.39$.
		Each spectrum is marked by scaling factor in the right side.
		The intensities were multiplied by this factor for a better readability.}
	\label{\figlabel{interpretation:gc}}
\end{figure}

\begin{table}
	\centering
	\input{results_and_discussion/assets/interpretation/gc/%
		assignment_table_gc1}
	\caption[%
		Assignments of the resonance Raman bands observable in measurements
	  of nucelic acids containing guanine (G) and cytosine (C) bases.
	]{%
		\captiontitle{%
			Assignments of the resonance Raman bands observable in measurements
			of nucelic acids containing guanine (G) and cytosine (C) bases.
		}%
		Positions of vibrations are in \icm{}:
			ss stands for GMP or polyC,
			G for polyG and
			ds for polyG$\cdot$polyC at 5 and 95\,\textdegree{}C, respectively.
		Relative intensities in per cents of strongest band from AMP at
		1338\,\icm{} are in parenthesees.
		Intensities in polyG$\cdot$polyC are doubled to compensate half
		concentration.
		Kekulé vibration is switching between two positions of double bonds in
		benzene.
		The vibrational motions use these abbreviations:
			\g{n} = stretching,
			\g{d} = bending,
			\g{d}\textsubscript{s} = scissoring,
			\g{r} = rocking,
			\g{w} = wagging,
			\g{t} = twisting,
			Pyr = pyrimidine,
			Pur = purine,
			Im = imidazole.
		Citations are marked by the samples which were used for the frequency
		determination:
			NA -- compilation of works on nucleic acids,
			Gua -- guanine,
			meGua -- 9-methylguanine,
			dGuo -- deoxyguanosine,
			Cyt -- cytosine,
			Ctd -- cytidine.
		Findings from this thesis and works resulting from this thesis are in bold.
		(Continued, 1 of 4.)
	}
	\label{\tablabel{interpretation:gc}}
\end{table}

\begin{table}
	\centering
	\input{results_and_discussion/assets/interpretation/gc/%
		assignment_table_gc2}
	\caption*{
		\captiontitle{%
			\tablename{}~\ref*{\tablabel{interpretation:gc}}
			Assignments of the resonance Raman bands observable in measurements
			of nucelic acids containing guanine (G) and cytosine (C) bases.
		}%
		(Continued, 2 of 4.)}
\end{table}

\begin{table}
	\centering
	\input{results_and_discussion/assets/interpretation/gc/%
		assignment_table_gc3}
	\caption*{
		\captiontitle{%
			\tablename{}~\ref*{\tablabel{interpretation:gc}}
			Assignments of the resonance Raman bands observable in measurements
			of nucelic acids containing guanine (G) and cytosine (C) bases.
		}%
		(Continued, 3 of 4.)}
\end{table}

\begin{table}
	\centering
	\input{results_and_discussion/assets/interpretation/gc/%
		assignment_table_gc4}
	\caption*{
		\captiontitle{%
			\tablename{}~\ref*{\tablabel{interpretation:gc}}
			Assignments of the resonance Raman bands observable in measurements
			of nucelic acids containing guanine (G) and cytosine (C) bases.
		}%
		(Continued, 4 of 4.)}
\end{table}

For better understanding of the spectra of complex NA, the spectra of single
mononucleotides AMP, TMP and GMP were measured first. Also spectrum of polyC
which contains just single nucleotide was added. These spectra were then
correlated with the spectra of poly(dAdT), polyG and polyG$\cdot$polyC where
poly(dAdT) and polyG$\cdot$polyC are known to form double helical structure on
lower temperatures
\parencite{Benevides2005}
and polyG forms quadrupex structure with sufficient concentration of kations in
the solution
\parencite{Simard1994}.

The last step was to measure the melting curves of the complexes and get
their spectra above the melting temperature. The safe temperature for all the
used complexes was estimated as 95\,\textdegree{}C.

The changes in the Raman spectra between high and low temperature spectrum
can be than used for characterization of the Raman bands because the changes
reflect the structural transition between the unfolded and folded forms.

Published calculations of the vibration modes of mostly simple molecules like
methylated bases, nucleosides and nucleotides and experimental studies on
structural information of NA were reviewed and used to construct the
\cref{%
	\tablabel{interpretation:at},%
	\tablabel{interpretation:gc}%
}
\parencite{%
	Benevides2005%
}.
Even though there were some differences between the published data, the
majority of Raman bands in the measured area could be assigned to fundamental
transitions of vibrational modes localized at least in large part on the
nucleobases.

The measured spectra can be generally divided into two regions. Raman bands
with frequencies below 1400\,\icm{} are usually more delocalized, coupled with
sugar vibrations which means that they are sensitive to the sugar puckering
and glycosidic angle
\parencite{%
	Benevides2005,%
	Nishimura1986b%
}.
The second, higher frequency region contains vibrational modes localized at
nucleobases sensitive to the nucleobase interaction with the environment.

It can be clearly seen from the measurements of NA complexes in
\cref{%
	\figlabel{interpretation:at},%
	\figlabel{interpretation:gc}%
}
that the Raman bands not only change their positions but also intensities are
impacted during complex formation. This phenomenon of intensity decrease
(hypochromism) upon complex formation is especially strongly visible in
RR in comparison to non-resonant Raman spectra because of the strong relation
between RR intensities and the electronic transitions. It means that the
stacking interactions between adjacent bases has strong effect on RR band
intensities.

Adenine and thymine bands below 1200\,\icm{} display mostly shifts in the
frequencies during complex formation. The thymine bands at 750 and 788\,\icm{}
which are coupled to the sugar vibrations
\parencite{Zhu2008}
shift to slightly higher frequencies.
Also the adenine ring deformation and ring stretching bands at 909 and
1011\,\icm{}
\parencite{Xue2000}
have higher frequency in folded form.
The interesting upshift combined with hypochromism can be observed for the
thymine at 1180\,\icm{}.
The bands in this region are rather weak in RR spectra but still some
hypochromism during duplex formation can be seen in bands at
	909\,\icm{} (A),
	961\,\icm{} (T),
	1009\,\icm{} (A),
	1165\,\icm{} (A),
	1180\,\icm{} (T).

The 1200 -- 1270\,\icm region contains many overlapping bands but only thymine
band at 1242\,\icm{} is strongly hypochromic.

The strong adenine bands at 1305 and 1336\,\icm{} are strongly hypochromic and
moreover the later band has slightly higher frequency at lower temperature.
Also the strong overlapping thymine and adenine band at 1375\,\icm{} is
hypochromic. The intensity of the resulting band is lower than the separate
intensities of the bands for just mononucleotides so we can conclude that both,
thymine and adenine, bands are hypochromic.

The strongly hypochromic band at 1421\,\icm{} is mostly due to the
conformationally sensitive adenine band
\parencite{%
	Tomkova1994,%
	Taillandier1989,%
}
even though it has slight contribution also from weaker thymine band.

The next, strongly hypochromic adenine bands are at 1481, 1506, 1579 and
1601\,\icm{}. The bands at 1806 and 1601 shifted to the higher frequencies
during complex formation, the band at 1579\,\icm{} slightly downshifted.

Finally the overlapping thymine bands in the range of 1620 -- 1700\,\icm{}
assigned to the vibrations of \ch{C=o} bonds slightly downshifted and lowered
their intensity and the tymine band at 1706\,\icm{} moved to the higher
frequencies.

The similar situation as in complexes with adenine and thymine is also in the
spectra of complexes containing guanine and cytosine in the region below
1200\,\icm{}.
There are mainly upshifts of 670 and 691\,\icm{} guanine bands with
hyperchromism in the later one.
The 717\,\icm{} band visible only in polyG$\cdot$polyC spectra is slightly
hypochromic.

The guanine band at 1232\,\icm{} and cytosine band at 1250\,\icm{} upshift
during duplex formation. The former one is also hyperchromic.
The later one is known to be sensitive to the nucleotide conformation
\parencite{%
	Benevides2005,%
	Hernandez2005%
}.

The guanine band in the range 1300 -- 1340\,\icm{} clearly consists of two
vibrations but they are strongly overlapped in the polyG$\cdot$polyC spectra.
The band is hypochromic and broadens visibly in duplex. This band is know to be
sensitive to nucleoside conformation
\parencite{Benevides2005}.

Tne next region between 1340 and 1400\,\icm also consist of more overlapped
bands with prominent downshift and hypochromism of the gunine band at
1363\,\icm{}.
This band downshifts even more in polyG folding and is known to be
conformationally sensitive
\parencite{Nishimura1986b}.

The band at 1411\,\icm{} is combination of cytosine and guanine band with
higher contribution of the guanine one.
Slight downshift is observable in this band during duplex formation.
Both of these bands were reported to be sensitive to nucleoside conformation
\parencite{Nishimura1986b}.

The spectra of nucleic acids excited at 244\,nm are dominated by the strongly
hypochromic band at 1483\,\icm{}.
This band downshifts during polyG folding which is in agreement with its
sensitivity to N7 Hoogsteen H-bonds
\parencite{Palacky2013}.

The next band in the 1500 -- 1555\,\icm{} region consists of cytosine band at
1529\,\icm{} and guanine band at 1537\,\icm{}. Both bands are hypochromic and
the later one also has higher frequency at low temperature.

The guanine band at 1573\,\icm{}, which is known to be sensitive to H-bonding
\parencite{Palacky2013}, is moderately hypochromic and slightly upshifts
during the duplex formation. Both of these effects are more prominent in
polyG folding.

The guanine band in the range between 1585 and 1630\,\icm{} consists of two
parts which are both hypochromic and upshift.
The upshift is even stronger in polyG folding.
This band is known to be sensitive to hydrogen bonding
\parencite{Miura1995}.

The bands above 1630\,\icm{} aren't well resolved but the guanine band at
1644\,\icm{} which overlaps the underlying cytosine band in the slightly
higher wavenumbers is strongly hypochromic during duplex formation.

The last interesting band is visible only in polyG at 1726\,\icm{} in the
folded form.
It is known that there is quadruplex formation marker in this region
\parencite{Palacky2013}.

These results were combined with the results from works
\textcite{Klener2015}
and
\textcite{Klener2021}
into
\cref{%
	\tablabel{interpretation:at},%
	\tablabel{interpretation:gc}%
}
to provide basis for interpretation of UVRR spectra of NA.
