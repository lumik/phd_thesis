\section{Interpretation of UV resonance Raman spectra of nucleic acids}

% setting global parameters for tables
\newlength{\assignwnl}
\settowidth{\assignwnl}{0000}
\newlength{\assignwnil}
\settowidth{\assignwnil}{(000)}
\newlength{\assignwnspl}
\setlength{\assignwnspl}{0.2cm}
\newlength{\assigntabrowindent}
\setlength{\assigntabrowindent}{.7em}

%%%


\begin{figure}
	\centering
	\begin{tikzpicture}[font=\sffamily]
	\node (Thymine) {
		\chemfig{
			N1*6([:30](-sugar)-C2(=O)-N3(-@{TH3}H)-C4(=@{TO4}O)-C5(-CH_3)=C6(-H)-)
		}
	};
	\node [below = 0 of Thymine] {Thymine};
	\node [right = of Thymine, shift = {(0,0.0cm)}] (Adenine) {
		\chemfig{
			[:90]N3*6(-C4*5(-N9(-sugar)-C8(-H)=N7-#(,2.5mm))=C5-C6(-@{AH6}H_2 N)=@{AN1}N1-C2(-H)=)
		}
	};
	\node [below = -0.7cm of Adenine] {Adenine};
\end{tikzpicture}

\chemmove{%  needs to be outside the tikzpicture so that it doesn't introduce
% infinite loop because the original tikzpicture is resized by the links
	\draw[-,dash pattern=on 2pt off 2pt] (AH6)--(TO4);
	\draw[-,dash pattern=on 2pt off 2pt] (AN1)--(TH3);
}

\begin{tikzpicture}[font=\sffamily]
	\node (Cytosine) {
		\chemfig{
			N1*6([:30](-sugar)-C2(=@{CO2}O)-@{CN3}N3=C4(-N@{CH4}H_2)-C5(-H)=C6(-H)-)
		}
	};
	\node [below = 0 of Cytosine] {Cytosine};
	\node [right = of Cytosine, shift = {(0,0.0cm)}] (Guanine) {
		\chemfig{
			[:90]N3*6(-C4*5(-N9(-sugar)-C8(-H)=N7-#(,2.5mm))=C5-C6(=@{GO6}O)-N1(-@{GH1}H)-C2(-@{GH2}H_2 N)=)
		}
	};
	\node [below = -0.7cm of Adenine] {Guanine};
\end{tikzpicture}

\chemmove{%  needs to be outside the tikzpicture so that it doesn't introduce
% infinite loop because the original tikzpicture is resized by the links
	\draw[-,dash pattern=on 2pt off 2pt] (GO6)--(CH4);
	\draw[-,dash pattern=on 2pt off 2pt] (GH1)--(CN3);
	\draw[-,dash pattern=on 2pt off 2pt] (GH2)--(CO2);
}

\begin{tikzpicture}[font=\sffamily]
	\node (Deoxyribose) {
		\chemfig[cram width=2pt]{
			C1'?(-[:90]nucleobase)%
				<[:-135,1]C2'%
					([:90,0.7]-H)%
					([:-90,0.7]-H)%
				-[:180,,,,line width = 2pt]C3'%
					([:90,0.7]-H)%
					([:-90,0.7]-O-[@{op,.5},1]\hphantom{P})%
				>[:135,1]C4'%
					([:90,1]-C5'%
						(-[:0]H)%
						(-[:90]H)%
						-[:180]O-[:180]P%
							(-[:-90]O^{-})%
							(=[:180]O)%
						-[@{cl,.5}]\hphantom{O}%
					)%
					([:-90,0.7]-H)%
				-[:25,1.35]O4'?
		}
	};
	\node [below = 0 of Deoxyribose] {2'-deoxyribose};
\end{tikzpicture}

\polymerdelim[delimiters = {[]}, h align = false, rotate = -90]{cl}{op}
	\caption{Nucleoside atom numbering and Watson-Crieg hydrogen
		bonding.}
	\label{\figlabel{interpretation:at_structure}}
\end{figure}

\begin{figure}
	\centering
	\input{results_and_discussion/assets/interpretation/cac/%
		interpretation_cac}
	\caption{Cacodylate 244\,nm excitation.}
	\label{\figlabel{interpretation:cac}}
\end{figure}

\begin{figure}
	\centering
	\input{results_and_discussion/assets/interpretation/at/%
		interpretation_at}
	\caption{Interpretation 244\,nm excitation.}
	\label{\figlabel{interpretation:at}}
\end{figure}

\begin{table}
	\centering
	\input{results_and_discussion/assets/interpretation/at/%
		assignment_table_at1}
	\caption{Assignments of the resonance Raman bands observable in measurements
	  of nucelic acids.
		Positions of vibrations are in \icm{}: ss stands for AMP or TMP and ds for
		poly(dAdT) at 5 and 95\,\textdegree{}C, respectively. Relative intensities
		in per cents of strongest band from AMP at 1338\,\icm{} are in
		parenthesees. Intensities in poly(dAdT) are doubled to compensate half
		concentration.
		The vibrational motions use these abbreviations:
			\g{n} = stretching,
			\g{d} = bending,
			\g{d}\textsubscript{s} = scissoring,
			\g{r} = rocking,
			\g{w} = wagging,
			\g{t} = twisting,
			Pyr = pyrimidine,
			Pur = purine,
			Im = imidazole.
		Citations are marked by the samples which were used for the frequency
		determination:
			Ade -- adenine,
			Ado -- adenosine,
			MeAde -- 9-methyladenine,
			AcAdo -- 2',3',5'-tri-O-acetyladenosin,
			DAcAdo -- [8-D]AcAdo,
			Thy -- thymine,
			Thd -- thymidine.
		(continued, 1 of 4)}
	\label{\tablabel{interpretation:at}}
\end{table}

\begin{table}
	\centering
	\input{results_and_discussion/assets/interpretation/at/%
		assignment_table_at2}
	\caption*{
	  \tablename{}~\ref*{\tablabel{interpretation:at}}
		Assignments of the resonance Raman bands observable in measurements
	  of nucelic acids (continued, 2 of 4).}
\end{table}

\begin{table}
	\centering
	\input{results_and_discussion/assets/interpretation/at/%
		assignment_table_at3}
	\caption*{
	  \tablename{}~\ref*{\tablabel{interpretation:at}}
		Assignments of the resonance Raman bands observable in measurements
	  of nucelic acids (continued, 3 of 4).}
\end{table}

\begin{table}
	\centering
	\input{results_and_discussion/assets/interpretation/at/%
		assignment_table_at4}
	\caption*{
	  \tablename{}~\ref*{\tablabel{interpretation:at}}
		Assignments of the resonance Raman bands observable in measurements
	  of nucelic acids (continued, 4 of 4).}
\end{table}

\begin{figure}
	\centering
	\input{results_and_discussion/assets/interpretation/gc/%
		interpretation_gc}
	\caption{Interpretation 244\,nm excitation.}
	\label{\figlabel{interpretation:gc}}
\end{figure}

\begin{table}
	\centering
	\input{results_and_discussion/assets/interpretation/gc/%
		assignment_table_gc1}
	\caption{Assignments of the resonance Raman bands observable in measurements
	  of nucelic acids.
		Positions of vibrations are in \icm{}:
			ss stands for GMP or polyC,
			G for polyG and
			ds for polyG$\cdot$polyC at 5 and 95\,\textdegree{}C, respectively.
		Relative intensities in per cents of strongest band from AMP at
		1338\,\icm{} are in parenthesees.
		Intensities in polyG$\cdot$polyC are doubled to compensate half
		concentration.
		Kekulé vibration is switching between two positions of double bonds in
		benzene.
		The vibrational motions use these abbreviations:
			\g{n} = stretching,
			\g{d} = bending,
			\g{d}\textsubscript{s} = scissoring,
			\g{r} = rocking,
			\g{w} = wagging,
			\g{t} = twisting,
			Pyr = pyrimidine,
			Pur = purine,
			Im = imidazole.
		Citations are marked by the samples which were used for the frequency
		determination:
			NA -- compilation of works on nucleic acids,
			Gua -- guanine,
			meGua -- 9-methylguanine,
			dGuo -- deoxyguanosine,
			Cyt -- cytosine,
			Ctd -- cytidine.
		(continued, 1 of 3)}
	\label{\tablabel{interpretation:gc}}
\end{table}

\begin{table}
	\centering
	\input{results_and_discussion/assets/interpretation/gc/%
		assignment_table_gc2}
	\caption*{
	  \tablename{}~\ref*{\tablabel{interpretation:gc}}
		Assignments of the resonance Raman bands observable in measurements
	  of nucelic acids (continued, 2 of 3).}
\end{table}

\begin{table}
	\centering
	\input{results_and_discussion/assets/interpretation/gc/%
		assignment_table_gc3}
	\caption*{
	  \tablename{}~\ref*{\tablabel{interpretation:gc}}
		Assignments of the resonance Raman bands observable in measurements
	  of nucelic acids (continued, 3 of 3).}
\end{table}
