\section{Interpretation of UV resonance Raman spectra of nucleic acids}

% setting global parameters for tables
\newlength{\assignwnl}
\settowidth{\assignwnl}{0000}
\newlength{\assignwnil}
\settowidth{\assignwnil}{(000)}
\newlength{\assignwnspl}
\setlength{\assignwnspl}{0.2cm}
\newlength{\assigntabrowindent}
\setlength{\assigntabrowindent}{.7em}

%%%

One of the major goals of this thesis was to summarize the current knowledge
about resonance Raman measurements of the nucleic acids and apply to a new
experiments.
Resonance Raman spectra of building blocks of nucleic acids AMP (250\,\g{m}M),
TMP (500\,\g{m}M), GMP (375\,\g{m}M) and polyC (500\,\g{m}M) and melting their
simple complexes poly(dAdT) (500\,\g{m}M), polyG (375\,\g{m}M) and
polyG$\cdot$polyC (500\,\g{m}M) were measured for that purpose.
The concentrations of the samples are in nucleic bases.

The Raman spectra were excited at 244\,nm and the samples were measured at
20\,\textdegree{}C or in the range from 5\,\textdegree{}C to 95\,\textdegree{}C
with the 5\,\textdegree{}C step. The samples were placed in 1 cm quartz cuvette
with constant stirring and the spectra were collected in backscattering
geometry.

The spectra were collected in 60 or 5 subsequent 60\,s frames for measurement
at one temperature or measurement of melting, respectively.
The data were then processed using the procedures described in
\cref{data_treatment}
which included also subtraction of the cacodylate buffer background.

Melting curves of the spectra were analyzed by means of PCA and the spectrum
at 5\,\textdegree{}C and 95\,\textdegree{}C was constructed from first 4
principal components.

The results of the measurements normalized to the same concentrations of
nucleobases are displayed in
\cref{%
	\figlabel{interpretation:cac},%
	\figlabel{interpretation:at},%
	\figlabel{interpretation:gc}%
}.
It means that the spectra of complexes which contain two types of nucleobases
were doubled in intensity to make them directly comparable.
The spectra were decomposed to the combination of Gaussian and Lorentzian curve
according to the
\eqnref{band_intensities:shape}
with experimentally determined
$c_\text{L} = 0.39$.
The positions of the bands which result from the fit are shown in the figures
and summarized together with the estimated intensities in the
\cref{%
	\tablabel{interpretation:at},%
	\tablabel{interpretation:gc}%
}.
The precision of the well resolved bands is better that 1\,\icm{} in position
and 10\,\% in intensity but significantly lowers on overlapped bands.
Finally the available literature was searched for the band assignments and
added to the tables.
This information was then used in the next scientific work in this thesis and
the table was enhanced also for the information from these works.

Vibration modes were labeled according the pertinent nucleobase.
Adenosine vibrations are numbered according to
\textcite{Toyama1994},
vibrations of other nucleobases are numbered analogously.
Standard numbering of atoms was used
\textcite{Bloomfield1999},
see
\figref{interpretation:structure}.

\begin{figure}
	\centering
	\input{results_and_discussion/assets/nucleoside_numbering}
	\caption{Nucleoside atom numbering and Watson-Crieg hydrogen
		bonding base on standard numbering of atoms from
		\textcite{Bloomfield1999}.}
	\label{\figlabel{interpretation:structure}}
\end{figure}

\begin{figure}
	\centering
	\input{results_and_discussion/assets/interpretation/cac/%
		interpretation_cac}
	\caption{Spectrum of cacodylate solvent used for background subtraction with
		244\,nm excitation.}
	\label{\figlabel{interpretation:cac}}
\end{figure}

\begin{figure}
	\centering
	\input{results_and_discussion/assets/interpretation/at/%
		interpretation_at}
	\caption{Spectra of AMP (a), TMP (b), poly(dAdT) (c) and (d) obtained at
		5\,\textdegree{}C and 95\,\textdegree{}C, repsectively.
		The spectra were excited at 244\,nm and spectral bands were decomposed
		according to the
		\eqnref{band_intensities:shape}
		with experimentally determined
		$c_\text{L} = 0.39$.
		Each spectrum is marked by scaling factor in the right side.
		The intensities were multiplied by this factor for a better readability.}
	\label{\figlabel{interpretation:at}}
\end{figure}

\begin{table}
	\centering
	\input{results_and_discussion/assets/interpretation/at/%
		assignment_table_at1}
	\caption[%
		Assignments of the resonance Raman bands observable in measurements
	  of nucelic acids containing adenine (A) and thymine (T) bases.
	]{
		\captiontitle{%
			Assignments of the resonance Raman bands observable in measurements
			of nucelic acids containing adenine (A) and thymine (T) bases.
		}%
		Positions of vibrations are in \icm{}: ss stands for AMP or TMP and ds for
		poly(dAdT) at 5 and 95\,\textdegree{}C, respectively. Relative intensities
		in per cents of strongest band from AMP at 1338\,\icm{} are in
		parenthesees. Intensities in poly(dAdT) are doubled to compensate half
		concentration.
		The vibrational motions use these abbreviations:
			\g{n} = stretching,
			\g{d} = bending,
			\g{d}\textsubscript{s} = scissoring,
			\g{r} = rocking,
			\g{w} = wagging,
			\g{t} = twisting,
			Pyr = pyrimidine,
			Pur = purine,
			Im = imidazole.
		Citations are marked by the samples which were used for the frequency
		determination:
			Ade -- adenine,
			Ado -- adenosine,
			MeAde -- 9-methyladenine,
			AcAdo -- 2',3',5'-tri-O-acetyladenosin,
			DAcAdo -- [8-D]AcAdo,
			Thy -- thymine,
			Thd -- thymidine.
		(continued, 1 of 4)
	}
	\label{\tablabel{interpretation:at}}
\end{table}

\begin{table}
	\centering
	\input{results_and_discussion/assets/interpretation/at/%
		assignment_table_at2}
	\caption*{
	  \tablename{}~\ref*{\tablabel{interpretation:at}}
		Assignments of the resonance Raman bands observable in measurements
	  of nucelic acids containing adenine (A) and thymine (T) bases
		(continued, 2 of 4).}
\end{table}

\begin{table}
	\centering
	\input{results_and_discussion/assets/interpretation/at/%
		assignment_table_at3}
	\caption*{
	  \tablename{}~\ref*{\tablabel{interpretation:at}}
		Assignments of the resonance Raman bands observable in measurements
	  of nucelic acids containing adenine (A) and thymine (T) bases
		(continued, 3 of 4).}
\end{table}

\begin{table}
	\centering
	\input{results_and_discussion/assets/interpretation/at/%
		assignment_table_at4}
	\caption*{
	  \tablename{}~\ref*{\tablabel{interpretation:at}}
		Assignments of the resonance Raman bands observable in measurements
	  of nucelic acids containing adenine (A) and thymine (T) bases
		(continued, 4 of 4).}
\end{table}

\begin{figure}
	\centering
	\input{results_and_discussion/assets/interpretation/gc/%
		interpretation_gc}
	\caption{Spectra of GMP (a), polyC (b), poly(G) (c) and (d) and
		polyG$\cdot$polyC (e) and (f) obtained at 5\,\textdegree{}C and
		95\,\textdegree{}C, repsectively.
		The spectra were excited at 244\,nm and spectral bands were decomposed
		according to the
		\eqnref{band_intensities:shape}
		with experimentally determined
		$c_\text{L} = 0.39$.
		Each spectrum is marked by scaling factor in the right side.
		The intensities were multiplied by this factor for a better readability.}
	\label{\figlabel{interpretation:gc}}
\end{figure}

\begin{table}
	\centering
	\input{results_and_discussion/assets/interpretation/gc/%
		assignment_table_gc1}
	\caption[%
		Assignments of the resonance Raman bands observable in measurements
	  of nucelic acids containing guanine (G) and cytosine (C) bases.
	]{%
		\captiontitle{%
			Assignments of the resonance Raman bands observable in measurements
			of nucelic acids containing guanine (G) and cytosine (C) bases.
		}%
		Positions of vibrations are in \icm{}:
			ss stands for GMP or polyC,
			G for polyG and
			ds for polyG$\cdot$polyC at 5 and 95\,\textdegree{}C, respectively.
		Relative intensities in per cents of strongest band from AMP at
		1338\,\icm{} are in parenthesees.
		Intensities in polyG$\cdot$polyC are doubled to compensate half
		concentration.
		Kekulé vibration is switching between two positions of double bonds in
		benzene.
		The vibrational motions use these abbreviations:
			\g{n} = stretching,
			\g{d} = bending,
			\g{d}\textsubscript{s} = scissoring,
			\g{r} = rocking,
			\g{w} = wagging,
			\g{t} = twisting,
			Pyr = pyrimidine,
			Pur = purine,
			Im = imidazole.
		Citations are marked by the samples which were used for the frequency
		determination:
			NA -- compilation of works on nucleic acids,
			Gua -- guanine,
			meGua -- 9-methylguanine,
			dGuo -- deoxyguanosine,
			Cyt -- cytosine,
			Ctd -- cytidine.
		(continued, 1 of 3)
	}
	\label{\tablabel{interpretation:gc}}
\end{table}

\begin{table}
	\centering
	\input{results_and_discussion/assets/interpretation/gc/%
		assignment_table_gc2}
	\caption*{
	  \tablename{}~\ref*{\tablabel{interpretation:gc}}
		Assignments of the resonance Raman bands observable in measurements
	  of nucelic acids containing guanine (G) and cytosine (C) bases
		(continued, 2 of 3).}
\end{table}

\begin{table}
	\centering
	\input{results_and_discussion/assets/interpretation/gc/%
		assignment_table_gc3}
	\caption*{
	  \tablename{}~\ref*{\tablabel{interpretation:gc}}
		Assignments of the resonance Raman bands observable in measurements
	  of nucelic acids containing guanine (G) and cytosine (C) bases
		(continued, 3 of 3).}
\end{table}
