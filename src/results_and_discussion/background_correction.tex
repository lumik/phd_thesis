\subsection{Background subtraction}
\label{background}

Raman spectra of liquid samples usually consist of many components, including
the signal of the analyte under investigation and the spectrum of the solvent.
In the case of absorbing analyte, the spectrum
is acquired from the near proximity of the sample cell wall, and so it
usually contains a signal of quartz.
The resulting spectrum can also contain unspecific background from different
sources like fluorescence from residuals from chemical synthesis and elastic
scattering of light from different light sources in the laboratory.

Therefore the background subtraction procedure took several steps.
Each spectrum was acquired as a series of frames, and all frames were stored
for further analysis.
The spectrum of solvent, buffer, and sample cell together with the intensity
normalization to the subtracted buffer signal was performed on each frame in
the first step using the \emph{Bgcor} program
\parencite{Bgcor2017}
(written in \cite{Matlab}),
developed as part of this thesis. The unspecific background was then removed
from each spectrum using background subtraction in frames decomposed to
spectral components by PCA
\parencite{Palacky2011}.
The thirds step consisted of extrapolation to zero time, where PCA decomposed
the frames to spectral components, and the spectrum at zero time
was reconstructed from significant ones.
The last optional step performed only when a series of spectra was
taken, for example, temperature-dependent measurement was to subtract the
background once more using PCA the same way as in step two but currently
applied to the whole series of the extrapolated spectra from the third step.
This section focuses mainly on the first step -- the subtraction of the
solvent, buffer, and sample cell, together with the intensity normalization.

Bgcor program allows specifying at most three background components that can
be used for subtraction.
Moreover, the unspecific polynomial background can be
simultaneously subtracted.
The background subtraction process is semiautomatic, where an automated
procedure finds the initial background removal draft, and manual adjustment
is then possible if needed.
The background subtraction mechanism is based on mean square fit, but it allows
to favor selected regions of the spectra by specifying different weights and
also can also highly penalize (by customizable factor) regions where the
subtracted background spectrum has a higher signal than the measured spectrum,
see
\figref{background:bgcor_main}.

\begin{figure}
	\centering
	\ig{1}{results_and_discussion/assets/background/bgcor_main}
	\caption[%
		Bgcor -- program for background correction.
	]{%
		\captiontitle{%
			Bgcor -- program for background correction.
		}
		The UVRR spectrum opened in the Bgcor program
		\parencite{Bgcor2017}.
		The automated subtraction draft was performed on the spectral region below
		1140\,\icm{}, negative bands in difference spectra were penalized by the
		factor of 100, the wavenumber scale was automatically adjusted at the
		position of the cacodylate band at 2932\,\icm{}, and the resulting spectrum
		intensity was normalized to the subtraction factor of the cacodylate
		spectrum.
		A cacodylate spectrum was used as \emph{bg1}, a spectrum of water as
		\emph{bg2}, and a spectrum of quartz sample cell as \emph{bg3}.
		The spectrum was acquired with 5\,mW of 257\,nm excitation laser at the
		sample from 500\,\g{m}M (in phosphates) poly(dAdT) at 10\,\textdegree{}C as
		the first frame with the 60\,s accumulation time.
	}
	\label{\figlabel{background:bgcor_main}}
\end{figure}

If requested, one of the backgrounds can also be used for intensity
normalization and fine-tuning of the wavenumber scale.
The correction of the wavenumber scale is performed by a fit (see
\cref{minimization})
of the gaussian curve with a polynomial background to a selected band and shift
of the wavenumber axis so that the band position from the background spectrum
matches the one from the measured spectrum.
The wavenumber scale shifts can also be further smoothed by polynomial fit, see
\figref{background:bgcor_xshift}.

\begin{figure}
	\centering
	\ig{0.7}{results_and_discussion/assets/background/bgcor_xshifts}
	\caption[%
		Bgcor -- wavenumber scale correction.
	]{%
		\captiontitle{%
			Bgcor -- wavenumber scale correction.
		}
		Overview of wavenumber scale correction of the spectrum from
		\figref{background:bgcor_main}.
		A linear trend was used for shifts approximation.
	}
	\label{\figlabel{background:bgcor_xshift}}
\end{figure}

The program also has a handy function for temperature measurements where it
allows to match a number from the file name containing the measured spectrum
with the number from a file name containing the background.
Intensities of background spectra from temperature measurements were corrected
for deviations of conditions between measurements.
The major band of each background spectrum used for intensity normalization
(cacodylate band at 606\,\icm{} in most of the measurements) was modeled as
described in
\cref{band_intensities}.
The resulting intensities were then fitted by a polynomial, and the values from
fit were used to normalize the intensities of the background spectra.

All the Bgcor settings can be seen in
\figref{background:bgcor_settings}.

\begin{figure}
	\centering
	\ig{1}{results_and_discussion/assets/background/bgcor_settings}
	\caption[%
		Bgcor -- settings.
	]{%
		\captiontitle{%
			Bgcor -- settings.
		}
		Overview of all the settings for the background correction for the spectrum
		from
		\figref{background:bgcor_main}.
	}
	\label{\figlabel{background:bgcor_settings}}
\end{figure}

\Figref{background:bgcor_main} shows that the spectrum also contains unspecific
fluorescence background, which was subtracted by the method described in
\textcite{Palacky2011}.
The resulting spectrum can be seen in
\figref{background:corrected_frame}.

\begin{figure}
	\centering
	\input{results_and_discussion/assets/background/corrected_frame}
	\vspace{3mm}
	\caption[%
		Single frame with a corrected background.
	]{%
		\captiontitle{%
			Single frame with a corrected background.
		}
		The spectrum from
		\figref{background:bgcor_main}
		with additionally corrected unspecific background by the method described
		in
		\textcite{Palacky2011}.
	}
	\label{\figlabel{background:corrected_frame}}
\end{figure}

Finally, the spectrum was extrapolated to zero time $t$ (to the middle of the
first spectrum accumulation) by using the \emph{Textrapol} program
\parencite{Textrapol2017} (developped in \cite{Matlab}).
The output from the program can be seen in
\figref{background:textrapol}.

\begin{figure}
	\centering
	\ig{1}{results_and_discussion/assets/background/textrapol}
	\caption[%
		Textrapol -- program for extrapolation and interpolation of series of
		spectra using PCA.
	]{%
		\captiontitle{%
			Textrapol -- program for extrapolation and interpolation of series of
			spectra using PCA.
		}
		The UVRR spectrum opened in the Textrapol program
		\parencite{Textrapol2017}.
		The spectral series of 12 consecutive frames with 5\,min acquisition time
		acquired with 5\,mW of 257\,nm excitation laser at the sample
		from 500\,\g{m}M (in phosphates) polyC at 20\,\textdegree{}C.
		The resulting spectrum was extrapolated to half of the accumulation of
		the first frame. Only the first principal component is shown, but the first
		two principal components were used.
	}
	\label{\figlabel{background:textrapol}}
\end{figure}
