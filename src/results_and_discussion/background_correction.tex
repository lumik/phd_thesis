\subsection{Subtraction of background}

Raman spectra of liquid samples consist usually of many components including
signal of the analyte under investigation and spectrum of the solvent.
RR spectroscopy also requires absorbing analyte which means that the spectrum
is usually acquired from the near proximity of the sample cell wall and the
spectrum usually contains also signal of quartz.
The resulting spectrum also contains unspecific background from different
sources like fluorescence from residuals from chemical synthesis and ellastic
scattering of light from different ligh sources in the laboratory.

Therefore the background subtraction procedure took several steps.
Each spectrum was acquired as a series of frames and all frames were stored
for further analysis.
The spectrum of solvent, buffer and sample cell together with the intensity
normalization to the subtracted buffer signal was performed on each frame in
the first step using \emph{Bgcor} program
\parencite{Bgcor2017},
developed as part of this thesis. The unspecific background was then removed
from each spectrum using background subtraction in frames decomposed to
spectral components by PCA
\parencite{Palacky2011}.
The thirds step consisted of extrapolation to zero time where the frames
were decomposed by PCA to spectral components and spectrum at zero time was
reconstructed from singificant ones.
The last optional step which was performed only when a series of spectra was
taken, for example temperature dependent measurement was to subtract the
background once more using PCA the same way as in step two but currently
applied to the whole series of the extrapolated spectra from the third step.
This section focuses mainly on the first step -- the subtraction of the
solvent, buffer and sample cell together with the intensity normalization.

Bgcor program allows to specify at most 3 background components which can be
used for subtraction. Moreover, unspecific polynomial background can be
simultaneously subtracted.
The background subtraction process is semiautomatic, where the initial
background removal draft is found by automated procedure and manual adjustment
is then possible if needed.
Tha background subtraction mechanism is based on mean square fit but it allows
to favor selected regions of the spectra by specifying different weights and
also can also highly penalize (by customizable factor) regions where the
subtracted background spectrum has higher signal than the measured spectrum,
see
\figref{background:bgcor_main}.

\begin{figure}
	\centering
	\ig{1}{results_and_discussion/assets/background/bgcor_main}
	\caption[%
		Bgcor -- program for background correction.
	]{%
		\captiontitle{%
			Bgcor -- program for background correction.
		}
		The UVRR spectrum opened in the Bgcor program
		\parencite{Bgcor2017}.
		The automated subtraction draft was performed on the spectral region below
		1140\,\icm{}, negative bands in difference spectra were penalized by the
		factor of 100, the wavenumber scale was automatically adjusted by the
		postion of cacodylate band at 2932\,\icm{} and the resulting spectrum
		intensity was normalized to the subtraction factor of the cacodylate
		spectrum.
		Cacodylate spectrum was used as \emph{bg1}, spectrum of water as \emph{bg2}
		and spectrum of quartz sample cell as \emph{bg3}.
		The spectrum was aquired with 5\,mW of 257\,nm excitation laser at sample
		from 500\,\g{m}M (in phosphates) poly(dAdT) at 10\,\textdegree{}C as the
		first frame with the 60\,s accumulation time.
	}
	\label{\figlabel{background:bgcor_main}}
\end{figure}

One of the backgrounds can be used also for intensity normalization and
fine tuning of wavenumber scale if requested.
The correction of wavenumber scale is performed by fit of gaussian curve with
polynomial background to a selected band and shift of the wavenumber axis so
that the band position from background spectrum matches the one from the
measured spectrum.
The wavenumber scale shifts can be further smoothed by polynomial fit, see
\figref{background:bgcor_xshift}.

\begin{figure}
	\centering
	\ig{0.7}{results_and_discussion/assets/background/bgcor_xshifts}
	\caption[%
		Bgcor -- wavenumber scale correction.
	]{%
		\captiontitle{%
			Bgcor -- wavenumber scale correction.
		}
		Overview of wavenumber scale correction of the spectrum from the
		\figref{background:bgcor_main}.
		The shifts were smoothed to linear trend.
	}
	\label{\figlabel{background:bgcor_xshift}}
\end{figure}

The program has also handy function for temperature measurements where it
allows to match number from file name containing measured spectrum with
the number from file name containing the background.
To make the intensities of spectra from temperature measurements comparable
the major band of background spectra used for intensity normalization
(cacodylate band at 606\,\icm{} in most of the measurements) was modeled as
described in
\cref{band_intensities}.
The resulting intensities were than fitted by polynome and the values from fit
was used to normalize the intensities of the background spectra.

All the Bgcor settings can be seen in
\figref{background:bgcor_settings}.

\begin{figure}
	\centering
	\ig{1}{results_and_discussion/assets/background/bgcor_settings}
	\caption[%
		Bgcor -- settings.
	]{%
		\captiontitle{%
			Bgcor -- settings.
		}
		Overview of all the settings for the background correction for the spectrum
		from the
		\figref{background:bgcor_main}.
	}
	\label{\figlabel{background:bgcor_settings}}
\end{figure}

\Figref{background:bgcor_main} shows that the spectrum contains also unspecific
fluorescence background which was subtracted by method described in
\textcite{Palacky2011}.
The resulting spectrum can be seen in
\figref{background:corrected_frame}.

\begin{figure}
	\centering
	\input{results_and_discussion/assets/background/corrected_frame}
	\vspace{3mm}
	\caption[%
		Single frame with corrected background.
	]{%
		\captiontitle{%
			Single frame with corrected background.
		}
		The spectrum from
		\figref{background:bgcor_main}
		with additionally corrected unspecific background by the method described
		in
		\textcite{Palacky2011}.
	}
	\label{\figlabel{background:corrected_frame}}
\end{figure}

Finally, the spectrum was extrapolated to zero time $t$ (to the middle of the
first spectrum accumulation) by using \emph{Textrapol} program
\parencite{Textrapol2017}.
The output from the program can be seen in
\figref{background:textrapol}.

\begin{figure}
	\centering
	\ig{1}{results_and_discussion/assets/background/textrapol}
	\caption[%
		Textrapol -- program for extrapolation and interpolation of series of
		spectra using PCA.
	]{%
		\captiontitle{%
			Textrapol -- program for extrapolation and interpolation of series of
			spectra using PCA.
		}
		The UVRR spectrum opened in the Textrapol program
		\parencite{Textrapol2017}.
		The spectral series of 12 consecutive frames with 5\,min acquisition time
		aquired with 5\,mW of 257\,nm excitation laser at sample
		from 500\,\g{m}M (in phosphates) polyC at 20\,\textdegree{}C.
		The resulting spectrum was extrapolated to the half of the accumulation of
		the first frame. Only the first principal component is shown but first
		two principal components were used.
	}
	\label{\figlabel{background:textrapol}}
\end{figure}