\section{%
	Analysis of temperature effects on the structural arrangement of DNA segments
}

Possibility to obtain reliable UV RRS spectra at various temperatures was
employed in study of a simple DNA structural transition.
It is known that UV absorption measurements reveal besides the temperature
region of the transition exhibiting typical sigmoidal temperature dependence,
also about linear temperature induced changes at temperatures below and above
the transition region
\parencite{%
	Owczarzy1997,%
	Owczarzy2005,%
	Mergny2009%
}.
Several studies
\parencite{%
	Leulliot1999b,%
	Stepanek2007,%
	Vachousek2008%
}
based on the temperature dependence of NA non-resonance Raman spectra and their
multivariate analysis have shown that the number of resolved spectral profiles
in these studies is less than would be expected analogously to UV absorption.
Our aim was to demonstrate that UV RRS combines reach informational content of
Raman spectrum with a high sensitivity to weak aspects of temperature induced
changes.
Beside this it was also desired to obtain UV RRS characteristics concerning
particular structural arrangements of DNA hairpin and double helix.

The experiments were carried out on DNA dodeca-deoxynucleotides
d(5'GCCG\allowbreak{}ATTACGGC3'), d(5'GCCGATTAGCCG3'), d(5'CGGCTAATCGGC3').
The first forms a hairpin while the latter two in a 1:1 molar mixture form a
fully complementary duplex.
In both cases, the total base composition was the same, containing (in the
folded state) a double-strand section of GC pairs and a central section of AT
pairs forming a flexible part of the double helix in the duplex or the loop in
the hairpin
(see \figref{dna_hairpins:structure})

\begin{figure}
	\centering
	\ig{1}{results_and_discussion/assets/dna_hairpins/structure}
	\caption[%
		Scheme of an arrangement of the studied dodecadeoxynucleotides in the
		folded state.
	]{%
		\captiontitle{%
			Scheme of an arrangement of the studied dodecadeoxynucleotides in the
			folded state: hairpin (left) and duplex (right).
		}
		Dashes represent the sugar-phosphate links in the chain and dots represent
		the hydrogen bonds connecting the W.-C. basepairs.
	}
	\label{\figlabel{dna_hairpins:structure}}
\end{figure}

Oligonucleotides were dissolved in a 40\,mM cacodylate buffer (pH 6.8) with
100\,mM \ch{NaCl}.
It was proved by gel electrophoresis that at ambient temperature no structural
forms of different molecularity than expected were formed.
UV RRS was measured on the spectrometer in its final configuration, excitation
at 257 and 244\,nm.
After each UV RRS experiment, the UV absorption spectrum of the sample was
recorded to check for possible photodecomposition.
No perceptible changes were observed.
Complementary temperature dependences of UV absorption and non-resonant Raman
spectra (532\,nm excitation, new automated control employed) were also
recorded.
For both UV RRS and UV absorption measurements, the total oligonucleotide
concentration was 25\,\g{m}M, while $200\times$ higher concentrations were used
for non-resonant Raman scattering.
Before the measurement, the solvents were annealed to 90\,\textdegree{}C.

Raman spectra were collected in a broad spectral range as a set of at least
five subsequent frames.
They were corrected and intensity-normalized in two steps.
The first step was the subtraction of the solvent and the cuvette wall signal
using multiplication coefficients optimized so that to remove the most
distinctive idle Raman features.
The optimal coefficient obtained for the solvent was also used for the
intensity normalization of individual frames.
Adjusted frames were then averaged.
Within the second step of the treatment, the baseline was straightened for the
set of spectra by using the procedure based on factor analysis
\parencite{Palacky2011}.
After that, the spectra were analyzed without any further adjustment.
The principal component analysis was performed for the series of
temperature-dependent Raman spectra in the region of the main UV RRS lines,
i.e. 1150 -- 1800\,\icm{}, and in the region of 220 -- 320\,nm for UV
absorption.

Comparison of UV RRS spectra recorded at various temperatures indicates
remarkable changes in the Raman spectra
(see \figref{dna_hairpins:spectra}).
It is worth noting that the spectra at the highest temperature of
95\,\textdegree{}C are effectively identical for the hairpin and duplex.
This shows that at this temperature, the structure of the oligonucleotide chain
is fully relaxed so that the spectrum is given only by the overall portions of
nucleobases, which are equal in both cases.

\begin{figure}
	\centering
	\ig{1}{results_and_discussion/assets/dna_hairpins/spectra}
	\caption[%
		UV RRS spectra of DNA dodecamers measured at various temperatures.
	]{%
		\captiontitle{%
			UV RRS spectra of DNA dodecamers measured at various temperatures on a
			rainbow color scale.
		}
	}
	\label{\figlabel{dna_hairpins:spectra}}
\end{figure}

To reveal the nature of the spectral changes, the UV RRS spectra and the
spectral series of complementary measurements of non-resonant RS excited and UV
absorption were subjected to Principal Component Analysis (PCA)
\parencite{Malinowski2002}.
For all spectral series, four PCA components were well resolvable while the
fifth and higher components primarily represented the noise contribution.
\Cref{%
	\figlabel{dna_hairpins:pca_duplex},%
	\figlabel{dna_hairpins:pca_hairpin}%
}
show temperature dependences of the relevant scores.
These results demonstrate that even for a simple temperature-induced structural
transition, the spectral changes of UV RRS spectra are, like UV absorption,
to be expressed as a combination of four spectral components.
The first component represents an average spectral shape and the corresponding
scores indicate the temperature-induced changes of the overall spectral
intensity.
It is dominant especially in the case of UV absorption where the curve of the
first scores is practically identical with the melting curve routinely measured
as a function of absorbance at 260\,nm on the temperature.
The next components describe the temperature-induced changes of the spectral
shape.
Their number (three) agrees well with the assumption that the changes in the
system are formed by the transition from the folded to the unfolded state, and
the effects of temperature increase for the folded and unfolded state i.e. it
proves that the temperature changes in the folded and unfolded state manifest
differently in the spectra (not only by the different asymptotes but also by
the different spectral shape).
While the transition shows a sigmoidal curve in the temperature dependence of
the scores, the latter two processes are approximately linearly dependent on
temperature.

\begin{figure}
	\centering
	\ig{1}{results_and_discussion/assets/dna_hairpins/pca_duplex}
	\caption[%
		PCA scores obtained for the duplex spectra.
	]{%
		\captiontitle{%
			PCA scores obtained for the duplex spectra (grey circles).
		}
		The black lines show the results of the fit of UV absorption and Raman
		spectra with the thermodynamic parameters of the structural transition
		fixed to those obtained from UV absorption.
		The grey lines show results of independent fits of Raman spectra.
	}
	\label{\figlabel{dna_hairpins:pca_duplex}}
\end{figure}

\begin{figure}
	\centering
	\ig{1}{results_and_discussion/assets/dna_hairpins/pca_hairpin}
	\caption[%
		PCA scores obtained for the hairpin spectra.
	]{%
		\captiontitle{%
			PCA scores obtained for the hairpin spectra (grey circles).
		}
		The black lines show results of the fit of UV absorption and Raman spectra
		with fixed thermodynamic parameters of the structural transition to those
		obtained from UV absorption.
		The grey lines show results of independent fits of Raman spectra (this
		fitting procedure aborted for non-resonance RS spectra).
	}
	\label{\figlabel{dna_hairpins:pca_hairpin}}
\end{figure}

The transition (i.e. temperature-induced dissociation of the strands or opening
of the hairpin) is dominant in the first and second score of UV absorption and
UV RRS.
It means that the transition causes main changes in the overall spectral
intensity and the spectral shape as well.
The differences between the spectral changes caused by the transition and those
attributed to uncorrelated relaxation, as well as the differences between the
uncorrelated relaxations in the folded and the unfolded form, are expressed by
the third and the fourth scores.
It is worth noting that the values of the fourth and even third scores are
relatively low (about 1\% for Raman and 0.1\% for UV absorption spectra of the
original spectral intensity), which means that sufficiently precise
measurements with high SNR are required to isolate the complete set of
four components.

In the case of the duplex, the melting transition is less visible in the
spectra of non-resonant RS.
This is partly because the melting temperature for the dissociation of the
duplex as a bimolecular reaction depends on the concentration, which moves it
to higher temperatures in the case of non-resonant RS ($200\times$ higher
concentration than used for UV absorption and UV RRS measurements).
This is why there is not a distinct upper linear portion of the curve.
But in the case of hairpin, the sigmoidal curve of which is remarkably
smoother, the transition sigmoid is in PCA results of non-resonance RS visible
only in the fourth scores.

The scores $P$ were least-square fitted according to a formula considering both
the ratio of the folded and unfolded form governed by the Van’t Hoff equation
and the linear temperature dependence of each form corresponding to the
structural relaxation (see
\eqnref{two_comp:sos})
as:

\begin{equation*}
	\sum_i\sum_{j=1}^4\left[
		P_{ij} - C_i^\text{fold}(A_{0j} + A_{1j}T_i)
		- C_i^\text{unfold}(B_{0j} + B_{1j}T_i)
	\right]^2 = \func{min}.
\end{equation*}

The index $i$ is referred to as the sample temperature and $j$ to the serial
number of the component.
The theoretical temperature dependences given by the values of the variables
obtained by the fit are shown in
\cref{%
	\figlabel{dna_hairpins:pca_duplex},%
	\figlabel{dna_hairpins:pca_hairpin}%
}
by solid lines.
The fit was simultaneously employed to all four scores of the given spectral
series.
The black lines denote the fit of UV absorption results and corresponding fits
of Raman spectra using the $\Delta{}H$ and $\Delta{}S$ fixed at the values
obtained from the fit of UV absorption.
The grey lines describe the results of the fits of Raman spectra with
$\Delta{}H$ and $\Delta{}S$ included among the fitted parameters.
To compare both fits from the viewpoint of the transition parameters,
\tabref{dna_hairpins:melting_temperatures}
presents obtained values of the transition melting temperature, i.e., the
temperature where the concentrations of the oligonucleotides in the folded and
the unfolded state are equal.

\begin{table}
	\centering
	\begin{tabular}{lllll}
\toprule
Sample  & UV absorption & UV RRS, 244\,nm & UV RRS 257\,nm & RS, 532\,nm \\
\midrule

duplex  & 60\,\textdegree{}C
                        & 59\,\textdegree{}C
												                  & 58\,\textdegree{}C
																					                 & 76\,\textdegree{}C* \\
hairpin & 66\,\textdegree{}C
                        & 65\,\textdegree{}C
												                  & 64\,\textdegree{}C
																					                 & NA \\

\bottomrule
\end{tabular}

	\caption[%
		Melting temperatures determined by the fit of the four relevant PCA
		scores.
	]{
		\captiontitle{%
			Melting temperatures determined by the fit of the four relevant PCA
			scores.
		}%
		Remark: * Non-resonant RS was measured in $200\times$ higher concentration
		then UV absorption and UV RRS.
		Increased concentration increases the melting temperature for the
		bimolecular reaction of the duplex dissociation.
		The thermodynamic parameters determined from UV absorption data predict a
		melting temperature of 73\,\textdegree{}C for this high concentration.}
	\label{\tablabel{dna_hairpins:melting_temperatures}}
\end{table}

The comparison of the obtained melting temperatures confirms good agreement
between the UV RRS and UV absorption characterization of the structural
transition.
The small decrease of the melting temperature for UV RRS can be explained in a
different way by the measurement of the sample temperature, but rather it is
related to weak heating of the sample in the focus of the exciting laser.
The agreement was somewhat poorer for non-resonance RS of the duplex, but it is
necessary to consider an error given by the recalculation of the melting
temperature value for the sample with 200 times higher concentration than that
used for UV absorption.
In the case of non-resonance RS of the hairpin, the structural transition was
pronounced so weakly that the fit with the thermodynamic parameters not fixed
to those obtained from UV absorption failed in all attempts.

The parameters of the sigmoidal asymptotes obtained by the fits allow us to
isolate spectra of both the folded and unfolded form at any temperature $T$
(assuming linearity of the noncoordinated temperature-induced changes is valid
in the studied temperature range).
The obtained UV RRS spectra and their differences reflecting particular types
of temperature-induced changes are shown in
\figref{dna_hairpins:forms_spectra}.

\begin{figure}
	\centering
	\ig{1}{results_and_discussion/assets/dna_hairpins/forms_spectra}
	\caption[%
		Isolated UV RRS spectra of the folded and unfolded form at several
		temperatures and their differences corresponding to the spectral changes
		caused by the structural transition from the folded to the unfolded form
		(melting) and those caused by a temperature increase when the structural
		form is maintained.
	]{%
		\captiontitle{%
			Isolated UV RRS spectra of the folded and unfolded form at several
			temperatures and their differences corresponding to the spectral changes
			caused by the structural transition from the folded to the unfolded form
			(melting) and those caused by a temperature increase when the structural
			form is maintained.
		}
		The chosen temperature increment of 25\,\textdegree{}C corresponds
		approximately to the width of the temperature interval of the structural
		transition.
	}
	\label{\figlabel{dna_hairpins:forms_spectra}}
\end{figure}

For all spectral sets, the dominating temperature-induced change is connected
with the structural transition from the folded to unfolded form (hereinafter
referred to as melting for shortening).
For a detailed analysis of the spectra, we relied on the published calculations
of the vibrational modes that have been at least partially confirmed by the
experiments monitoring the influence of the environmental conditions on these
modes in simple molecules, mostly methylated bases, nucleosides, or
nucleotides.
The basic characteristics of the Raman bands in the measured area were assigned
to fundamental transitions of vibrational modes localized at least in large
part on the nucleobases.
Most of the detected Raman bands belong to vibrations of purine bases, i.e.,
guanine and adenine.
It is assumed that in the spectral region below 1400\,\icm{} the vibrational
modes are coupled with sugar vibrations and are sensitive to the nucleoside
conformation (the type of sugar puckering and glycosidic angle -- syn or anti)
\parencite{%
	Benevides2005,%
	Nishimura1986b%
}.
For Raman bands above this wavenumber, the vibrational modes are localized at
the nucleobases and are sensitive to their interactions with the environment
via hydrogen bonding.
Besides the above-mentioned sensitivity that primarily causes spectral shifts
noticeable also in the non-resonant spectra, the resonance enhancements of the
Raman band intensities are strongly influenced by the stacking interactions
between adjacent bases oriented in a parallel fashion.
Similar to UV absorbance, NA folding brings about the stacked arrangement of
bases that leads to intensity decreases (hypochromism) in numerous UV RRS
bands.

Interpretation of the observed spectral changes were somewhat complicated by an
overlap of guanine and adenine bands, often belonging to similar vibrational
modes of the purine cycle.
To better estimate the proportion of overlapping bands, we relied on our UV RRS
measurements of mononucleotides and polynucleotides, which enabled us to
clarify the composition of these bands.

The detailed spectral analysis was published by
\textcite{Klener2021}.
The dominant effect of both duplex and hairpin melting was an increase in
Raman intensity due to the hypochromism of numerous Raman bands, including the
strong peaks at 1600, 1575, and 1485\,\icm{} (overlapping guanine and adenine
bands) and at 1529\,\icm{} (cytosine, only for excitation at 244\,nm) as well
as the Raman bands in the region of 1150 -- 1280\,\icm{}.

Although the change in resonant enhancement induced by melting is principally
hypochromic, i.e. formation of the duplex decreases RRS intensity, we also
observed a significant hyperchromic band where the intensity change is
opposite.
It is the adenine band at 1627\,\icm{} of the vibrational mode consisting of
\ch{NH2} scissoring vibration coupled with stretching of the adjacent part of
the pyrimidine ring.

The melting difference spectrum of the hairpin is similar to the duplex, i.e.
B-form double-helical structure of the stem, probably with slightly more
relaxed geometry (lower hypochromism of guanine bands at 1322 and
1485\,\icm{}).
On the other hand, vastly different positions with respect to adjacent bases
were indicated for adenines and thymines forming the loop.
Exceptionally low hypochromism means that stacking is minimal for the
nucleobases in the loop. On the other hand, the contribution of the
hyperchromic adenine band at 1627\,\icm{} is comparable for the hairpin, which
might indicate certain proximities of adenines and other bases, presumably
those from the GC basepair closing the loop.

The temperature increase outside the melting region (warming) influences UV RRS
remarkably weaker than the melting.
In contrast to the melting effect, the warming effect is much more similar for
both samples, the duplex and the hairpin.
The warming of the folded forms causes intensity increases of hypochromic
guanine and cytosine bands.
On the other hand, no changes of the 1650\,\icm{} peak, which is formed by
overlapping bands of CO stretching vibrations of cytosine and thymine, can be
seen.
The new spectral feature is an intensity decrease at 1583\,\icm{}.
It indicates an unusual adenine position or its interactions with the
surrounding at low temperature as this band is reported as hypochromic
\parencite{%
	Klener2015,%
	Jolles1985%
}.
The only resolvable spectral change in the low-wavenumber portion of the
differential spectra is the peak at 1326\,\icm{}.
It can be attributed to an intensity increase of the conformationally
sensitive guanine band, the position of which indicates prevailing
C2'-endo/anti conformation in the folded form.
To conclude, warming of the oligonucleotides in their folded form influences
surprisingly their double-helical segments consisting of GC basepairs, which
are expected to be the most stable parts of their folded structures.
The adenine bands seem to indicate that lower temperatures support unusual
adenine positions and/or interactions in both duplex and hairpin structures.

The amplitudes of differential spectra that describe the effects of warming on
the oligonucleotides in the unfolded form were comparable to those concerning
the folded form.
The main spectral feature was connected with the downshift of the 1485\,\icm{}
peak.
Broad and unstructured maxima were seen in the region 1280 -- 1350\,\icm{}
where the conformationally sensitive bands of guanine were present.
Other peaks in the differential spectrum were attributed to intensity changes
of hypochromic or hyperchromic guanine and adenine bands, except for the very
weak intensity increase of the cytosine band at 1529\,\icm{}.
The spectral changes were explained by a tendency of separated and open
oligonucleotide chains to prefer temporarily limited local geometry with
stacked neighboring, preferably purine nucleobases.
The broad peak in the 1280 -- 1350\,\icm{} region indicates that there was no
preferred conformation for the unfolded form (at least for guanosines).

We can conclude that this study demonstrated the ability of UV resonance Raman
spectroscopy to resolve and determine, in contrast to non-resonance RS, all
three types of temperature-induced effects of DNA oligonucleotides involving
more flexible central AT tract.
The reason is in the dominant effect of the oligonucleotide structure
disintegration on the resonance enhancement, which is pronounced via remarkable
changes of the Raman band intensities.
This enables us to distinguish the temperature-induced structural transition
(melting) from the effects of the temperature increase on the oligonucleotide
in both the folded and unfolded forms.

The difference spectra corresponding to melting confirmed similar B-form
double-helical structures of the segments consisting of GC basepairs for both
the duplex and hairpin with a slightly more relaxed geometry in the latter
case.
Minimal stacking has been indicated for the nucleobases in the loop.
Nevertheless, certain interactions of the adenine rings with neighboring
nucleobases have been detected.
Warming of both oligonucleotides in their folded state primarily influences
their segments consisting of GC basepairs.
An unusual adenine position or its unusual interaction with surroundings at low
temperatures has been also indicated.
Warming of the oligonucleotides in their unfolded state seems to demonstrate
the tendency of separated and open oligonucleotide chains to prefer momentary
local geometry with stacked neighboring purine nucleobases.
