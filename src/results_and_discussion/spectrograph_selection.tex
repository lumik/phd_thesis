\subsection{Choice of spectrograph parameters}
\begin{docitemize}
	\item Introduce dispersion according to focal length and groove density of
	grating.
	\begin{docitemize}
		\item How the theoretical values were computed using theoretical dispersion
			from Horiba manual.
		\item How the experimental values were measured and how we extrapolated
			for possible new grating selection
	\end{docitemize}
	\item Produce table of possible scenarios:
	\begin{docitemize}
		\item Theoretical values \tabref{spectrograph_selection:dispersion_theory}
		\item Experimental values
	\end{docitemize}
	\item Present decision criteria.
\end{docitemize}


\begin{table}
	\centering
	\input{results_and_discussion/assets/spectrograph_dispersion_theory}
	\caption{Spectrograph dispersion -- theory. Gratins are denoted by number
		of grooves per mm, lowest and highest mean the lowest and highest detected
		frequency in \icm, respectively. The disp. denotes average dispersion in
		\icm/px.}
	\label{\tablabel{spectrograph_selection:dispersion_theory}}
\end{table}

Finaly, we chose grating with 300 gr/mm for possible fluorescence measurements
1200 gr/mm for the measurement of full range including valence hydrogen
stretching vibrations at higher wavelengths (e.g. 257\,nm excitation) and
2400 gr/mm for the most precise measurements.