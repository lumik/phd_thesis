\subsection{Choice of spectrograph parameters}

As a spectrograph, we selected Horiba iHR550 Imaging Spectrometer with aperture
f/6.4, focal length 550 mm, magnification 1.1 and triple-grating turret so we
needed to select three gratings with the best parameters for our purpose. One
grating position we reserved for fluorescence measurements with 300 gr/mm, but
gratings for the other two positions needed to be selected based on the
required spectral range and dispersion. For the spectral range estimation
we neglected all the nonlinearities of dispersion in dependence on wavelength
and used dispersion from spectrograph specifications which are shown in the
\tabref{spectrograph_selection:dispersion_spec}
for available gratings.


\begin{table}
	\centering
	\begin{tabular}{l|cccc}
\toprule
grating (gr/mm)    & 1200 & 1800 & 2400 & 3600 \\
$\frac{\text{d}\lambda}{\text{d}x}$ (nm/mm) & 1.34 & 0.81 & 0.53 & 0.16 \\
\bottomrule
\end{tabular}
	\caption{Grating dispersion specifications taken from Horiba iHR550
		specification document. The linear dispersion
		$\frac{\text{d}\lambda}{\text{d}x}$ defines the extent to which a spectral
		interval is spread out across the focal field.}
	\label{\tablabel{spectrograph_selection:dispersion_spec}}
\end{table}


Then we used the equation for Raman wavenumber $\tilde{\nu}$ calculation from
the laser excitation wavelength in the air $\lambda_\text{e}$ and Stokes Raman
signal wavelength $\lambda_\text{R}$
\begin{equation}
	\tilde{\nu} = \frac{1}{\lambda_\text{e}} - \frac{1}{\lambda_\text{R}}
	\label{\eqnlabel{spectrograph_selection:nu}}
\end{equation}
to calculate the maximal measured wavenumber $\tilde{\nu}_2$ from the
$\tilde{\nu}_1 = 200$\,\icm{} for all the available excitation wavelengths
from our Ar ion frequency double laser according to formula
\begin{equation}
	\tilde{\nu}_2 = \lambda_\text{e}^{-1}
		- \left(\frac{1}{\lambda_\text{e}^{-1} - \tilde{\nu}_1}
			+ w\frac{\text{d}\lambda}{\text{d}x}\right)^{-1},
	\label{\eqnlabel{spectrograph_selection:spectral_range_max_est}}
\end{equation}
where $w$ denotes width of a CCD camera and $\frac{\text{d}\lambda}{\text{d}x}$
grating linear diespersion which defines the extent to which a spectral
interval is spread out across the focal field. We also calculated minimal
measured wavenumber $\tilde{\nu}_1$ from the maximal
$\tilde{\nu}_2 = 4000$\,\icm{} using the formula
\begin{equation}
	\tilde{\nu}_1 = \lambda_\text{e}^{-1}
		- \left(\frac{1}{\lambda_\text{e}^{-1} - \tilde{\nu}_2}
			- w\frac{\text{d}\lambda}{\text{d}x}\right)^{-1}.
	\label{\eqnlabel{spectrograph_selection:spectral_range_min_est}}
\end{equation}

From these values, the average wavenumber dispersion per pixel can be
calculated as
\begin{equation}
	\bar{d} = \frac{\tilde{\nu}_2 - \tilde{\nu}_1}{N_w},
	\label{\eqnlabel{spectrograph_selection:dispersion_est}}
\end{equation}
where $N_w = 2048$ is number of camera width pixels. The results of these
calculations can be seen in
\tabref{spectrograph_selection:dispersion_est}.

\begin{table}
	\centering
	\begin{tabular}{crrrrrrrr}
\toprule
$\lambda$ (nm)
	& grating & 1200 & \multicolumn{2}{c}{1800}
		& \multicolumn{2}{c}{2400} & \multicolumn{2}{c}{3600} \\
\midrule
\multirow{3}{*}{228.962}
	& $\tilde{\nu}_1$ &   200 &   200 &   131 &   200 &  1551 &   200 &  3291 \\
	& $\tilde{\nu}_2$ &  6231 &  4057 &  4000 &  2804 &  4000 &  1020 &  4000 \\
	& $\bar{d}$       & 2.945 & 1.883 & 1.889 & 1.271 & 1.196 & 0.401 & 0.346 \\
\midrule
\multirow{3}{*}{238.238}
	& $\tilde{\nu}_1$ &   200 &   200 &   470 &   200 &  1762 &   200 &  3351 \\
	& $\tilde{\nu}_2$ &  5799 &  3774 &  4000 &  2610 &  4000 &   958 &  4000 \\
	& $\bar{d}$       & 2.734 & 1.745 & 1.723 & 1.177 & 1.093 & 0.370 & 0.317 \\
\midrule
\multirow{3}{*}{243.989}
	& $\tilde{\nu}_1$ &   200 &   200 &   660 &   200 &  1881 &   200 &  3385 \\
	& $\tilde{\nu}_2$ &  5554 &  3613 &  4000 &  2500 &  4000 &   923 &  4000 \\
	& $\bar{d}$       & 2.614 & 1.667 & 1.631 & 1.123 & 1.035 & 0.353 & 0.300 \\
\midrule
\multirow{3}{*}{248.250}
	& $\tilde{\nu}_1$ &   200 &   200 &   791 &   200 &  1963 &   200 &  3408 \\
	& $\tilde{\nu}_2$ &  5382 &  3502 &  4000 &  2424 &  4000 &   898 &  4000 \\
	& $\bar{d}$       & 2.531 & 1.612 & 1.567 & 1.086 & 0.995 & 0.341 & 0.289 \\
\midrule
\multirow{3}{*}{250.854}
	& $\tilde{\nu}_1$ &   200 &   200 &   868 &   200 &  2011 &   200 &  3422 \\
	& $\tilde{\nu}_2$ &  5282 &  3436 &  4000 &  2379 &  4000 &   884 &  4000 \\
	& $\bar{d}$       & 2.481 & 1.580 & 1.529 & 1.064 & 0.971 & 0.334 & 0.282 \\
\midrule
\multirow{3}{*}{257.261}
	& $\tilde{\nu}_1$ &   200 &   200 &  1046 &   200 &  2122 &   200 &  3454 \\
	& $\tilde{\nu}_2$ &  5046 &  3282 &  4000 &  2274 &  4000 &   850 &  4000 \\
	& $\bar{d}$       & 2.366 & 1.505 & 1.442 & 1.013 & 0.917 & 0.318 & 0.267 \\
\midrule
\multirow{3}{*}{264.345}
	& $\tilde{\nu}_1$ &   200 &   200 &  1227 &   200 &  2236 &   200 &  3486 \\
	& $\tilde{\nu}_2$ &  4804 &  3125 &  4000 &  2166 &  4000 &   816 &  4000 \\
	& $\bar{d}$       & 2.248 & 1.428 & 1.354 & 0.960 & 0.862 & 0.301 & 0.251 \\
\bottomrule
\end{tabular}

	\caption{Spectrograph dispersion estimation. Gratings are denoted by number
		of grooves per mm, $\tilde{\nu}_1$ and $\tilde{\nu}_2$ are lowest and
		highest detected frequencies in \icm calculated according to
		\cref{%
			\eqnlabel{spectrograph_selection:spectral_range_max_est},%
			\eqnlabel{spectrograph_selection:spectral_range_min_est}%
		},
		respectively. The $\bar{d}$ denotes average dispersion in \icm/px
		calculated from
		\eqnref{spectrograph_selection:dispersion_est}.}
	\label{\tablabel{spectrograph_selection:dispersion_est}}
\end{table}

We can see, that if we want to see the full Raman vibration range from the
low frequency fibrations from 200\,\icm{} to valence hydrogen strething
vibrations to 4000\,\icm{} at all possible excitation wavelengths we need to
select grating with 1200\,gr/mm. For the Raman fingerprint region below
1800\,\icm{} we need to chose the grating with 2400\,gr/mm.

So, finally we chose grating with 300\,gr/mm for possible fluorescence
measurements 1200\,gr/mm for the measurement of full range including valence
hydrogen stretching vibrations at higher wavelengths (e.g. 257\,nm excitation)
and 2400\,gr/mm for the Raman fingerprint region measurements.

After the gratings were installed the predictions from spectrograph
specifications were evaluated on actual experiment and linear interpolation to
1800\,gr/mm and extrapolation to 3600\,gr/mm gratings were done for possible
future evaluation of extension of our experimental capabilities. We measured
spectra of Pt lamp for different positions of spectrograph and calibrated the
spectra to themselves
(\REFERENCE{} to the chapter describing calibration).
Then the lower ($\lambda_1$) and upper ($\lambda_2$) bound wavelengths from
each spectrum was taken and dependences of the measured range
($\lambda_2 - \lambda_1$) on these bounds were investigated. From the
\figref{spectrograph_selection:dispersion_range}
we see that they can be well approximated by linear curve, only in the plot
for 2400\,gr/mm grating a slight curvature can be seen (the residuals from
the beginning and end of the plot are slightly negative and positive in the
center). So we fitted the ranges following the equation
\begin{equation}
	\lambda_2 - \lambda_1 = p_0 \lambda_1 + p_1,
	\label{\eqnlabel{spectrograph_selection:spectral_range_max_meas}}
\end{equation}
where $p_i$ are the parameters of the fit. By using the simple transformations
we can derive the following linear dependences for the $\lambda_2$ on
$\lambda_1$
\begin{equation}
	\lambda_2 = (p_0 + 1) \lambda_1 + p_1
\end{equation}
and the range $\lambda_2 - \lambda_1$ on $\lambda_2$
\begin{align}
	\lambda_2 - \lambda_1 &= q_0 \lambda_2 + q_1,
	\label{\eqnlabel{spectrograph_selection:spectral_range_min_meas}} \\
	q_0 &= \frac{p_0}{1 + p_0}, \nonumber\\
	q_1 &= \frac{p_1}{1 + p_0}. \nonumber
\end{align}

\begin{figure}
	\centering
	\begin{subfigure}[b]{1\textwidth}
		\centering
		\input{results_and_discussion/assets/spectrograph_dispersion_meas/%
bounds_1200_range}
		\caption{The grating with 1200\,gr/mm.}
		\label{\figlabel{spectrograph_selection:dipsersion_range_1200}}
	\end{subfigure}
	\\
	\begin{subfigure}[b]{1\textwidth}
		\centering
		\input{results_and_discussion/assets/spectrograph_dispersion_meas/%
bounds_2400_range}
		\caption{The grating with 2400\,gr/mm.}
		\label{\figlabel{spectrograph_selection:dipsersion_range_2400}}
	\end{subfigure}
	\caption{Plots of spectral ranges captured by the detector in dependence on
		the lowest detected wavelength $\lambda_1$ for different gratings, where
		$\lambda_2$ denotes the largest detected wavelength. Solid lines
		represents linear fits.}
	\label{\figlabel{spectrograph_selection:dispersion_range}}
\end{figure}

From the results of the fit we calculated the same table of ranges, this time
based on measurements with real gratings. As we know that the dipsersion range
is inverse proportional to the grating number $a$, we can also estimate
them for the 1800 and 3600\,gr/mm gratings by for example interpolating
and extrapolating reciprocal dispersion (px/mm) from the values measured for
1200 and 2400\,gr/mm gratings. All these results are in
\tabref{spectrograph_selection:dispersion_meas}.

\begin{table}
	\centering
	\begin{tabular}{crrrrrrrr}
\toprule
$\lambda$ (nm)
	& grating & 1200 & \multicolumn{2}{c}{1800}
		& \multicolumn{2}{c}{2400} & \multicolumn{2}{c}{3600} \\
\midrule
\multirow{3}{*}{228.962}
	& $\tilde{\nu}_1$ &   200 &   200 &  -403 &   200 &   854 &   200 &  2001 \\
	& $\tilde{\nu}_2$ &  6500 &  4484 &  4000 &  3445 &  4000 &  2385 &  4000 \\
	& $\bar{d}$       & 3.076 & 2.092 & 2.150 & 1.585 & 1.536 & 1.067 & 0.976 \\
\midrule
\multirow{3}{*}{238.238}
	& $\tilde{\nu}_1$ &   200 &   200 &     8 &   200 &  1148 &   200 &  2188 \\
	& $\tilde{\nu}_2$ &  6042 &  4155 &  4000 &  3190 &  4000 &  2209 &  4000 \\
	& $\bar{d}$       & 2.852 & 1.931 & 1.949 & 1.460 & 1.392 & 0.981 & 0.885 \\
\midrule
\multirow{3}{*}{243.989}
	& $\tilde{\nu}_1$ &   200 &   200 &   237 &   200 &  1313 &   200 &  2293 \\
	& $\tilde{\nu}_2$ &  5782 &  3970 &  4000 &  3046 &  4000 &  2110 &  4000 \\
	& $\bar{d}$       & 2.614 & 1.841 & 1.837 & 1.389 & 1.312 & 0.932 & 0.834 \\
\midrule
\multirow{3}{*}{248.250}
	& $\tilde{\nu}_1$ &   200 &   200 &   396 &   200 &  1426 &   200 &  2365 \\
	& $\tilde{\nu}_2$ &  5600 &  3840 &  4000 &  2945 &  4000 &  2040 &  4000 \\
	& $\bar{d}$       & 2.637 & 1.777 & 1.760 & 1.340 & 1.257 & 0.899 & 0.798 \\
\midrule
\multirow{3}{*}{250.854}
	& $\tilde{\nu}_1$ &   200 &   200 &   488 &   200 &  1492 &   200 &  2407 \\
	& $\tilde{\nu}_2$ &  5492 &  3764 &  4000 &  2886 &  4000 &  2000 &  4000 \\
	& $\bar{d}$       & 2.585 & 1.740 & 1.715 & 1.312 & 1.225 & 0.879 & 0.778 \\
\midrule
\multirow{3}{*}{257.261}
	& $\tilde{\nu}_1$ &   200 &   200 &   702 &   200 &  1646 &   200 &  2505 \\
	& $\tilde{\nu}_2$ &  5244 &  3587 &  4000 &  2749 &  4000 &  1905 &  4000 \\
	& $\bar{d}$       & 2.463 & 1.653 & 1.610 & 1.244 & 1.150 & 0.833 & 0.730 \\
\midrule
\multirow{3}{*}{264.345}
	& $\tilde{\nu}_1$ &   200 &   200 &   919 &   200 &  1801 &   200 &  2605 \\
	& $\tilde{\nu}_2$ &  4987 &  3404 &  4000 &  2608 &  4000 &  1808 &  4000 \\
	& $\bar{d}$       & 2.338 & 1.565 & 1.504 & 1.176 & 1.074 & 0.785 & 0.681 \\
\bottomrule
\end{tabular}

	\caption{Spectrograph dispersion estimation. These are the same values as in
		\tabref{spectrograph_selection:dispersion_est}
		but now calculated by fit of
		\cref{%
			\eqnlabel{spectrograph_selection:nu},%
			\eqnlabel{spectrograph_selection:spectral_range_max_meas},%
			\eqnlabel{spectrograph_selection:spectral_range_min_meas}%
		}
		to measured dispersions of 1200 and 2400\,gr/mm gratings which were then
		interpolated and extrapolated to 1800 and 3600\,gr/mm gratings,
		respectively. Gratings are denoted by number of grooves per mm,
		$\tilde{\nu}_1$ and $\tilde{\nu}_2$ are lowest and highest detected
		frequencies in \icm. The $\bar{d}$ denotes average dispersion in \icm/px
		calculated from
		\eqnref{spectrograph_selection:dispersion_est}.}
	\label{\tablabel{spectrograph_selection:dispersion_meas}}
\end{table}

These results from the
\tabref{spectrograph_selection:dispersion_meas}
shows that the real values are significantly higher compared to the values
estimated from the spectrograph vendor's manual
(\tabref{spectrograph_selection:dispersion_est}),
especially for the 2400\,gr/mm grating and therefore the extrapolation to the
3600\,gr/mm grating differs even more from the estimated values. If this
extrapolation is correct it can be seen that the 3600\,gr/mm grating is more
suitable for the Raman fingerprint region than the 2400\,gr/mm one.
