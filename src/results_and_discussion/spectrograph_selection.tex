\subsection{Choice of spectrograph parameters}
\begin{docitemize}
	\item Introduce dispersion according to focal length and groove density of
	grating.
	\begin{docitemize}
		\item How the theoretical values were computed using theoretical dispersion
			from Horiba manual.
		\item How the experimental values were measured and how we extrapolated
			for possible new grating selection
	\end{docitemize}
	\item Produce table of possible scenarios:
	\begin{docitemize}
		\item Theoretical values \tabref{spectrograph_selection:dispersion_est}
		\item Experimental values
	\end{docitemize}
	\item Present decision criteria.
\end{docitemize}


As a spectrograph, we selected Horiba iHR550 Imaging Spectrometer with aperture
f/6.4, focal length 550 mm, magnification 1.1 and triple-grating turret so we
needed to select three gratings with the best parameters for our purpose. One
grating position we reserved for fluorescence measurements with 300 gr/mm, but
gratings for the other two positions needed to be selected based on the
required spectral range and dispersion. For the spectral range estimation
we neglected all the nonlinearities of dispersion in dependence on wavelength
and used dispersion from spectrograph specifications which are shown in the
\tabref{spectrograph_selection:dispersion_spec} for available gratings.


\begin{table}
	\centering
	\begin{tabular}{l|cccc}
\toprule
grating (gr/mm)    & 1200 & 1800 & 2400 & 3600 \\
$\frac{\text{d}\lambda}{\text{d}x}$ (nm/mm) & 1.34 & 0.81 & 0.53 & 0.16 \\
\bottomrule
\end{tabular}
	\caption{Grating dispersion specifications taken from Horiba iHR550
		specification document. The linear dispersion
		$\frac{\text{d}\lambda}{\text{d}x}$ defines the extent to which a spectral
		interval is spread out across the focal field.}
	\label{\tablabel{spectrograph_selection:dispersion_spec}}
\end{table}


Then we calculated the maximal measured wavenumber $\tilde{\nu}_2$ from the
$\tilde{\nu}_1 = 200$\,\icm{} for all the available excitation wavelengths
$\lambda_\text{e}$ from our Ar ion frequency double laser in Stokes Raman
according to formula
\begin{equation}
	\tilde{\nu}_2 = \lambda_\text{e}^{-1}
		- \left(\frac{1}{\lambda_\text{e}^{-1} - \tilde{\nu}_1}
			+ w\frac{\text{d}\lambda}{\text{d}x}\right)^{-1},
	\label{\eqnlabel{spectrograph_selection:spectral_range_max_est}}
\end{equation}
where $w$ denotes width of a CCD camera and $\frac{\text{d}\lambda}{\text{d}x}$
grating linear diespersion which defines the extent to which a spectral
interval is spread out across the focal field. We also calculated minimal
measured wavenumber $\tilde{\nu}_1$ from the maximal
$\tilde{\nu}_2 = 4000$\,\icm{} using the formula
\begin{equation}
	\tilde{\nu}_1 = \lambda_\text{e}^{-1}
		- \left(\frac{1}{\lambda_\text{e}^{-1} - \tilde{\nu}_2}
			- w\frac{\text{d}\lambda}{\text{d}x}\right)^{-1}.
	\label{\eqnlabel{spectrograph_selection:spectral_range_min_est}}
\end{equation}

From these values, the average wavenumber dispersion per pixel
\begin{equation}
	\bar{d} = \frac{\tilde{\nu}_2 - \tilde{\nu}_1}{N_w},
	\label{\eqnlabel{spectrograph_selection:dispersion_est}}
\end{equation}
where $N_w = 2048$ is number of camera width pixels. The results of these
calculations can be seen in \tabref{spectrograph_selection:dispersion_est}

\begin{table}
	\centering
	\begin{tabular}{crrrrrrrr}
\toprule
$\lambda$ (nm)
	& grating & 1200 & \multicolumn{2}{c}{1800}
		& \multicolumn{2}{c}{2400} & \multicolumn{2}{c}{3600} \\
\midrule
\multirow{3}{*}{228.962}
	& $\tilde{\nu}_1$ &   200 &   200 &   131 &   200 &  1551 &   200 &  3291 \\
	& $\tilde{\nu}_2$ &  6231 &  4057 &  4000 &  2804 &  4000 &  1020 &  4000 \\
	& $\bar{d}$       & 2.945 & 1.883 & 1.889 & 1.271 & 1.196 & 0.401 & 0.346 \\
\midrule
\multirow{3}{*}{238.238}
	& $\tilde{\nu}_1$ &   200 &   200 &   470 &   200 &  1762 &   200 &  3351 \\
	& $\tilde{\nu}_2$ &  5799 &  3774 &  4000 &  2610 &  4000 &   958 &  4000 \\
	& $\bar{d}$       & 2.734 & 1.745 & 1.723 & 1.177 & 1.093 & 0.370 & 0.317 \\
\midrule
\multirow{3}{*}{243.989}
	& $\tilde{\nu}_1$ &   200 &   200 &   660 &   200 &  1881 &   200 &  3385 \\
	& $\tilde{\nu}_2$ &  5554 &  3613 &  4000 &  2500 &  4000 &   923 &  4000 \\
	& $\bar{d}$       & 2.614 & 1.667 & 1.631 & 1.123 & 1.035 & 0.353 & 0.300 \\
\midrule
\multirow{3}{*}{248.250}
	& $\tilde{\nu}_1$ &   200 &   200 &   791 &   200 &  1963 &   200 &  3408 \\
	& $\tilde{\nu}_2$ &  5382 &  3502 &  4000 &  2424 &  4000 &   898 &  4000 \\
	& $\bar{d}$       & 2.531 & 1.612 & 1.567 & 1.086 & 0.995 & 0.341 & 0.289 \\
\midrule
\multirow{3}{*}{250.854}
	& $\tilde{\nu}_1$ &   200 &   200 &   868 &   200 &  2011 &   200 &  3422 \\
	& $\tilde{\nu}_2$ &  5282 &  3436 &  4000 &  2379 &  4000 &   884 &  4000 \\
	& $\bar{d}$       & 2.481 & 1.580 & 1.529 & 1.064 & 0.971 & 0.334 & 0.282 \\
\midrule
\multirow{3}{*}{257.261}
	& $\tilde{\nu}_1$ &   200 &   200 &  1046 &   200 &  2122 &   200 &  3454 \\
	& $\tilde{\nu}_2$ &  5046 &  3282 &  4000 &  2274 &  4000 &   850 &  4000 \\
	& $\bar{d}$       & 2.366 & 1.505 & 1.442 & 1.013 & 0.917 & 0.318 & 0.267 \\
\bottomrule
\end{tabular}

	\caption{Spectrograph dispersion estimation. Gratins are denoted by number
		of grooves per mm, $\tilde{\nu}_1$ and $\tilde{\nu}_2$ are lowest and
		highest detected frequencies in \icm calculated according to
		\cref{%
			\eqnlabel{spectrograph_selection:spectral_range_max_est},%
			\eqnlabel{spectrograph_selection:spectral_range_min_est}%
		},
		respectively. The $\bar{d}$ denotes average dispersion in \icm/px
		calculated from \eqnref{spectrograph_selection:dispersion_est}.}
	\label{\tablabel{spectrograph_selection:dispersion_est}}
\end{table}

We can see, that if we want to see the full Raman vibration range from the
low frequency fibrations from 200\,\icm to valence hydrogen strething
vibrations to 4000\,\icm at all possible excitation wavelengths we need to
select grating with 1200 gr/mm. For the Raman fingerprint region below
1800\,\icm we need to chose the grating with 2400 gr/mm.

So, finally we chose grating with 300 gr/mm for possible fluorescence
measurements 1200 gr/mm for the measurement of full range including valence
hydrogen stretching vibrations at higher wavelengths (e.g. 257\,nm excitation)
and 2400 gr/mm for the Raman fingerprint region measurements.
