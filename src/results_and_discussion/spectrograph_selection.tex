\subsection{Choice of spectrograph parameters}

As a spectrograph, we selected Horiba iHR550 Imaging Spectrometer with aperture
f/6.4, focal length 550 mm, magnification 1.1 and triple-grating turret so we
needed to select three gratings with the best parameters for our purpose. One
grating position we reserved for fluorescence measurements with 300 gr/mm, but
gratings for the other two positions needed to be selected based on the
required spectral range and dispersion. For the spectral range estimation
we neglected all the nonlinearities of dispersion in dependence on wavelength
and used dispersions from spectrograph sale materials which are shown in the
\tabref{spectrograph_selection:dispersion_spec}
for available gratings.

\begin{table}
	\centering
	\begin{tabular}{l|cccc}
\toprule
grating (gr/mm)    & 1200 & 1800 & 2400 & 3600 \\
$\frac{\text{d}\lambda}{\text{d}x}$ (nm/mm) & 1.34 & 0.81 & 0.53 & 0.16 \\
\bottomrule
\end{tabular}
	\caption{Grating dispersion specifications taken from Horiba iHR550
		specification document. The linear dispersion
		$\frac{\text{d}\lambda}{\text{d}x}$ defines the extent to which a spectral
		interval is spread out across the focal field.}
	\label{\tablabel{spectrograph_selection:dispersion_spec}}
\end{table}

Then we used the equation for Raman wavenumber $\tilde{\nu}$ calculation from
the laser excitation wavelength in the air $\lambda_\text{e}$ and Stokes Raman
signal wavelength $\lambda_\text{R}$
\begin{equation}
	\tilde{\nu} = \frac{1}{\lambda_\text{e}} - \frac{1}{\lambda_\text{R}}
	\label{\eqnlabel{spectrograph_selection:nu}}
\end{equation}
to calculate the maximal measured wavenumber $\tilde{\nu}_2$ from the
$\tilde{\nu}_1 = 200$\,\icm{} for all the available excitation wavelengths
from our Ar ion frequency double laser according to formula
\begin{equation}
	\tilde{\nu}_2 = \lambda_\text{e}^{-1}
		- \left(\frac{1}{\lambda_\text{e}^{-1} - \tilde{\nu}_1}
			+ w\frac{\text{d}\lambda}{\text{d}x}\right)^{-1},
	\label{\eqnlabel{spectrograph_selection:spectral_range_max_est}}
\end{equation}
where $w$ denotes width of a CCD camera and $\frac{\text{d}\lambda}{\text{d}x}$
grating linear diespersion which defines the extent to which a spectral
interval is spread out across the focal field. We also calculated minimal
measured wavenumber $\tilde{\nu}_1$ from the maximal
$\tilde{\nu}_2 = 4000$\,\icm{} using the formula
\begin{equation}
	\tilde{\nu}_1 = \lambda_\text{e}^{-1}
		- \left(\frac{1}{\lambda_\text{e}^{-1} - \tilde{\nu}_2}
			- w\frac{\text{d}\lambda}{\text{d}x}\right)^{-1}.
	\label{\eqnlabel{spectrograph_selection:spectral_range_min_est}}
\end{equation}

From these values, the average wavenumber dispersion per pixel can be
calculated as
\begin{equation}
	\bar{d} = \frac{\tilde{\nu}_2 - \tilde{\nu}_1}{N_w},
	\label{\eqnlabel{spectrograph_selection:dispersion_est}}
\end{equation}
where $N_w = 2048$ is number of camera width pixels. The results of these
calculations can be seen in
\tabref{spectrograph_selection:dispersion_est}.

\begin{table}
	\centering
	\begin{tabular}{crrrrrrrr}
\toprule
$\lambda$ (nm)
	& grating & 1200 & \multicolumn{2}{c}{1800}
		& \multicolumn{2}{c}{2400} & \multicolumn{2}{c}{3600} \\
\midrule
\multirow{3}{*}{228.962}
	& $\tilde{\nu}_1$ &   200 &   200 &   131 &   200 &  1551 &   200 &  3291 \\
	& $\tilde{\nu}_2$ &  6231 &  4057 &  4000 &  2804 &  4000 &  1020 &  4000 \\
	& $\bar{d}$       & 2.945 & 1.883 & 1.889 & 1.271 & 1.196 & 0.401 & 0.346 \\
\midrule
\multirow{3}{*}{238.238}
	& $\tilde{\nu}_1$ &   200 &   200 &   470 &   200 &  1762 &   200 &  3351 \\
	& $\tilde{\nu}_2$ &  5799 &  3774 &  4000 &  2610 &  4000 &   958 &  4000 \\
	& $\bar{d}$       & 2.734 & 1.745 & 1.723 & 1.177 & 1.093 & 0.370 & 0.317 \\
\midrule
\multirow{3}{*}{243.989}
	& $\tilde{\nu}_1$ &   200 &   200 &   660 &   200 &  1881 &   200 &  3385 \\
	& $\tilde{\nu}_2$ &  5554 &  3613 &  4000 &  2500 &  4000 &   923 &  4000 \\
	& $\bar{d}$       & 2.614 & 1.667 & 1.631 & 1.123 & 1.035 & 0.353 & 0.300 \\
\midrule
\multirow{3}{*}{248.250}
	& $\tilde{\nu}_1$ &   200 &   200 &   791 &   200 &  1963 &   200 &  3408 \\
	& $\tilde{\nu}_2$ &  5382 &  3502 &  4000 &  2424 &  4000 &   898 &  4000 \\
	& $\bar{d}$       & 2.531 & 1.612 & 1.567 & 1.086 & 0.995 & 0.341 & 0.289 \\
\midrule
\multirow{3}{*}{250.854}
	& $\tilde{\nu}_1$ &   200 &   200 &   868 &   200 &  2011 &   200 &  3422 \\
	& $\tilde{\nu}_2$ &  5282 &  3436 &  4000 &  2379 &  4000 &   884 &  4000 \\
	& $\bar{d}$       & 2.481 & 1.580 & 1.529 & 1.064 & 0.971 & 0.334 & 0.282 \\
\midrule
\multirow{3}{*}{257.261}
	& $\tilde{\nu}_1$ &   200 &   200 &  1046 &   200 &  2122 &   200 &  3454 \\
	& $\tilde{\nu}_2$ &  5046 &  3282 &  4000 &  2274 &  4000 &   850 &  4000 \\
	& $\bar{d}$       & 2.366 & 1.505 & 1.442 & 1.013 & 0.917 & 0.318 & 0.267 \\
\midrule
\multirow{3}{*}{264.345}
	& $\tilde{\nu}_1$ &   200 &   200 &  1227 &   200 &  2236 &   200 &  3486 \\
	& $\tilde{\nu}_2$ &  4804 &  3125 &  4000 &  2166 &  4000 &   816 &  4000 \\
	& $\bar{d}$       & 2.248 & 1.428 & 1.354 & 0.960 & 0.862 & 0.301 & 0.251 \\
\bottomrule
\end{tabular}

	\caption{Spectrograph dispersion estimation. Gratings are denoted by number
		of grooves per mm, $\tilde{\nu}_1$ and $\tilde{\nu}_2$ are lowest and
		highest detected frequencies in \icm calculated according to
		\cref{%
			\eqnlabel{spectrograph_selection:spectral_range_max_est},%
			\eqnlabel{spectrograph_selection:spectral_range_min_est}%
		},
		respectively. The $\bar{d}$ denotes average dispersion in \icm/px
		calculated from
		\eqnref{spectrograph_selection:dispersion_est}.}
	\label{\tablabel{spectrograph_selection:dispersion_est}}
\end{table}

We can see, that if we want to see the full Raman vibration range from the
low frequency vibrations from 200\,\icm{} to valence hydrogen stretching
vibrations to 4000\,\icm{} at all possible excitation wavelengths we need to
select grating with 1200\,gr/mm. For the Raman fingerprint region below
1800\,\icm{} we need to chose the grating with 2400\,gr/mm.

So, finally we chose grating with 300\,gr/mm for possible fluorescence
measurements 1200\,gr/mm for the measurement of full range including valence
hydrogen stretching vibrations at higher wavelengths (e.g. 257\,nm excitation)
and 2400\,gr/mm for the Raman fingerprint region measurements.

After the gratings were installed the predictions from spectrograph vendor
sale materials were evaluated on actual experiment and with help of
diffraction grating theory and spectrograph and CCD camera specifications.
The values in the table were adjusted for possible future evaluation of
extension of our experimental capabilities. We measured spectra of Pt lamp for
different positions of spectrograph and calibrated the spectra to themselves
(\cref{wavenumber_calibration}).
Then the lower ($\lambda_1$) and upper ($\lambda_2$) bound wavelengths from
each spectrum were taken and dependences of the measured range
($\Delta\lambda = \lambda_2 - \lambda_1$)
on these bounds were investigated.

The spectrograph uses a grating to separate different light energies. The
diffraction of light on the grating follows grating equation
\begin{equation}
	a(\sin\alpha_\text{i} + \sin\alpha_m) = m\lambda,
	\label{\eqnlabel{spectrograph_selection:grating_equation}}
\end{equation}
where $\alpha_\text{i}$ is incident light angle, $\alpha_m$ is diffraction
angle to the $m$-th diffraction order, $\lambda$ is diffracted light wavelength
and $a$ is grating constant. The directions of the angles $\alpha$ can be seen
in the
\figref{spectrograph_selection:configuration_schema}.

\begin{figure}
	\centering
	\begin{tikzpicture}[font=\sffamily, >=Latex]

\tikzset{
	mirror element/.style={color=black, line width=2*\pgflinewidth},
	axis/.style={color=black, dash pattern=on 3pt off 3pt},
	ray arrow/.tip={Latex[length=3mm]},
	angle arrow/.tip={Latex[length=2mm]},
	incident ray/.style={->,color=black, line width=2*\pgflinewidth,%
		>=ray arrow},
	difracted ray/.style={->,color=cyan, line width=2*\pgflinewidth,%
		>=angle arrow},
	directed angle/.style={->,color=black, >=angle arrow},
};

\clip (1,-.1) rectangle (11,4.1);

\pgfdeclarelayer{bg}    % declare background layer
\pgfsetlayers{bg,main}  % set the order of the layers (main is the standard layer)

\newcommand*{\gratingStepA}{0.1}
\newcommand*{\gratingStepB}{0.3}
\newcommand*{\gratingStepY}{0.1}
\newcommand*{\halfGrooveRepetition}{15}
\newcommand*{\canvasHeight}{4}
\newcommand*{\angleRadiusA}{1.0}
\newcommand*{\angleRadiusB}{1.7}
\newcommand*{\angleRadiusC}{1.8}
\newcommand*{\angleRadiusD}{2.5}

% draw the grating
\draw[mirror element] (0,0)
	% first half of the grating
	\foreach \x in {1,...,\halfGrooveRepetition} {
		-- ++(\gratingStepA,\gratingStepY) -- ++(\gratingStepB,-\gratingStepY)}
	coordinate (gratingMiddle)
	% second half of the grating
	\foreach \x in {1,...,\halfGrooveRepetition} {
		-- ++(\gratingStepA,\gratingStepY) -- ++(\gratingStepB,-\gratingStepY)};

% draw axis perpendicular to the grating offseted from the bottom of the groove
\draw[axis] ($ (gratingMiddle) + (-\gratingStepB/2,\gratingStepY/2) $)
	coordinate (zero)
	-- ++(0,10);

% draw the rays in background for them not to overlay the grating
\begin{pgfonlayer}{bg}    % select the background layer
	\draw[difracted ray] (zero) -- ($ (zero) + (5,\canvasHeight) $)
		coordinate (difractedEnd);
	\draw[incident ray] ($ (zero) + (-1.7,\canvasHeight) $)
		coordinate (incidentStart) -- (zero);
\end{pgfonlayer}

% calculate incident and difracted light angles
\mypgfextractangle{\incidentAngle}{zero}{incidentStart}
\mypgfextractangle{\diffractedAngle}{zero}{difractedEnd}

% calculate middle of Ebert angle \varphi
\pgfmathparse{\incidentAngle/2 + \diffractedAngle/2}
\let\diffractionAxisAngle\pgfmathresult

% draw axis in the middle of Ebert angle \varphi
\draw[axis] (zero) -- ++(\diffractionAxisAngle:10);

% draw angles
\draw[directed angle] ($ (zero) + (\incidentAngle:\angleRadiusA) $)
	arc (\incidentAngle:\diffractedAngle:\angleRadiusA)
	node[above,pos=0.85] {$\varphi$};
\draw[directed angle] ($ (zero) + (90:\angleRadiusB) $)
	arc (90:\incidentAngle:\angleRadiusB)
	node[above,pos=0.5] {$\alpha_i$};
\draw[directed angle] ($ (zero) + (90:\angleRadiusC) $)
	arc (90:\diffractedAngle:\angleRadiusC)
	node[above,pos=0.7] {$\alpha_m$};
\draw[directed angle] ($ (zero) + (90:\angleRadiusD) $)
	arc (90:\diffractionAxisAngle:\angleRadiusD)
	node[above,pos=0.5] {$\vartheta_m$};
\draw[directed angle] ($ (zero) + (\diffractionAxisAngle:\angleRadiusD) $)
	arc (\diffractionAxisAngle:\diffractedAngle:\angleRadiusD)
	node[above,pos=0.5] {$\frac{\varphi}{2}$};

\end{tikzpicture}

	\caption{Spectrograph configuration schema.}
	\label{\figlabel{spectrograph_selection:configuration_schema}}
\end{figure}

The grating constant can be calculated from the number of grooves per mm
$N_\text{gr}$ as

\begin{equation*}
	a = \frac{1}{N_\text{gr}}.
\end{equation*}

We can also derive angular dispersion
$\frac{\text{d}\alpha_m}{\text{d}\lambda}$
of the grating from the
\eqnref{spectrograph_selection:grating_equation}
for the fixed incident light angle $\alpha_\text{i}$ as

\begin{equation}
	\frac{\text{d}\alpha_m}{\text{d}\lambda} = \frac{m}{a\cos\alpha_m}.
	\label{\eqnlabel{spectrograph_selection:angular_dispersion}}
\end{equation}

The selected spectrograph uses Czerny-Turner configuration in which the
diffracted light is focused onto detector plane by focussing mirror with
effective focal length $f$ which converts angular dispersion to dispersion on
the detector plane with coordinate $x$ as
\begin{equation*}
	\frac{\text{d}x}{\text{d}\lambda} =
		f\frac{\text{d}\alpha_m}{\text{d}\lambda}
\end{equation*}
and inverting the equation and using
\eqnref{spectrograph_selection:angular_dispersion}
we get linear dispersion
\begin{equation}
	\frac{\text{d}\lambda}{\text{d}x} = \frac{a\cos\alpha_m}{fm}.
	\label{\eqnlabel{spectrograph_selection:linear_dispersion}}
\end{equation}

We can see, that linear dispersion for fixed grating position is independent
of the light wavelength but, for a typical spectrograph, the angle between
incident and diffracted light $\varphi$, called \emph{Ebert angle}, is fixed by
the spectrograph geometry and the diffracted light wavelength is selected by
rotation of diffraction grating by angle $\vartheta_m$. So, it is useful to
transform the
\eqnref{spectrograph_selection:grating_equation}
to these variables. Following the direction of angles denoted in the
\figref{spectrograph_selection:configuration_schema}
we can derive
\begin{equation}
	\alpha_\text{i} = \vartheta_m - \frac{\varphi}{2}\ ,
	\alpha_m = \vartheta_m + \frac{\varphi}{2}
	\label{\eqnlabel{spectrograph_selection:ebert_transformation}}
\end{equation}
and putting that into
\eqnref{spectrograph_selection:grating_equation}
and using goniometric formula for summation of two sine functions we get
\begin{equation}
	2a\sin\vartheta_m\cos\frac{\varphi}{2} = m\lambda.
	\label{\eqnlabel{spectrograph_selection:grating_equation_ebert}}
\end{equation}

As we previously said, for typical spectrograph, we usually do not know
diffraction angle $\alpha_m$ but rather the Ebert angle $\phi$ and we can
measure the diffracted light wavelength $\lambda$ by comparison to the source
of light with known wavelength so using
\eqnref{spectrograph_selection:grating_equation_ebert}
we can calculate grating rotation angle
\begin{equation}
	\vartheta_m = \text{arcsin}\left(
		\frac{m\lambda}{2a\cos\frac{\varphi}{2}}\right)
	\label{\eqnlabel{spectrograph_selection:grating_rotation_angle}}
\end{equation}
which can be used in combination with
\cref{%
	\eqnlabel{spectrograph_selection:ebert_transformation},%
	\eqnlabel{spectrograph_selection:linear_dispersion}%
}
for linear dispersion calculation.

All the parameters for the linear dispersion calculation can be retrieved from
the spectrograph documentation and are summarized in
\tabref{spectrograph_selection:dispersion_params}.

\begin{table}
	\centering
	\begin{tabular}{lll}
\toprule
quantity name          & symbol         & value \\
\midrule
effective focal length & $f$            & 568.72\,mm \\
Ebert angle            & $\varphi$      & $12.996^\circ$ \\
diffraction order      & $m$            & 1 \\
length of CCD          & $l_\text{CCD}$ & 27.6 \\
\bottomrule
\end{tabular}

	\caption{Spectrograph and CCD parameters for dispersion calculation. All the
		values are taken from the spectrograph and CCD specification documents.}
	\label{\tablabel{spectrograph_selection:dispersion_params}}
\end{table}

We can't measure linear dispersion directly but we can measure it for example
from the spectral range captured by the used CCD camera
\begin{equation*}
	\Delta\lambda = \lambda_2 - \lambda_1
\end{equation*}
divided by the length of the CCD camera $l_\text{CCD}$, because in the first
approximation of
\eqnref{spectrograph_selection:linear_dispersion}
the linear dispersion is not dependent on the diffracted light wavelength. So,
we can relate experimentally measured spectral range to the dispersion by the
equation
\begin{equation}
	\Delta\lambda = l_\text{CCD}\frac{\text{d}\lambda}{\text{d}x}.
	\label{\eqnlabel{spectrograph_selection:measured_range}}
\end{equation}

Last unknown in this equation is the wavelength $\lambda$ for the Ebert angle
$\varphi$ but we can calculate it in the linear dispersion approximation as the
center of the measured spectral range
\begin{equation*}
	\lambda = \frac{\lambda_1 + \lambda_2}{2}.
\end{equation*}

As we want to compare the theoretical and experimental values the least precise
parameter from the description of the instrument was the camera size, so we
fitted
\eqnref{spectrograph_selection:measured_range}
to the measured ranges with the camera length $l_\text{CCD}$ as a parameter for
grating with 1200 and 2400\,gr/mm, the results can be seen in the
\tabref{spectrograph_selection:detector_length_fits}
and dependences of the spectral ranges on wavelength are plotted in the
\figref{spectrograph_selection:dispersion_range}.
It can be seen from the
\tabref{spectrograph_selection:detector_length_fits}
that the detector length estimation difference between grating
with 1200\,gr/mm and 2400\,gr/mm is slightly larger thant the estimated
standard deviations. It can be attributed to the slight inaccuracy in used
Ebert angle but the difference is so small that we decided not to pursue its
further refinement by more complicated nonlinear fit.

\begin{table}
	\centering
	\begin{tabular}{ll}
\toprule
grating (gr/mm) & $l_\text{CCD}$ (mm) \\
\midrule
1200            & $27.55889 \pm 0.00040$ \\
2400            & $27.56333 \pm 0.00055$ \\
weighted mean   & $27.56043 \pm 0.00032$ \\
specification   & $27.6$ \\
\bottomrule
\end{tabular}

	\caption{Results of fits of dispersion in dependence on wavelength with
		detector length as a parameter for different gratings.}
	\label{\tablabel{spectrograph_selection:detector_length_fits}}
\end{table}

\begin{figure}
	\centering
	\begin{subfigure}[b]{1\textwidth}
		\centering
		\input{results_and_discussion/assets/spectrograph_dispersion_meas/%
bounds_1200_range}
		\caption{The grating with 1200\,gr/mm.}
		\label{\figlabel{spectrograph_selection:dipsersion_range_1200}}
	\end{subfigure}
	\\
	\begin{subfigure}[b]{1\textwidth}
		\centering
		\input{results_and_discussion/assets/spectrograph_dispersion_meas/%
bounds_2400_range}
		\caption{The grating with 2400\,gr/mm.}
		\label{\figlabel{spectrograph_selection:dipsersion_range_2400}}
	\end{subfigure}
	\caption{Plots of spectral ranges captured by the detector in dependence on
		the central detected wavelength $\lambda$ for different gratings. Solid
		lines represents theoretical curves following
		\eqnref{spectrograph_selection:measured_range}
		with parameters from
		\tabref{spectrograph_selection:dispersion_params}
		but for lenght of CCD chip $l_\text{CCD}$ which is average of estimates
		from fits summarized in
		\tabref{spectrograph_selection:detector_length_fits}.}
	\label{\figlabel{spectrograph_selection:dispersion_range}}
\end{figure}

We would like to calculate the similar table to
\tabref{spectrograph_selection:dispersion_est}
now. For that purpose, we need to be able to estimate the $\lambda_2$ if we
know $\lambda_1$ or vice versa. We can use the grating equation
\eqnref{spectrograph_selection:grating_equation}
using
\cref{%
	\eqnlabel{spectrograph_selection:ebert_transformation},%
	\eqnlabel{spectrograph_selection:grating_rotation_angle}%
}
with Ebert angle $\phi$ modified by view angle of detector length
$\beta$ for the rays pointing to the detector edges instead of the center
\begin{align}
	\begin{split}
		\varphi_1 = \varphi - \beta / 2    &= 11.606^\circ, \\
		\varphi_2 = \varphi + \beta / 2    &= 14.386^\circ, \\
		\beta     = \frac{l_\text{CCD}}{f} &=  2.781^\circ,
	\end{split}
	\label{\eqnlabel{spectrograph_selection:modified_ebert_angles}}
\end{align}
where $\varphi_1$ and $\varphi_2$ stay for Ebert angles to the beginning and
end of the detector respectively. The results are summarized in the
\tabref{spectrograph_selection:dispersion_meas}.

\begin{table}
	\centering
	\begin{tabular}{crrrrrrrr}
\toprule
$\lambda$ (nm)
	& grating & 1200 & \multicolumn{2}{c}{1800}
		& \multicolumn{2}{c}{2400} & \multicolumn{2}{c}{3600} \\
\midrule
\multirow{3}{*}{228.962}
	& $\tilde{\nu}_1$ &   200 &   200 &  -403 &   200 &   854 &   200 &  2001 \\
	& $\tilde{\nu}_2$ &  6500 &  4484 &  4000 &  3445 &  4000 &  2385 &  4000 \\
	& $\bar{d}$       & 3.076 & 2.092 & 2.150 & 1.585 & 1.536 & 1.067 & 0.976 \\
\midrule
\multirow{3}{*}{238.238}
	& $\tilde{\nu}_1$ &   200 &   200 &     8 &   200 &  1148 &   200 &  2188 \\
	& $\tilde{\nu}_2$ &  6042 &  4155 &  4000 &  3190 &  4000 &  2209 &  4000 \\
	& $\bar{d}$       & 2.852 & 1.931 & 1.949 & 1.460 & 1.392 & 0.981 & 0.885 \\
\midrule
\multirow{3}{*}{243.989}
	& $\tilde{\nu}_1$ &   200 &   200 &   237 &   200 &  1313 &   200 &  2293 \\
	& $\tilde{\nu}_2$ &  5782 &  3970 &  4000 &  3046 &  4000 &  2110 &  4000 \\
	& $\bar{d}$       & 2.614 & 1.841 & 1.837 & 1.389 & 1.312 & 0.932 & 0.834 \\
\midrule
\multirow{3}{*}{248.250}
	& $\tilde{\nu}_1$ &   200 &   200 &   396 &   200 &  1426 &   200 &  2365 \\
	& $\tilde{\nu}_2$ &  5600 &  3840 &  4000 &  2945 &  4000 &  2040 &  4000 \\
	& $\bar{d}$       & 2.637 & 1.777 & 1.760 & 1.340 & 1.257 & 0.899 & 0.798 \\
\midrule
\multirow{3}{*}{250.854}
	& $\tilde{\nu}_1$ &   200 &   200 &   488 &   200 &  1492 &   200 &  2407 \\
	& $\tilde{\nu}_2$ &  5492 &  3764 &  4000 &  2886 &  4000 &  2000 &  4000 \\
	& $\bar{d}$       & 2.585 & 1.740 & 1.715 & 1.312 & 1.225 & 0.879 & 0.778 \\
\midrule
\multirow{3}{*}{257.261}
	& $\tilde{\nu}_1$ &   200 &   200 &   702 &   200 &  1646 &   200 &  2505 \\
	& $\tilde{\nu}_2$ &  5244 &  3587 &  4000 &  2749 &  4000 &  1905 &  4000 \\
	& $\bar{d}$       & 2.463 & 1.653 & 1.610 & 1.244 & 1.150 & 0.833 & 0.730 \\
\midrule
\multirow{3}{*}{264.345}
	& $\tilde{\nu}_1$ &   200 &   200 &   919 &   200 &  1801 &   200 &  2605 \\
	& $\tilde{\nu}_2$ &  4987 &  3404 &  4000 &  2608 &  4000 &  1808 &  4000 \\
	& $\bar{d}$       & 2.338 & 1.565 & 1.504 & 1.176 & 1.074 & 0.785 & 0.681 \\
\bottomrule
\end{tabular}

	\caption{Spectrograph dispersion estimation. These are the same values as in
		\tabref{spectrograph_selection:dispersion_est}
		but now calculated from
		\cref{%
			\eqnlabel{spectrograph_selection:grating_equation},%
			\eqnlabel{spectrograph_selection:ebert_transformation},%
			\eqnlabel{spectrograph_selection:grating_rotation_angle},%
			\eqnlabel{spectrograph_selection:nu}%
		}
		with Ebert angle values calculated in
		\eqnref{spectrograph_selection:modified_ebert_angles}.
		Gratings are denoted by number of grooves per mm, $\tilde{\nu}_1$ and
		$\tilde{\nu}_2$ are lowest and highest detected frequencies in \icm. The
		$\bar{d}$ denotes average dispersion in \icm/px calculated from
		\eqnref{spectrograph_selection:dispersion_est}.}
	\label{\tablabel{spectrograph_selection:dispersion_meas}}
\end{table}

These results from the
\tabref{spectrograph_selection:dispersion_meas}
show that the real ranges are significantly higher compared to the values
estimated from the spectrograph sale materials
(\tabref{spectrograph_selection:dispersion_est}),
especially for the gratings with higher groove densities. It can be seen from
these results that the 3600\,gr/mm grating would be more suitable for the Raman
fingerprint region than the 2400\,gr/mm one.
