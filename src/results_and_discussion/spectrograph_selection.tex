\subsection{Choice of spectrograph parameters}
\begin{docitemize}
	\item Introduce dispersion according to focal length and groove density of
	grating.
	\begin{docitemize}
		\item How the theoretical values were computed using theoretical dispersion
			from Horiba manual.
		\item How the experimental values were measured and how we extrapolated
			for possible new grating selection
	\end{docitemize}
	\item Produce table of possible scenarios:
	\begin{docitemize}
		\item Theoretical values \tabref{spectrograph_selection:dispersion_theory}
		\item Experimental values
	\end{docitemize}
	\item Present decision criteria.
\end{docitemize}


\begin{table}
	\centering
	\begin{tabular}{crrrrrrrr}
\toprule
$\lambda$ (nm)
    & grating & 1200 & \multicolumn{2}{c}{1800}
        & \multicolumn{2}{c}{2400} & \multicolumn{2}{c}{3600} \\
\midrule
\multirow{3}{*}{228.962}
    & lower &   200 &   200 &   131 &   200 &  1551 &   200 &  3291 \\
    & upper &  6231 &  4057 &  4000 &  2804 &  4000 &  1020 &  4000 \\
    & disp. & 2.945 & 1.883 & 1.889 & 1.271 & 1.196 & 0.401 & 0.346 \\
\midrule
\multirow{3}{*}{238.238}
    & lower &   200 &   200 &   470 &   200 &  1762 &   200 &  3351 \\
    & upper &  5799 &  3774 &  4000 &  2610 &  4000 &   958 &  4000 \\
    & disp. & 2.734 & 1.745 & 1.723 & 1.177 & 1.093 & 0.370 & 0.317 \\
\midrule
\multirow{3}{*}{243.989}
    & lower &   200 &   200 &   660 &   200 &  1881 &   200 &  3385 \\
    & upper &  5554 &  3613 &  4000 &  2500 &  4000 &   923 &  4000 \\
    & disp. & 2.614 & 1.667 & 1.631 & 1.123 & 1.035 & 0.353 & 0.300 \\
\midrule
\multirow{3}{*}{248.250}
    & lower &   200 &   200 &   791 &   200 &  1963 &   200 &  3408 \\
    & upper &  5382 &  3502 &  4000 &  2424 &  4000 &   898 &  4000 \\
    & disp. & 2.531 & 1.612 & 1.567 & 1.086 & 0.995 & 0.341 & 0.289 \\
\midrule
\multirow{3}{*}{250.854}
    & lower &   200 &   200 &   868 &   200 &  2011 &   200 &  3422 \\
    & upper &  5282 &  3436 &  4000 &  2379 &  4000 &   884 &  4000 \\
    & disp. & 2.481 & 1.580 & 1.529 & 1.064 & 0.971 & 0.334 & 0.282 \\
\midrule
\multirow{3}{*}{257.261}
    & lower &   200 &   200 &  1046 &   200 &  2122 &   200 &  3454 \\
    & upper &  5046 &  3282 &  4000 &  2274 &  4000 &   850 &  4000 \\
    & disp. & 2.366 & 1.505 & 1.442 & 1.013 & 0.917 & 0.318 & 0.267 \\
\midrule
\multirow{3}{*}{264.345}
    & lower &   200 &   200 &  1227 &   200 &  2236 &   200 &  3486 \\
    & upper &  4804 &  3125 &  4000 &  2166 &  4000 &   816 &  4000 \\
    & disp. & 2.248 & 1.428 & 1.354 & 0.960 & 0.862 & 0.301 & 0.251 \\
\bottomrule
\end{tabular}

	\caption{Spectrograph dispersion -- theory. Gratins are denoted by number
		of grooves per mm, lowest and highest mean the lowest and highest detected
		frequency in \icm, respectively. The disp. denotes average dispersion in
		\icm/px.}
	\label{\tablabel{spectrograph_selection:dispersion_theory}}
\end{table}

Finaly, we chose grating with 300 gr/mm for possible fluorescence measurements
1200 gr/mm for the measurement of full range including valence hydrogen
stretching vibrations at higher wavelengths (e.g. 257\,nm excitation) and
2400 gr/mm for the most precise measurements.