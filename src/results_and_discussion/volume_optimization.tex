\subsection{Volume optimization}

Results from the laser power and concentration optimization measurements
led to the conclusion that the main influence on the photodegradation of
sample has the number of illuminated molecules. So, we tried to verify the
hypothesis by next measurement where we kept the number of molecules in the
sample at the same level so we varied concentration and volume of the samples
and preserved the amount of the analyte, concretely we used 3\,mL of
125\,g{m}M sample, 2\,mL of 188\g{m}M sample, 1\,mL of 375\,\g{m}M sample and
0.5\,mL of 750\,g{m}M sample. The results are displayed in
\figref{vol_optim:hairpins}
and
\tabref{vol_optim:lifetimes_hairpins}.

\begin{figure}
	\centering
	\input{results_and_discussion/assets/volume_optimization_hairpins/%
		volume_optimization_hairpins}
	\caption{Decrease of integral intensity of polyU band at 1231\,\icm{}
		normalized to the integral intensity of cacodylate band at 607\,\icm{}
		which was used as the internal intensity standard. The values were fitted
		by exponential decay curves \eqnref{power_optim:decay} and subtracted
		by the baseline constant $b$ from the fit.}
	\label{\figlabel{vol_optim:hairpins}}
\end{figure}

\begin{table}
	\centering
	\begin{tabular}{cr@{$\,\pm\,$}lr@{$\,\pm\,$}l}
\toprule
c (\g{m}M)
     & \multicolumn{2}{c}{$\tau$\,(min)}
                  & \multicolumn{2}{c}{$E_0$} \\
\midrule

 125 & 10.3 & 0.3 &  1.00 & 0.11 \\
 188 &  8.3 & 0.2 &  0.80 & 0.08 \\
 375 &  8.2 & 0.1 &  0.80 & 0.08 \\
 750 & 10.2 & 0.4 &  0.99 & 0.11 \\
\bottomrule
\end{tabular}

	\caption{Lifetimes $\tau$ of the polyU in dependence on concentration
		$c$. $E_0$ are total energies accumulated by detector divided by maximal
		value across all the concentrations $c$.
	}
	\label{\tablabel{vol_optim:lifetimes_hairpins}}
\end{table}

We can see, that the longest lifetime was achieved at 125\,\g{m}M and
750\,\g{m}M concentrations. The lowest concentration can have longer lifetime
probably because of less absorbed laser power in the sample. The result for
750\,\g{m}M concentratin can be more influenced by the slow decay component as
was shown in the measurements with spinning cell. But overall, the results
supports the hypothesis that with sufficient stirring the sample
photodecomposition is inversely proportional to the number of the illuminated
molecules. We therefore tried to use cuvettes fully filled by samples
(3\,mL in the case of thermostated sample cell and 150\,\g{m}L for the
spinning cell).
