\subsection{Volume optimization}

Results from the laser power and concentration optimization measurements
concluded that the main factor influencing the photodegradation of the sample
is the number of illuminated molecules.
So, we tried to verify the hypothesis by an experiment where we kept the
number of molecules in the sample at the same level.
We varied the concentration and volume of the samples and preserved the amount
of the analyte, concretely we used
	3\,mL of 125\,g{m}M sample,
	2\,mL of 188\,\g{m}M sample,
	1\,mL of 375\,\g{m}M sample and
	0.5\,mL of 750\,g{m}M sample.
The results are displayed in
\figref{vol_optim:hairpins}
and
\tabref{vol_optim:lifetimes_hairpins}.

\begin{figure}
	\centering
	\input{results_and_discussion/assets/volume_optimization_hairpins/%
		volume_optimization_hairpins}
	\caption[%
		Decrease of the integral intensity of the polyU band at 1231\,\icm{}
		for different volumes with a preserved number of molecules in raw spectra
		using backscattering geometry.%
	]{%
		\captiontitle{%
			Decrease of the integral intensity of the polyU band at 1231\,\icm{}
			for different volumes with a preserved number of molecules in raw spectra
			using backscattering geometry.%
		}
		It was normalized to the integral intensity of cacodylate band at
		607\,\icm{}, which was used as the internal intensity standard.
		The values were fitted	by exponential decay curves
		\eqnref{power_optim:decay}.
		The baseline constant $b$ from the fit was subtracted from the plots.
	}
	\label{\figlabel{vol_optim:hairpins}}
\end{figure}

\begin{table}
	\centering
	\begin{tabular}{cr@{$\,\pm\,$}lr@{$\,\pm\,$}l}
\toprule
c (\g{m}M)
     & \multicolumn{2}{c}{$\tau$\,(min)}
                  & \multicolumn{2}{c}{$E_0$} \\
\midrule

 125 & 10.3 & 0.3 &  1.00 & 0.11 \\
 188 &  8.3 & 0.2 &  0.80 & 0.08 \\
 375 &  8.2 & 0.1 &  0.80 & 0.08 \\
 750 & 10.2 & 0.4 &  0.99 & 0.11 \\
\bottomrule
\end{tabular}

	\caption[%
		Lifetimes of the polyU in dependence on concentration
		for different volumes with a preserved number of molecules in raw spectra
		using backscattering geometry.%
	]{%
		\captiontitle{%
			Lifetimes $\tau$ of the polyU in dependence on concentration $c$
			for different volumes with a preserved number of molecules in raw spectra
			using backscattering geometry.%
		}
		$E_0$ are total energies accumulated by the detector divided by the
		maximal value across all the concentrations $c$ and normalized to the
		concentration.
	}
	\label{\tablabel{vol_optim:lifetimes_hairpins}}
\end{table}

We can see that the longest lifetime was achieved at 125\,\g{m}M and
750\,\g{m}M concentrations.
The lowest concentration probably can have a longer lifetime because of less
absorbed laser power in the sample.
The result for 750\,\g{m}M concentration can be more influenced by the slow
decay component, as shown in the spinning cell measurements.
However, overall, the results support the hypothesis that the sample
photodecomposition is inversely proportional to the number of illuminated
molecules with sufficient stirring.
We, therefore, tried to use fully-filled cuvettes by samples (3\,mL in the case
of the thermostated sample cell and 150\,\g{m}L for the spinning cell).
