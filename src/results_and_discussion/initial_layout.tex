\subsection{First layout of the instrument}

At first, we built a proof of concept apparatus to test if we selected the
right optical elements. Optical schema is depicted in
\figref{initial_layout:apparatus_schema}.
The laser beam is emitted with horizontal polarization and we decided, that we
want to guide it through a sample in vertical direction and gather the
scattered light in right angle geometry. Because of the steric restrictions in
our laboratory we needed to place the laser in such a way that it produced
laser beam in the oposit direction then we gathered the scattered light.
Because of that we needed to guide the beam in the loop using laser mirrors
enhanced for 244\,nm \emph{M1} and \emph{M2} (Thorlabs) to maintain the right
polarization direction because we wanted to be able to measure not only
depolarized Raman spectra. The beam was then turned up by another laser mirror
enhanced for 244\,nm \emph{M3} and hit the bottom of the sample cell, having
polarization perpendicular to the direction of scattered signal acquisition.
The excitation beam was focused onto the sample by the plano-convex lens
\emph{L1} with focal length 100\,mm. The most practical place for this lens
was between M2 and M3 so we couldn't select shorter focal length as is proposed
in \cref{subsec:focus_optimization}. The excitation laser beam could be attenuated by
the neutral density filters \emph{ND} with selection of 0.5, 1, 2, 3 and 4
optical densities (Thorlabs) which were placed inside carousel (Thorlabs) for
easy switching. We also old single blade shutter which we found in our
laboratory. The shutter was controlled from the CCD port for external shutter
by the WinSpec software provided with the detector. We needed to insert
condenser into the path because the leading edge of output signal from CCD was
too sharp. The shutter was open during measurement and closed rest of the time
to prevent photodecomposition of samples by the UV laser beam.

\begin{figure}
	\centering
	\begin{tikzpicture}[font=\sffamily]

% settings
\newcommand*{\cellBorderWidth}{3\pgflinewidth}
\newcommand*{\cellLineWidth}{1.5\pgflinewidth}
\definecolor{glassBorderColor}{RGB}{0,128,255}
\definecolor{glassFillColor}{RGB}{230,242,255}
\definecolor{waterFillColor}{RGB}{0,128,255}
\tikzset{
	clip
}
\tikzset{
	mirror element/.style={color=black, line width=2*\pgflinewidth},
	laser beam/.style={color=cyan, line width=2*\pgflinewidth},
	scattered ray/.style={color=red!60, line width=1.5*\pgflinewidth},
	scattered fill/.style={fill=red, draw=none, fill opacity=0.2},
	glass/.style={color=glassBorderColor, opacity=0.5,%
		fill=glassFillColor, fill opacity=0.5},
	sample cell/.style={color=glassBorderColor, opacity=0.5,%
		double=glassFillColor, double distance=\cellBorderWidth,
		line width=\cellLineWidth, line cap=rect},
	water fill/.style={fill=waterFillColor, fill opacity=0.1},
	mirror surface/.style={color=black!20, fill=black!10},
	nd filter/.style={color=black, opacity=0.2, fill=black, fill opacity=0.1},
	nd carousel/.style={color=black, opacity=0.4, fill=black, fill opacity=0.2},
	notch/.style={color=black, opacity=0.4, fill=black, fill opacity=0.1},
	aperture/.style={color=black, line width=2*\pgflinewidth},
	aperture filldraw/.style={color=black!40, fill=black!20, clip even odd rule},
	clip even odd rule/.code={\pgfseteorule},
	shutter/.style={nd carousel},
	shutter blade/.style={dashed, line width = 1.5*\pgflinewidth, opacity = 0.4}
};
\clip (-.1,-2.1) rectangle (14,6.6);
\coordinate (shutter) at (11,6);
\newcommand*{\samplePosWidth}{10}
\newcommand*{\samplePosHeight}{3}
\coordinate (samplePos) at (\samplePosWidth,\samplePosHeight);
\newcommand*{\cellWidth}{1};
\coordinate (cassegrainM1Center) at (\samplePosWidth - 0.5,\samplePosHeight);
\newcommand*{\cassegrainMARadius}{1.5}
\coordinate (cassegrainM2Center) at (\samplePosWidth - 0.7,\samplePosHeight);
\newcommand*{\cassegrainMBRadius}{0.4}
\newcommand*{\sqrttwo}{1.414213}
\coordinate (MS1Edge1) at (4.5,\samplePosHeight + 0.5);
\coordinate (MS1Edge2) at (3.5,\samplePosHeight - 0.5);
\coordinate (parabolaFocus) at (3,0.4);
\coordinate (MS2Edge1) at (4.5,1);
\coordinate (MS2EdgeControl1) at (245:0.5);
\coordinate (MS2Edge2) at (3.5,0);
\coordinate (MS2EdgeControl2) at (17:0.3);
\newcommand*{\apertureOuterRadius}{0.3}
\newcommand*{\apertureInnerRadius}{0.1}

% laser
\draw (0,5.5) rectangle ++(3,1) node[pos=.5] {laser};

% Mirror 3 - it should be under the beam so we have to draw it first
\draw[mirror surface] ($ (samplePos) + (-0.3,-\sqrttwo * 0.3) $)
	rectangle ++(0.6,\sqrttwo*0.6);
\node[above] at ($ (samplePos) + (0,0.5) $) {M3};

% laser beam
\draw[laser beam] (3,6) -- (13,6) coordinate (M1)
	-- (13,\samplePosHeight) coordinate (M2) -- (samplePos);

% aperture 2
\draw[aperture filldraw]
	(samplePos) circle (\apertureOuterRadius)
	(samplePos) circle (\apertureInnerRadius);

% neutral density filters
\newcommand*{\ndfilterA}{(3.5,5.8) rectangle ++(0.2,0.4)}
\newcommand*{\ndfilterB}{(3.5,5) rectangle ++(0.2,0.4)}
\draw[nd filter] \ndfilterA;
\draw[nd filter] \ndfilterB;
\draw[nd carousel] (3.45,4.9) rectangle ++(0.3,0.1);
\draw[nd carousel] (3.45,5.4) rectangle ++(0.3,0.4);
\draw[nd carousel] (3.45,6.2) rectangle ++(0.3,0.1);
\node[below] at (3.6,4.9) {ND};

\draw[shutter blade] ($ (shutter) + (0,0.3) $) -- ++(0,-0.6);
\draw[shutter] ($ (shutter) + (-0.1,-0.3) $) -- ++(0.2,0) -- ++(0,-0.4)
	-- ++(-0.2,0) -- cycle;
\node[below] at ($ (shutter) + (0,-0.7) $) {shutter};

% Mirror 1 and 2
\draw[mirror element] ($ (M1) + (-0.3,0.3) $)
	-- node[above,shift={(0.35cm,-0.05cm)}]{M1} ($ (M1) + (0.3,-0.3) $);
\draw[mirror element] ($ (M2) + (0.3,0.3) $)
	-- node[below,shift={(0.35cm,0.05cm)}]{M2} ($ (M2) + (-0.3,-0.3) $);

% Aperture 1
\draw[aperture] ($ (M2) + (-0.5,\apertureOuterRadius) $)
	node[above]{A1}
	-- ++(0,-\apertureOuterRadius + \apertureInnerRadius);
\draw[aperture] ($ (M2) + (-0.5,-\apertureOuterRadius) $)
	-- ++(0,\apertureOuterRadius - \apertureInnerRadius);

% Draw laser focusing lens
\newcommand*{\LARadius}{0.7}
\coordinate (L1Center) at (\samplePosWidth+1.6-\LARadius,\samplePosHeight);
\path[name path=L1Arc, shift={(L1Center)}]
	(270:\LARadius) arc (-90:90:\LARadius);
\path[name path=toL1Arc1] ($ (samplePos)  + (0,0.4) $) -- ++(10,0);
\path[name path=toL1Arc2] ($ (samplePos)  + (0,-0.4) $) -- ++(10,0);
\path[name intersections={of=L1Arc and toL1Arc1, by=L11}];
\mypgfextractangle{\LAAAngle}{L1Center}{L11}
\path[name intersections={of=L1Arc and toL1Arc2, by=L12}];
\mypgfextractangle{\LABAngle}{L1Center}{L12}
\draw[glass] (L11) arc (\LAAAngle:\LABAngle-360:\LARadius) -- ++(-0.05,0)
	-- ($ (L11) + (-0.05,0) $) -- cycle;
\node[above] at (L11) {L1};


%%%%%%%%%%%%%%%%%%%%%
% draw the cassegrain

% calculate intersections with mirror 1 (the objective mirror)
% mirror1 arc
\path[name path=M1arc, shift={(cassegrainM1Center)}]
	(90:\cassegrainMARadius) arc (90:270:\cassegrainMARadius);
% upper top ray
\path[name path=toCassegrainM11] (samplePos) -- ++(135:5);
\path[name intersections={of=M1arc and toCassegrainM11, by=cassegrainM11}];
\mypgfextractangle{\cassegrainMAAAngle}{cassegrainM1Center}{cassegrainM11}
% upper bottom ray
\path[name path=toCassegrainM12] (samplePos) -- ++(165:5);
\path[name intersections={of=M1arc and toCassegrainM12, by=cassegrainM12}];
\mypgfextractangle{\cassegrainMABAngle}{cassegrainM1Center}{cassegrainM12}
% lower top ray
\path[name path=toCassegrainM13] (samplePos) -- ++(195:5);
\path[name intersections={of=M1arc and toCassegrainM13, by=cassegrainM13}];
\mypgfextractangle{\cassegrainMACAngle}{cassegrainM1Center}{cassegrainM13}
% lower bottom ray
\path[name path=toCassegrainM14] (samplePos) -- ++(225:5);
\path[name intersections={of=M1arc and toCassegrainM14, by=cassegrainM14}];
\mypgfextractangle{\cassegrainMADAngle}{cassegrainM1Center}{cassegrainM14}

\draw[scattered ray] (samplePos) -- (cassegrainM11);
\draw[scattered ray] (samplePos) -- (cassegrainM12);
\draw[scattered fill] (samplePos) -- (cassegrainM11)
	arc (\cassegrainMAAAngle:\cassegrainMABAngle:\cassegrainMARadius) -- cycle;
\draw[scattered ray] (samplePos) -- (cassegrainM13);
\draw[scattered ray] (samplePos) -- (cassegrainM14);
\draw[scattered fill] (samplePos) -- (cassegrainM13)
	arc (\cassegrainMACAngle:\cassegrainMADAngle:\cassegrainMARadius) -- cycle;

% draw the cell
\draw[sample cell, water fill]
	($ (samplePos)%
		+ (-0.5 * \cellBorderWidth - \cellLineWidth,-0.5 * \cellWidth) $)
		rectangle ++(\cellWidth,\cellWidth);
\node[below] at ($ (samplePos) + (0.5 * \cellWidth,-0.5 * \cellWidth)%
	+ (-0.5 * \cellBorderWidth,0) + (-0.5 * \cellLineWidth,0) $) {S};

% calculate intersections with mirror 2 (the objective mirror)
% mirror2 arc
\path[
	name path=M2arc, shift={(cassegrainM2Center)}] (90:\cassegrainMBRadius)
		arc (90:270:\cassegrainMBRadius);
% calculate cassegrain mirror2 edges
\path[
	name intersections={of=M2arc and toCassegrainM12, by=cassegrainM2Edge1}];
\mypgfextractangle{\cassegrainMBAAngle}{cassegrainM2Center}{cassegrainM2Edge1}
\path[
	name intersections={of=M2arc and toCassegrainM13, by=cassegrainM2Edge2}];
\mypgfextractangle{\cassegrainMBDAngle}{cassegrainM2Center}{cassegrainM2Edge2}
% upper bottom ray
\path[name path=toCassegrainM22] ($ (samplePos) + (0,0.1) $) -- ++(-10,0);
\path[name intersections={of=M2arc and toCassegrainM22, by=cassegrainM22}];
\mypgfextractangle{\cassegrainMBBAngle}{cassegrainM2Center}{cassegrainM22}
% lower top ray
\path[name path=toCassegrainM23] ($ (samplePos) + (0,-0.1) $) -- ++(-10,0);
\path[name intersections={of=M2arc and toCassegrainM23, by=cassegrainM23}];
\mypgfextractangle{\cassegrainMBCAngle}{cassegrainM2Center}{cassegrainM23}

\draw[scattered ray] (cassegrainM11) -- (cassegrainM2Edge1);
\draw[scattered ray] (cassegrainM12) -- (cassegrainM22);
\draw[scattered fill] (cassegrainM2Edge1)
	arc (\cassegrainMBAAngle:\cassegrainMBBAngle:\cassegrainMBRadius)
		-- (cassegrainM12)
	arc (\cassegrainMABAngle:\cassegrainMAAAngle:\cassegrainMARadius) -- cycle;
\draw[scattered ray] (cassegrainM13) -- (cassegrainM23);
\draw[scattered ray] (cassegrainM14) -- (cassegrainM2Edge2);
\draw[scattered fill] (cassegrainM23)
	arc (\cassegrainMBCAngle:\cassegrainMBDAngle:\cassegrainMBRadius)
		-- (cassegrainM14)
	arc (\cassegrainMADAngle:\cassegrainMACAngle:\cassegrainMARadius) -- cycle;

% to mirror MS1
% path representing the mirror
\path[name path=MS1Path] (MS1Edge1) -- (MS1Edge2);
% intersections with the mirror
% ray 1
\path[name path=toCassegrainM21] (cassegrainM2Edge1) -- ++(-10,0);
\path[name intersections={of=MS1Path and toCassegrainM21, by=MS11}];
% ray 2
\path[name intersections={of=MS1Path and toCassegrainM22, by=MS12}];
% ray 3
\path[name intersections={of=MS1Path and toCassegrainM23, by=MS13}];
% ray 4
\path[name path=toCassegrainM24] (cassegrainM2Edge2) -- ++(-10,0);
\path[name intersections={of=MS1Path and toCassegrainM24, by=MS14}];

\draw[scattered ray] (cassegrainM2Edge1) -- (MS11);
\draw[scattered ray] (cassegrainM22) -- (MS12);
\draw[scattered fill] (cassegrainM2Edge1)
	arc (\cassegrainMBAAngle:\cassegrainMBBAngle:\cassegrainMBRadius) -- (MS12)
		-- (MS11) -- cycle;
\draw[scattered ray] (cassegrainM23) -- (MS13);
\draw[scattered ray] (cassegrainM2Edge2) -- (MS14);
\draw[scattered fill] (cassegrainM23)
	arc (\cassegrainMBCAngle:\cassegrainMBDAngle:\cassegrainMBRadius) -- (MS14)
		-- (MS13) -- cycle;

% draw first mirror of cassegrain
\draw[mirror element]
	(cassegrainM11)
		arc (\cassegrainMAAAngle:\cassegrainMABAngle:\cassegrainMARadius)
			node[left,pos=0.5] {O};
\draw[mirror element]
	(cassegrainM13)
		arc (\cassegrainMACAngle:\cassegrainMADAngle:\cassegrainMARadius);
% mirror 2
\draw[mirror element]
	(cassegrainM2Edge1)
		arc (\cassegrainMBAAngle:\cassegrainMBDAngle:\cassegrainMBRadius);

% draw notch
\draw[notch] ($ (samplePos) + (-3,-0.5) $) rectangle ++(0.2,1);
\node[above] at ($ (samplePos) + (-2.9,0.5) $) {EF};

% Parabolic mirror MS2
\newcommand*{\parabolicMirrorDef}{%
		(MS2Edge2)
		.. controls ($ (MS2Edge2) + (MS2EdgeControl2) $)
			and ($ (MS2Edge1) + (MS2EdgeControl1) $)
		.. (MS2Edge1)
}
\path[name path=MS2Path] \parabolicMirrorDef;
% ray 1
\path[name path=toMS21] (MS11) -- ++(0,-10);
\path[name intersections={of=MS2Path and toMS21, by=MS21}];
% ray 2
\path[name path=toMS22] (MS12) -- ++(0,-10);
\path[name intersections={of=MS2Path and toMS22, by=MS22}];
% ray 3
\path[name path=toMS23] (MS13) -- ++(0,-10);
\path[name intersections={of=MS2Path and toMS23, by=MS23}];
% ray 4
\path[name path=toMS24] (MS14) -- ++(0,-10);
\path[name intersections={of=MS2Path and toMS24, by=MS24}];

\draw[scattered ray] (MS11) -- (MS21);
\draw[scattered ray] (MS12) -- (MS22);
\begin{scope}
	\clip (MS11) -- ($ (MS21) + (0,-1) $) -- ($ (MS22) + (0,-1) $) -- (MS12)
		-- cycle;
	\draw[scattered fill] (MS12) -- \parabolicMirrorDef -- (MS11) -- cycle;
\end{scope}
\draw[scattered ray] (MS13) -- (MS23);
\draw[scattered ray] (MS14) -- (MS24);
\begin{scope}
	\clip (MS13) -- ($ (MS23) + (0,-1) $) -- ($ (MS24) + (0,-1) $) -- (MS14)
		-- cycle;
	\draw[scattered fill] (MS14) -- \parabolicMirrorDef -- (MS13) -- cycle;
\end{scope}

% draw MS1 over all incident rays on that mirror
\draw[mirror element]
	(MS1Edge1) -- node[above,shift={(-0.5cm,-0.1cm)}]{MS1} (MS1Edge2);

\draw[scattered ray] (MS21) -- (parabolaFocus);
\draw[scattered ray] (MS22) -- (parabolaFocus);
\begin{scope}
	\clip (parabolaFocus) -- (MS21) -- ++(1,0) -- ($ (MS22) + (1,0) $) -- (MS22)
		-- cycle;
	\draw[scattered fill] (parabolaFocus) -- \parabolicMirrorDef -- cycle;
\end{scope}
\draw[scattered ray] (MS23) -- (parabolaFocus);
\draw[scattered ray] (MS24) -- (parabolaFocus);
\begin{scope}
	\clip (parabolaFocus) -- (MS23) -- ++(1,0) -- ($ (MS24) + (1,0) $) -- (MS24)
		-- cycle;
	\draw[scattered fill] (parabolaFocus) -- \parabolicMirrorDef -- cycle;
\end{scope}

% draw the parabolic mirror MS2
\draw[mirror element]
	(MS2Edge2)
	.. controls ($ (MS2Edge2) + (MS2EdgeControl2) $)
		and ($ (MS2Edge1) + (MS2EdgeControl1) $)
	.. node[below,shift={(0.5cm,0.1cm)}]{MS2} (MS2Edge1);

\draw (0,0) rectangle ++(3,2.5) node[pos=.5] {spectrograph};

% side view
\begin{scope}[shift={(\samplePosWidth - 2,-1.7)}]

\newcommand*{\samplePosSideWidth}{2}
\newcommand*{\samplePosSideHeight}{2.5}
\coordinate (samplePosSide) at (\samplePosSideWidth,\samplePosSideHeight);
\coordinate (cassegrainM1SideCenter)
	at (\samplePosSideWidth - 0.5,\samplePosSideHeight);
\coordinate (cassegrainM2SideCenter)
	at (\samplePosSideWidth - 0.7,\samplePosSideHeight);

% clip the view
\clip (-.05,-.35) rectangle ++(3.3,3.4);
\draw (-.05,-.35) rectangle ++(3.3,3.4);

% laser beam
\draw[laser beam] (4,0) -- (\samplePosSideWidth,0) coordinate (M3Side)
	-- (samplePosSide);

% mirror 3
\draw[mirror element] ($ (M3Side) + (-0.3,0.3) $)
	-- node[above,shift={(0.35cm,-0.05cm)}]{M3} ($ (M3Side) + (0.3,-0.3) $);

% aperture 2
\draw[aperture] ($ (M3Side) + (\apertureOuterRadius,0.8) $)
	node[above, shift={(0.05,0)}]{A2}
	-- ++(-\apertureOuterRadius + \apertureInnerRadius,0);
\draw[aperture] ($ (M3Side) + (-\apertureOuterRadius,0.8) $)
	-- ++(\apertureOuterRadius - \apertureInnerRadius,0);

% draw the cassegrain
% calculate intersections with mirror 1 (the objective mirror)
% mirror1 arc
\path[name path=M1SideArc, shift={(cassegrainM1SideCenter)}]
	(90:\cassegrainMARadius) arc (90:270:\cassegrainMARadius);
% upper top ray
\path[name path=toCassegrainM11Side] (samplePosSide) -- ++(135:5);
\path[name intersections={of=M1SideArc and toCassegrainM11Side,
	by=cassegrainM11Side}];
% upper bottom ray
\path[name path=toCassegrainM12Side] (samplePosSide) -- ++(165:5);
\path[name intersections={of=M1SideArc and toCassegrainM12Side,
	by=cassegrainM12Side}];
% lower top ray
\path[name path=toCassegrainM13Side] (samplePosSide) -- ++(195:5);
\path[name intersections={of=M1SideArc and toCassegrainM13Side,
	by=cassegrainM13Side}];
% lower bottom ray
\path[name path=toCassegrainM14Side] (samplePosSide) -- ++(225:5);
\path[name intersections={of=M1SideArc and toCassegrainM14Side,
	by=cassegrainM14Side}];

\draw[scattered ray] (samplePosSide) -- (cassegrainM11Side);
\draw[scattered ray] (samplePosSide) -- (cassegrainM12Side);
\draw[scattered fill] (samplePosSide) -- (cassegrainM11Side)
	arc (\cassegrainMAAAngle:\cassegrainMABAngle:\cassegrainMARadius) -- cycle;
\draw[scattered ray] (samplePosSide) -- (cassegrainM13Side);
\draw[scattered ray] (samplePosSide) -- (cassegrainM14Side);
\draw[scattered fill] (samplePosSide) -- (cassegrainM13Side)
	arc (\cassegrainMACAngle:\cassegrainMADAngle:\cassegrainMARadius) -- cycle;

% draw the cell
\draw[sample cell, water fill]
	($ (samplePosSide) + (-0.5 * \cellBorderWidth - \cellLineWidth,%
		-0.5 * \cellBorderWidth - \cellLineWidth) $)
		rectangle ++(\cellWidth,10);

% calculate intersections with mirror 2 (the objective mirror)
% mirror2 arc
\path[
	name path=M2SideArc, shift={(cassegrainM2SideCenter)}] (90:\cassegrainMBRadius)
		arc (90:270:\cassegrainMBRadius);
% calculate cassegrain mirror2 edges
\path[
	name intersections={%
		of=M2SideArc and toCassegrainM12Side, by=cassegrainM2SideEdge1}];
\path[
	name intersections={%
		of=M2SideArc and toCassegrainM13Side, by=cassegrainM2SideEdge2}];
% upper bottom ray
\path[name path=toCassegrainM22Side]%
	($ (samplePosSide) + (0,0.1) $) -- ++(-10,0);
\path[name intersections={%
	of=M2SideArc and toCassegrainM22Side, by=cassegrainM22Side}];
% lower top ray
\path[name path=toCassegrainM23Side]
	($ (samplePosSide) + (0,-0.1) $) -- ++(-10,0);
\path[name intersections={%
	of=M2SideArc and toCassegrainM23Side, by=cassegrainM23Side}];

\draw[scattered ray] (cassegrainM11Side) -- (cassegrainM2SideEdge1);
\draw[scattered ray] (cassegrainM12Side) -- (cassegrainM22Side);
\draw[scattered fill] (cassegrainM2SideEdge1)
	arc (\cassegrainMBAAngle:\cassegrainMBBAngle:\cassegrainMBRadius)
		-- (cassegrainM12Side)
	arc (\cassegrainMABAngle:\cassegrainMAAAngle:\cassegrainMARadius) -- cycle;
\draw[scattered ray] (cassegrainM13Side) -- (cassegrainM23Side);
\draw[scattered ray] (cassegrainM14Side) -- (cassegrainM2SideEdge2);
\draw[scattered fill] (cassegrainM23Side)
	arc (\cassegrainMBCAngle:\cassegrainMBDAngle:\cassegrainMBRadius)
		-- (cassegrainM14Side)
	arc (\cassegrainMADAngle:\cassegrainMACAngle:\cassegrainMARadius) -- cycle;

% to mirror MS1
% ray 1
\draw[scattered ray] (cassegrainM2SideEdge1) -- ++(-10,0);
\draw[scattered ray] (cassegrainM22Side) -- ++(-10,0);
\draw[scattered fill] (cassegrainM2SideEdge1)
	arc (\cassegrainMBAAngle:\cassegrainMBBAngle:\cassegrainMBRadius)
		-- ++(-10,0) -- ($ (cassegrainM2SideEdge1) + (-10,0) $) -- cycle;
\draw[scattered ray] (cassegrainM23Side) -- ++(-10,0);
\draw[scattered ray] (cassegrainM2SideEdge2) -- ++(-10,0);
\draw[scattered fill] (cassegrainM23Side)
	arc (\cassegrainMBCAngle:\cassegrainMBDAngle:\cassegrainMBRadius) --
		++(-10,0) -- ($ (cassegrainM2SideEdge1) + (-10,0) $) -- cycle;

% draw first mirror of cassegrain
\draw[mirror element]
	(cassegrainM11Side)
		arc (\cassegrainMAAAngle:\cassegrainMABAngle:\cassegrainMARadius)
			node[left,pos=0.5] {O};
\draw[mirror element]
	(cassegrainM13Side)
		arc (\cassegrainMACAngle:\cassegrainMADAngle:\cassegrainMARadius);
% mirror 2
\draw[mirror element]
	(cassegrainM2SideEdge1)
		arc (\cassegrainMBAAngle:\cassegrainMBDAngle:\cassegrainMBRadius);

\end{scope}

\end{tikzpicture}
	\caption{Top-view schema of the apparatus with side-view inset of the sample
		space. A laser beam is emitted with horizontal polarization. It is guided
		through carousel with neutral density (ND) filters and shutter and by laser
		mirrors M1 -- M3 and laser focusing lens (L1) to the sample cell (S)
		through apertures A1 and A2. The scattered light is gathered by Cassegrain
		objective (O) and reflected by the mirror MS1 and focused on the entrance
		slit of spectrograph by parabolic mirror MS2. Edge filter (EF) suppresses
		elastically scattered light.}
	\label{\figlabel{initial_layout:apparatus_schema}}
\end{figure}

For proper laser beam path setup we needed visualize the beam because the
wavelengths in use are outside of visible light range. Business cards showed
to be the best equipment for the beam visualization because the paper used for
them usually has strong fluorescence. Excitation beam adjustment procedure
uses 2 apertures \emph{A1}, \emph{A2} and cross hair reticle printed on the
paper fixed to the ceiling of the room. At first, the kinematic holder of
mirror M1 is adjusted so the beam passes through aperture A1. If the beam is
blocked by some of the other optical elements on it's way to the ceiling
reticle, we can use business card to detect if it passes through the center
of the aperture A1. Then the mirror M2 is adjusted for the beam to pass
through A2 and finally the mirror M3 is adjusted to hit the center of the
reticle. The procedure can be iteratively repeated if the result is not
satisfactory but usually the good alignment was achieved after the first
iteration.

Raman scattered light is no longer monochromatic so we decided to use
reflective optics in the light gathering part of the apparatus to avoid
chromatic aberration. We selected Cassegrain objective \emph{O} (Newport) with
$f_o = 13$\,mm effective front focal length, back focal length in infinity,
0.4 numerical aperture and 9.72\,mm diameter of output beam. The elastically
scattered light is removed by edge filter \emph{EF} (Semrock). The rest of the
collimated signal is then guided by UV enhanced broadband aluminum mirror
\emph{MS1} (Newport) to focusing off-axis parabolic mirror \emph{MS2}
(Newport) with $f_c = 101.6$\,mm effective focal length to the spectrograph
slit. The spectrograph has f/6.4 aperture whereas the focusing lens with the
beam diameter determined by objective has aperture $101.6 / 9.72 = f/10.5$
which means that we slightly under fill the spectrometer's angle of acceptance.

(\INCONSISTENCY{} - this means that the spectrograph has f/10.5 in reality so
this value should be probably used in \cref{subsec:focus_optimization} about
focusing lens focal length estimation)
