\subsection{Length of accumulation}

A Raman spectrum is usually acquired as a consecutive series of spectra,
further called \emph{frames}, which are taken at the same experimental
conditions and with the constant accumulation time. This approach has many
reasons. Amongst the most important is that measurement of Raman spectra
requires highly sensitive detectors which means that you easily reach the
detector saturation with stronger signal.

Secondly, such a sensitive detectors are susceptible to the cosmic ray
artifacts which can significantly damage the spectra. This cosmic ray signal
can be much easily subtracted in the series of spectra taken with the same
measurement condition then from single spectrum because the subsequent
spectra should be almost identical and you can replace the spectral regions
with cosmic ray signal by the average of the surrounding spectra.

Furthermore, other temporal effects can affect the spectra, like small
temperature variations, mechanical movements etc.

RRS is also affected by photodecomposition and increasing presence of
photoproducts which can directly or indirectly influence the measured spectra.
Direct presence of signal of photoproducts in the detected spectra is usually
negligible because Raman scattering of the photoproducts isn't usually
resonantly enhanced but indirect effects through chemical interactions of
photoproducts with the system under investigation are always possible.

So, obviously, measurement of more frames with shorter accumulation time would
be favored for better understanding of all these processes during accumulation
of the spectrum.

But there are also opposite effects that support longer measurements. The most
dominant is a ratio between signal of the sample and noise. The noise in the
spectra can have a number of origins but one of the dominant ones is the one
connected with accquiring the signal on the CCD detector. The first one is low
frequency (almost constant) background which is usually used in
\emph{signal-to-noise} (SNR) calculation. Best results with the most linear
response to the intensity of the gathered light are achieved with a signal in
the range $\approx 10 - 80\,\%$ of the maximal signal limit.

The second component is high frequency noise which lowers the quality of the
spectra, reliability of the band position detection or can even hide some
spectral features completely.

We tried to estimate some guiding principle about the balance between a number
of frames and an accumulation time. We measured background spectrum of
deionized water with high laser power (100\,mW) which meant short accumulation
times (5\,s) taken in 100 frames. All these frames were then averaged to obtain
one high quality Raman spectrum of water $I_\text{water}(\wn)$ as a reference.
Then we measured the same water with 1\,mW excitation in 30 frames with 20\,s
accumulation time and 10 frames with 60\,s accumulation time which are
parameters of our typical RRS measurement. We summed all the frames in these
two measurements to obtain single Raman spectrum $I_{20}(\wn)$ and
$I_{60}(\wn)$ respectively which represented 10\,min total accumulation.

We decided to assess the SNR to estimate the quality of the measured spectra.
A noise height was estimated by a fit of the high quality water spectrum
$I_\text{water}(\wn)$ plus constant background which resulted in the model
intensity function
\begin{equation}
	I_\text{model}(\wn) = a_0 I_\text{water}(\wn) + a_1.
	\label{\eqnlabel{accum_length:sn_model}}
\end{equation}
An example fit can be seen in \figref{accum_length:sn_ratio}.

\begin{figure}
	\centering
	\input{results_and_discussion/assets/sn_ratio/sn_ratio}
	\caption{Signal-to-noise calculation from a fit of high quality spectrum
		and constant background addition. The original \emph{spectrum} (here we are
		using the spectrum $I_20(\wn)$ with 20\,s accumulation time) was modeled
		according to \emph{model} \eqnref{accum_length:sn_model}. The spectra on
		the image were then normalized to the \emph{constant} $a_1$.}
	\label{\figlabel{accum_length:sn_ratio}}
\end{figure}

The low frequency SNR was then estimated as
\begin{equation*}
	SNR_\text{low} = \frac{I_\text{model}(1637)}{a_1}.
\end{equation*}
which leads to
\begin{align*}
	SNR_{\text{low},20} &= 1.3841 \pm 0.0004 \\
	SNR_{\text{low},60} &= 2.6112 \pm 0.0026
\end{align*}

We also estimated the high frequency noise. We subtracted the constant $a_1$
from the measured spectra and normalized them to the maximum of the
$I_\text{model}(1637\,\text{cm}^{-1})$
so that we have a comparison of the noise size to the size of this band. We
than calculated spectrum $I_\text{SG}$ smoothed by Savitzky-Golay filter
\CITATION,
see
\figref{accum_length:sn_ratio_sg}.

\begin{figure}
	\centering
	\input{results_and_discussion/assets/sn_ratio/sn_ratio_sg}
	\caption{SN ratio SG}
	\label{\figlabel{accum_length:sn_ratio_sg}}
\end{figure}

Finally, we estimated the high frequency $SNR_\text{high}$ from the standard
deviation $\sigma$ of the original spectrum from the smoothed spectrum
normalized to intensity if water band at 1637\,\icm{} (which means that the
normalized intensity maximum at 1637\,\icm{} is equal to 1)
\begin{equation*}
	SNR_\text{high} = \frac{1}{\sigma}.
\end{equation*}

\begin{align*}
	SNR_{\text{high},20} &= 240 \\
	SNR_{\text{high},60} &= 297
\end{align*}

It can be clearly seen that longer accumulation times of the single frame gives
better quality spectra in these experimental conditions. It is more obvious
from the low frequency SNR even though the high frequency SNR difference is
also significant. Regarding the disadvantages of the
longer accumulation discussed at the beginning of this section and speed of
the sample degradation with lifetimes in the range of minutes we decided that
the optimal accumulation time is 60\,s.
