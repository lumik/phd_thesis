\subsection{Spinning cell rotation speed optimization}

All the previously described experiments used the maximal rotation speed of
the spinning cell.
Next question was if slower rotation couldn't produce better results.
Slower rotation could provide more time for reversible changes and damaged
sample diffusion from the focus.
Faster rotation can also produce unwanted vibrations in the spectrometer.
We used the same experimental setup as in previous measurements with 5\,mW of
excitation power and 500\,mM polyU samples.
We varied the rotation speed where we tried to measure the sample at slow
rotation of 4\,Hz and full rotation of 160\,Hz.
The time needed to adjust the samples was 100\,s for slow rotation case and
30\,s for fast rotation one.
We analyzed decay curves using the same methods as in
\cref{subsec:power_optim}
where fast decay curve was fitted to first 15 and 25 frames for the slow and
fast rotation respectively and slow decay curve was fitted for the rest of
the spectrum.
Results of the fit can be seen in
\figref{rotation_optim:fast_decay},
\figref{rotation_optim:slow_decay}
and
\tabref{rotation_optim:lifetimes}.

\begin{figure}
	\centering
	\input{results_and_discussion/assets/rotation_optimization/%
		rotation_optimization}
	\caption{Decrease of integral intensity of polyU band at 1231\,\icm{}
		normalized to the integral intensity of cacodylate band at 607\,\icm{}
		which was used as the internal intensity standard. The values were fitted
		by exponential decay curves \eqnref{power_optim:decay} and subtracted
		by the baseline constant $b$ from the fit.}
	\label{\figlabel{rotation_optim:fast_decay}}
\end{figure}

\begin{figure}
	\centering
	\input{results_and_discussion/assets/rotation_optimization/%
		rotation_optimization_kor}
	\caption{Decrease of integral intensity of polyU band at 1231\,\icm{}
		normalized to the subtracted spectrum of cacodylate buffer which was used
		as the internal intensity standard. The values were fitted by exponential
		decay curves \eqnref{power_optim:decay} and subtracted by the baseline
		constant $b$ from the fit.}
	\label{\figlabel{rotation_optim:slow_decay}}
\end{figure}

\begin{table}
	\centering
	\begin{tabular}{ccr@{$\,\pm\,$}lr@{$\,\pm\,$}lr@{$\,\pm\,$}lr@{$\,\pm\,$}l}
\toprule
f (Hz)
    & decay
		       &\multicolumn{2}{c}{$\tau$\,(min)}
                        & \multicolumn{2}{c}{$E_0$}
                                       & \multicolumn{2}{c}{$E$}
                                                      & \multicolumn{2}{c}{$r$} \\
\midrule

  4 & fast &  2.2 & 0.3 &  0.26 & 0.05 &  0.17 & 0.05 &  0.64 & 0.10 \\
120 & fast &  2.5 & 0.2 &  1.00 & 0.10 &  0.67 & 0.12 &  0.67 & 0.09 \\
  4 & slow & 24.3 & 1.4 &  0.96 & 0.07 &  0.91 & 0.07 &  0.96 & 0.01 \\
120 & slow & 22.0 & 2.0 &  1.00 & 0.12 &  0.96 & 0.12 &  0.96 & 0.01 \\
\bottomrule
\end{tabular}

	\caption{Lifetimes $\tau$ of slow and fast decay components of the polyU in
		dependence on rotation frequency $f$. $E_0$ are total energies accumulated
		by detector divided by maximal value accross all the excitation powers $P$
		and $E$ are energies accumulated from the time $T = 60\pm20$\,s which was
		needed for the adjustment of the samples before the acquisition can even
		start but the sample needs to be irradiated by the excitation laser. The
		last column contains fractions of the samples $r$ which were not destroyed
		by photodecomposition after the time $T$.
	}
	\label{\tablabel{rotation_optim:lifetimes}}
\end{table}

Even though the results in this experiment are internaly consistent, we once
more see different lifetimes in comparison with the results in
\tabref{power_optim:lifetimes_triplexes}
and
\tabref{power_optim:lifetimes_triplexes}.
This can be caused by the different laser focus setup for the experiments.

The effect of fast rotation is visible only in the fast decay curves.
Analysis of the results of fits displayed in
\tabref{rotation_optim:lifetimes}
shows that the speed of rotation doesn't affect lifetimes of slow nor fast
decay components but it significantly affects the Raman signal intensity
gathered from the fast decay component.
It means that the numbers of damaged molecules are the same for both rotation
speed scenarios because the lifetimes are the same but more undamaged molecules
are available for the measurement with fast rotation speed.
It also means that the speed of rotation has lower impact on the measurements
with lower excitation laser power where the contribution of the fast decay
component to the resulting spectra is weaker.

Overall, the fast rotation seems to be better than slower rotation with
stronger signal and no impact on the sample lifetime.
Therefore we decided to use the maximal rotation speed in all the experiments
which used the spinning sample cell.