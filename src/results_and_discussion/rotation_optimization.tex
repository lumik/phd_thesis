\subsection{Optimization of spinning cell rotation speed}

All the previously described experiments used the maximal rotation speed of
the spinning cell.
The next question was if slower rotation could not produce better results.
It could, for example, provide more time for reversible changes and damaged
sample diffusion from the focus.
Faster rotation can also produce unwanted vibrations in the spectrometer.
We used the same experimental setup as previous measurements with 5\,mW of
excitation power and 500\,mM polyU samples.
We varied the rotation speed where we tried to measure the sample at the slow
rotation of 4\,Hz and the full rotation of 160\,Hz.
The time needed to adjust the samples was 100\,s for slow rotation case and
30\,s for fast rotation one.
We analyzed decay curves using the same methods as in
\cref{subsec:power_optim},
where the fast decay curve was fitted to the first 15 and 25 frames for the
slow and fast rotation respectively.
The slow decay curve was fitted for the rest of the spectrum.
Results of the fit can be seen in
\cref{%
	\figlabel{rotation_optim:fast_decay},%
	\figlabel{rotation_optim:slow_decay}%
}
and
\tabref{rotation_optim:lifetimes}.

\begin{figure}
	\centering
	\input{results_and_discussion/assets/rotation_optimization/%
		rotation_optimization}
	\caption[%
		Decrease of the integral intensity of the polyU band at 1231\,\icm{} for
		different sample cell rotation speeds in raw spectra.%
	]{%
		\captiontitle{%
			Decrease of the integral intensity of the polyU band at 1231\,\icm{} for
			different sample cell rotation speeds in raw spectra.%
		}
		It was normalized to the integral intensity of the cacodylate band at
		607\,\icm{}, which was used as the internal intensity standard.
		The values were fitted by exponential decay curves
		\eqnref{power_optim:decay}.
		The baseline constant $b$ from the fit was subtracted from the plots.
	}
	\label{\figlabel{rotation_optim:fast_decay}}
\end{figure}

\begin{figure}
	\centering
	\input{results_and_discussion/assets/rotation_optimization/%
		rotation_optimization_kor}
	\caption[%
		Decrease of the integral intensity of the polyU band at 1231\,\icm{}
		for different sample cell rotation speeds in background-corrected spectra.%
	]{%
		\captiontitle{%
			Decrease of the integral intensity of polyU band at 1231\,\icm{}
			for different sample cell rotation speeds in background-corrected
			spectra.%
		}
		The intensity was normalized to the subtracted spectrum of cacodylate
		buffer, which was used as the internal intensity standard.
		The values were fitted by exponential decay curves
		\eqnref{power_optim:decay}.
		The baseline constant $b$ from the fit was subtracted from the plots.
	}
	\label{\figlabel{rotation_optim:slow_decay}}
\end{figure}

\begin{table}
	\centering
	\begin{tabular}{ccr@{$\,\pm\,$}lr@{$\,\pm\,$}lr@{$\,\pm\,$}lr@{$\,\pm\,$}l}
\toprule
f (Hz)
    & decay
		       &\multicolumn{2}{c}{$\tau$\,(min)}
                        & \multicolumn{2}{c}{$E_0$}
                                       & \multicolumn{2}{c}{$E$}
                                                      & \multicolumn{2}{c}{$r$} \\
\midrule

  4 & fast &  2.2 & 0.3 &  0.26 & 0.05 &  0.17 & 0.05 &  0.64 & 0.10 \\
120 & fast &  2.5 & 0.2 &  1.00 & 0.10 &  0.67 & 0.12 &  0.67 & 0.09 \\
  4 & slow & 24.3 & 1.4 &  0.96 & 0.07 &  0.91 & 0.07 &  0.96 & 0.01 \\
120 & slow & 22.0 & 2.0 &  1.00 & 0.12 &  0.96 & 0.12 &  0.96 & 0.01 \\
\bottomrule
\end{tabular}

	\caption[%
		Lifetimes of slow and fast decay components of the polyU in
		dependence on rotation frequency.%
	]{%
		\captiontitle{%
			Lifetimes $\tau$ of slow and fast decay components of the polyU in
			dependence on rotation frequency $f$.%
		}
		$E_0$ are total energies accumulated by the detector divided by the maximal
		value across all the excitation powers $P$, and $E$ are energies
		accumulated from the time $T = 60\pm20$\,s needed for the adjustment of the
		samples before the acquisition can even start, but the sample needs to be
		irradiated by the excitation laser.
		The last column contains fractions of the samples $r$ that were not
		destroyed by photodecomposition after the time $T$.
	}
	\label{\tablabel{rotation_optim:lifetimes}}
\end{table}

Even though the results in this experiment are internally consistent, we once
more see different lifetimes compared to the results in
\tabref{power_optim:lifetimes_triplexes}
and
\tabref{power_optim:lifetimes_triplexes2}.
This can be caused by the different laser setup of focus for the experiments.

The effect of fast rotation is visible only in the fast decay curves.
Analysis of the fits displayed in
\tabref{rotation_optim:lifetimes}
shows that the rotation speed does not affect the lifetimes of slow nor fast
decay components, but it significantly affects the Raman signal intensity
gathered from the fast decay component.
It means that the numbers of damaged molecules are the same for both rotation
speed scenarios because the lifetimes are the same, but more undamaged
molecules are available for measurement with fast rotation speed.
It also means that the rotation speed has lower impact on the measurements
with lower excitation laser power where the contribution of the fast decay
component to the resulting spectra is weaker.

Overall, the fast rotation seems better than the slower rotation with a
stronger signal and no impact on the sample lifetime.
Therefore we decided to use the maximal rotation speed in all the experiments
which used the spinning sample cell.
