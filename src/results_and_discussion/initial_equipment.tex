\subsection{Initial equipment}

As an excitation source, we used the Coherent Innova 300C
MotoFreD\texttrademark{} Ion Laser which enables intracavity frequency
doubling~\parencite{Asher1993b} of fundamental Ar ion emission lines by
nonlinear crystal. As a frequency doubling crystal we used BBO crystal (also
from Coherent) designed for doubling 488\,nm. Later we extended our
experimental options with additional two crystals designed for doubling 514 and
457\,nm. The BBO crystals are hygroscopic so they needed to be purged by steady
flow of 0.25--0.5 l/min of dry nitrogen of at least grade 5 purity (99.999\%)
\CITATION(laser manual). As a source of this nitrogen we decided to use
nitrogen generator NG1/1 from Gas Generation LTD which uses the pressure
swing absorption technology.

The above mentioned crystals could be used for
frequency doubling also the adjacent Ar ion lines. All the frequency doubled
wavelengths and expected and measured output laser beam powers which we could
tune with these three crystals are listed in
\tabref{initial_equipment:laser_power_spec}. The output wavelengths were
measured with HighFinesse High Precision Wavelength Meter WS5 UV-II and output
powers were measured with Power Meter from Thorlabs. The output powers in the
table were somewhat maximal which we could achieve, usually the power was
lower depending on laser condition.

\begin{table}
	\centering
	\newcommand*{\nodotcell}[1]{\multicolumn{1}{r@{\hphantom{.}}}{#1}}

\begin{tabular}{c@{\hspace{5mm}}r@{\hspace{7mm}}>{\hspace{6mm}}r@{.}l}
\toprule
$\lambda$ (nm) & \multicolumn{1}{c}{$P_\text{e}$ (mW)}
                     & \multicolumn{2}{c}{$P_\text{m}$ (mW)} \\
\midrule

228.962        &  10 &              2&37 \\
238.238        &  30 &             33&5  \\
243.989        & 100 & \nodotcell{110}   \\
248.250        &  60 &             33&2  \\
250.854        &  15 &             11&4  \\
257.261        & 100 & \nodotcell{170}   \\

\bottomrule
\end{tabular}

	\caption{Specifications of laser power of frequency doubled lines.
	$P_\text{e}$ denotes expected laser power and $P_\text{m}$ measured
	laser powers. The wavelengths are measured in air at $20\,^\circ$C.}
	\label{\tablabel{initial_equipment:laser_power_spec}}
\end{table}

As a detector, we used liquid nitrogen cooled Princeton Instruments
SPEC-10:2KBUV/LN backiluminated CCD camera enhanced for UV light detection.
The camera has $2048\times512$ pixels of $13.5\times13.5$\,\g{m}m and can
be controlled with WinSpec software through ST133B/U camera controller unit.
