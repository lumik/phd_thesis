\subsection{Low volume artifacts}

The spinning cell can be used not only for measurements in the right-angle
geometry, but also for measurements in the backscattering configuration.
We observed two sharp lines at 1555 and 2329\,\icm{} in these measurements
when very small sample volumes were used.
It can be seen, for example, in the spectrum of the samples extracted from alga
amphidinium in
\figref{artefact:artefact_amphidinium}.
50\,\g{m}l of sample was measured in the spinning cell in backscattering
geometry with 10\,mW of 257\,nm excitation laser.

\begin{figure}
	\centering
	\input{results_and_discussion/assets/artefact/%
		artefact_amphidinium}
	\caption[%
		Spectrum of extract from alga amphidinium in the spinning cell in
		backscattering geometry with 257\,nm excitation laser.%
	]{%
		\captiontitle{%
			Spectrum of extract from alga amphidinium in the spinning cell in
			backscattering geometry with 257\,nm excitation laser.%
		}
	}
	\label{\figlabel{artefact:artefact_amphidinium}}
\end{figure}

The unknown bands have a narrow shape which leads to a hypothesis that they
do not originate from the liquid samples but could come from some outside
source.
So the next step was to try to measure the spectrum without the sample in the
sample area.
The 120\,mW of excitation laser power was used to measure high-quality spectrum
in 10 frames and 120\,s accumulation of each of them.
The resulting spectrum can be seen in
\figref{artifact:artifact}.

\begin{figure}
	\centering
	\input{results_and_discussion/assets/artefact/%
		artefact}
	\caption[%
		High-quality spectrum of the artifact.%
	]{%
		\captiontitle{%
			High-quality spectrum of the artifact.%
		}
	}
	\label{\figlabel{artifact:artifact}}
\end{figure}

The rotational bands around the main Raman bands even narrowed the searching
for the origin of the artifact to some molecular spectrum.
Further investigation and searching through the databases and atlases of
Raman spectra identified the cause of the spectra as molecules of gas \ch{O2}
at 1556\,\icm{} and \ch{N2} at 2330\,\icm{}.

The 50\,\g{m}l of liquid sample rotated at 9600\,rpm results in the layer
thinner than 500\,\g{m}m at the call wall, which is much lower than the focal
region defined in
\cref{subsec:focus_optimization}
in
\eqnref{focus_optimization:L_E}
for the excitation laser focusing lens focal length of 100\,mm
\begin{equation*}
L_\text{E} \doteq 11.9\,\text{mm}
\end{equation*}
used in our experiments.
It means that the signal is also gathered from the central cylinder of the air
inside the spinning cell resulting in the contribution of gas \ch{O2} and
\ch{N2} in the spectra.
