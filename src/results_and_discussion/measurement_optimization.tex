\section{Optimization of the experiment}

Optimal measurement conditions are one of the requirements for high quality
scientific results. There are many adjustable parameters which can influence
quality of the measurement and their optimal values highly depend on the
samples under measurement. Further analysis is focused on optimization of
measurement parameters of nucleic acids using 244 and 257-nm excitation
wavelengths. PolyU was chosen as the sample for these experiments because
uracil is known to be the most susceptible base for photodecomposition and our
further experiments were conducted on olygo and polynucleotides and thus polyU
is better model molecule than for example mononucleotide UMP. The main
parameters which needed to be determined were excitation laser power, length of
accumulation, sample volume and concentration. Moreover, the optimal sample
rotation speed needed to be found for the experiments using spinning cell.

\subsection{Excitation laser power optimization}
\label{subsec:power_optim}

Firstly, we tried to establish the optimal laser power for excitation in
right-angle geometry using spinning cell. We used commercial samples of polyU
(Sigma) dissolved in 80 mM cacodylate buffer with ph 6.4 to the concentration
of 500\,\g{m}M per nucleobase. We chose 244 nm excitation laser wavelength with
10, 2, and 1 mW at samples. The 150\,\g{m}L of samples were rotated at
9600\,rpm during the measurement. The spectral series were measured with 30\,s
(10 and 2\,mW excitation) and 60\,s (1\,mW excitation) accumulation time per
spectrum. Beginning of each measurement was slightly delayed from the start of
the exposure to the laser because the samples needed to be adjusted for
optimal signal. The adjustment took 26\,s for 10\,mW excitation, 90\,s for
2\,mW, and 69\,s for 1\,mW excitation.

Further analysis was performed on the integral intensity of measured Raman
bands. The band shapes were modeled as described in
\cref{band_intensities}
with the experimentally estimated Lorentzian curve fraction coefficient
$c_\text{L} = 0.5$, which visually gave the best band shape fit. The
linear function $a_0 + a_1\wn$ was added to the shape function
\eqnref{band_intensities:shape}
as the model of the spectral background.

The analysis was performed on the polyU integral intensity represented by
it's band at 1231\,\icm{} compared to the intensity of cacodylate band at
607\,\icm{} to account for the excitation laser power and sample adjustment
variations during the measurement.
\Figref{power_optim:triplexes_pU}
clearly shows that the polyU band contains also shoulder at 1248\,\icm{} and
is overlapping with cacodylate band at 1276\,\icm. The band was therefore
modeled by the shape function
(\eqnref{band_intensities:shape})
with three components but it's intensity was calculated as a sum of the pulyU
bands without the cacodylate one. The cacodylate band was modeled as having
two components, at 607 and 640\,\icm, but only the first one was used for the
integral intensity calculation.

\begin{figure}
	\centering
	\input{results_and_discussion/assets/power_optimization_triplexes/%
		power_optimization_triplexes_pU}
	\caption{Example UV RR spectrum of polyU dissolved in cacodylate buffer. The
		fits of shape functions (\eqnref{band_intensities:shape}) to the bands of
		polyU and cacodylate used in intensity estimation fit are marked by green
		and red line respectively and the areas used for intensity calculation are
		filled with the corresponding colors.}
	\label{\figlabel{power_optim:triplexes_pU}}
\end{figure}

The time dependence of the intensity of polyU band at 1231\,\icm{} (normalized
to the intensity of cacodylate band at 607\,\icm) was modeled by the
exponential decay curve
\begin{equation}
	I = I_0 \text{e}^{-\lambda t} + b,
	\label{\eqnlabel{power_optim:decay}}
\end{equation}
where $I_0$ is initial intensity, $\lambda$ is decay constant and $b$ is
baseline constant. The normalized results (subtracted by $b$) can be seen in
\figref{power_optim:triplexes}

\begin{figure}
	\centering
	\input{results_and_discussion/assets/power_optimization_triplexes/%
		power_optimization_triplexes}
	\caption{Decrease of integral intensity of polyU band at 1231\,\icm{}
		normalized to the integral intensity of cacodylate band at 607\,\icm{}
		which was used as the internal intensity standard. The values were fitted
		by exponential decay curves \eqnref{power_optim:decay} and subtracted
		by the baseline constant $b$ from the fit.}
	\label{\figlabel{power_optim:triplexes}}
\end{figure}

We can see, that the lines intersect almost at the same point which is
expected outcome but that this point is not at 0\,s. This is caused by the fact
that the time axis denotes the ends of the spectrum accumulation whereas the
intensity comes from the whole time period of the spectrum acqusition which
is even more enhanced by the fact, that the intensity of the band lowers
during the measurement. The lifetimes $\tau$ can be calculated from decay
constants $\lambda$ from the equation
\eqnref{power_optim:decay}
\begin{equation}
	\tau = \frac{1}{\lambda}
	\label{\eqnlabel{power_optim:lifetime}}
\end{equation}
We can also calculate total accumulated relative energy from the decay curve
by
\begin{equation*}
	E_{0,\text{r}}
		= \int_0^{\infty}{I_0 \text{e}^{-\lambda t}\text{d}t}
		= \frac{I_0}{\lambda}
\end{equation*}
and energy accumulated from time $T_0$ by
\begin{equation*}
	E_r	= \int_{T_0}^{\infty}{I_0 \text{e}^{-\lambda t}\text{d}t}
		= \frac{I_0}{\lambda} \text{e}^{-\lambda T_0}.
\end{equation*}
Energy fractions can be then calculated by dividing the relative energies by
the maximal total relative energy. The resulting lifetimes, total accumulated
energy fractions $E_0$ and energy fractions $E$ accumulated from time
$T = 60\pm20$\,s, which was regarded as reasonable approximate for the time
needed for adjustment of the sample before measurement could start, can be
seen in
\tabref{power_optim:lifetimes_triplexes}.

\begin{table}
	\centering
	\begin{tabular}{cr@{$\,\pm\,$}lr@{$\,\pm\,$}lr@{$\,\pm\,$}lr@{$\,\pm\,$}l}
\toprule
P (mW)
   & \multicolumn{2}{c}{$\tau$\,(min)}
                 & \multicolumn{2}{c}{$E_0$}
                                & \multicolumn{2}{c}{$E$}
                                               & \multicolumn{2}{c}{$r$} \\
\midrule

10 &   2.3 & 0.1 &  1.00 & 0.11 &  0.64 & 0.12 &  0.64 & 0.10 \\
 2 &   7.6 & 0.6 &  0.67 & 0.08 &  0.59 & 0.08 &  0.88 & 0.04 \\
 1 &  16.8 & 4.5 &  0.74 & 0.21 &  0.70 & 0.21 &  0.94 & 0.02 \\
\bottomrule
\end{tabular}

	\caption{Lifetimes $\tau$ of the polyU in dependence on excitation power
		$P$. $E_0$ are total energies accumulated by detector divided by maximal
		value accross all the excitation powers $P$ and $E$ are energies
		accumulated from the time $T = 60\pm20$\,s which was needed for the
		adjustment of the samples before the acquisition can even start but
		the sample needs to be irradiated by the excitation laser. The last column
		contains fractions of the samples $r$ which were not destroyed by
		photodecomposition after the time $T$.
	}
	\label{\tablabel{power_optim:lifetimes_triplexes}}
\end{table}

The results show, that the 10\,mW power gave the best results if we don't take
into account adjustment of the samples but as you can see in the
\figref{power_optim:triplexes}
the signal intensity quickly lowers with the time. The lifetimes can also
vary significantly in dependence on a sample which can lower the reliability of
the band intensity estimations. The table shows that, after
1-minute adjustment of the sample, the 10\,mW excitation photodecomposes
about 35\% of the sample whereas with the 1\,mW excitation only around 6\% of
the sample is destroyed. The energies accumulated are calculated as integral
to infinity but with lower powers we are effectively able to measure only
smaller parts of the decay which would clearly favor measurements with higher
powers.

Further investigation of the spectral decay behavior shows that this simple
approach doesn't reveal that such a complex system can't be modeled by single
decay curve. This effect is clearly visible if we subtract the cacodylate
buffer background from the spectra. The resulting spectrum can be seen in
\figref{power_optim:triplexes2_pU}.

\begin{figure}
	\centering
	\input{results_and_discussion/assets/power_optimization_triplexes2/%
		power_optimization_triplexes2_pU}
	\caption{Example UV RR spectrum of polyU dissolved in cacodylate buffer with
	  the cacodylate buffer background subtracted. The fit of shape function
		(\eqnref{band_intensities:shape})
		to the band of polyU used in intensity estimation fit is marked by green
		line together with the area used for the intensity calculation.}
	\label{\figlabel{power_optim:triplexes2_pU}}
\end{figure}

The polyU band at 1230\,\icm{} can be then modeled directly without the
interference of the cacodylate band at 1276\,\icm{}. It can be seen from the
polyU band intensity dependence plot in
\figref{power_optim:triplexes2}
that the background subtraction process slightly obscures the intensity
normalization so that the fast decay curves can't be modeled here. But we can
also see a slower decay trend which isn't obvious from the fit to the
intensities of the polyU band calculated from the spectra without subtracted
background. This is caused by the fact that it is harder to reliably decouple
the contribution of the overlapping polyU and cacodylate bands with the
lowering polyU band intensity.

\begin{figure}
	\centering
	\input{results_and_discussion/assets/power_optimization_triplexes2/%
		power_optimization_triplexes2}
	\caption{Decrease of integral intensity of polyU band at 1231\,\icm{}
		normalized to the subtracted spectrum of cacodylate buffer which was used
		as the internal intensity standard. The values were fitted by exponential
		decay curves \eqnref{power_optim:decay} and subtracted by the baseline
		constant $b$ from the fit.}
	\label{\figlabel{power_optim:triplexes2}}
\end{figure}

For the purposes of analysis of the optimal measurement conditions we decided
to just simply ignore the first 3, 3 and 10 spectra for excitations at 1, 2 and
10\,mW respectively which were visibly deviating from the slow decay curves. We
used the eqution
\eqnref{power_optim:decay}
to estimate the lifetime $\tau$, total accumulated energy fractions $E_0$ and
energy fractions $E$ accumulated from time $T = 60\pm20$\,s the same way as
for the samples without subtracted background, see
\tabref{power_optim:lifetimes_triplexes2}. The baseline constant $b$ was used
only for the power of 10 mW where it decayed quickly to constant noise caused
by the fit for almost undetectable band. The table clearly shows that the slow
decay process depends linearly on excitation power, in other words that the
total accumulated energy is the same for all the excitation powers and that
the lifetime is inversely related to the excitation power.

\begin{table}
	\centering
	\begin{tabular}{cr@{$\,\pm\,$}lr@{$\,\pm\,$}lr@{$\,\pm\,$}lr@{$\,\pm\,$}l}
\toprule
P (mW)
   & \multicolumn{2}{c}{$\tau$\,(min)}
                 & \multicolumn{2}{c}{$E_0$}
                                & \multicolumn{2}{c}{$E$}
                                               & \multicolumn{2}{c}{$r$} \\
\midrule

10 &   7.1 & 0.8 &  0.90 & 0.14 &  0.79 & 0.14 &  0.87 & 0.04 \\
 2 &  39.5 & 2.4 &  1.00 & 0.12 &  0.98 & 0.12 &  0.98 & 0.01 \\
 1 &  70.3 & 7.3 &  0.89 & 0.13 &  0.88 & 0.13 &  0.99 & 0.01 \\
\bottomrule
\end{tabular}

	\caption{Lifetimes $\tau$ of the polyU in dependence on excitation power
		$P$. $E_0$ are total energies accumulated by detector divided by maximal
		value accross all the excitation powers $P$ and $E$ are energies
		accumulated from the time $T = 60\pm20$\,s which was needed for the
		adjustment of the samples before the acquisition can even start but
		the sample needs to be irradiated by the excitation laser. The last column
		contains fractions of the samples $r$ which were not destroyed by
		photodecomposition after the time $T$.
	}
	\label{\tablabel{power_optim:lifetimes_triplexes2}}
\end{table}

We can see that the system is complex and the proper analysis of the
underlaying processes is not possible with our current methods. The resulting
spectra can be influenced by the photoproducts and therefore we decided to use
rather lower excitation powers like 1\,mW or 0.5\,mW even thought the higher
excitation power could probably give higher quality spectra.

We also tried to optimize the excitation laser power for the backscattering
measurements in thermostated cell holder. We used 3\,mL of 125\,\g{m}M polyU
samples dissolved in 40\,mM cacodylate buffer with pH 6.8. 257\,nm was used as
an excitation laser wavelength and the samples were stirred by a magnetic
stirrer. The spectral series was measured in 20\,s accumulation time per
spectrum. The measurements in thermostated cell holder were easier to set up
because the backscattering geometry and robust construction of the holder
were less sensitive to displacements which meant that the adjustments can be
performed on helper samples and the real samples weren't exposed to the
laser power before measurement.

The analysis of the results was performed with the same steps as the analysis
of the measurements with spinning cell using the integral intensity of polyU
band at 1231\,\icm{} and cacodylate band at 607\,\icm{}. As we used 20 times
larger volumes of samples and 4 times lower concentrations, we should roughly
expect 5 times longer lifetimes which means that we could observe only the fast
decay components, see
\figref{power_optim:hairpins}
and
\tabref{power_optim:lifetimes_hairpins}.

\begin{figure}
	\centering
	\input{results_and_discussion/assets/power_optimization_hairpins/%
		power_optimization_hairpins}
	\caption{Decrease of integral intensity of polyU band at 1231\,\icm{}
		normalized to the integral intensity of cacodylate band at 607\,\icm{}
		which was used as the internal intensity standard. The values were fitted
		by exponential decay curves \eqnref{power_optim:decay} and subtracted
		by the baseline constant $b$ from the fit.}
	\label{\figlabel{power_optim:hairpins}}
\end{figure}

\begin{table}
	\centering
	\begin{tabular}{cr@{$\,\pm\,$}lr@{$\,\pm\,$}l}
\toprule
P (mW)
   & \multicolumn{2}{c}{$\tau$\,(min)}
                 & \multicolumn{2}{c}{$E_0$} \\
\midrule

 4 &  15.8 & 0.1 &  1.00 & 0.10 \\
 2 &  30.5 & 0.8 &  0.95 & 0.10 \\
 1 &  61.0 & 5.5 &  0.94 & 0.13 \\
\bottomrule
\end{tabular}

	\caption{Lifetimes $\tau$ of the polyU in dependence on excitation power
		$P$. $E_0$ are total energies accumulated by detector divided by maximal
		value accross all the excitation powers.
	}
	\label{\tablabel{power_optim:lifetimes_hairpins}}
\end{table}

It can be seen, that the lifetimes are proportionally longer with lower
excitation laser power and no further special effects are observable. The
lifetimes are slightly lower than the expected 5 times longer lifetimes as
compared to the \tabref{power_optim:lifetimes_triplexes} but almost in the
range of estimated errors. It can be caused by many effects like underestimated
errors, different excitation laser wavelength (257\,nm instead of 244\,nm),
less effective stirring in larger volumes or longer laser path in the samples.
We can also see that higher intensity gives more reliable results with
lower errors because of better signal to noise ratio. We decided to perform
further measurements with 4\,mW of excitation power at 257\,nm and 5\,mW at
244\,nm.

\subsection{Concentration optimization}

The next step was to determine the optimal concentration of the measured NA
sample in respect to its photodamage.
First, we tried to estimate the dependence of the lifetime on concentration. We
used the same sample preparation as described in
\cref{subsec:power_optim}
but with variable concentration of samples of 500\,\g{m}M and 1000\,\g{m}M per
nucleobase and 1\,mW of excitation laser power. We also used the same data
treatment as in the previous section. We weren't able to detect fast decay
component for 500\,\g{m}M samples in this measurement so we tried to compare
at least the slow decay components the same way as in the previous section.
The results can be seen in
\figref{conc_optim:triplexes}
and
\tabref{conc_optim:lifetimes_triplexes}.

\begin{figure}
	\centering
	\input{results_and_discussion/assets/concentration_optimization_triplexes/%
		concentration_optimization_triplexes}
	\caption{Decrease of integral intensity of polyU band at 1231\,\icm{}
		normalized to the subtracted spectrum of cacodylate buffer which was used
		as the internal intensity standard. The values were fitted by exponential
		decay curves \eqnref{power_optim:decay} and subtracted by the baseline
		constant $b$ from the fit.}
	\label{\figlabel{conc_optim:triplexes}}
\end{figure}

\begin{table}
	\centering
	\begin{tabular}{cr@{$\,\pm\,$}lr@{$\,\pm\,$}lr@{$\,\pm\,$}lr@{$\,\pm\,$}l}
\toprule
c (\g{m}M)
   & \multicolumn{2}{c}{$\tau$\,(min)}
                & \multicolumn{2}{c}{$E_0$}
                               & \multicolumn{2}{c}{$E$}
                                              & \multicolumn{2}{c}{$r$} \\
\midrule

 500 &   53 & 3 &  0.55 & 0.06 &  0.54 & 0.06 &  0.981 & 0.006 \\
1000 &   96 & 6 &  1.00 & 0.12 &  0.99 & 0.12 &  0.990 & 0.003 \\
\bottomrule
\end{tabular}

	\caption{Lifetimes $\tau$ of the polyU in dependence on concentration
		$c$. $E_0$ are total energies accumulated by detector divided by maximal
		value accross all the excitation powers $P$ and $E$ are energies
		accumulated from the time $T = 60\pm20$\,s which was needed for the
		adjustment of the samples before the acquisition can even start but
		the sample needs to be irradiated by the excitation laser. The last column
		contains fractions of the samples $r$ which were not destroyed by
		photodecomposition after the time $T$.
	}
	\label{\tablabel{conc_optim:lifetimes_triplexes}}
\end{table}

First of all, it is important to notice that there is variation in the
estimated lifetime between this measurement and results from previous section
\tabref{power_optim:lifetimes_triplexes2}. The difference between these two
measurements was that we used larger slit width (70 \g{m}M instead of 50 \g{m}m
used in the previous section) which could have impact on the decay curve
because as we showed in the previous section the photodecomposition process
is very complex and involves changes in the sample absorbance which can have
large impact on the focus because of the anomalous dispersion.

The second observation is, that the lifetime seems to depend almost linearly on
the concentration which is proportional to the number of nucleotides in the
sample. This means that in our right angle experimental configuration all the
excitation laser energy is absorbed in the sample for both concentrations and
that the number of photodecomposed molecules is proportional to the absorbed
energy in this concentration range and the excitation laser power density
doesn't significantly influence the photodecomposition process.

This analysis means that samples can endure longer measurements on higher
concentrations. On the other hand, higher concentrations are also less cost
effective because they require more samples and it is harder to adjust focus
for them, the spectra are also more influenced by the signal of the sample cell
because you need to focus closer to the cell wall and floor. 1000\,\g{m}M
concentration was the highest practical value, higher concentrations had
excesive absorbance and were hard to set up for the measurement.

Later,
the concentration was also optimalized for the measurements in backscattering
geometry in thermostated cell holder. The 500\,\g{m}L of samples with
75\,\g{m}M, 250\,\g{mM} and 750\,\g{m}M concentrations were used with 5\,mW of
excitation laser power. Only fast decay components were investigated. You can
see the results in
\figref{conc_optim:hairpins}
and
\tabref{conc_optim:lifetimes_hairpins}.

\begin{figure}
	\centering
	\input{results_and_discussion/assets/concentration_optimization_hairpins/%
		concentration_optimization_hairpins}
	\caption{Decrease of integral intensity of polyU band at 1231\,\icm{}
		normalized to the integral intensity of cacodylate band at 607\,\icm{}
		which was used as the internal intensity standard. The values were fitted
		by exponential decay curves \eqnref{power_optim:decay} and subtracted
		by the baseline constant $b$ from the fit.}
	\label{\figlabel{conc_optim:hairpins}}
\end{figure}

\begin{table}
	\centering
	\begin{tabular}{cr@{$\,\pm\,$}lr@{$\,\pm\,$}l}
\toprule
c (\g{m}M)
     & \multicolumn{2}{c}{$\tau$\,(min)}
                  & \multicolumn{2}{c}{$E_0$} \\
\midrule

  75 &  1.3 & 0.1 &  1.00 & 0.11 \\
 250 &  3.3 & 0.1 &  0.74 & 0.07 \\
 750 & 10.2 & 0.4 &  0.76 & 0.08 \\
\bottomrule
\end{tabular}

	\caption{Lifetimes $\tau$ of the polyU in dependence on concentration
		$c$. $E_0$ are total energies accumulated by detector divided by maximal
		value across all the concentrations $c$ and normalized to the
		concentration.
	}
	\label{\tablabel{conc_optim:lifetimes_hairpins}}
\end{table}

The results clearly show that all laser power is not absorbed in the samples
with extremely low concentrations and therefore the lifetime is longer.
Unfortunately also the signal of the sample and lifetime itself are
insufficiently low in this case. We can also see that at least from
250\,\g{m}M concentration the concentration doesn't affect the lifetime
unexpectedly which means that higher concentrations are better up to maximal
practical values which were estimated at 1\,mM for nucleic acids.

For some measurements, it can be also beneficial to measure samples with the
same concentration as in UV absorption measurements so that these two can
complement each other under the same measurement conditions. Therefore the
lower concentrations than 1\,mM were usually chosen in this study.
\subsection{Volume optimization}

Results from the laser power and concentration optimization measurements
concluded that the main factor influencing the photodegradation of the sample
is the number of illuminated molecules.
So, we tried to verify the hypothesis by an experiment where we kept the
number of molecules in the sample at the same level.
We varied the concentration and volume of the samples and preserved the amount
of the analyte, concretely we used
	3\,mL of 125\,g{m}M sample,
	2\,mL of 188\,\g{m}M sample,
	1\,mL of 375\,\g{m}M sample and
	0.5\,mL of 750\,g{m}M sample.
The results are displayed in
\figref{vol_optim:hairpins}
and
\tabref{vol_optim:lifetimes_hairpins}.

\begin{figure}
	\centering
	\input{results_and_discussion/assets/volume_optimization_hairpins/%
		volume_optimization_hairpins}
	\caption[%
		Decrease of the integral intensity of the polyU band at 1231\,\icm{}
		for different volumes with a preserved number of molecules in raw spectra
		using backscattering geometry.%
	]{%
		\captiontitle{%
			Decrease of the integral intensity of the polyU band at 1231\,\icm{}
			for different volumes with a preserved number of molecules in raw spectra
			using backscattering geometry.%
		}
		It was normalized to the integral intensity of cacodylate band at
		607\,\icm{}, which was used as the internal intensity standard.
		The values were fitted	by exponential decay curves
		\eqnref{power_optim:decay}.
		The baseline constant $b$ from the fit was subtracted from the plots.
	}
	\label{\figlabel{vol_optim:hairpins}}
\end{figure}

\begin{table}
	\centering
	\begin{tabular}{cr@{$\,\pm\,$}lr@{$\,\pm\,$}l}
\toprule
c (\g{m}M)
     & \multicolumn{2}{c}{$\tau$\,(min)}
                  & \multicolumn{2}{c}{$E_0$} \\
\midrule

 125 & 10.3 & 0.3 &  1.00 & 0.11 \\
 188 &  8.3 & 0.2 &  0.80 & 0.08 \\
 375 &  8.2 & 0.1 &  0.80 & 0.08 \\
 750 & 10.2 & 0.4 &  0.99 & 0.11 \\
\bottomrule
\end{tabular}

	\caption[%
		Lifetimes of the polyU in dependence on concentration
		for different volumes with a preserved number of molecules in raw spectra
		using backscattering geometry.%
	]{%
		\captiontitle{%
			Lifetimes $\tau$ of the polyU in dependence on concentration $c$
			for different volumes with a preserved number of molecules in raw spectra
			using backscattering geometry.%
		}
		$E_0$ are total energies accumulated by the detector divided by the
		maximal value across all the concentrations $c$ and normalized to the
		concentration.
	}
	\label{\tablabel{vol_optim:lifetimes_hairpins}}
\end{table}

We can see that the longest lifetime was achieved at 125\,\g{m}M and
750\,\g{m}M concentrations.
The lowest concentration probably can have a longer lifetime because of less
absorbed laser power in the sample.
The result for 750\,\g{m}M concentration can be more influenced by the slow
decay component, as shown in the spinning cell measurements.
However, overall, the results support the hypothesis that the sample
photodecomposition is inversely proportional to the number of illuminated
molecules with sufficient stirring.
We, therefore, tried to use fully-filled cuvettes by samples (3\,mL in the case
of the thermostated sample cell and 150\,\g{m}L for the spinning cell).

\subsection{The course of the signal accumulation}

A Raman spectrum is usually acquired as a consecutive series of spectra,
further called \emph{frames}, taken at the same experimental conditions and
with the constant accumulation time.
This approach has many reasons.
Amongst the most important is that measurement of Raman spectra requires highly
sensitive detectors which means that detector saturation can be easily reached
with a stronger signal.

Secondly, such sensitive detectors are susceptible to cosmic ray artifacts,
which can significantly damage the spectra, see
\cref{subsec:spike_removal}.
It is easier to subtract the cosmic ray signal in the series of spectra taken
with the same measurement condition than from a single spectrum because the
subsequent spectra should be almost identical, and the average of the
surrounding spectra can replace the spectral regions with cosmic ray signal.

Furthermore, other temporal effects can affect the spectra, like slight
temperature variations, mechanical movements etc.

RRS is also affected by photodecomposition and an increasing presence of
photoproducts which can directly or indirectly influence the measured spectra.
The direct presence of a signal of photoproducts in the detected spectra is
usually negligible because Raman scattering of the photoproducts is not usually
resonantly enhanced.
However, indirect effects through chemical interactions of photo products with
the system under investigation are always possible.

So, measurement of more frames with a shorter accumulation time would be
favored for better monitoring of all these processes during the accumulation of
the spectrum.

However, there are also opposite effects that support longer measurements.
The most dominant is the ratio between the signal from the sample and noise.
The noise in the spectra can have many origins, but dominant are those
connected with acquiring the signal on the CCD detector.
The first one is low-frequency (almost constant) background which is usually
used in \emph{signal-to-noise ratio} (SNR) calculation.
Best results with the most linear response to the intensity of the gathered
light are achieved with a signal in the range $\approx 10 -- 80\,\%$ of the
maximal signal limit.
The second component is high-frequency noise which lowers the quality of the
spectra, reliability of the band position detection, or can even hide some
spectral features completely.

We tried to estimate some guiding principles about a balance between a number
of frames and an accumulation time.
We measured the background spectrum of deionized water with high laser power
(100\,mW), which meant short accumulation times (5\,s) taken in 100 frames.
All these frames were then averaged to obtain one high quality Raman spectrum
of water $I_\text{water}(\wn)$ as a reference.
Then we measured the same water with 1\,mW excitation in 30 frames with 20\,s
accumulation time and 10 frames with 60\,s accumulation time which are
parameters of our typical RRS measurement. We summed all the frames to obtain a
single Raman spectrum $I_{20}(\wn)$ and $I_{60}(\wn)$, so both of them
represented 10 min total accumulation.

We decided to assess the SNR to estimate the quality of the measured spectra.
The noise height was estimated by a fit of the high-quality water spectrum
$I_\text{water}(\wn)$ plus constant background, which resulted in the model
intensity function
\begin{equation}
	I_\text{model}(\wn) = a_0 I_\text{water}(\wn) + a_1.
	\label{\eqnlabel{accum_length:sn_model}}
\end{equation}
An example fit can be seen in
\figref{accum_length:sn_ratio}.

\begin{figure}
	\centering
	\input{results_and_discussion/assets/sn_ratio/sn_ratio}
	\caption[%
		Signal-to-noise ratio calculation from a fit of the high-quality spectrum
		and constant background addition.%
	]{%
		\captiontitle{%
			Signal-to-noise ratio calculation from a fit of the high-quality spectrum
			and constant background addition.%
		}
		The original \emph{spectrum} (here, we are using the spectrum $I_{20}(\wn)$
		with 20\,s accumulation time) was modeled by \emph{model}
		\eqnref{accum_length:sn_model}.
		The spectra on the image were then normalized to the \emph{constant} $a_1$.
	}
	\label{\figlabel{accum_length:sn_ratio}}
\end{figure}

The low-frequency SNR was then estimated as
\begin{equation*}
	SNR_\text{low} = \frac{I_\text{model}(1637)}{a_1},
\end{equation*}
which leads to
\begin{align*}
	SNR_{\text{low},20} &= 1.3841 \pm 0.0004, \\
	SNR_{\text{low},60} &= 2.6112 \pm 0.0026,
\end{align*}

We also estimated the high-frequency noise.
We subtracted the constant $a_1$ from the measured spectra and normalized them
to the maximum of the $I_\text{model}(1637\,\text{cm}^{-1})$ to compare the
noise size to the size of this band.
We then calculated spectrum $I_\text{SG}$ smoothed by Savitzky-Golay filter
\parencite{Savitzky1964},
see
\figref{accum_length:sn_ratio_sg}.

\begin{figure}
	\centering
	\input{results_and_discussion/assets/sn_ratio/sn_ratio_sg}
	\caption[%
		Signal-to-noise ratio calculation using standard deviation from the
	  spectrum smoothed by the Savitzky-Golay filter.
	]{%
		\captiontitle{%
			Signal-to-noise ratio calculation using standard deviation from the
			spectrum smoothed by the Savitzky-Golay filter.
		}
		The spectrum is normalized to the water band at 1637\,\icm{}.
		The figure shows the original \emph{spectrum} and the spectrum smoothed by
		the Savitzky-Golay filter (\emph{SG filter}).
	}
	\label{\figlabel{accum_length:sn_ratio_sg}}
\end{figure}

Finally, we estimated the high-frequency $SNR_\text{high}$ from the standard
deviation $\sigma$ of the original spectrum from the smoothed spectrum
normalized to the intensity of the water band at 1637\,\icm{} (which means that
the normalized intensity maximum at 1637\,\icm{} is equal to 1),
\begin{equation*}
	SNR_\text{high} = \frac{1}{\sigma},
\end{equation*}
\begin{align*}
	SNR_{\text{high},20} &= 240, \\
	SNR_{\text{high},60} &= 297.
\end{align*}

It can be seen that longer single-frame accumulation times give better quality
spectra in these experimental conditions.
The difference is larger in the low-frequency SNR even though the
high-frequency SNR difference is also significant.
Regarding the disadvantages of the
longer accumulation discussed at the beginning of this section and the speed of
the sample degradation with lifetimes in the range of minutes, we decided that
the optimal accumulation time is 60\,s.

\subsection{Optimization of spinning cell rotation speed}

All the previously described experiments used the maximal rotation speed of
the spinning cell.
The next question was if slower rotation could not produce better results.
It could, for example, provide more time for reversible changes and damaged
sample diffusion from the focus.
Faster rotation can also produce unwanted vibrations in the spectrometer.
We used the same experimental setup as previous measurements with 5\,mW of
excitation power and 500\,mM polyU samples.
We varied the rotation speed where we tried to measure the sample at the slow
rotation of 4\,Hz and the full rotation of 160\,Hz.
The time needed to adjust the samples was 100\,s for slow rotation case and
30\,s for fast rotation one.
We analyzed decay curves using the same methods as in
\cref{subsec:power_optim},
where the fast decay curve was fitted to the first 15 and 25 frames for the
slow and fast rotation respectively.
The slow decay curve was fitted for the rest of the spectrum.
Results of the fit can be seen in
\cref{%
	\figlabel{rotation_optim:fast_decay},%
	\figlabel{rotation_optim:slow_decay}%
}
and
\tabref{rotation_optim:lifetimes}.

\begin{figure}
	\centering
	\input{results_and_discussion/assets/rotation_optimization/%
		rotation_optimization}
	\caption[%
		Decrease of the integral intensity of the polyU band at 1231\,\icm{} for
		different sample cell rotation speeds in raw spectra.%
	]{%
		\captiontitle{%
			Decrease of the integral intensity of the polyU band at 1231\,\icm{} for
			different sample cell rotation speeds in raw spectra.%
		}
		It was normalized to the integral intensity of the cacodylate band at
		607\,\icm{}, which was used as the internal intensity standard.
		The values were fitted by exponential decay curves
		\eqnref{power_optim:decay}.
		The baseline constant $b$ from the fit was subtracted from the plots.
	}
	\label{\figlabel{rotation_optim:fast_decay}}
\end{figure}

\begin{figure}
	\centering
	\input{results_and_discussion/assets/rotation_optimization/%
		rotation_optimization_kor}
	\caption[%
		Decrease of the integral intensity of the polyU band at 1231\,\icm{}
		for different sample cell rotation speeds in background-corrected spectra.%
	]{%
		\captiontitle{%
			Decrease of the integral intensity of polyU band at 1231\,\icm{}
			for different sample cell rotation speeds in background-corrected
			spectra.%
		}
		The intensity was normalized to the subtracted spectrum of cacodylate
		buffer, which was used as the internal intensity standard.
		The values were fitted by exponential decay curves
		\eqnref{power_optim:decay}.
		The baseline constant $b$ from the fit was subtracted from the plots.
	}
	\label{\figlabel{rotation_optim:slow_decay}}
\end{figure}

\begin{table}
	\centering
	\begin{tabular}{ccr@{$\,\pm\,$}lr@{$\,\pm\,$}lr@{$\,\pm\,$}lr@{$\,\pm\,$}l}
\toprule
f (Hz)
    & decay
		       &\multicolumn{2}{c}{$\tau$\,(min)}
                        & \multicolumn{2}{c}{$E_0$}
                                       & \multicolumn{2}{c}{$E$}
                                                      & \multicolumn{2}{c}{$r$} \\
\midrule

  4 & fast &  2.2 & 0.3 &  0.26 & 0.05 &  0.17 & 0.05 &  0.64 & 0.10 \\
120 & fast &  2.5 & 0.2 &  1.00 & 0.10 &  0.67 & 0.12 &  0.67 & 0.09 \\
  4 & slow & 24.3 & 1.4 &  0.96 & 0.07 &  0.91 & 0.07 &  0.96 & 0.01 \\
120 & slow & 22.0 & 2.0 &  1.00 & 0.12 &  0.96 & 0.12 &  0.96 & 0.01 \\
\bottomrule
\end{tabular}

	\caption[%
		Lifetimes of slow and fast decay components of the polyU in
		dependence on rotation frequency.%
	]{%
		\captiontitle{%
			Lifetimes $\tau$ of slow and fast decay components of the polyU in
			dependence on rotation frequency $f$.%
		}
		$E_0$ are total energies accumulated by the detector divided by the maximal
		value across all the excitation powers $P$, and $E$ are energies
		accumulated from the time $T = 60\pm20$\,s needed for the adjustment of the
		samples before the acquisition can even start, but the sample needs to be
		irradiated by the excitation laser.
		The last column contains fractions of the samples $r$ that were not
		destroyed by photodecomposition after the time $T$.
	}
	\label{\tablabel{rotation_optim:lifetimes}}
\end{table}

Even though the results in this experiment are internally consistent, we once
more see different lifetimes compared to the results in
\tabref{power_optim:lifetimes_triplexes}
and
\tabref{power_optim:lifetimes_triplexes2}.
This can be caused by the different laser setup of focus for the experiments.

The effect of fast rotation is visible only in the fast decay curves.
Analysis of the fits displayed in
\tabref{rotation_optim:lifetimes}
shows that the rotation speed does not affect the lifetimes of slow nor fast
decay components, but it significantly affects the Raman signal intensity
gathered from the fast decay component.
It means that the numbers of damaged molecules are the same for both rotation
speed scenarios because the lifetimes are the same, but more undamaged
molecules are available for measurement with fast rotation speed.
It also means that the rotation speed has lower impact on the measurements
with lower excitation laser power where the contribution of the fast decay
component to the resulting spectra is weaker.

Overall, the fast rotation seems better than the slower rotation with a
stronger signal and no impact on the sample lifetime.
Therefore we decided to use the maximal rotation speed in all the experiments
which used the spinning sample cell.

\subsection{Low volume artefacts}

The spinning cell can be used not only for measurements in the right angle
geometry but also for measurements in the back-scattering configuration.
We observed the two sharp lines at 1555 and 2329\,\icm{} in these measurements
when very small sample volumes were used.
You can see example spectrum of samples extracted from the alga amphidinium in
\figref{artefact:artefact_amphidinium}.
50\,\g{m}l of sample was measured in spinning cell in back-scattering geometry
with 10\,mW of 257\,nm excitation laser.

\begin{figure}
	\centering
	\input{results_and_discussion/assets/artefact/%
		artefact_amphidinium}
	\caption{Spectrum of extract from alga aphidinium in spinning cell in
		backscattering geometry with 257\,nm excitation laser.}
	\label{\figlabel{artefact:artefact_amphidinium}}
\end{figure}

The unknown bands have narrow shape which leads to a hypothesis that they
don't originate from the liquid samples but could come from some outside
source. So the next step was to try to measure the spectrum without the
sample in the sample area. The 120\,mW of excitation laser power was used to
measure high quality spectrum in 10 frames and 120\,s accumulation of each of
them. The resulting spectrum can be seen in
\figref{artefact:artefact}.

\begin{figure}
	\centering
	\input{results_and_discussion/assets/artefact/%
		artefact}
	\caption{High quality spectrum of the artefact.}
	\label{\figlabel{artefact:artefact}}
\end{figure}

The rotational bands around the main Raman bands even narrowed the searching
for the origin of the artefact to some molecular spectrum. Further
investigation and searching through the databases and atlases of Raman spectra
identified the cause of the spectra as molecules of gas \ch{O2} at 1556\,\icm{}
and \ch{N2} at 2330\,\icm{}.

The 50\,\g{m}l of liquid sample rotated at 9600\,rpm results in the layer
thinner than 500\,\g{m}m at the call wall which is much lower than the focal
region defined in
\cref{subsec:focus_optimization}
in
\eqnref{focus_optimization:L_E}
for the excitation laser focusing lens focal length of 100\,mm
\begin{equation*}
L_\text{E} \doteq 11.9\,\text{mm}
\end{equation*}
used in our experiments.
It means that the signal is also gathered from the central cylinder of the air
inside the spinning cell resulting in the contribution of gas \ch{O2} and
\ch{N2} in the spectra.

