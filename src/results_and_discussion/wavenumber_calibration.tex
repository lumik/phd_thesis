\subsection{Wavenumber calibration}
\label{wavenumber_calibration}

For correct Raman spectra interpretation, we needed precise wavenumber
calibration.
It can be done by measuring spectrum with known line positions in the same
spectrometer configuration as for the spectrum of the sample and then
interpolating and extrapolating the wavenumbers from these known line positions
to all detector pixels.
One group of calibration spectra is Raman spectra of organic solvents, but one
cannot use pure indene, which is widely used in visible Raman because it
strongly absorbs UV light, and its lines can be shifted if it is diluted.
So other solvents need to be found.
Organic solvents were used as wavenumber calibration for the first UV Raman
spectra measurements.
For example, \textcite{Harada1975} calibrated to Raman line of \ch{CCl4}.
However, early, it was clear that a single organic solvent was not
satisfactory, and therefore some teams started to use mixtures of solvents.
However, it is unclear how the components influence each other if a mixture of
solvents is used.
They can evaporate at different rates, so their ratio can be changing which can
shift their bands.
This problem can be solved by measuring the solvents separately, but it
prolongs the time spent by measuring the calibration spectra.
Some selected solvents are listed in
\tabref{wavenumber_calibration:solvents}.

\begin{table}
	\centering
	\begin{tabular}{m{8.7cm}l}
\toprule
solvent & citation \\
\midrule
\ch{CCl4}                                   & \textcite{Harada1975} \\
\addlinespace[.3em]
cyclohexane                                 & \textcite{Myers1988} \\
\addlinespace[.3em]
acetonitrile                                & \textcite{Gustafson1988} \\
\addlinespace[.3em]
ethanol                                     & \textcite{Su1990} \\
\addlinespace[.3em]
ethyl acetate/dioxane (2:1 v/v)             & \textcite{Toyama1991} \\
\addlinespace[.3em]
cyclohexane, acetone                        & \textcite{Kaminaka1992} \\
\addlinespace[.3em]
acetonitrile, acetone                       & \textcite{Benson1992} \\
\addlinespace[.3em]
cyclohexanone/acetonitrile (1:1 v/v)        & \textcite{Hashimoto1993} \\
\addlinespace[.3em]
ethanol, \textit{n}-penthane                & \textcite{Mukerji1995} \\
\addlinespace[.3em]
dioxane, \ch{CCl4}, acetonitrile            & \textcite{Russell1995} \\
\addlinespace[.3em]
ethyl acetate/dioxane (2:1 v/v) (257\,nm)
                                            & \textcite{Fujimoto1998} \\
\addlinespace[.3em]
cyclohexanone/acetonitrile (1:1 v/v) (213\,nm)
                                            & \textcite{Fujimoto1998} \\
\addlinespace[.3em]
acetone, ethanol, pentane                   & \textcite{Mukerji1998} \\
\addlinespace[.3em]
\textit{n}-pentane, cyclehexane, dimethylformamide, ethanol, acetonitrile,
acetic acid                                 & \textcite{Billinghurst2006} \\
\addlinespace[.3em]
dimethylsulfoxide                           & \textcite{Srivastava2008} \\
\addlinespace[.3em]
dimethyl formamide, acetonitrile, trichloroethylene, isopropanol, indene,
cyclohexane                                 & \textcite{Jayanth2009} \\
\addlinespace[.3em]
cyclohexane, \textit{N,N}-dimethylformamide, methanol, acetonitrile,
dimethyl sylfoxide, acetic acid             & \textcite{Oladepo2011} \\
\addlinespace[.3em]
acetonitrile, isopropanol, cyclohexane and dimethyl formamide
                                            & \textcite{Mondal2016} \\
\bottomrule
\end{tabular}

	\caption[%
		Selection of organic solvents which were used for UV Raman spectra
		wavenumber calibration in literature.%
	]{%
		\captiontitle{%
			Selection of organic solvents which were used for UV Raman spectra
			wavenumber calibration in literature.%
		}
	}
	\label{\tablabel{wavenumber_calibration:solvents}}
\end{table}

Some laboratories were not satisfied with the separate calibration and used
internal standards to improve spectra' calibration.
For example,
\textcite{Wen1998}
used \ch{Na2SO4} as internal frequency (981\,\icm) standard, and Raman
frequencies were calibrated to $\pm1 $\,\icm{} using \ch{CCl4}/\ch{CH3CN}
mixture.

The other approach was also to use the elastically scattered laser line.
\textcite{Kumamoto2012}
calibrated grating dispersion with the laser line (0\,\icm) and Raman bands of
boron nitride (1370\,\icm) and acetonitrile (2249\,\icm).

%\textcite{Myers1988} calibrated wavenumbers using cyclohexane in
combination with various \ch{H2} Raman-shifted laser lines and mercury
emission lines. \textcite{Su1990} calibrated Raman spectra in frequency with
ethanol. \textcite{Toyama1991, Toyama1996} used for wavenumber calibration
the Raman bands of ethyl acetate/dioxane (2:1 v/v), which had been measured
with visible-laser excitation on a spectrometer calibrated with indene Raman
bands and neon emission lines. \textcite{Kaminaka1992} calibrated Raman
shifts with cyclohexane and acetone as the secondary standard, while their
Raman bands were calibrated with indene, with the use of visible excitation.
\textcite{Benson1992} calibrated the wavenumber shift with the use of the
Raman spectra of acetonitrile and acetone. \textcite{Benson1993} used carbon
tetrachloride for wavenumber shift calibration standard.
\textcite{Hashimoto1993} made wavenumber calibration with the Raman bands of
cyclohexanone/acetonitrile (1:1 v/v). \textcite{Mukerji1995} calibrated spectra
against neat solution of ethanol and \textit{n}-penthane.
\textcite{Russell1995} performed wavenumber calibration in UV-laser excited
Raman spectra using the well known Raman lines of dioxane, carbon
tetrachloride, and acetonitrile. \textcite{Gustafson1988} calibrated
wavenumber shift using the Raman spectrum of acetonitrile.
\textcite{Leonard1994} calibrated wavenumber shift with the use of the Raman
spectrum of cyclohexane. \textcite{Takeuchi1990} made wavenumber calibration
by using the Raman bands of cyclohexanone/acetonitrile (1:1, v/v), the
wavenumber of which had been established with visible-laser excitation on a
spectrometer calibrated with indene Raman bands and neon emission lines.
\textcite{Takeuchi1995} calibrated wavenumber axis of the spectra by using
the spectrum of a mixture solution of ethyl acetate/dioxane (2:1 v/v) as a
wavenumber standard. \textcite{Tuma1995} performed wavenumber
calibration in UVRR spectra using well-known Raman lines of dioxane, carbon
tetrachloride, and acetonitrile. \textcite{Wen1997} calibrated Raman
frequencies using a standard liquid mixture of carbon tetrachloride and
acetonitrile. \textcite{Fujimoto1998} made wavenumber calibration of the
spectra by use of the Raman bands of a 2:1 (v/v) mixture of ethyl
acetate/1,4-dioxane (257\,nm excitation) or of a 1:1 (v/v) mixture of
cyclohexanone/acetonitrile (213\,nm). \textcite{Mukerji1998} calibrated data
with acetone, ethanol, and pentane. \textcite{Wen1998} used \ch{Na2SO4} as
internal frequency (981\,\icm) standard, and Raman frequencies were
calibrated to $\pm1 $\,\icm{} using \ch{ccl4}/\ch{CH3CN} mixture.
\textcite{Suen1999} calibrated against neat solutions of ethanol and acetone,
for which the Raman band frequencies were known. \textcite{Toyama1999}
performed wavenumber calibraion by use of the Raman bands of
ethylacetate--dioxane (2:1, v/v). \textcite{Wen1999} calibrated Raman
frequencies by using a standarad liquid mixture of carbon tetrachloride and
acetonitrile. \textcite{Sokolov2000} calibrated data with acetone, ethanol and
pentane. \textcite{Toyama2001} made wavenumber calibration of Raman spectra
by using the Raman spectrum of a 2:1 (v/v) mixture of ethylacetate/1,4-dioxane.
\textcite{Fujimoto2002} calibrated Raman spectrometer by using the Raman bands
of a 2:1 (v/v) mixture of ethylacetate and dioxane for 257-nm excitation or a
1:1 (v/v) mixture of cyclohexanone and acetonitrile for 229-nm excitation.
\textcite{Toyama2002} performed wavenumber calibration of Raman spectra by use
of the Raman bands of a 2:1 (v/v) mixture of ethyl acetate and 1,4-dioxane.
\textcite{Nelson2004} calibrated the Raman shift axis with pure ethanol.
\textcite{Toyama2005,Toyama2005a} made wavenumber calibration of Raman spectra
by use of Raman bands of a 2:1 (v/v) mixture of ethyl acetate and 1,4-dioxane.
\textcite{Billinghurst2006,Billinghurst2006a} perforemd frequency calibration
by measuring the Raman scattering of organic solvents for whom the peak
positions are known (\textit{n}-pentane, cyclehexane, dimethylformamide,
ethanol, acetonitrile, and acetic acid). \textcite{Jirasek2006} wavenumber
calibrated abscissa of each acquired spectrum using an ethanol spectrum as the
calibration standard. \textcite{Kundu2007} performed frequency calibration by
measuring the resonance Raman spectra of organic solvents whose peak positions
were known (acetonitrile, dimethylformamide, \textit{n}-hexane,
\textit{n}-pentane, cyclohexane and acetic acid). \textcite{Yarasi2007}
performed frequency calibration by measuring Raman scattering of solvents of
known frequencies (acetonitrile, dimethylformamide, carbon tetrachloride,
ethanol, and pentane). \textcite{Knee2008} calibrated spectra against ethanol,
acetone and pentane. \textcite{Billinghurst2009} performed calibration by
measuring the Raman scattering of solvents for which the peak positions are
known (\textit{n}-pentane, cyclohexane, \textit{N,N}-dimethilformamide,
ethanol, acetonitrile, and acetic acid). \textcite{Jayanth2009} calibrated
spectra using known solvent bands (dimethyl formamide, acetonitrile,
trichloroethylene, isopropanol, indene, and cyclohexane). \textcite{Shaw2009}
used a spectrum of ethanol to calibrate the abscissa of each spectrum.
\textcite{Jayanth2011} carried out calibration using spectra of standard
solvents: dimethyl formamide, acetonitrile, trichloroethylene, isopropanol,
indene, and cyclohexane. \textcite{Oladepo2011} performed frequency calibration
by measuring the Raman scattering of solvents for which the peak positions were
known (cyclohexane, \textit{N,N}-dimethylformamide, methanol, acetonitrile,
dimethyl sylfoxide and acetic acid). \textcite{Billinghurst2012} performed
frequency calibration by measuring the Raman scattering of organic solvents for
which the peak positions are known (\textit{n}-pentane, cyclohexane,
dimethylformamide, ethanol, acetonitrile and acetic acid).
\textcite{Kumamoto2012} calibrated grating dispersion with the laser line
(0\,\icm), and Raman bands of boron nitride (1370\,\icm) and acetonitrile
(2249\,\icm). \textcite{Srivastava2008,Muntean2013} used Raman spectra of
dimethylsulfoxide as a standard for wavenumber calibration.
\textcite{Mondal2016} calibrated the recorded spectra using known bands of
standard solvents, e.g. acetonitrile, isopropanol, cyclohexane and dimethyl
formamide.


Calibration to the Raman spectrum of organic solvent has the advantage that the
exact wavelength of excitation beam typically to precision in the order of
$10^{-3}$\,nm does not need to be known.
The wavelength of the source also does not need to be stable to this precision
between measurements.
The calibration spectrum can also be measured from the same place and in the
same configuration as the sample, so there is no danger of some offsets caused
by changes in the geometrical alignment of the spectrometer, for example, when
the objective is moved while focusing the light gathering part of the
spectrometer on the sample or filters are changed in the gathered light path.
The significant disadvantage of calibration to the Raman spectra of organic
solvents is that their Raman bands are broad and overlapping, and therefore
their positions cannot be determined with sufficiently high precision.
We can achieve the precision of $\pm1$ \,\icm{} for well-resolved dominant and
a few \icm{} for shoulders or weak bands.
Organic solvents also do not usually have many bands in the region between
1800 and 2800\,\icm{} and do not have a sufficiently large number of usable
Raman lines, especially if solvents usable in deep UV are wanted.

We had a good experience with the other approach to calibration, which uses
atomic emission spectra of standard lamps.
Usually, as is also the case in our laboratory, the neon lamp is used.
However, this lamp does not cover the deep UV range of light, so we decided to
try some organic solvents for UV Raman calibration as shown in
\tabref{wavenumber_calibration:solvents}.

We selected cyclohexane because it has a pretty rich Raman spectrum.
We also wanted to solve the problem that most standard organic solvents do not
have many bands between 1800 and 2800\,\icm{}, so we decided to use deuterated
cyclohexane-d12, which nicely fills that region.
To our knowledge, the cyclohexane-d12 was not used for calibration of UV Raman
spectra of nucleic acids before.

The spectra of cyclohexane and cyclohexane-d12 were measured on a homemade
Raman spectrometer using visible excitation of 532\,nm
\parencite{Palacky2011}
and calibrated to the spectrum of neon calibration lamp. At least three and
typically eight spectra for each Raman line with different grating positions
were taken.
The top halves of the well-resolved peaks were fitted to the gaussian
curve, and their precise positions were estimated.
All the measured positions were then averaged.
The results can be seen in
\tabref{wavenumber_calibration:cyclohexane_wavenumbers}.

\begin{table}
	\centering
	\begin{tabular}{lr@{\,$\pm$\,}lr@{.}l}
\toprule
\multicolumn{1}{c}{name} & \multicolumn{2}{c}{$\bar{\nu}$ (\icm)}
	& \multicolumn{2}{c}{x (px)} \\
\midrule

D1   &  298.55 & 0.08 & 2002&68 \\
D2   &  373.77 & 0.06 & 1952&43 \\
H1   &  384.19 & 0.15 & 1943&28 \\
H2   &  426.75 & 0.13 & 1914&58 \\
D3a  &  636.15 & 0.09 & 1775&71 \\
D3   &  724.48 & 0.10 & 1715&56 \\
D5   &  795.93 & 0.09 & 1666&70 \\
H3   &  802.41 & 0.13 & 1660&07 \\
D6   &  939.10 & 0.12 & 1568&22 \\
D7   & 1014.07 & 0.12 & 1516&56 \\
H5   & 1028.91 & 0.14 & 1503&78 \\
D8   & 1073.70 & 0.11 & 1474&89 \\
D9   & 1120.05 & 0.10 & 1442&68 \\
H6   & 1158.77 & 0.13 & 1413&54 \\
D10  & 1214.12 & 0.08 & 1376&89 \\
H7   & 1267.25 & 0.11 & 1337&14 \\
H8   & 1444.81 & 0.11 & 1211&48 \\
D11  & 1706.70 & 0.04 & 1026&44 \\
H13  & 1720.30 & 0.06 & 1014&25 \\
D12  & 1753.42 & 0.04 &  992&68 \\
H13a & 1765.45 & 0.03 &  981&85 \\
D16  & 1979.06 & 0.06 &  828&31 \\
D17  & 2004.03 & 0.06 &  810&09 \\
H14  & 2052.09 & 0.03 &  772&50 \\
D18  & 2083.28 & 0.05 &  751&69 \\
D19  & 2106.41 & 0.06 &  734&77 \\
H14a & 2117.79 & 0.05 &  723&54 \\
D21  & 2154.45 & 0.05 &  699&13 \\
D23  & 2198.41 & 0.05 &  666&48 \\
D25  & 2322.52 & 0.04 &  574&26 \\
H15  & 2350.57 & 0.03 &  550&63 \\
H16  & 2634.46 & 0.03 &  336&50 \\
H20  & 2853.91 & 0.05 &  168&13 \\
D28  & 2885.28 & 0.05 &  146&18 \\
D29  & 2914.56 & 0.06 &  123&83 \\
H21  & 2924.74 & 0.04 &  113&34 \\
H22  & 2939.47 & 0.05 &  102&30 \\

\bottomrule
\end{tabular}

	\caption[%
		Estimated wavenumbers $\tilde{\nu}$ of cyclohexane and cyclohexane-d12.%
	]{%
		\captiontitle{%
			Estimated wavenumbers $\tilde{\nu}$ of cyclohexane (H) and
			cyclohexane-d12 (D) with the corresponding pixel positions $x$ from UV
			Raman measurement.%
		}
	}
	\label{\tablabel{wavenumber_calibration:cyclohexane_wavenumbers}}
\end{table}

The spectrum of cyclohexane and cyclohexane-d12 can then be measured after each
UV Raman measurement in such a way that the measured sample is replaced by the
calibration sample so that the calibration samples are measured from the same
place as the measured sample (with the same grating position of spectrograph).
The precise pixel positions of their Raman lines were estimated from the
spectrum in the same way as in the visible Raman measurements.
These pixel positions can then be matched to known approximate pixel positions
of cyclohexane and cyclohexane-d12 using an appropriate weight function which
favors matches with more bands by moving the measured spectrum in the table
with calibration spectrum with, for example, 0.5 px step.
The outcome of this procedure is calibration line wavenumbers and their pixel
positions.
They can be fitted by a 3rd-degree polynomial to assign the wavenumbers to all
the 2048 detector pixels.
The spectrum is then linearly interpolated to have 1\,\icm{} steps.
Example UV Raman spectra of cyclohexane and cyclohexane-d12 calibrated by our
homemade calibration program with highlighted band positions are shown in
\figref{wavenumber_calibration:cyclohexane_spc}.

\begin{figure}
	\centering
	\input{results_and_discussion/assets/cyclohexane_calibration/%
cyclohexane_calibration}
	\caption[%
		UV Raman spectrum of cyclohexane and cyclohexane-d12 with 244\,nm
		excitation.%
	]{%
		\captiontitle{%
			UV Raman spectrum of cyclohexane and cyclohexane-d12 with 244\,nm
			excitation.%
		}
		Grey lines denote the positions of Raman lines used for calibration (see
		\tabref{wavenumber_calibration:cyclohexane_wavenumbers}).
	}
	\label{\figlabel{wavenumber_calibration:cyclohexane_spc}}
\end{figure}

As was written above, a different method for sub-\icm{} calibration precision
than calibration to organic solvents needs to be used.
Some authors proposed calibration to atomic lines of a mercury lamp.
For example,
\textcite{%
	Manoharan1990,%
	Efremov1991%
}
calibrated the spectrograph with the 253.6, 312.6, and 365 nm lines of a
low-pressure Hg lamp.
The disadvantage of low-pressure Hg-lamp usage is that its lines are relatively
sparse so other approaches, which combined the Hg spectrum with other methods
were proposed.
For example, \textcite{Myers1988} calibrated wavenumbers using cyclohexane
combined with various \ch{H2} Raman-shifted laser lines and mercury emission
lines.

We tried to use a mercury lamp too, but it did not bring any new precision to
the calibrated spectra, so we searched for the different calibration sources.
There was a similar effort in
\textcite{Wert2014},
who proposed calibration with zinc (202.5, 206.2, and 213.9\,nm) and cadmium
(214.4, 226.5, and 228.8\,nm) standard lamps and Positive Light
Indigo-S 210 -- 240\,nm tunable Ti:Sa laser if necessary.
However, we pursued a different approach and took inspiration from the
calibration of space instruments which broadly use hollow cathode platinum
lamps
\parencite{%
	Mount1977,%
	Reader1990,%
	Sansonetti1992%
}
for calibration in UV spectral range.

\emph{Hollow cathode lamps} (HCL) are routinely used in \emph{atomic absorption
spectroscopy} (AAS), and therefore they are commercially available on the
market together with power supplies because HCL requires a precision current
source.
We used P209 HCL Power Supply (Photron) with P840 Hollow Cathode Lamp -- Pt
(Photron) which provided a stable platinum atomic spectrum right after start.
The new spectrometer schema with calibration source is in
\figref{wavenumber_calibration:apparatus_schema}.
Right-angle prisms (MC1 and MC2) guided the calibration beam in total internal
reflection mode.
We chose prisms instead of mirrors because they are much more stable in time,
and we do not need to have any concerns about their aging.
Prism MC2 was placed on a motorized flip mount to easily switch between
measurement of calibration spectra and Raman spectra from samples.
The calibration beam was collimated by the lens LC1 with focal length of
10\,cm (Thorlabs) and could be attenuated by the aperture AC1 represented by
the iris diaphragm (Thorlabs).

\begin{figure}
	\centering
	\input{results_and_discussion/assets/calibration_schema}
	\caption[%
		Top-view schema of the apparatus with wavenumber calibration lamp
		and with side-view inset of the sample space.%
	]{%
		\captiontitle{%
			Top-view schema of the apparatus with wavenumber calibration lamp
			and with side-view inset of the sample space.%
		}
		The calibration beam from Pt \emph{hollow cathode lamp} (HCL) is collimated
		by calibration beam lens LC1 and guided by the calibration right-angle
		prisms (MC1 and MC2) in total internal reflection mode and and focused on
		the entrance slit of spectrograph by the parabolic mirror MS2 the same way
		as the scattered signal from samples.
		The calibration lamp signal can be attenuated by modifying the size of the
		calibration beam aperture AC1.
		The mirror MC2 was placed on a motorized flip mount to enable easy
		insertion for calibration spectra measurement.
		The explanation of the rest of the symbols is the same as in
		\figref{initial_layout:apparatus_schema}.
	}
	\label{\figlabel{wavenumber_calibration:apparatus_schema}}
\end{figure}

\begin{figure}
	\centering
	\input{results_and_discussion/assets/pt_calibration/pt_calibration}
	\caption[%
		UV Raman spectrum of platinum hollow cathode lamp.%
	]{%
		\captiontitle{%
			UV Raman spectrum of platinum hollow cathode lamp with the same grating
			position as in
			\figref{wavenumber_calibration:cyclohexane_spc}.%
		}
		Grey lines denote the positions of Raman lines used for calibration.
	}
	\label{\figlabel{wavenumber_calibration:pt_spc}}
\end{figure}

As can be seen in
\figref{wavenumber_calibration:pt_spc},
the Pt lamp has many sharp well-resolved lines with reasonable intensity.
The spectrum acquisition is also relatively fast; the spectrum was usually
acquired as an average of 100 of 0.1s scans.
Because of these parameters, the measured Raman spectra can be calibrated with
a precision as high as $\pm0.1$\,\icm.
