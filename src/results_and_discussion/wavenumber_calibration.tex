\subsection{Wavenumber calibration}

For correct Raman spectra interpretation we needed precise wavenumber
calibration. It can be done by measuring spectrum with known line positions
in the same spectrometer configuration as for the spectrum of the sample and
then interpolating and extrapolating the wavenumbers from these known line
positions to all detector pixels. One group of calibration spectra are Raman
spectra of organic solvents but you can't use pure indene which is widely
used in visible Raman because it strongly absorbs in UV and its lines can be
shifted if you dilute it. So other solvents need to be found. Organic
solvents were used as wavenumber calibration for first UV Raman spectra
measurements, for example \textcite{Harada1975} calibrated to Raman line of
\ch{CCl4}. But early, it was clear, that single organic solvent is not
satisfactory and therefore some teams started to use mixtures of solvents.
If you use mixture of solvents, you don't know, how the components influece
each other, they can evaporate in different rate so their ratio can be
changing which can shift their bands. You can solve this problem by measuring
the solvents separately but then you prolong the time spend for calibration
spectra measurement. Some selected solvents are listed in
\tabref{wavenumber_calibration:solvents}.

\begin{table}
	\centering
	\begin{tabular}{m{8.7cm}l}
\toprule
solvent & citation \\
\midrule
\ch{CCl4}                                   & \textcite{Harada1975} \\
\addlinespace[.3em]
cyclohexane                                 & \textcite{Myers1988} \\
\addlinespace[.3em]
acetonitrile                                & \textcite{Gustafson1988} \\
\addlinespace[.3em]
ethanol                                     & \textcite{Su1990} \\
\addlinespace[.3em]
ethyl acetate/dioxane (2:1 v/v)             & \textcite{Toyama1991} \\
\addlinespace[.3em]
cyclohexane, acetone                        & \textcite{Kaminaka1992} \\
\addlinespace[.3em]
acetonitrile, acetone                       & \textcite{Benson1992} \\
\addlinespace[.3em]
cyclohexanone/acetonitrile (1:1 v/v)        & \textcite{Hashimoto1993} \\
\addlinespace[.3em]
ethanol, \textit{n}-penthane                & \textcite{Mukerji1995} \\
\addlinespace[.3em]
dioxane, \ch{CCl4}, acetonitrile            & \textcite{Russell1995} \\
\addlinespace[.3em]
ethyl acetate/dioxane (2:1 v/v) (257\,nm)
                                            & \textcite{Fujimoto1998} \\
\addlinespace[.3em]
cyclohexanone/acetonitrile (1:1 v/v) (213\,nm)
                                            & \textcite{Fujimoto1998} \\
\addlinespace[.3em]
acetone, ethanol, pentane                   & \textcite{Mukerji1998} \\
\addlinespace[.3em]
\textit{n}-pentane, cyclehexane, dimethylformamide, ethanol, acetonitrile,
acetic acid                                 & \textcite{Billinghurst2006} \\
\addlinespace[.3em]
dimethylsulfoxide                           & \textcite{Srivastava2008} \\
\addlinespace[.3em]
dimethyl formamide, acetonitrile, trichloroethylene, isopropanol, indene,
cyclohexane                                 & \textcite{Jayanth2009} \\
\addlinespace[.3em]
cyclohexane, \textit{N,N}-dimethylformamide, methanol, acetonitrile,
dimethyl sylfoxide, acetic acid             & \textcite{Oladepo2011} \\
\addlinespace[.3em]
acetonitrile, isopropanol, cyclohexane and dimethyl formamide
                                            & \textcite{Mondal2016} \\
\bottomrule
\end{tabular}

	\caption{%
		Selection of organic solvents which were used for UV Raman spectra
		wavenumber calibration in literature.%
	}
	\label{\tablabel{wavenumber_calibration:solvents}}
\end{table}

Some laboratories were not satisfied with the separate calibration and used
also internal standard to improve the calibration of spectra. For example,
\textcite{Wen1998} used \ch{Na2SO4} as internal frequency (981\,\icm) standard,
and Raman frequencies were calibrated to $\pm1 $\,\icm{} using
\ch{ccl4}/\ch{CH3CN} mixture.

The other approach was also to use the elastically scattered laser line.
 \textcite{Kumamoto2012} calibrated grating dispersion with the laser line
(0\,\icm), and Raman bands of boron nitride (1370\,\icm) and acetonitrile
(2249\,\icm).

%\textcite{Myers1988} calibrated wavenumbers using cyclohexane in
combination with various \ch{H2} Raman-shifted laser lines and mercury
emission lines. \textcite{Su1990} calibrated Raman spectra in frequency with
ethanol. \textcite{Toyama1991, Toyama1996} used for wavenumber calibration
the Raman bands of ethyl acetate/dioxane (2:1 v/v), which had been measured
with visible-laser excitation on a spectrometer calibrated with indene Raman
bands and neon emission lines. \textcite{Kaminaka1992} calibrated Raman
shifts with cyclohexane and acetone as the secondary standard, while their
Raman bands were calibrated with indene, with the use of visible excitation.
\textcite{Benson1992} calibrated the wavenumber shift with the use of the
Raman spectra of acetonitrile and acetone. \textcite{Benson1993} used carbon
tetrachloride for wavenumber shift calibration standard.
\textcite{Hashimoto1993} made wavenumber calibration with the Raman bands of
cyclohexanone/acetonitrile (1:1 v/v). \textcite{Mukerji1995} calibrated spectra
against neat solution of ethanol and \textit{n}-penthane.
\textcite{Russell1995} performed wavenumber calibration in UV-laser excited
Raman spectra using the well known Raman lines of dioxane, carbon
tetrachloride, and acetonitrile. \textcite{Gustafson1988} calibrated
wavenumber shift using the Raman spectrum of acetonitrile.
\textcite{Leonard1994} calibrated wavenumber shift with the use of the Raman
spectrum of cyclohexane. \textcite{Takeuchi1990} made wavenumber calibration
by using the Raman bands of cyclohexanone/acetonitrile (1:1, v/v), the
wavenumber of which had been established with visible-laser excitation on a
spectrometer calibrated with indene Raman bands and neon emission lines.
\textcite{Takeuchi1995} calibrated wavenumber axis of the spectra by using
the spectrum of a mixture solution of ethyl acetate/dioxane (2:1 v/v) as a
wavenumber standard. \textcite{Tuma1995} performed wavenumber
calibration in UVRR spectra using well-known Raman lines of dioxane, carbon
tetrachloride, and acetonitrile. \textcite{Wen1997} calibrated Raman
frequencies using a standard liquid mixture of carbon tetrachloride and
acetonitrile. \textcite{Fujimoto1998} made wavenumber calibration of the
spectra by use of the Raman bands of a 2:1 (v/v) mixture of ethyl
acetate/1,4-dioxane (257\,nm excitation) or of a 1:1 (v/v) mixture of
cyclohexanone/acetonitrile (213\,nm). \textcite{Mukerji1998} calibrated data
with acetone, ethanol, and pentane. \textcite{Wen1998} used \ch{Na2SO4} as
internal frequency (981\,\icm) standard, and Raman frequencies were
calibrated to $\pm1 $\,\icm{} using \ch{ccl4}/\ch{CH3CN} mixture.
\textcite{Suen1999} calibrated against neat solutions of ethanol and acetone,
for which the Raman band frequencies were known. \textcite{Toyama1999}
performed wavenumber calibraion by use of the Raman bands of
ethylacetate--dioxane (2:1, v/v). \textcite{Wen1999} calibrated Raman
frequencies by using a standarad liquid mixture of carbon tetrachloride and
acetonitrile. \textcite{Sokolov2000} calibrated data with acetone, ethanol and
pentane. \textcite{Toyama2001} made wavenumber calibration of Raman spectra
by using the Raman spectrum of a 2:1 (v/v) mixture of ethylacetate/1,4-dioxane.
\textcite{Fujimoto2002} calibrated Raman spectrometer by using the Raman bands
of a 2:1 (v/v) mixture of ethylacetate and dioxane for 257-nm excitation or a
1:1 (v/v) mixture of cyclohexanone and acetonitrile for 229-nm excitation.
\textcite{Toyama2002} performed wavenumber calibration of Raman spectra by use
of the Raman bands of a 2:1 (v/v) mixture of ethyl acetate and 1,4-dioxane.
\textcite{Nelson2004} calibrated the Raman shift axis with pure ethanol.
\textcite{Toyama2005,Toyama2005a} made wavenumber calibration of Raman spectra
by use of Raman bands of a 2:1 (v/v) mixture of ethyl acetate and 1,4-dioxane.
\textcite{Billinghurst2006,Billinghurst2006a} perforemd frequency calibration
by measuring the Raman scattering of organic solvents for whom the peak
positions are known (\textit{n}-pentane, cyclehexane, dimethylformamide,
ethanol, acetonitrile, and acetic acid). \textcite{Jirasek2006} wavenumber
calibrated abscissa of each acquired spectrum using an ethanol spectrum as the
calibration standard. \textcite{Kundu2007} performed frequency calibration by
measuring the resonance Raman spectra of organic solvents whose peak positions
were known (acetonitrile, dimethylformamide, \textit{n}-hexane,
\textit{n}-pentane, cyclohexane and acetic acid). \textcite{Yarasi2007}
performed frequency calibration by measuring Raman scattering of solvents of
known frequencies (acetonitrile, dimethylformamide, carbon tetrachloride,
ethanol, and pentane). \textcite{Knee2008} calibrated spectra against ethanol,
acetone and pentane. \textcite{Billinghurst2009} performed calibration by
measuring the Raman scattering of solvents for which the peak positions are
known (\textit{n}-pentane, cyclohexane, \textit{N,N}-dimethilformamide,
ethanol, acetonitrile, and acetic acid). \textcite{Jayanth2009} calibrated
spectra using known solvent bands (dimethyl formamide, acetonitrile,
trichloroethylene, isopropanol, indene, and cyclohexane). \textcite{Shaw2009}
used a spectrum of ethanol to calibrate the abscissa of each spectrum.
\textcite{Jayanth2011} carried out calibration using spectra of standard
solvents: dimethyl formamide, acetonitrile, trichloroethylene, isopropanol,
indene, and cyclohexane. \textcite{Oladepo2011} performed frequency calibration
by measuring the Raman scattering of solvents for which the peak positions were
known (cyclohexane, \textit{N,N}-dimethylformamide, methanol, acetonitrile,
dimethyl sylfoxide and acetic acid). \textcite{Billinghurst2012} performed
frequency calibration by measuring the Raman scattering of organic solvents for
which the peak positions are known (\textit{n}-pentane, cyclohexane,
dimethylformamide, ethanol, acetonitrile and acetic acid).
\textcite{Kumamoto2012} calibrated grating dispersion with the laser line
(0\,\icm), and Raman bands of boron nitride (1370\,\icm) and acetonitrile
(2249\,\icm). \textcite{Srivastava2008,Muntean2013} used Raman spectra of
dimethylsulfoxide as a standard for wavenumber calibration.
\textcite{Mondal2016} calibrated the recorded spectra using known bands of
standard solvents, e.g. acetonitrile, isopropanol, cyclohexane and dimethyl
formamide.


Calibration to the Raman spectrum of organic solvent has an advantage that
you don't need to know the exact wavelength of excitation beam typically to
precision in the order of $10^{-3}$\,nm. The wavelength of the source also
does not need to be stable to this precision between measurements. You can
also measure the calibration spectrum from the same place and in the same
configuration as the sample so there is no danger of some offsets caused by
changes in the geometrical alignment of spectrometer, for example when you
move with objective during focusing the light gathering part of the
spectrometer on the sample or change filters in the gathered light path. The
big disadvantage of calibration to the Raman spectra of organic solvents is
that their Raman bands are broad and overlapping and therefore their
positions can't be determined with sufficiently high precision. We can
achieve precision of $\pm1$ \,\icm{} for well-resolved dominant and a few
\icm{} for shoulders or weak bands. Organic solvents also do not usually have
much bands in the region between 1800 and 2800\,\icm{} and don't have
sufficiently large number of usable Raman lines especially if you want
solvents which are usable in deep UV.

We had good experience with the other approach to calibration which uses
atomic emission spectra of some standard lamp. Usually, as it is also the
case in our laboraty, the neon lamp is used. But this lamp do not cover deep
UV range of light so we decided to try some organic solvents for UV Raman
calibration as they were broadly used as you can see from the
\tabref{wavenumber_calibration:solvents}.

We selected cyclohexane because it has pretty rich Raman spectrum. We also
wanted to solve the problem, that most of the standard organic solvents
does not have much bands in the region between 1800 and 2800\,\icm{} so we
decided that we also use deuterated cyclohexane-d12 which nicely fills that
region. To our knowledge, the cyclohexane-d12 was not used for calibration of
UV Raman spectra of nucleic acids before.

The spectra of cyclohexane and cyclohexane-d12 was measured on home-made Raman
spectrometer using visible excitation of 532\,nm \REFERENCE{}
\textcolor{red}{(description should be in materials and methods)}
and calibrated to the spectrum of neon calibration lamp. At least 3 and
typically 8 spectra for each Raman line with different grating position were
taken. The top half of the well resolved peaks was fitted to gaussian
curve and its precise postion was estimated. All the measured positions were
then averaged. The results can be seen in
\tabref{wavenumber_calibration:cyclohexane_wavenumbers}.

\begin{table}
	\centering
	\begin{tabular}{lr@{\,$\pm$\,}lr@{.}l}
\toprule
\multicolumn{1}{c}{name} & \multicolumn{2}{c}{$\bar{\nu}$ (\icm)}
	& \multicolumn{2}{c}{x (px)} \\
\midrule

D1   &  298.55 & 0.08 & 2002&68 \\
D2   &  373.77 & 0.06 & 1952&43 \\
H1   &  384.19 & 0.15 & 1943&28 \\
H2   &  426.75 & 0.13 & 1914&58 \\
D3a  &  636.15 & 0.09 & 1775&71 \\
D3   &  724.48 & 0.10 & 1715&56 \\
D5   &  795.93 & 0.09 & 1666&70 \\
H3   &  802.41 & 0.13 & 1660&07 \\
D6   &  939.10 & 0.12 & 1568&22 \\
D7   & 1014.07 & 0.12 & 1516&56 \\
H5   & 1028.91 & 0.14 & 1503&78 \\
D8   & 1073.70 & 0.11 & 1474&89 \\
D9   & 1120.05 & 0.10 & 1442&68 \\
H6   & 1158.77 & 0.13 & 1413&54 \\
D10  & 1214.12 & 0.08 & 1376&89 \\
H7   & 1267.25 & 0.11 & 1337&14 \\
H8   & 1444.81 & 0.11 & 1211&48 \\
D11  & 1706.70 & 0.04 & 1026&44 \\
H13  & 1720.30 & 0.06 & 1014&25 \\
D12  & 1753.42 & 0.04 &  992&68 \\
H13a & 1765.45 & 0.03 &  981&85 \\
D16  & 1979.06 & 0.06 &  828&31 \\
D17  & 2004.03 & 0.06 &  810&09 \\
H14  & 2052.09 & 0.03 &  772&50 \\
D18  & 2083.28 & 0.05 &  751&69 \\
D19  & 2106.41 & 0.06 &  734&77 \\
H14a & 2117.79 & 0.05 &  723&54 \\
D21  & 2154.45 & 0.05 &  699&13 \\
D23  & 2198.41 & 0.05 &  666&48 \\
D25  & 2322.52 & 0.04 &  574&26 \\
H15  & 2350.57 & 0.03 &  550&63 \\
H16  & 2634.46 & 0.03 &  336&50 \\
H20  & 2853.91 & 0.05 &  168&13 \\
D28  & 2885.28 & 0.05 &  146&18 \\
D29  & 2914.56 & 0.06 &  123&83 \\
H21  & 2924.74 & 0.04 &  113&34 \\
H22  & 2939.47 & 0.05 &  102&30 \\

\bottomrule
\end{tabular}

	\caption{%
		Estimated wavenumbers $\tilde{\nu}$ of cyclohexane (H) and cyclohexane-d12
		(D) with the corresponding pixel positions $x$ from UV Raman measurement.%
	}
	\label{\tablabel{wavenumber_calibration:cyclohexane_wavenumbers}}
\end{table}

Spectrum of cyclohexane and cyclohexane-d12 can be then measured after each
UV Raman measurement in such a way, that measured sample is replaced by
calibration sample so that the calibration samples are measured from the same
place as the measured sample (with the same grating position of spectrograph).
The precise pixel positions of their Raman lines was estimated from the
spectrum in the same way as in the visible Raman measurements. These pixel
positions can be then matched to known approximate pixel positions of
cyclohexane and cyclohexane-d12 using appropriate weight function which
favors matches with more bands then less by moving the measured spectrum along
the table with calibration spectrum with for example 0.5-px step. The outcome
of this step is calibration lines wavenumbers in dependence on their pixel
positions. They can be fitted by 3rd degree polynomial to assign the
wavenumbers to all the 2048 detector pixels. The spectrum is then linearly
interpolated to have 1-\icm{} steps. Example UV Raman spectra of cyclohexane
and cyclohexane-d12 calibrated by our home-made calibration program with
highlighted band positions can be seen in
\figref{wavenumber_calibration:cyclohexane_spc}.

\begin{figure}
	\centering
	\input{results_and_discussion/assets/cyclohexane_calibration/%
cyclohexane_calibration}
	\caption{%
		UV Raman spectrum of cyclohexane and cyclohexane-d12 with 244-nm
		excitation. Grey lines denotes the positions of Raman lines used for
		calibratin (see
		\tabref{wavenumber_calibration:cyclohexane_wavenumbers}).%
	}
	\label{\figlabel{wavenumber_calibration:cyclohexane_spc}}
\end{figure}

As was written above, for sub-\icm{} calibration precision, the different
method then calibration to organic solvents needs to be used. Some authors
proposed calibration to atomic lines of mercury lamp, for example
\textcite{Manoharan1990,Efremov1991} calibrated the spectrograph with the
253.6-, 312.6-, and 365-nm lines of a low-pressure Hg lamp. The disadvantage
of usage of low pressure Hg-lamp is that its lines are rather sparse so
other approaches, which combined the Hg spectrum with other methods were
proposed. For example, \textcite{Myers1988} calibrated wavenumbers using
cyclohexane in combination with various \ch{H2} Raman-shifted laser lines and
mercury emission lines.

We tried to use mercury lamp too but it didn't bring any new precision to the
calibrated spectra so we tried to search for the different calibration source.
There was similar effort in \textcite{Wert2014}, who proposed calibration with
zinc (202.5, 206.2, and 213.9\,nm) and cadmium (214.4, 226.5, and 228.8\,nm)
standard lamps and Positive Light Indigo-S 210 -- 240\,nm tunable Ti:Sa laser
if necessary. But we pursued different approach and took inspiration from
calibration of space instruments which broadly use hollow cathode platinum
lamp \parencite{Mount1977,Reader1990,Sansonetti1992} for calibration in UV
spectral range.

\emph{Hollow cathode lamps} (HCL) are routinely used in \emph{atomic absorption
spectroscopy} (AAS) and therefore they are commercially available on the market
together with power supplies because HCL requires precision current source. We
used P209 HCL Power Supply (Photron) with P840 Hollow Cathode Lamp -- Pt
(Photron) which provided stable platinum atomic spectrum right after start. The
new spectrometer schema with calibration source is in
\figref{wavenumber_calibration:aparatus_schema}.
The callibration beam was guided by right-angle prisms in total internal
reflection mode (MC1 and MC2). We chose prisms instead of mirrors because they
are much more stable in time and we don't need to have any concerns about their
aging. Prism MC2 was placed on motorized flip mount so we can easily switch
between measurement of calibration spectra and Raman spectra from samples.
The calibration beam was colimated by the lens LC1 with focal length of
10\,cm (Thorlabs) and could be attenuated by the aperture AC1 which was
represented by iris diaphragm (Thorlabs).

\begin{figure}
	\centering
	\begin{tikzpicture}[font=\sffamily]

% settings
\newcommand*{\cellBorderWidth}{3\pgflinewidth}
\newcommand*{\cellLineWidth}{1.5\pgflinewidth}
\definecolor{glassBorderColor}{RGB}{0,128,255}
\definecolor{glassFillColor}{RGB}{230,242,255}
\definecolor{waterFillColor}{RGB}{0,128,255}
\definecolor{hclColor}{RGB}{255,128,0}
\tikzset{
	clip
}
\tikzset{
	mirror element/.style={color=black, line width=2*\pgflinewidth},
	real laser beam/.style={color=cyan, line width=2*\pgflinewidth},
	laser beam/.style={color = black, opacity = 0.2,%
		line width = 2*\pgflinewidth},
	scattered ray/.style={color = black, opacity = 0.2,%
		line width = 1.5*\pgflinewidth},
	scattered fill/.style={fill=black, draw=none, fill opacity=0.1},
	glass/.style={color=glassBorderColor, opacity=0.5,%
		fill=glassFillColor, fill opacity=0.5},
	sample cell/.style={color=glassBorderColor, opacity=0.5,%
		double=glassFillColor, double distance=\cellBorderWidth,
		line width=\cellLineWidth, line cap=rect},
	water fill/.style={fill=waterFillColor, fill opacity=0.1},
	mirror surface/.style={color=black!20, fill=black!10},
	nd filter/.style={color=black, opacity=0.2, fill=black, fill opacity=0.1},
	nd carousel/.style={color=black, opacity=0.4, fill=black, fill opacity=0.2},
	notch/.style={color=black, opacity=0.4, fill=black, fill opacity=0.1},
	aperture/.style={color=black, line width=2*\pgflinewidth},
	aperture filldraw/.style={color=black!40, fill=black!20, clip even odd rule},
	clip even odd rule/.code={\pgfseteorule},
	shutter/.style={nd carousel},
	shutter blade/.style={solid, line width = 1.5*\pgflinewidth, opacity = 0.4},
	hcl/.style={nd carousel},
	hcl lamp/.style = {color = hclColor, line width = 3 * \pgflinewidth,%
		line cap = round},
	hcl ray/.style = {color = hclColor, opacity = 0.4,%
		line width = 1.5 * \pgflinewidth},
	hcl beam/.style = {draw = none, fill = hclColor, opacity = 0.3}
};
\clip (-.1,-2.1) rectangle (14,6.6);
\coordinate (shutter) at (11,6);
\newcommand*{\samplePosWidth}{10}
\newcommand*{\samplePosHeight}{3}
\coordinate (samplePos) at (\samplePosWidth,\samplePosHeight);
\newcommand*{\cellWidth}{1};
\coordinate (cassegrainM1Center) at (\samplePosWidth - 0.5,\samplePosHeight);
\newcommand*{\cassegrainMARadius}{1.5}
\coordinate (cassegrainM2Center) at (\samplePosWidth - 0.7,\samplePosHeight);
\newcommand*{\cassegrainMBRadius}{0.4}
\newcommand*{\sqrttwo}{1.414213}
\coordinate (HCL) at (6,0);
\coordinate (AC1) at (6,1.2);  % calibration aperture
\coordinate (LC1) at (6,0.6);  % calibration lens
\coordinate (MS1Edge1) at (4.5,\samplePosHeight + 0.5);
\coordinate (MS1Edge2) at (3.5,\samplePosHeight - 0.5);
\coordinate (MC1Edge1) at (5.5,2.5);
\coordinate (MC1Edge2) at (6.5,1.5);
\coordinate (MC1Edge3) at (5.5,1.5);
\coordinate (MC2Edge1) at (4.5,2.5);
\coordinate (MC2Edge2) at (3.5,1.5);
\coordinate (MC2Edge3) at (4.5,1.5);
\coordinate (parabolaFocus) at (3,0.4);
\coordinate (MS2Edge1) at (4.5,1);
\coordinate (MS2EdgeControl1) at (245:0.5);
\coordinate (MS2Edge2) at (3.5,0);
\coordinate (MS2EdgeControl2) at (17:0.3);
\newcommand*{\apertureOuterRadius}{0.3}
\newcommand*{\apertureInnerRadius}{0.1}

% laser
\draw (0,5.5) rectangle ++(3,1) node[pos=.5] {laser};

% Mirror 3 - it should be under the beam so we have to draw it first
\draw[mirror surface] ($ (samplePos) + (-0.3,-\sqrttwo * 0.3) $)
	rectangle ++(0.6,\sqrttwo*0.6);
\node[above] at ($ (samplePos) + (0,0.5) $) {M3};

% laser beam
\draw[real laser beam] (3,6) -- (shutter);
\draw[laser beam] (shutter) -- (13,6) coordinate (M1)
	-- (13,\samplePosHeight) coordinate (M2) -- (samplePos);

% aperture 2
\draw[aperture filldraw]
	(samplePos) circle (\apertureOuterRadius)
	(samplePos) circle (\apertureInnerRadius);

% neutral density filters
\newcommand*{\ndfilterA}{(3.5,5.8) rectangle ++(0.2,0.4)}
\newcommand*{\ndfilterB}{(3.5,5) rectangle ++(0.2,0.4)}
\draw[nd filter] \ndfilterA;
\draw[nd filter] \ndfilterB;
\draw[nd carousel] (3.45,4.9) rectangle ++(0.3,0.1);
\draw[nd carousel] (3.45,5.4) rectangle ++(0.3,0.4);
\draw[nd carousel] (3.45,6.2) rectangle ++(0.3,0.1);
\node[below] at (3.6,4.9) {ND};

\draw[shutter blade] ($ (shutter) + (0,0.3) $) -- ++(0,-0.6);
\draw[shutter] ($ (shutter) + (-0.1,-0.3) $) -- ++(0.2,0) -- ++(0,-0.4)
	-- ++(-0.2,0) -- cycle;
\node[below] at ($ (shutter) + (0,-0.7) $) {shutter};

% Mirror 1 and 2
\draw[mirror element] ($ (M1) + (-0.3,0.3) $)
	-- node[above,shift={(0.35cm,-0.05cm)}]{M1} ($ (M1) + (0.3,-0.3) $);
\draw[mirror element] ($ (M2) + (0.3,0.3) $)
	-- node[below,shift={(0.35cm,0.05cm)}]{M2} ($ (M2) + (-0.3,-0.3) $);

% Aperture 1
\draw[aperture] ($ (M2) + (-0.5,\apertureOuterRadius) $)
	node[above]{A1}
	-- ++(0,-\apertureOuterRadius + \apertureInnerRadius);
\draw[aperture] ($ (M2) + (-0.5,-\apertureOuterRadius) $)
	-- ++(0,\apertureOuterRadius - \apertureInnerRadius);

% Draw laser focusing lens
\newcommand*{\LARadius}{0.7}
\coordinate (L1Center) at (\samplePosWidth+1.6-\LARadius,\samplePosHeight);
\path[name path=L1Arc, shift={(L1Center)}]
	(270:\LARadius) arc (-90:90:\LARadius);
\path[name path=toL1Arc1] ($ (samplePos)  + (0,0.4) $) -- ++(10,0);
\path[name path=toL1Arc2] ($ (samplePos)  + (0,-0.4) $) -- ++(10,0);
\path[name intersections={of=L1Arc and toL1Arc1, by=L11}];
\mypgfextractangle{\LAAAngle}{L1Center}{L11}
\path[name intersections={of=L1Arc and toL1Arc2, by=L12}];
\mypgfextractangle{\LABAngle}{L1Center}{L12}
\draw[glass] (L11) arc (\LAAAngle:\LABAngle-360:\LARadius) -- ++(-0.05,0)
	-- ($ (L11) + (-0.05,0) $) -- cycle;
\node[above] at (L11) {L1};


%%%%%%%%%%%%%%%%%%%%%
% draw the cassegrain

% calculate intersections with mirror 1 (the objective mirror)
% mirror1 arc
\path[name path=M1arc, shift={(cassegrainM1Center)}]
	(90:\cassegrainMARadius) arc (90:270:\cassegrainMARadius);
% upper top ray
\path[name path=toCassegrainM11] (samplePos) -- ++(135:5);
\path[name intersections={of=M1arc and toCassegrainM11, by=cassegrainM11}];
\mypgfextractangle{\cassegrainMAAAngle}{cassegrainM1Center}{cassegrainM11}
% upper bottom ray
\path[name path=toCassegrainM12] (samplePos) -- ++(165:5);
\path[name intersections={of=M1arc and toCassegrainM12, by=cassegrainM12}];
\mypgfextractangle{\cassegrainMABAngle}{cassegrainM1Center}{cassegrainM12}
% lower top ray
\path[name path=toCassegrainM13] (samplePos) -- ++(195:5);
\path[name intersections={of=M1arc and toCassegrainM13, by=cassegrainM13}];
\mypgfextractangle{\cassegrainMACAngle}{cassegrainM1Center}{cassegrainM13}
% lower bottom ray
\path[name path=toCassegrainM14] (samplePos) -- ++(225:5);
\path[name intersections={of=M1arc and toCassegrainM14, by=cassegrainM14}];
\mypgfextractangle{\cassegrainMADAngle}{cassegrainM1Center}{cassegrainM14}

\draw[scattered ray] (samplePos) -- (cassegrainM11);
\draw[scattered ray] (samplePos) -- (cassegrainM12);
\draw[scattered fill] (samplePos) -- (cassegrainM11)
	arc (\cassegrainMAAAngle:\cassegrainMABAngle:\cassegrainMARadius) -- cycle;
\draw[scattered ray] (samplePos) -- (cassegrainM13);
\draw[scattered ray] (samplePos) -- (cassegrainM14);
\draw[scattered fill] (samplePos) -- (cassegrainM13)
	arc (\cassegrainMACAngle:\cassegrainMADAngle:\cassegrainMARadius) -- cycle;

% draw the cell
\draw[sample cell, water fill]
	($ (samplePos)%
		+ (-0.5 * \cellBorderWidth - \cellLineWidth,-0.5 * \cellWidth) $)
		rectangle ++(\cellWidth,\cellWidth);
\node[below] at ($ (samplePos) + (0.5 * \cellWidth,-0.5 * \cellWidth)%
	+ (-0.5 * \cellBorderWidth,0) + (-0.5 * \cellLineWidth,0) $) {S};

% calculate intersections with mirror 2 (the objective mirror)
% mirror2 arc
\path[
	name path=M2arc, shift={(cassegrainM2Center)}] (90:\cassegrainMBRadius)
		arc (90:270:\cassegrainMBRadius);
% calculate cassegrain mirror2 edges
\path[
	name intersections={of=M2arc and toCassegrainM12, by=cassegrainM2Edge1}];
\mypgfextractangle{\cassegrainMBAAngle}{cassegrainM2Center}{cassegrainM2Edge1}
\path[
	name intersections={of=M2arc and toCassegrainM13, by=cassegrainM2Edge2}];
\mypgfextractangle{\cassegrainMBDAngle}{cassegrainM2Center}{cassegrainM2Edge2}
% upper bottom ray
\path[name path=toCassegrainM22] ($ (samplePos) + (0,0.1) $) -- ++(-10,0);
\path[name intersections={of=M2arc and toCassegrainM22, by=cassegrainM22}];
\mypgfextractangle{\cassegrainMBBAngle}{cassegrainM2Center}{cassegrainM22}
% lower top ray
\path[name path=toCassegrainM23] ($ (samplePos) + (0,-0.1) $) -- ++(-10,0);
\path[name intersections={of=M2arc and toCassegrainM23, by=cassegrainM23}];
\mypgfextractangle{\cassegrainMBCAngle}{cassegrainM2Center}{cassegrainM23}

\draw[scattered ray] (cassegrainM11) -- (cassegrainM2Edge1);
\draw[scattered ray] (cassegrainM12) -- (cassegrainM22);
\draw[scattered fill] (cassegrainM2Edge1)
	arc (\cassegrainMBAAngle:\cassegrainMBBAngle:\cassegrainMBRadius)
		-- (cassegrainM12)
	arc (\cassegrainMABAngle:\cassegrainMAAAngle:\cassegrainMARadius) -- cycle;
\draw[scattered ray] (cassegrainM13) -- (cassegrainM23);
\draw[scattered ray] (cassegrainM14) -- (cassegrainM2Edge2);
\draw[scattered fill] (cassegrainM23)
	arc (\cassegrainMBCAngle:\cassegrainMBDAngle:\cassegrainMBRadius)
		-- (cassegrainM14)
	arc (\cassegrainMADAngle:\cassegrainMACAngle:\cassegrainMARadius) -- cycle;

% to mirror MS1
% path representing the mirror
\path[name path=MS1Path] (MS1Edge1) -- (MS1Edge2);
% intersections with the mirror
% ray 1
\path[name path=toCassegrainM21] (cassegrainM2Edge1) -- ++(-10,0);
\path[name intersections={of=MS1Path and toCassegrainM21, by=MS11}];
% ray 2
\path[name intersections={of=MS1Path and toCassegrainM22, by=MS12}];
% ray 3
\path[name intersections={of=MS1Path and toCassegrainM23, by=MS13}];
% ray 4
\path[name path=toCassegrainM24] (cassegrainM2Edge2) -- ++(-10,0);
\path[name intersections={of=MS1Path and toCassegrainM24, by=MS14}];

\draw[scattered ray] (cassegrainM2Edge1) -- (MS11);
\draw[scattered ray] (cassegrainM22) -- (MS12);
\draw[scattered fill] (cassegrainM2Edge1)
	arc (\cassegrainMBAAngle:\cassegrainMBBAngle:\cassegrainMBRadius) -- (MS12)
		-- (MS11) -- cycle;
\draw[scattered ray] (cassegrainM23) -- (MS13);
\draw[scattered ray] (cassegrainM2Edge2) -- (MS14);
\draw[scattered fill] (cassegrainM23)
	arc (\cassegrainMBCAngle:\cassegrainMBDAngle:\cassegrainMBRadius) -- (MS14)
		-- (MS13) -- cycle;

% draw first mirror of cassegrain
\draw[mirror element]
	(cassegrainM11)
		arc (\cassegrainMAAAngle:\cassegrainMABAngle:\cassegrainMARadius)
			node[left,pos=0.5] {O};
\draw[mirror element]
	(cassegrainM13)
		arc (\cassegrainMACAngle:\cassegrainMADAngle:\cassegrainMARadius);
% mirror 2
\draw[mirror element]
	(cassegrainM2Edge1)
		arc (\cassegrainMBAAngle:\cassegrainMBDAngle:\cassegrainMBRadius);

% draw notch
\draw[notch] ($ (samplePos) + (-3,-0.5) $) rectangle ++(0.2,1);
\node[above] at ($ (samplePos) + (-2.9,0.5) $) {EF};

% Parabolic mirror MS2
\newcommand*{\parabolicMirrorDef}{%
		(MS2Edge2)
		.. controls ($ (MS2Edge2) + (MS2EdgeControl2) $)
			and ($ (MS2Edge1) + (MS2EdgeControl1) $)
		.. (MS2Edge1)
}
\path[name path=MS2Path] \parabolicMirrorDef;
% ray 1
\path[name path=toMS21] (MS11) -- ++(0,-10);
\path[name intersections={of=MS2Path and toMS21, by=MS21}];
% ray 2
\path[name path=toMS22] (MS12) -- ++(0,-10);
\path[name intersections={of=MS2Path and toMS22, by=MS22}];
% ray 3
\path[name path=toMS23] (MS13) -- ++(0,-10);
\path[name intersections={of=MS2Path and toMS23, by=MS23}];
% ray 4
\path[name path=toMS24] (MS14) -- ++(0,-10);
\path[name intersections={of=MS2Path and toMS24, by=MS24}];

\draw[scattered ray] (MS11) -- (MS21);
\draw[scattered ray] (MS12) -- (MS22);
\begin{scope}
	\clip (MS11) -- ($ (MS21) + (0,-1) $) -- ($ (MS22) + (0,-1) $) -- (MS12)
		-- cycle;
	\draw[scattered fill] (MS12) -- \parabolicMirrorDef -- (MS11) -- cycle;
\end{scope}
\draw[scattered ray] (MS13) -- (MS23);
\draw[scattered ray] (MS14) -- (MS24);
\begin{scope}
	\clip (MS13) -- ($ (MS23) + (0,-1) $) -- ($ (MS24) + (0,-1) $) -- (MS14)
		-- cycle;
	\draw[scattered fill] (MS14) -- \parabolicMirrorDef -- (MS13) -- cycle;
\end{scope}

% draw MS1 over all incident rays on that mirror
\draw[mirror element]
	(MS1Edge1) -- node[above,shift={(-0.5cm,-0.1cm)}]{MS1} (MS1Edge2);

% Draw calibration source
\newcommand*{\hclBeamHalfWidth}{0.25}
\draw [hcl] ($ (HCL) + (-0.3,0) $) rectangle ++(0.6,-1);
\node[right] at ($ (HCL) + (0.3,-0.5) $) {HCL};
\draw [hcl lamp] ($ (HCL) + (-0.1 + 0.02, 0) $) --
	++(0.2 - 0.04,0);

% path representing calibration lens surface
\newcommand*{\LCARadius}{0.8}
\coordinate (LC1Center) at ($ (LC1) + (0,\LCARadius) $);
\path[name path=LC1Arc, shift={(LC1Center)}]
	(180:\LCARadius) arc (-180:0:\LCARadius);
% ray 1
\path[name path = toMC11] ($ (HCL) + (-\hclBeamHalfWidth, 0) $) -- ++(0,5);
\path[name intersections = {of = LC1Arc and toMC11, by = LC11}];
\draw[hcl ray] (HCL) -- (LC11);
% ray 2
\path[name path = toMC12] ($ (HCL) + (\hclBeamHalfWidth, 0) $) -- ++(0,5);
\path[name intersections = {of = LC1Arc and toMC12, by = LC12}];
\draw[hcl ray] (HCL) -- (LC12);
 MC1
% path representing the mirror MC1
\path[name path = MC1Path] (MC1Edge1) -- (MC1Edge2);
% ray 1
\path[name intersections = {of = MC1Path and toMC11, by = MC11}];
\draw[hcl ray] (LC11) -- (MC11);
% ray 2
\path[name intersections = {of = MC1Path and toMC12, by = MC12}];
\draw[hcl ray] (LC12) -- (MC12);

% fill the beam from source to prism MC1
\draw[hcl beam] (HCL) -- (LC11) -- (MC11) -- (MC12) -- (LC12) -- cycle;

% Draw calibration lens
\path[name path=toLC1Arc1] ($ (LC1) + (-0.5,0) $) -- ++(0,1);
\path[name path=toLC1Arc2] ($ (LC1)  + (0.5,0) $) -- ++(0,1);
\path[name intersections={of=LC1Arc and toLC1Arc1, by=LC11}];
\mypgfextractangle{\LCAAAngle}{LC1Center}{LC11}
\path[name intersections={of=LC1Arc and toLC1Arc2, by=LC12}];
\mypgfextractangle{\LCABAngle}{LC1Center}{LC12}
\draw[glass] (LC11) arc (\LCAAAngle:\LCABAngle:\LCARadius) -- ++(0,0.05)
	-- ($ (LC11) + (0,0.05) $) -- cycle;
\node[right] at ($ (LC1) + (0.5,0.1) $) {LC1};

% Calibration aperture 1
\draw[aperture] ($ (AC1) + (-0.5,0) $) -- ($ (AC1) + (-0.3,0) $);
\draw[aperture] ($ (AC1) + (0.3,0) $) -- ($ (AC1) + (0.5,0) $)
	node[right]{AC1};

% path representing the mirror MC2
\path[name path = MC2Path] (MC2Edge1) -- (MC2Edge2);
% ray 1
\path[name path = toMC21] (MC11) -- ++(-4,0);
\path[name intersections = {of = MC2Path and toMC21, by = MC21}];
\draw[hcl ray] (MC11) -- (MC21);
% ray 2
\path[name path = toMC22] (MC12) -- ++(-4,0);
\path[name intersections = {of = MC2Path and toMC22, by = MC22}];
\draw[hcl ray] (MC12) -- (MC22);
% fill the beam to prism MC2
\draw[hcl beam] (MC11) -- (MC21) -- (MC22) -- (MC12) -- cycle;

% draw the calibration prism MC1
\draw[glass] (MC1Edge1) -- (MC1Edge2) -- (MC1Edge3) -- cycle;
\node[right] at ($ (MC1Edge1) + (0.5,-0.4) $) {MC1};


% beam from calibration prism MC2 to parabolic mirror MS2
% ray 1
\path[name path=toMS21C] (MC21) -- ++(0,-10);
\path[name intersections={of=MS2Path and toMS21C, by=MS21C}];
% ray 2
\path[name path=toMS22C] (MC22) -- ++(0,-10);
\path[name intersections={of=MS2Path and toMS22C, by=MS22C}];

\draw[hcl ray] (MC21) -- (MS21C);
\draw[hcl ray] (MC22) -- (MS22C);
\begin{scope}
	\clip (MC21) -- ($ (MS21C) + (0,-1) $) -- ($ (MS22C) + (0,-1) $) -- (MC22)
		-- cycle;
	\draw[hcl beam] (MC22) -- \parabolicMirrorDef -- (MC21) -- cycle;
\end{scope}

% draw the calibration prism MC2
\draw[glass] (MC2Edge1) -- (MC2Edge2) -- (MC2Edge3) -- cycle;
\node[below right, shift = {(-0.1,0.1)}] at (MC2Edge3) {MC2};

% draw scattered beam from MS1 to parabolic mirror MS2
\draw[scattered ray] (MS21) -- (parabolaFocus);
\draw[scattered ray] (MS22) -- (parabolaFocus);
\begin{scope}
	\clip (parabolaFocus) -- (MS21) -- ++(1,0) -- ($ (MS22) + (1,0) $) -- (MS22)
		-- cycle;
	\draw[scattered fill] (parabolaFocus) -- \parabolicMirrorDef -- cycle;
\end{scope}
\draw[scattered ray] (MS23) -- (parabolaFocus);
\draw[scattered ray] (MS24) -- (parabolaFocus);
\begin{scope}
	\clip (parabolaFocus) -- (MS23) -- ++(1,0) -- ($ (MS24) + (1,0) $) -- (MS24)
		-- cycle;
	\draw[scattered fill] (parabolaFocus) -- \parabolicMirrorDef -- cycle;
\end{scope}

% draw calibration beam from parabolic mirror to spectrograph
\draw[hcl ray] (MS21C) -- (parabolaFocus);
\draw[hcl ray] (MS22C) -- (parabolaFocus);
\begin{scope}
	\clip (parabolaFocus) -- (MS21C) -- ++(1,0) -- ($ (MS22C) + (1,0) $)
		-- (MS22C) -- cycle;
	\draw[hcl beam] (parabolaFocus) -- \parabolicMirrorDef -- cycle;
\end{scope}

% draw the parabolic mirror MS2
\draw[mirror element]
	(MS2Edge2)
	.. controls ($ (MS2Edge2) + (MS2EdgeControl2) $)
		and ($ (MS2Edge1) + (MS2EdgeControl1) $)
	.. node[below,shift={(0.5cm,0.1cm)}]{MS2} (MS2Edge1);

\draw (0,0) rectangle ++(3,2.5) node[pos=.5] {spectrograph};

% side view
\begin{scope}[shift={(\samplePosWidth - 2,-1.7)}]

\newcommand*{\samplePosSideWidth}{2}
\newcommand*{\samplePosSideHeight}{2.5}
\coordinate (samplePosSide) at (\samplePosSideWidth,\samplePosSideHeight);
\coordinate (cassegrainM1SideCenter)
	at (\samplePosSideWidth - 0.5,\samplePosSideHeight);
\coordinate (cassegrainM2SideCenter)
	at (\samplePosSideWidth - 0.7,\samplePosSideHeight);

% clip the view
\clip (-.05,-.35) rectangle ++(3.3,3.4);
\draw (-.05,-.35) rectangle ++(3.3,3.4);

% laser beam
\draw[laser beam] (4,0) -- (\samplePosSideWidth,0) coordinate (M3Side)
	-- (samplePosSide);

% mirror 3
\draw[mirror element] ($ (M3Side) + (-0.3,0.3) $)
	-- node[above,shift={(0.35cm,-0.05cm)}]{M3} ($ (M3Side) + (0.3,-0.3) $);

% aperture 2
\draw[aperture] ($ (M3Side) + (\apertureOuterRadius,0.8) $)
	node[above, shift={(0.05,0)}]{A2}
	-- ++(-\apertureOuterRadius + \apertureInnerRadius,0);
\draw[aperture] ($ (M3Side) + (-\apertureOuterRadius,0.8) $)
	-- ++(\apertureOuterRadius - \apertureInnerRadius,0);

% draw the cassegrain
% calculate intersections with mirror 1 (the objective mirror)
% mirror1 arc
\path[name path=M1SideArc, shift={(cassegrainM1SideCenter)}]
	(90:\cassegrainMARadius) arc (90:270:\cassegrainMARadius);
% upper top ray
\path[name path=toCassegrainM11Side] (samplePosSide) -- ++(135:5);
\path[name intersections={of=M1SideArc and toCassegrainM11Side,
	by=cassegrainM11Side}];
% upper bottom ray
\path[name path=toCassegrainM12Side] (samplePosSide) -- ++(165:5);
\path[name intersections={of=M1SideArc and toCassegrainM12Side,
	by=cassegrainM12Side}];
% lower top ray
\path[name path=toCassegrainM13Side] (samplePosSide) -- ++(195:5);
\path[name intersections={of=M1SideArc and toCassegrainM13Side,
	by=cassegrainM13Side}];
% lower bottom ray
\path[name path=toCassegrainM14Side] (samplePosSide) -- ++(225:5);
\path[name intersections={of=M1SideArc and toCassegrainM14Side,
	by=cassegrainM14Side}];

\draw[scattered ray] (samplePosSide) -- (cassegrainM11Side);
\draw[scattered ray] (samplePosSide) -- (cassegrainM12Side);
\draw[scattered fill] (samplePosSide) -- (cassegrainM11Side)
	arc (\cassegrainMAAAngle:\cassegrainMABAngle:\cassegrainMARadius) -- cycle;
\draw[scattered ray] (samplePosSide) -- (cassegrainM13Side);
\draw[scattered ray] (samplePosSide) -- (cassegrainM14Side);
\draw[scattered fill] (samplePosSide) -- (cassegrainM13Side)
	arc (\cassegrainMACAngle:\cassegrainMADAngle:\cassegrainMARadius) -- cycle;

% draw the cell
\draw[sample cell, water fill]
	($ (samplePosSide) + (-0.5 * \cellBorderWidth - \cellLineWidth,%
		-0.5 * \cellBorderWidth - \cellLineWidth) $)
		rectangle ++(\cellWidth,10);

% calculate intersections with mirror 2 (the objective mirror)
% mirror2 arc
\path[
	name path=M2SideArc, shift={(cassegrainM2SideCenter)}] (90:\cassegrainMBRadius)
		arc (90:270:\cassegrainMBRadius);
% calculate cassegrain mirror2 edges
\path[
	name intersections={%
		of=M2SideArc and toCassegrainM12Side, by=cassegrainM2SideEdge1}];
\path[
	name intersections={%
		of=M2SideArc and toCassegrainM13Side, by=cassegrainM2SideEdge2}];
% upper bottom ray
\path[name path=toCassegrainM22Side]%
	($ (samplePosSide) + (0,0.1) $) -- ++(-10,0);
\path[name intersections={%
	of=M2SideArc and toCassegrainM22Side, by=cassegrainM22Side}];
% lower top ray
\path[name path=toCassegrainM23Side]
	($ (samplePosSide) + (0,-0.1) $) -- ++(-10,0);
\path[name intersections={%
	of=M2SideArc and toCassegrainM23Side, by=cassegrainM23Side}];

\draw[scattered ray] (cassegrainM11Side) -- (cassegrainM2SideEdge1);
\draw[scattered ray] (cassegrainM12Side) -- (cassegrainM22Side);
\draw[scattered fill] (cassegrainM2SideEdge1)
	arc (\cassegrainMBAAngle:\cassegrainMBBAngle:\cassegrainMBRadius)
		-- (cassegrainM12Side)
	arc (\cassegrainMABAngle:\cassegrainMAAAngle:\cassegrainMARadius) -- cycle;
\draw[scattered ray] (cassegrainM13Side) -- (cassegrainM23Side);
\draw[scattered ray] (cassegrainM14Side) -- (cassegrainM2SideEdge2);
\draw[scattered fill] (cassegrainM23Side)
	arc (\cassegrainMBCAngle:\cassegrainMBDAngle:\cassegrainMBRadius)
		-- (cassegrainM14Side)
	arc (\cassegrainMADAngle:\cassegrainMACAngle:\cassegrainMARadius) -- cycle;

% to mirror MS1
% ray 1
\draw[scattered ray] (cassegrainM2SideEdge1) -- ++(-10,0);
\draw[scattered ray] (cassegrainM22Side) -- ++(-10,0);
\draw[scattered fill] (cassegrainM2SideEdge1)
	arc (\cassegrainMBAAngle:\cassegrainMBBAngle:\cassegrainMBRadius)
		-- ++(-10,0) -- ($ (cassegrainM2SideEdge1) + (-10,0) $) -- cycle;
\draw[scattered ray] (cassegrainM23Side) -- ++(-10,0);
\draw[scattered ray] (cassegrainM2SideEdge2) -- ++(-10,0);
\draw[scattered fill] (cassegrainM23Side)
	arc (\cassegrainMBCAngle:\cassegrainMBDAngle:\cassegrainMBRadius) --
		++(-10,0) -- ($ (cassegrainM2SideEdge1) + (-10,0) $) -- cycle;

% draw first mirror of cassegrain
\draw[mirror element]
	(cassegrainM11Side)
		arc (\cassegrainMAAAngle:\cassegrainMABAngle:\cassegrainMARadius)
			node[left,pos=0.5] {O};
\draw[mirror element]
	(cassegrainM13Side)
		arc (\cassegrainMACAngle:\cassegrainMADAngle:\cassegrainMARadius);
% mirror 2
\draw[mirror element]
	(cassegrainM2SideEdge1)
		arc (\cassegrainMBAAngle:\cassegrainMBDAngle:\cassegrainMBRadius);

\end{scope}

\end{tikzpicture}

	\caption{Top-view schema of the apparatus with wavenumber calibration lamp
		and with side-view inset of the sample space. The calibration beam from Pt
		\emph{hollow cathode lamp} (HCL) is collimated by calibration beam lens
		LC1 and guided by calibration right-angle prisms in total internal
		reflection mode (MC1 and MC2) and and focused on the entrance slit of
		spectrograph by parabolic mirror MS2 the same way as scattered signal from
		samples. The calibration lamp signal can be attenuated by modifying the
		size of the calibration beam aperture AC1. The mirror MC2 was placed on
		motorized flip mount to enable easy insertion for measurement of
		calibration spectra. The explanation of rest of the symbols is the same as
		in
		\figref{initial_layout:aparatus_schema}.}
	\label{\figlabel{wavenumber_calibration:aparatus_schema}}
\end{figure}

\begin{figure}
	\centering
	\input{results_and_discussion/assets/pt_calibration/pt_calibration}
	\caption{%
		UV Raman spectrum of platinum hollow cathode lamp with the same grating
		position as in
		\figref{wavenumber_calibration:cyclohexane_spc}.
		excitation. Grey lines denotes the positions of Raman lines used for
		calibratin.%
	}
	\label{\figlabel{wavenumber_calibration:pt_spc}}
\end{figure}

As you can see in
\figref{wavenumber_calibration:pt_spc}
the Pt lamp has many sharp well-resolved lines with reasonable intensity. The
acquisition of the spectrum is also relatively fast, the spectrum was usually
acquired as average of 100 of 0.1s scans. Because of these parameters, the
measured Raman spectra can be calibrated with the precision as high as
$\pm0.1$\,\icm.
