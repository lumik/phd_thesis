%%% Šablona pro jednoduchý soubor formátu PDF/A, jako treba samostatný abstrakt práce.

\documentclass[12pt]{report}

\usepackage[a4paper, hmargin=1in, vmargin=1in]{geometry}
\usepackage[a-2u]{pdfx}
\usepackage[czech]{babel}
\usepackage[utf8]{inputenc}
\usepackage[T1]{fontenc}
\usepackage{lmodern}
\usepackage{textcomp}

\begin{document}

%% Nezapomeňte upravit abstrakt.xmpdata.

Navzdory desetiletím intenzivního výzkumu jsou nukleové kyseliny (NK) stále
předmětem strukturních studií.
V průběhu této doktorské práce byl postaven a optimalizován přístroj na měření
rezonančních Ramanových spekter buzených v UV oblasti (UV RRS).
Na základě analýzy publikovaných dat a rozsáhlého souboru měření UV RRS
modelových struktur NK, mononukleotidů a plynukleotidů, byla vytvořena
komplexní interpretační tabulka.
Navržená metodologie byla ověřena nasazením v několika strukturních studiích
nukleových kyselin, především při studiu vlivu hořečnatých iontů na chemickou
rovnováhu mezi duplexy a triplexy tvořenými PolyA a PolyU homopolynukleotidy,
studiu teplotně indukovaných strukturních změn u DNA dvoušroubovice a DNA
vlásenek a výzkumu pomalého konformačního přechodu guaninového kvadruplexu
kontrolovaného přítomností draselných iontů.
Výsledky ověřovacích měření a výše zmíněných studií prokázaly že navržená
metodologie pro studium UV RRS NK umožňuje využití většiny předností
rezonančního buzení: možnosti měření Ramanova rozptylu při stejných
koncentracích jako pro měření UV absorpce, velké citlivosti na drobné
strukturní změny způsobené změnou teploty a dobré interpretovatelnosti
naměřených spekter.

\end{document}
