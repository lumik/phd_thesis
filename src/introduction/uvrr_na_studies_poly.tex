\subsection{Polynucleotides}

Polynucleotides have been popular molecular models of nucleic acids for their
accessibility and limited number of different nucleobases, which simplified the
assignment of Raman lines.
They can be used to simulate various forms of folded NA structures like double
helix of A, B or Z conformation, triplex or ordered single-helical structure.
Early UV RRS studies were focused on the effect of resonance Raman
hypochromism.
This effect was first evidenced in preRRS and RRS spectra of polyA$\cdot$polyU
duplex
\parencite{Pezolet1975}.
Following works reported the UV RRS hypochromism of certain Raman lines much
more intensive (160 -- 200\,\%) than that of UV absorption or RS excited at
visible (e.g. thymine lines at 1245, 1380 and 1665\,\icm{} in poly(dAdT)
excited with 300\,nm
\cite{Chinsky1982a}).
On the other hand some Raman bands of NA were found to be (at certain
conditions) insensitive to the conformational transition of NA or to be even
hyperchromic
\parencite{Chinsky1980}.
First measurements of complete temperature spectral dependences, i.e. UV RRS
melting curves of NA, were carried out on a set of polynucleotides forming A/T
or A/U base pairs
\parencite{Jolles1985}.

Changes of Raman line positions caused by the duplex formation were measured by
RRS of
	poly(dA)$\cdot$poly(dT),
	\ch{"poly(dAdT)" {}_2},
	poly(rA)$\cdot$poly(rU)
	and \ch{"poly(dAdU)" {}_2}
excited with 200, 218, 253 and 266\,nm wavelengths and compared with those of
equimolar mixtures of mononucleotides.
Downshift up to 10\,\icm{} was seen for the adenine ring modes at 730, 1310,
and 1580\,\icm{} in poly(rA)$\cdot$poly(rU)
\parencite{Grygon1990}.
The temperature induced changes in UV RRS (260\,nm) spectra of
poly(dA)$\cdot$poly(dT) were also measured in the "premelting" temperature
region
\parencite{Chan1997}.
They observed changes in the vibrational frequencies associated with the
C4-carbonyl stretching mode (thymine) and the N6 scissors mode of the adenine
amine in the spectral region 1580 -- 1690\,\icm{}.
This seemed to indicate cross-strand bifurcated hydrogen bonds between
consecutive dA:dT pairs at low temperatures (bent form).

The effect of pH induced poly(rA) transition from a single to a double helix
form was studied via 308-nm excited RRS
\parencite{Gfrorer1989}.
Main effects were related to the change of adenine protonation.
The effect of deuteration on vibrational frequencies of purine nucleotides and
polynucleotides was systematically studied in order to provide an experimental
evidence that the conformational sensitivity of purine in-plane modes originate
in their coupling with the sugar vibrations (as previous theoretical
calculations had indicated)
\parencite{Toyama1993}.
They revealed that the deuteration at C1' caused upshift of 1208, 1321, and
1414\,\icm{} and downshift of 825, 1026, and 1081\,\icm{} guanine lines and
upshift of 1174, 1213, 1309, 1375\,\icm{} and downshift of 730, 1009, and
1254\,\icm{} lines of adenine.

RRS conformational markers of the triple helix
poly(U)$\cdot$poly(A)$\cdot$poly(U), the double helix poly(A)$\cdot$poly(U) and
a random copolymer poly(AU) (260 and 220\,nm excitation) were studied from
neutral to low pH (down to 2.5)
\parencite{Gfrorer1993a}.
It was suggested that uracil band around 1452\,\icm{}, which was prominent in
220\,nm excited triplex spectra, might be an H-bond marker.
Triplex RRS spectra showed a very strong decrease in 1341, 1480, 1573\,\icm{}
adenine lines.
The only significant frequency change was the slight upshift of the adenine
band at 1338\,\icm{}.

Several studies dealt with the B to Z conformation transition.
UV RRS excited at 257 and 295\,nm wavelengths of poly(dGdC) duplex were
measured for different salt concentrations.
Main spectral changes were the intensity increases of the 1322\,\icm{} guanine
line with regard to its 1355\,\icm{} line and of the 1579 (guanine) and
1630\,\icm{} (cytosine) lines in respect to the guanine 1490\,\icm{} line
\parencite{Jolles1984}.
Similar studies were later carried out for other nucleobase compositions that
are considered not to be typical for the transition,
poly(dAdC)$\cdot$poly(dGdT)
\parencite{Miskovsky1992},
poly(dI-dC)
\parencite{Tomkova1993},
and poly(rArU)
\parencite{Tomkova1994}.
[Tomkova et al., 1994].
The structural transition was supposedly invoked by adding \ch{Ni^{2+}} ions or
\ch{NH4F}.

B to X transition in poly(dAdT) induced by high concentration of CsF was also
observed by means of UV RRS (257 and 281\,nm excitation)
\parencite{Tomkova1995}.
Main effect was obtained for thymine lines of the ring and sugar coupled modes
at 665, 745, and 784\,\icm{} (interpreted in terms of a C3'-endo/anti
reorganization), 1370\,\icm{}, and 1660\,\icm{}, which split.
The changes of the latter two were explained by modifications in the thymine
\ch{"C4" \bond{db}O} hydrogen-bond network with opposite adenine bases.

Some published works concerned also poly(rI) -- basic UV RRS characterization
\parencite{Ulicny1994},
effects of \ch{Na^+} versus \ch{K^+} ions on the conformation and stability
\parencite{Mukerji1998}
or demonstration of quadruplex structures present at high ionic strength
\parencite{Wheeler1996}.
