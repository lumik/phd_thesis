\subsubsection{Nucleic acid complexes with other molecules}

In addition to the studies of nucleic acids themselves and their components, a
number of works were published on their interactions with other molecules.
Numerous works concerned assorted biologically significant molecules, which
interacted with NA preferentially via intercalation between stacked
nucleobases.
Such interaction caused significant changes of UV RRS spectra.
The list of studied interacting molecules contains actinomycin D
\parencite{Chinsky1978},
adriamycin
\parencite{Manfait1982},
\g{b}-propiolactone
\parencite{Jolles1988},
pyrene
\parencite{Jones1988},
distamycin
\parencite{Grygon1989},
hypericin
\parencite{Miskovsky1995},
Cu(TMpy-P4) porphyrin
\parencite{Tomkova1995},
CuTMPyP4 porphyrin
\parencite{Wheeler1995},
camphotecin
\parencite{Feofanov1996},
cinchonine
\parencite{Weselucha-Birczynska1996},
nickel, cobalt, and zinc derivatives of TMPyP4 porphyrin
\parencite{Wheeler1997},
amantadine
\parencite{Stanicova1999},
bleomycin
\parencite{Weselucha-Birczynska2000},
and free-base \ch{H2TMPy "P4"} porphyrin
\parencite{Wei2006}.

A few UV RRS works were devoted to NA interaction with peptides or protein
segments.
DNA complexes with peptides, namely arginine-methylester, lysine-methylester
and arginyl-arginine were studied by RRS excited at 300 and 257\,nm
\parencite{Laigle1982a}.
Other works concern NA interaction with histones.
In the first work, DNA interaction with the four major histone subunits was
investigated
\parencite{Laigle1982b}.
Complexes of poly(dAdT) with poly(dAdT)-RNAse and with histone H1 were measured
as models of destabilized and stabilized DNA structure
\parencite{Chinsky1982a}.
Later binding of serin-prolin-lysin-lysin tetrapeptide (binding motif of H1
histones) to various deoxypolynucleotides was studied by using UV RRS
\parencite{Takeuchi1995}.
The effect was well pronounced for DNA duplexes built of A/T basepairs but not
for those with G/C basepairs.

A special group of works are the studies of nucleic acid interactions with
cis-platinum complexes.
In these works, nucleotides were employed as NA models.
UV RRS of dGMP and its complexes with \ch{cis-(NH3)2Pt^{2+}}, \ch{Ni^{2+}}, and
\ch{H^+} in complexes with dGMP
\parencite{Perno1987},
d(GpG), and d(GpA)
\parencite{Perno1988}
was measured to clarify the mode of interaction.
Strong effect of platinum on RRE was reported.
cis-platin and carboplatin binding to all main mononucleotides was also studied
\parencite{%
	Benson1992,%
	Benson1993%
}.
The interaction effected intensities of Raman bands but only very slightly the
frequencies.
Later, cis-platin and trans-platin interaction with genomic DNA was
investigated
\parencite{Geng2017}.
