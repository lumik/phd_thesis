\subsection{Nucleobases, nucleosides and nucleotides}

First UV RRS experiments concerned mainly the simple NA components.
The published works from the seventies and eighties were devoted primarily to
nucleotides, probably because of their good solubility.
Typical was the use of several excitation wavelengths in the resonance or
preresonance spectral region.
The resonance enhancement of Raman signal was first reported for 5'-AMP and its
analogs
\parencite{%
	Tsuboi1974,%
	Pezolet1975,%
	Blazej1977%
}.
It was found out that while the resonance enhancement of some adenine lines
(1484 and 1580\,\icm{}) was associated with the lowest-frequency absorption
band, the ring-breathing vibration (730\,\icm{}) belongs to the
higher-frequency band (210\,nm).
Soon the RRS active vibrations of NA assigned to nucleotides containing all
nucleobases were reported
\parencite{%
	Chinsky1978,%
	Nishimura1977%
}.
It was found out, that for a proper excitation wavelength the resonance
enhancement of some Raman lines is so strong that the sample concentration
needed is as low as what is required for the ultraviolet absorbance measurement
\parencite{Nishimura1977}.

UV RRS investigations of nucleotides continued in eighties, usually with the
aim to specify precisely the dependence of the resonance enhancement on the
excitation wavelength.
This effort was probably contributed by the finding that some previously
achieved results do not match each other and thus cannot be considered as
reliable
\parencite{Bushaw1980}.
It was concluded that maximal resonance enhancement can be achieved for about
260\,nm, but deeper UV excitations may be more sensitive to some changes,
which was demonstrated on adenine protonation by
\textcite{Kubasek1985}.
Especially Raman lines of pyrimidine vibrations were found to be stronger
enhanced for excitation in deep UV
\parencite{%
	Ziegler1984,%
	Fodor1985%
}.

The depolarization ratios for RS of CMP and UMP excited by 514 and 300\,nm were
measured by
\textcite{Blazej1980}.
For the most intensive lines the depolarization ratios were close to 1/3 for
300\,nm excitation even though they somewhat differed for excitation at
514.5\,nm.
Detailed differences in the UVRR spectra of guanosine and adenosine nucleotides
with varying types of furanose rings were reported for 266\,nm and 218\,nm
excitations
\parencite{Nishimura1987}.
UV RRS of main deoxynucleosides and aromatic aminoacids were measured in
\ch{H2O} and \ch{D2O} using various excitations (257, 244, 238, and 229\,nm) to
demonstrate the suitability of UV RRS for study of nucleoprotein assemblies
including viruses
\parencite{Wen1998}.

From the half of eighties, relatively precise UV RRS measurements of
nucleobases and their analogs appeared, usually with aim to enlarge the
knowledge of the nucleobase vibrational spectra in order to support
interpretation of vibrational lines.
These works include RRS spectra of uracil and its \ch{^{18}O} substituted
derivatives
\parencite{Ghomi1986},
adenine and 2-aminoadenine
\parencite{Dhaouadi1994},
thymine, thymidine and its stereoisotopomer
\parencite{Tsuboi1997},
guanosine and its seven isotope-substituted analogs
(%
	\ch{^{13} "C2"},
	\ch{^{15} "N2"},
	\ch{^{18} "O6"},
	\ch{^{15} "N7"},
	\ch{^{13} "C8"},
	\ch{^{15} "N9"},
	and \ch{^{13} "C1'"}%
)
in \ch{H2O} and \ch{D2O}
\parencite{%
	Toyama1999,%
	Toyama2002%
},
1-methyldeoxyadenosine and 3-methyldeoxycytidine
\parencite{Jayanth2011}.

Another studies focused on the effect of nucleobase protonation state and/or
hydrogen bonds (both donor/acceptor type) on UV RRS spectra of nucleotides.
They concerned major nucleobases and 5-methyl-cytosine in monophosphate
nucleotides dissolved in \ch{H2O} or \ch{D2O}
\parencite{Gfrorer1991},
2'-deoxy-3',5'-bis(triisopropylsilyl)\allowbreak{}guanosine
in various types ofsolvents
\parencite{Toyama1996},
acetyl derivative of adenosine and its C8-deuterated analog
\parencite{Fujimoto1998},
pH effect on UV RRS of dAMP, dGMP, and dIMP
\parencite{Sokolov2000},
2',3',5-tri-O-acetyladenosine in various solvents
\parencite{Toyama2005a},
1-methyldeoxyadenosine and 3-methyldeoxycytidine at various pH
\parencite{Jayanth2011}.
