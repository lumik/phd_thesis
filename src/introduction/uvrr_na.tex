\chapter{UV RRS measurements of nucleic acids}

Despite the decades of intensive research, \emph{nucleic acids} (NA) represent
still a permanent object of the structural studies, which focus on
characterizing NA local geometric arrangement, its formation and collapse
depending on the internal (nucleobase sequence) and external (environmental
conditions, intermolecular interactions) parameters.
The reason is the rich structural polymorphism of nucleic acids, which are in
addition to the classic Watson-Crick duplex capable to adopt a variety of
non-canonical structures, such as hairpins, cruciforms, triplexes, quadruplexes
or i-motifs.
It has been also shown that the processes of the expression of genetic
information, of its modification and transfer are controlled through specific
nucleic acid-protein interactions.
However, detailed molecular mechanisms of these key biological processes are
often unknown
\parencite{%
	Bansal2014,%
	Kaushik2016,%
	Yamamoto2021%
}

A variety of experimental techniques has been used to investigate the
structural arrangement of nucleic acids in their natural environment.
Many of them were gradually developed to achieve credible and
well-interpretable results.
One of the relatively new techniques that was considered as perspective is the
resonance Raman scattering excited in the ultraviolet spectral region (UV RRS).
UV RRS of NA and their components have been measured since the beginning of
seventies.
In contrast to the original optimistic expectations, this method is though
currently applied in nucleic acid studies only rarely.

This chapter represents a brief overview of the published studies, summarizes
the main assets to NA research and tries to draw attention to possible
pitfalls.
In addition to the results achieved, great attention is paid to the used
apparatus and the methodology of the experiment, data treatment and
interpretation.


\subsection{Experimental Approach}

Within the about fifty years from the first published works dealing with UV RRS
measurements of NA, the devices used have change accordingly to the technical
progress but also to the growth experience with this type of experiments.
We concentrate on the main parts of UV RRS spectrometers including the way of
the sample placement.


\subsection{Excitation sources}

It is typical for the first two decades that a great number of various laser
sources were employed, to reach an intensive excitation at various UV
wavelengths.
As a rule, pulse lasers with high peak power were chosen in order to reach high
efficiency of nonlinear effects serving for the wavelength transformation.

Frequency-doubled pulsed high-power Ar\textsuperscript{+} laser was used for
excitation at 257\,nm
\parencite{%
	Harada1975,%
	Pezolet1975,%
	Nishimura1977%
}.
Pulsed Nd:YAG lasers, as a rule of low repetition frequency, were employed
mainly to obtain 266\,nm or 213\,nm wavelengths as the forth and the fifth
harmonic, respectively
\parencite{%
	Ziegler1981,%
	Toyama1993%
}.
Other wavelengths were obtained using higher harmonics
\parencite{Kubasek1985}
and \ch{H2} Raman shift cell
\parencite{%
	Fodor1985,%
	Nishimura1987%
}.
211\,nm radiation was obtained by using sum frequency mixing of the second and
third harmonics of a
\emph{continuous-wave} (cw)
mode-locked Nd:YLF laser operating at 76\,MHz
\parencite{Leonard1994}.
Excimer lasers were also employed
\parencite{Ziegler1983}.

Continuously tunable UV radiation was obtained by frequency-doubled pulsed dye
lasers, firstly pumped by a flash lamp
\parencite{%
	Asher1977,%
	Blazej1977%
},
which enables to obtain wavelengths above 265\,nm.
Later on frequency doubled synchronously pumped dye lasers providing high
repetition ps pulses were used for an excitation with relatively low average
power tunable down to 285\,nm
\parencite{%
	Bushaw1978,%
	Samanta1982,%
	Benson1992%
}.
An excitation source tunable in a large range of 217–750 nm was designed using
a frequency doubled or tripled 20\,Hz Nd:YAG laser to pump a dye laser
\parencite{Asher1983}.
UV wavelengths were obtained by a nonlinear mixing or frequency doubling.
An excimer-pumped dye-laser system with subsequent frequency doubling provided
wavelengths were continuously adjustable down to 205\,nm
\parencite{Gfrorer1993a}.

At the beginning of nineties, the quantity of so far obtained experimental data
has allowed for general conclusions concerning UV RRS spectra of NA obtained by
using pulse lasers.
Although the strong resonance enhancement led to high signal-to-noise (S/N),
the typical S/N ratios remained far below the theoretical limit for high-pulse
energy excitation.
Moreover, disaccords of Raman intensity data, in particular the excitation RRS
profiles, obtained in different laboratories were presented (e.g.
\cite{Bushaw1980}).
The optical excitation sampling limitations, dielectric breakdown and other
nonlinear optical phenomena
\parencite{Teraoka1990},
and Raman saturation phenomena
\parencite{%
	Harmon1990,%
	Johnson1986,%
	Ludwig1988a,%
	Song1991a,%
	Sweeney1990,%
	Teraoka1990%
}
were specified as the reasons.

In the same time, the cw Ar\textsuperscript{+} ion laser with intracavity
doubling has appeared as a new UV source providing 257, 248, 244, 238 and
228.9\,nm excitations
\parencite{%
	Asher1993b,%
	Russell1995%
}.
Due to the cw regime, its use lowered nonlinear sample photochemistry, sample
thermal degradation, and ground-state saturation.
The beam could be focused to a small sample volume, which could be efficiently
collected; this capability resulted in very high spectral S/N ratios.

\subsection{Scattering geometry, optical paths and spectral analysis}

Both 90-degree and backscattering geometries were employed. An exception is a
grazing incidence geometry used in the case of the sample cuvette with a side
opening
\parencite{Jolles1984}
and a 120\textdegree geometry used for measurement from a jet stream
\parencite{Fodor1985}.
The excitation beam was directed by means of planar mirrors and focused by a
quartz lens.
More diverse were the arrangements of the collecting paths.
Besides the quartz lenses, mirrors (concave mirrors
\cite{Blazej1977},
or later Cassegrain reflective objectives were
\cite{%
	Toyama1991,%
	Russell1995%
})
were used to avoid the chromatic aberration.

Spectral analysis was first performed by using double or triple spectrographs
\parencite{%
	Harada1975,%
	Gfrorer1993a,%
	Toyama1993%
},
from the beginning of nineties equipped with a cooled CCD detector
\parencite{%
	Gfrorer1993a,%
	Toyama1993%
}.
It was later demonstrated that a single-stage spectrograph (which is of higher
throughput) is sufficient to reject the elastically scattered radiation when it
is placed behind a simple prefiltering element.
The premonochromators, used for this purpose
\parencite{%
	Hashimoto1993,%
	Russell1995%
},
were after replaced by proper optical filters
\parencite{%
	Munro1997,%
	Bykov2013%
}.

\subsubsection{Sample placement}

Except some cases when UV RRS spectra of NA were measured in stationary quartz
capillaries or tubes
\parencite{%
	Blazej1977,%
	Asher1983%
},
three main concepts has been designed to reduce the NA photodamage during the
experiment.
All these concepts are used up to now without any clear evaluation, which of
them should be preferred.

Quartz rotating cell enables to enlarge the sample volume exposed by the
radiation.
The rotating cells were usually constructed for sample volumes of units of mL
\parencite{%
	Kiefer1971,%
	Kiefer1971a,%
	Pezolet1973,%
	Nishimura1977,%
	Laigle1982a,%
	Asher1983,%
	Benson1992,%
	Toyama1993,%
	Russell1995%
},
but a low-volume (down to 2\g{m}L) construction was also published
\parencite{Gfrorer1993a}.

Another possibility is a flow system with a reservoir, where the sample is
circulating.
UV RRS is measured from a jet stream
\parencite{%
	Ziegler1981,%
	Asher1983,%
	Fodor1985,%
	Russell1995,%
	Toyama1993%
}
or a properly designed flow cell
\parencite{%
	Blazej1980,%
	Gfrorer1989%
}.
The sample volume depending on the particular design was from 2 to 50\,mL.

Relatively simple but acceptable variant is a standard cuvette of about 1\,mL
or somewhat more volume with constantly stirring.
In this case, the backscattering geometry is usually used (except for the
mentioned above the grazing incidence
\parencite{Jolles1984}).
The advantage of this approach is a relatively easy temperature control of the
sample
\parencite{Mukerji1995}.

It is worth to mention that a combination of the rotating cuvette and stirring
has also been published
\parencite{Rodgers1992}.
The authors used a Suprasil NMR tube (0.5\,mL of sample) spun at approximately
10\,Hz around a stationary helical stirring wire.

\subsubsection{Calibration}

During the first period when the main aim was to determine the UV RRS
excitation profiles of various NA components, great attention was paid to an
intensity calibration.
The calibration procedures served mainly for intensity normalization of Raman
spectra obtained for different excitation wavelengths.
Usually Raman lines (mostly only the most intensive ones) of pure simple
compounds or their relatively concentrated solutions, which did not absorb in
the near UV, were taken as standards.
The dependence of their UV RS intensity on the excitation wavelength was
omitted or simply approximated.
The list of employed standards include cyclohexane
\parencite{Hamaguchi1974},
water (3400\,\icm{} band)
\parencite{%
	Chinsky1978,%
	Fodor1985%
},
cacodylate ion (608\,\icm{} line)
\parencite{%
	Blazej1977,%
	Samanta1982%
},
phosphate ion (994\,\icm{} line)
\parencite{Kubasek1985},
sulphate ion (981\,\icm{} line)
\parencite{%
	Perno1989,%
	Benson1992,%
	Wen1998%
},
and perchlorate (934\,\icm{} line)
\parencite{Rodgers1992}.
Some authors employed the Raman lines naturally occurring in the measured
spectrum but without resonance enhancement, e.g. 980\,\icm{} line of the
\ch{PO3^-}
vibration and the water band at 1640\,\icm{} in the spectrum of nucleotide
solution
\parencite{Tsuboi1974}.

The only work was devoted to an intensity correction through the measured
UV RRS spectrum
\parencite{Chinsky1983a}.
The authors aimed to correct the spectral dependence of the monochromator
throughput in the region of 2000 -- 5000\,\icm.
They used UV radiation from deuterium lamp passing through a double
monochromator or only through its first stage, assuming the same transmission
for the second stage.

Surprisingly only a few published UV RRS studies of NA specified how the
wavenumber scale was calibrated.
Usually the calibrations were based on known positions of Raman lines of
simple compounds.
Carbon tetrachloride was used by
\textcite{Hamaguchi1974}.
Later on, mixtures or sets of separately measured chemicals were employed,
namely ethyl acetate/dioxane
\parencite{Toyama1991},
ethanol and n-pentane
\parencite{Mukerji1995},
dioxane, carbon tetrachloride and acetonitrile
\parencite{Russell1995}.
Spectral calibration with the lines of a low-pressure Hg lamp was presented by
\textcite{Efremov1991}.

\subsection{Data treatment and analysis}

Within the published UV RRS studies of nucleic acids, some advanced
mathematical methods of data treatment and analysis have been applied.
A method of noise and background correction by a filtration in the Fourier
space was developed and applied by
\textcite{Laigle1982}.
To decompose overlapping UV RRS lines a combination of maximum entropy and
Fourier self-deconvolution was employed
\parencite{Efremov1991}.
To visualize the coordination of the effects of a photo- or thermally induced
damage of 12- and 18-mer AT oligomers, the method of two-dimensional
correlation spectroscopy was used
\parencite{Jirasek2006}.
Another approach for analysis of a spectral set, the principal component
analysis was employed to extract the features of interest from UV RRS spectra
of ethidium bromide intercalating supercoiled DNA plasmid
\parencite{Neugebauer2007}.


\subsection{Published UV RRS studies of NA}


\subsubsection{Nucleobases, nucleosides and nucleotides}

First UV RRS experiments concerned mainly the simple NA components.
The published works from the seventies and eighties were devoted primarily to
nucleotides, probably because of their good solubility.
Typical was the use of several excitation wavelengths in the resonance or
preresonance spectral region.
The resonance enhancement of Raman signal was first reported for 5'-AMP and its
analogs
\parencite{%
	Tsuboi1974,%
	Pezolet1975,%
	Blazej1977%
}.
It was found out that while the resonance enhancement of some adenine lines
(1484 and 1580\,\icm{}) was associated with the lowest-frequency absorption
band, the ring-breathing vibration (730\,\icm{}) belongs to the
higher-frequency band (210\,nm).
Soon the RRS active vibrations of NA assigned to nucleotides containing all
nucleobases were reported
\parencite{%
	Chinsky1978,%
	Nishimura1977%
}.
It was found out, that for a proper excitation wavelength the resonance
enhancement of some Raman lines is so strong that the sample concentration
needed is as low as what is required for the ultraviolet absorbance measurement
\parencite{Nishimura1977}.

UV RRS investigations of nucleotides continued in eighties, usually with the
aim to specify precisely the dependence of the resonance enhancement on the
excitation wavelength.
This effort was probably contributed by the finding that some previously
achieved results do not match each other and thus cannot be considered as
reliable
\parencite{Bushaw1980}.
It was concluded that maximal resonance enhancement can be achieved for about
260\,nm, but deeper UV excitations may be more sensitive to some changes,
which was demonstrated on adenine protonation by
\textcite{Kubasek1985}.
Especially Raman lines of pyrimidine vibrations were found to be stronger
enhanced for excitation in deep UV
\parencite{%
	Ziegler1984,%
	Fodor1985%
}.

The depolarization ratios for RS of CMP and UMP excited by 514 and 300\,nm were
measured by
\textcite{Blazej1980}.
For the most intensive lines the depolarization ratios were close to 1/3 for
300\,nm excitation even though they somewhat differed for excitation at
514.5\,nm.
Detailed differences in the UVRR spectra of guanosine and adenosine nucleotides
with varying types of furanose rings were reported for 266\,nm and 218\,nm
excitations
\parencite{Nishimura1987}.
UV RRS of main deoxynucleosides and aromatic aminoacids were measured in
\ch{H2O} and \ch{D2O} using various excitations (257, 244, 238, and 229\,nm) to
demonstrate the suitability of UV RRS for study of nucleoprotein assemblies
including viruses
\parencite{Wen1998}.

From the half of eighties, relatively precise UV RRS measurements of
nucleobases and their analogs appeared, usually with aim to enlarge the
knowledge of the nucleobase vibrational spectra in order to support
interpretation of vibrational lines.
These works include RRS spectra of uracil and its \ch{^{18}O} substituted
derivatives
\parencite{Ghomi1986},
adenine and 2-aminoadenine
\parencite{Dhaouadi1994},
thymine, thymidine and its stereoisotopomer
\parencite{Tsuboi1997},
guanosine and its seven isotope-substituted analogs
(%
	\ch{^{13} "C2"},
	\ch{^{15} "N2"},
	\ch{^{18} "O6"},
	\ch{^{15} "N7"},
	\ch{^{13} "C8"},
	\ch{^{15} "N9"},
	and \ch{^{13} "C1'"}%
)
in \ch{H2O} and \ch{D2O}
\parencite{%
	Toyama1999,%
	Toyama2002%
},
1-methyldeoxyadenosine and 3-methyldeoxycytidine
\parencite{Jayanth2011}.

Another studies focused on the effect of nucleobase protonation state and/or
hydrogen bonds (both donor/acceptor type) on UV RRS spectra of nucleotides.
They concerned major nucleobases and 5-methyl-cytosine in monophosphate
nucleotides dissolved in \ch{H2O} or \ch{D2O}
\parencite{Gfrorer1991},
2'-deoxy-3',5'-bis(triisopropylsilyl)\allowbreak{}guanosine
in various types ofsolvents
\parencite{Toyama1996},
acetyl derivative of adenosine and its C8-deuterated analog
\parencite{Fujimoto1998},
pH effect on UV RRS of dAMP, dGMP, and dIMP
\parencite{Sokolov2000},
2',3',5-tri-O-acetyladenosine in various solvents
\parencite{Toyama2005a},
1-methyldeoxyadenosine and 3-methyldeoxycytidine at various pH
\parencite{Jayanth2011}.

\subsection{Polynucleotides}

Polynucleotides have been popular molecular models of nucleic acids for their
accessibility and the limited number of different nucleobases, which simplified
the assignment of Raman lines.
They can simulate various forms of folded NA structures like a double
helix of A, B, or Z conformation, triplex, or ordered single-helical structure.
Early UV RRS studies were focused on the effect of resonance Raman
hypochromism.
This effect was first evidenced in preRRS and RRS spectra of polyA$\cdot$polyU
duplex
\parencite{Pezolet1975}.
Following works reported the UV RRS hypochromism of certain Raman lines much
more intensive (160 -- 200\,\%) than that of UV absorption or RS excited at
visible (e.g., thymine lines at 1245, 1380 and 1665\,\icm{} in poly(dAdT)
excited with 300\,nm
\cite{Chinsky1982a}).
On the other hand, some Raman bands of NA were found to be (at certain
conditions) insensitive to the conformational transition of NA or to be even
hyperchromic
\parencite{Chinsky1980}.
First measurements of complete temperature spectral dependences, i.e., UV RRS
melting curves of NA, were carried out on a set of polynucleotides forming A/T
or A/U base pairs
\parencite{Jolles1985}.

Changes of Raman line positions caused by the duplex formation were measured by
RRS of
	poly(dA)$\cdot$poly(dT),
	\ch{"poly(dAdT)" {}_2},
	poly(rA)$\cdot$poly(rU)
	and \ch{"poly(dAdU)" {}_2}
excited with 200, 218, 253, and 266\,nm wavelengths and compared with those of
equimolar mixtures of mononucleotides.
Downshift up to 10\,\icm{} was seen for the adenine ring modes at 730, 1310,
and 1580\,\icm{} in poly(rA)$\cdot$poly(rU)
\parencite{Grygon1990}.
The temperature-induced changes in UV RRS (260\,nm) spectra of
poly(dA)$\cdot$poly(dT) were also measured in the “premelting” temperature
region
\parencite{Chan1997}.
They observed changes in the vibrational frequencies associated with the
C4-carbonyl stretching mode (thymine) and the N6 scissors mode of the adenine
amine in the spectral region 1580 -- 1690\,\icm{}.
This seemed to indicate cross-strand bifurcated hydrogen bonds between
consecutive dA:dT pairs at low temperatures (bent form).

The pH induced poly(rA) transition from a single to a double helix
form was studied via 308\,nm excited RRS
\parencite{Gfrorer1989}.
The main effects were related to the change of adenine protonation.
The effect of deuteration on vibrational frequencies of purine nucleotides and
polynucleotides was systematically studied to provide experimental
evidence that the conformational sensitivity of purine in-plane modes originate
in their coupling with the sugar vibrations (as previous theoretical
calculations had indicated)
\parencite{Toyama1993}.
They revealed that the deuteration at C1' caused upshift of 1208, 1321, and
1414\,\icm{} and downshift of 825, 1026, and 1081\,\icm{} guanine lines and
upshift of 1174, 1213, 1309, 1375\,\icm{} and downshift of 730, 1009, and
1254\,\icm{} lines of adenine.

RRS conformational markers of the triple helix
poly(U)$\cdot$poly(A)$\cdot$poly(U), the double helix poly(A)$\cdot$poly(U),
and a random copolymer poly(AU) (260 and 220\,nm excitation) were studied from
neutral to low pH (down to 2.5)
\parencite{Gfrorer1993a}.
It was suggested that the uracil band around 1452\,\icm{}, which was prominent
in 220\,nm excited triplex spectra, might be an H-bond marker.
Triplex RRS spectra showed a very strong decrease in 1341, 1480, 1573\,\icm{}
adenine lines.
The only significant frequency change was the slight upshift of the adenine
band at 1338\,\icm{}.

Several studies dealt with the B to Z conformation transition.
UV RRS excited at 257 and 295\,nm wavelengths of poly(dGdC) duplex were
measured for different salt concentrations.
Main spectral changes were the intensity increases of the 1322\,\icm{} guanine
line with regard to its 1355\,\icm{} line and of the 1579 (guanine) and
1630\,\icm{} (cytosine) lines in respect to the guanine 1490\,\icm{} line
\parencite{Jolles1984}.
Similar studies were later carried out for other nucleobase compositions that
are considered not to be typical for the transition,
poly(dAdC)$\cdot$poly(dGdT)
\parencite{Miskovsky1992},
poly(dI-dC)
\parencite{Tomkova1993},
and poly(rArU)
\parencite{Tomkova1994}.
The structural transition was supposedly invoked by adding \ch{Ni^{2+}} ions or
\ch{NH4F}.

B to X transition in poly(dAdT) induced by high a concentration of CsF was also
observed by means of UV RRS (257 and 281\,nm excitation)
\parencite{Tomkova1995}.
The main effect was obtained for thymine lines of the ring and sugar coupled
modes at 665, 745, and 784\,\icm{} (interpreted in terms of a C3'-endo/anti
reorganization), 1370\,\icm{}, and 1660\,\icm{}, which split.
The changes of the latter two were explained by modifications in the thymine
\ch{"C4" \bond{db}O} hydrogen-bond network with opposite adenine bases.

Some published works also concerned poly(rI) -- basic UV RRS characterization
\parencite{Ulicny1994},
effects of \ch{Na^+} versus \ch{K^+} ions on the conformation and stability
\parencite{Mukerji1998}
or demonstration of quadruplex structures present at high ionic strength
\parencite{Wheeler1996}.

\subsubsection{Oligonucleotides and natural NA}

The published UV RRS measurements of natural NA are relatively rare.
In eighties, two measurements on calf thymus DNA (the most common natural NA)
were carried out and Raman spectra assigned by a comparison with nucleotide
spectra
\parencite{%
	Laigle1982a,%
	Fodor1986a%
}.
In the latter work, poly(dAdT), poly(dGdC) were also measured as references and
the spectral shifts caused by the duplex formation were identified.
UV RRS (248, 257, and 264\,nm) of calf-thymus DNA was later also measured to
determine the damage due to ionizing radiation
\parencite{Shaw2009}.

UV RRS of filamentous viruses Pf1 and fd were measured with various excitation
wavelengths (257, 244, 238, and 229\,nm)
\parencite{Wen1999}.
The results indicated markedly different modes of organization of ssDNA in Pf1
and fd virions, despite similar environments for coat protein tyrosines.

UV RRS was also used to observe effect of supercoiling on DNA plasmid pBR322
\parencite{Neugebauer2007}.
In comparison with relaxed form changes were seen in the \ch{C=O} and exocyclic
\ch{NH2} vibration above 1600\,\icm{}, CN stretching mode of adenosine and
guanosine at 1487\,\icm{}, and in the spectral region 1320 -- 1360\,\icm{}
sensitive to deoxyribose conformation.

Oligonucleotides as NA segments with a customer specified nucleobase sequences
can serve as realistic molecular models of important NA structural elements.
The published UV RRS studies concern mainly duplex structures, hairpins and
guanine quadruplexes.

The effect of cytosine and guanosine methylation on the thermal stability of
DNA duplexes was studied via UV RRS of DNA hexamers with alternating C/G
sequences
\parencite{%
	Laigle1986,%
	Chinsky1987%
}.
\textcite{Mukerji2002}
The premelting transition of A/T tracts in DNA duplexes was studied
via UV RRS (and ECD) of DNA 12-mers with palindrome sequences containing
central A/T 6-membered tract consisting of alternating A and T or two triads of
A and T
\parencite{Mukerji2002}.
The premelting effect was observed in the latter case only being pronounced by
frequency shifts of dA exocyclic \ch{NH2} and dT \ch{"C4" \bond{db}O}
vibrational modes.
This suggested formation of three-centered hydrogen bonds at low temperatures.
UV RRS (210, 240, 260\,nm) was also used to monitor A to B transition in three
DNA duplexes --
	\ch{d(CGCGAATTCGCG)2},
	\ch{d(CGCGAATTGCGC) + "complementary~strand"},
	and \ch{d(CGCAAATTTCGC) + "complementary~strand"}
\parencite{Knee2008}.
The A form existed at low water activity for solutions in trifluoroethanol.
The effect of pH (6.4 and 3.45) in combination with possible presence of
\ch{Mn^{2+}} and \ch{Ca^{2+}} ions on UV RRS (275\,nm) spectra of LacDNA 22-mer
duplex
	d(TAATGTGAGTT\allowbreak{}AGCTCACTCAT)%
	$\cdot$%
	d(ATGAGTGAGCT\allowbreak{}AACTCACATTA)
was measured
and compared with that of its individual strands
\parencite{Muntean2013}.

Hairpin studies are represented by UV RRS (257\,nm) spectral measurements of
DNA 12-mer d(5'TTCTCGCGAAGC3'), which forms a stable hairpin at its
5'GCGAAGC3' part, and d(5'TTCTCCGCAAGC3') as a reference without the hairpin
formation
\parencite{Refregiers1997}.
The difference spectrum showed marker peaks of the hairpin formation.
The oligonucleotides were measured with and without the fluorescence labeling
used in previously published FRET study.
The results proved that the labeling did not prevent the hairpin formation.

As for the guanine quadruplexes, UV RRS (240 and 250\,nm) spectral features
related to non-canonical structural arrangement of \ch{d(AG)10} were
determined at different temperatures
\parencite{%
	Mukerji1995,%
	Mukerji1996%
}.
The observed changes of guanine modes exhibited behavior corresponding to
melting transition but the changes of adenine vibrational modes were linear.
The results support the model, in which two hairpins interact to form a helical
structure with G-tetrads and intercalated dA residues.
UV RRS (257\,nm) of several DNA oligonucleotides with telometric sequences (12,
16, and 22-mers containing dT and dG) have confirmed formation of guanine
quadruplexes connected by thymidine loops
\parencite{Krafft2002}.

\subsubsection{Nucleic acid complexes with other molecules}

In addition to the studies of nucleic acids themselves and their components, a
number of works were published on their interactions with other molecules.
Numerous works concerned assorted biologically significant molecules, which
interacted with NA preferentially via intercalation between stacked
nucleobases.
Such interaction caused significant changes of UV RRS spectra.
The list of studied interacting molecules contains actinomycin D
\parencite{Chinsky1978},
adriamycin
\parencite{Manfait1982},
\g{b}-propiolactone
\parencite{Jolles1988},
pyrene
\parencite{Jones1988},
distamycin
\parencite{Grygon1989},
hypericin
\parencite{Miskovsky1995},
Cu(TMpy-P4) porphyrin
\parencite{Tomkova1995},
CuTMPyP4 porphyrin
\parencite{Wheeler1995},
camphotecin
\parencite{Feofanov1996},
cinchonine
\parencite{Weselucha-Birczynska1996},
nickel, cobalt, and zinc derivatives of TMPyP4 porphyrin
\parencite{Wheeler1997},
amantadine
\parencite{Stanicova1999},
bleomycin
\parencite{Weselucha-Birczynska2000},
and free-base \ch{H2TMPy "P4"} porphyrin
\parencite{Wei2006}.

A few UV RRS works were devoted to NA interaction with peptides or protein
segments.
DNA complexes with peptides, namely arginine-methylester, lysine-methylester
and arginyl-arginine were studied by RRS excited at 300 and 257\,nm
\parencite{Laigle1982a}.
Other works concern NA interaction with histones.
In the first work, DNA interaction with the four major histone subunits was
investigated
\parencite{Laigle1982b}.
Complexes of poly(dAdT) with poly(dAdT)-RNAse and with histone H1 were measured
as models of destabilized and stabilized DNA structure
\parencite{Chinsky1982a}.
Later binding of serin-prolin-lysin-lysin tetrapeptide (binding motif of H1
histones) to various deoxypolynucleotides was studied by using UV RRS
\parencite{Takeuchi1995}.
The effect was well pronounced for DNA duplexes built of A/T basepairs but not
for those with G/C basepairs.

A special group of works are the studies of nucleic acid interactions with
cis-platinum complexes.
In these works, nucleotides were employed as NA models.
UV RRS of dGMP and its complexes with \ch{cis-(NH3)2Pt^{2+}}, \ch{Ni^{2+}}, and
\ch{H^+} in complexes with dGMP
\parencite{Perno1987},
d(GpG), and d(GpA)
\parencite{Perno1988}
was measured to clarify the mode of interaction.
Strong effect of platinum on RRE was reported.
cis-platin and carboplatin binding to all main mononucleotides was also studied
\parencite{%
	Benson1992,%
	Benson1993%
}.
The interaction effected intensities of Raman bands but only very slightly the
frequencies.
Later, cis-platin and trans-platin interaction with genomic DNA was
investigated
\parencite{Geng2017}.


\subsection{The most important results}


\subsection{Analysis of nucleic acids UV RRS spectra}

UV RRS measurements of NA components providing first data concerning the
resonance Raman enhancement (RRE) of nucleobase vibrational modes and its
dependence on the excitation wavelength stimulated intensive efforts for RRE
theoretical prediction.
It was assumed that the comparison with experimental results would enable
determination of the equilibrium value shift of the vibrational coordinates
connected with a particular electronic excitation.
It was expected to estimate in this way the changes in the nucleobase geometry
caused by the low-energy electron absorption transitions and to obtain
information significant to understand the photochemical processes in nucleic
acids.

The first applicable theoretical relationships were derived by
\textcite{Peticolas1970}
by an extensive simplification of the theory of Raman intensities developed by
\textcite{Albrecht1961}.
According this theory the resonantly enhanced bands when excited at the
low-energy absorption band should also exhibit hypochromism like the UV
absorption in this spectral region.
First quantitative measurements of preRRE
\parencite{Kalantar1972}
seemed to agree with the theory predicting RRE in the low-energy absorption
region for all hypochromic Raman bands.
Further detailed studies showed though only partly agreement with this
statement:
The vibration of uracil at 1230\,\icm{} is strongly hypochromic
\parencite{%,
	Small1971,%
	Small1971a%
}
and exhibits preRRE
\parencite{Tsuboi1971}
but calculations did not indicate the derivation of this band intensity from
the low-energy absorption band.
On the other hand, the 1338\,\icm{} band of adenine with strong hypochromism
\parencite{%
	Aylward1970,%
	Lafleur1972,%
	Small1971%
}
gains its intensity from the 260\,nm transition.
Adenine Raman bands at 1583 and 1485\,\icm{} were not hypochromic but show RRE.
Later on this theory was improved (but still omitting the effect of
Frank-Condon factors for other vibrational modes then the one active in the
Raman transition)
\parencite{%
	Blazej1977,%
	Warshel1977a%
}.
Nevertheless only partial agreement with the new experimental data was achieved
\parencite{%
	Peticolas1979,%
	Peticolas1980%
}.

Further works revealed the need to consider also the electron transitions lying
at higher energies.
In enabled to explain the decrease of the hypochromism of some Raman bands of
poly(rU) when the excitation was changed from 363\,nm to 290\,nm
\parencite{Chinsky1980}.
Raman excitation profiles of GMP, CMP, and TMP (and their protonated forms)
measured to 285\,nm revealed that while GMP 1324 and 1488\,\icm{} bands derive
their intensity mainly from the same electronic state near 276\,nm (but for
1577\,\icm{}, small contributions from other excited states cannot be
excluded), for all TMP bands the RRE was connected with 267\,nm absorption band
and for CMP all bends seemed to have origin in 175\,nm state
\parencite{Samanta1982}.
It was also demonstrated that in some cases the presence of n-\g{p}* electronic
transitions perpendicular to the nucleobase plane had to be taken into account
besides the generally considered \g{p}-\g{p}* electronic transitions, the
transition dipole of which is parallel to this plane.
In this way the twice greater intensity of the cytosine band at 778\,\icm{} in
Z conformation of poly(dGdC) duplex in respect to B conformation (295\,nm
excitation) was explained
\parencite{Chinsky1984}.
The UV RRS intensities were used to improve the molecular force fields for
nucleic acid bases, guanine and cytosine, so that an agreement with the
frequencies and intensities of in-plane vibrations active in UV RRS spectra was
obtained considering the two low-energy \g{p}-\g{p}* transitions
\parencite{Lagant1991}.

Obviously the most precise theoretical calculations of UV RRS spectra of NA
components published so far are the works of
\textcite{%
	Sun2014,%
	Sun2015,%
	Sun2017%
}.
They used the Herzberg-Teller short-time dynamics formalism considering three
electron transitions --
	the low-energy \g{p}-\g{p}* transition,
	the low-energy n-\g{p}* transition
	and Rydberg transition at higher energy
-- and made their calculation for nucleobase complexes with explicit water
molecules.
In their calculations, several quantum chemical methods were employed and the
results compared with experimental spectra.
Although only dealing with the nucleobases having the simplest absorption
spectrum, i.e., uracil, 5-halogenated uracil, and thymine, they gained the so
far best, but still not ideal agreement with the experiment.
The authors revealed strong effect of hydrogen bonds on the vibrational
frequencies and RRS intensities as well.
The optimal inclusion of explicit hydrogen bonding might be the way to reach
better match with the experiment.

In parallel to the first direct UV RRS calculations, another idea how to
analyze UV RRS spectra has appeared.
It was based on a match between the theoretical expressions for UV RRS and for
electron absorption.
This enabled construction of semiempirical approaches when the shape of the
electronic absorption spectrum was employed to substitute a significant part of
the UV RRS calculation.
Peticolas group developed a method using the absorption spectrum and its
Kramers-Kronig transform
\parencite{Blazej1980},
which was also adopted to an overtone spectrum
\parencite{Chinsky1982a}.
A certain similarity to this approach is the method elaborated by the Loppnow
group, which is based on the similarity of the expressions for the
time-developed perturbation before the integration by time.
This analysis, called by the authors "time-dependent wave packet formalism" was
applied to UV RRS of numerous NA components and their derivatives, in
particular 5-fluorouracil
\parencite{Billinghurst2006a},
cytosine
\parencite{Billinghurst2006},
thymine
\parencite{Yarasi2007},
uracil
\parencite{Yarasi2009},
5-deuterouracil
\parencite{Ng2011},
N1-methylthymine, thymidine, and thymidine 5'-monophosphate
\parencite{Billinghurst2012},
2'-deoxyguanosine
\parencite{El-Yazbi2011},
and methyladenine
\parencite{Oladepo2011}.
The authors made numerous detail conclusions about the structural dynamics of
nucleobases upon photoexcitation.
Despite the obviously higher quality of UV RRS spectra measured by using
up-to-date equipments, the agreement between the predicted and measured
excitation profiles was not always satisfying.
Nevertheless, this series of papers brings about at least empirical indications
of different effects of the electron excitation on individual nucleobases.

Muntean with coworkers investigated lineshapes of individual UV RRS lines of
DNA at various conditions (pH, added cations)
\parencite{%
	Muntean2017,%
	Muntean2020%
}.
The lineshapes were characterized by vibrational correlation function obtained
as Fourier transform of the lineshape and the results interpreted in terms of
vibrational relaxation in picosecond timescale.

\subsection{Asset of UV RRS to the interpretation of NA vibrational modes}

The resonance enhancement concerns almost exclusively in-plane vibrational
modes of nucleobases.
UV RRS can thus serve as an elementary verification of the proposed
interpretation of vibrational transitions observed in non-resonant RS or IR
absorption spectra.
On the other hand, frequencies of some NA lines active in RRS are sensitive to
conformational changes, which indicate their coupling to the sugar vibrational
modes.
For instance, the shifts of 1310\,\icm{} line to 1303\,\icm{} and that at
730\,\icm{} to 725\,\icm{} were found to correlate with anti versus syn
adenosine conformation.
Similarly, the frequency shifts of guanosine lines from 1319 to 1325\,\icm{}
and 1360 to 1365\,\icm{} correspond to the conformational change from
C4'-exo-syn to C4'-exo-anti
\parencite{Nishimura1987}.
Frequency shifts were also observed for guanine modes at lower frequencies for
B to Z transition of poly(dGdC), when 682\,\icm{} band was shifted to
624\,\icm{} and 862 to 842\,\icm{}
\parencite{Fodor1986a}.
These spectral changes were similar to those previously observed with visible
excitation, but with 218-nm excitation were seen in detail.

The obtained RRE can also be employed to confirm the vibrational mode
interpretation provided the characteristics of the respective electron
transition are known or at least realistically estimated.
So changes of RRS intensities with excitation wavelength enabled the assignment
of dUMP \ch{"C4" \bond{db}O} and \ch{"C2" \bond{db}O} stretches at 1674 and
1686\,\icm{}
\parencite{Fodor1985}.
The same study revealed that the strengths of lines at 1230 (dUMP) and
1244\,\icm{} (dTMP) were anomalous and thus a need for normal mode revisions
was suggested.
Use of UV RRS additionally to non-resonant RS and IR absorption enabled
confirmation of the planar modes of thymine and tracking the effect of
deuteration.
Novel assignment of spectral bands was then proposed for several vibrational
modes of the methylene group: \ch{"C2'" H2} antisymmetric stretching
(2995\,\icm{}), symmetric stretching (2956\,\icm{}), scissoring (1404\,\icm{}),
and wagging (1174\,\icm{})
\parencite{Tsuboi1997}.

A significant advantage of the resonance RS over the classical one is that it
is relatively simple (lower number of active fundamental transitions).
Moreover, the use of several excitation wavelengths allows for the separation
of overlapping bands.
This makes it possible to accurately monitor the frequency changes caused by
various molecular states or identify individual bands in complex NA samples
containing all types of nucleobases.
In this way, the sensitivity of guanine vibrational modes to both
donor/acceptor type hydrogen bonds was investigated by means of UV RRS spectra
recorded in various solvents
\parencite{Toyama1996}.
Adenine protonated form was identified via UV RRS in non-canonical \ch{d(AG)10}
single helix
\parencite{Mukerji1995}.
UV RRS was also successfully employed to analyze premelting conformational
changes in poly(dA)$\cdot$poly(dT) and poly(dAdT).
The observed changes of the thymine \ch{"C4" \bond{db}O} stretching and the
adenine N6 scissoring mode indicated cross-strand bifurcated hydrogen bonds
between consecutive A:T pairs at low temperatures
\parencite{Chan1997}.
UV RRS spectrum of guanosine was compared with those of its seven
isotope-substituted analogs, and the observed isotope shifts were used to
assign the Raman bands to vibrations of the peripheral sites
	(\ch{"N1" \bond{sb}H},
	\ch{"C2" \bond{sb}NH2}
	and \ch{"C6" \bond{db}O}),
the pyrimidine ring and/or the imidazole ring
\parencite{Toyama1999}.
The suitability of UV RRS to study complex NA samples was demonstrated by the
study of tandem repeats of telomeric DNA, when numerous markers of guanine
quadruplexes were established
\parencite{Krafft2002},
or by monitoring A to B transition in DNA dodecamers
\parencite{Knee2008}.


