\subsubsection{Analysis of nucleic acids UV RRS spectra}

UV RRS measurements of NA components providing first data concerning the
resonance Raman enhancement (RRE) of nucleobase vibrational modes and its
dependence on the excitation wavelength stimulated intensive efforts for RRE
theoretical prediction.
It was assumed that the comparison with experimental results would enable
determination of the equilibrium value shift of the vibrational coordinates
connected with a particular electronic excitation.
It was expected to estimate in this way the changes in the nucleobase geometry
caused by the low-energy electron absorption transitions and to obtain
information significant to understand the photochemical processes in nucleic
acids.

The first applicable theoretical relationships were derived by
\textcite{Peticolas1970}
by an extensive simplification of the theory of Raman intensities developed by
\textcite{Albrecht1961}.
According this theory the resonantly enhanced bands when excited at the
low-energy absorption band should also exhibit hypochromism like the UV
absorption in this spectral region.
First quantitative measurements of preRRE
\parencite{Kalantar1972}
seemed to agree with the theory predicting RRE in the low-energy absorption
region for all hypochromic Raman bands.
Further detailed studies showed though only partly agreement with this
statement:
The vibration of uracil at 1230\,\icm{} is strongly hypochromic
\parencite{%,
	Small1971,%
	Small1971a%
}
and exhibits preRRE
\parencite{Tsuboi1971}
but calculations did not indicate the derivation of this band intensity from
the low-energy absorption band.
On the other hand, the 1338\,\icm{} band of adenine with strong hypochromism
\parencite{%
	Aylward1970,%
	Lafleur1972,%
	Small1971%
}
gains its intensity from the 260\,nm transition.
Adenine Raman bands at 1583 and 1485\,\icm{} were not hypochromic but show RRE.
Later on this theory was improved (but still omitting the effect of
Frank-Condon factors for other vibrational modes then the one active in the
Raman transition)
\parencite{%
	Blazej1977,%
	Warshel1977a%
}.
Nevertheless only partial agreement with the new experimental data was achieved
\parencite{%
	Peticolas1979,%
	Peticolas1980%
}.

Further works revealed the need to consider also the electron transitions lying
at higher energies.
In enabled to explain the decrease of the hypochromism of some Raman bands of
poly(rU) when the excitation was changed from 363\,nm to 290\,nm
\parencite{Chinsky1980}.
Raman excitation profiles of GMP, CMP, and TMP (and their protonated forms)
measured to 285\,nm revealed that while GMP 1324 and 1488\,\icm{} bands derive
their intensity mainly from the same electronic state near 276\,nm (but for
1577\,\icm{}, small contributions from other excited states cannot be
excluded), for all TMP bands the RRE was connected with 267\,nm absorption band
and for CMP all bends seemed to have origin in 175\,nm state
\parencite{Samanta1982}.
It was also demonstrated that in some cases the presence of n-\g{p}* electronic
transitions perpendicular to the nucleobase plane had to be taken into account
besides the generally considered \g{p}-\g{p}* electronic transitions, the
transition dipole of which is parallel to this plane.
In this way the twice greater intensity of the cytosine band at 778\,\icm{} in
Z conformation of poly(dGdC) duplex in respect to B conformation (295\,nm
excitation) was explained
\parencite{Chinsky1984}.
The UV RRS intensities were used to improve the molecular force fields for
nucleic acid bases, guanine and cytosine, so that an agreement with the
frequencies and intensities of in-plane vibrations active in UV RRS spectra was
obtained considering the two low-energy \g{p}-\g{p}* transitions
\parencite{Lagant1991}.

Obviously the most precise theoretical calculations of UV RRS spectra of NA
components published so far are the works of
\textcite{%
	Sun2014,%
	Sun2015,%
	Sun2017%
}.
They used the Herzberg-Teller short-time dynamics formalism considering three
electron transitions --
	the low-energy \g{p}-\g{p}* transition,
	the low-energy n-\g{p}* transition
	and Rydberg transition at higher energy
-- and made their calculation for nucleobase complexes with explicit water
molecules.
In their calculations, several quantum chemical methods were employed and the
results compared with experimental spectra.
Although only dealing with the nucleobases having the simplest absorption
spectrum, i.e., uracil, 5-halogenated uracil, and thymine, they gained the so
far best, but still not ideal agreement with the experiment.
The authors revealed strong effect of hydrogen bonds on the vibrational
frequencies and RRS intensities as well.
The optimal inclusion of explicit hydrogen bonding might be the way to reach
better match with the experiment.

In parallel to the first direct UV RRS calculations, another idea how to
analyze UV RRS spectra has appeared.
It was based on a match between the theoretical expressions for UV RRS and for
electron absorption.
This enabled construction of semiempirical approaches when the shape of the
electronic absorption spectrum was employed to substitute a significant part of
the UV RRS calculation.
Peticolas group developed a method using the absorption spectrum and its
Kramers-Kronig transform
\parencite{Blazej1980},
which was also adopted to an overtone spectrum
\parencite{Chinsky1982a}.
A certain similarity to this approach is the method elaborated by the Loppnow
group, which is based on the similarity of the expressions for the
time-developed perturbation before the integration by time.
This analysis, called by the authors "time-dependent wave packet formalism" was
applied to UV RRS of numerous NA components and their derivatives, in
particular 5-fluorouracil
\parencite{Billinghurst2006a},
cytosine
\parencite{Billinghurst2006},
thymine
\parencite{Yarasi2007},
uracil
\parencite{Yarasi2009},
5-deuterouracil
\parencite{Ng2011},
N1-methylthymine, thymidine, and thymidine 5'-monophosphate
\parencite{Billinghurst2012},
2'-deoxyguanosine
\parencite{El-Yazbi2011},
and methyladenine
\parencite{Oladepo2011}.
The authors made numerous detail conclusions about the structural dynamics of
nucleobases upon photoexcitation.
Despite the obviously higher quality of UV RRS spectra measured by using
up-to-date equipments, the agreement between the predicted and measured
excitation profiles was not always satisfying.
Nevertheless, this series of papers brings about at least empirical indications
of different effects of the electron excitation on individual nucleobases.

Muntean with coworkers investigated lineshapes of individual UV RRS lines of
DNA at various conditions (pH, added cations)
\parencite{%
	Muntean2017,%
	Muntean2020%
}.
The lineshapes were characterized by vibrational correlation function obtained
as Fourier transform of the lineshape and the results interpreted in terms of
vibrational relaxation in picosecond timescale.
