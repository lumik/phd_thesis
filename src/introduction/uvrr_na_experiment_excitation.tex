\subsection{Excitation sources}

It is typical for the first two decades that a great number of various laser
sources were employed, to reach an intensive excitation at various UV
wavelengths.
As a rule, pulse lasers with high peak power were chosen in order to reach high
efficiency of nonlinear effects serving for the wavelength transformation.

Frequency-doubled pulsed high-power Ar\textsuperscript{+} laser was used for
excitation at 257\,nm
\parencite{%
	Harada1975,%
	Pezolet1975,%
	Nishimura1977%
}.
Pulsed Nd:YAG lasers, as a rule of low repetition frequency, were employed
mainly to obtain 266\,nm or 213\,nm wavelengths as the forth and the fifth
harmonic, respectively
\parencite{%
	Ziegler1981,%
	Toyama1993%
}.
Other wavelengths were obtained using higher harmonics
\parencite{Kubasek1985}
and \ch{H2} Raman shift cell
\parencite{%
	Fodor1985,%
	Nishimura1987%
}.
211\,nm radiation was obtained by using sum frequency mixing of the second and
third harmonics of a
\emph{continuous-wave} (cw)
mode-locked Nd:YLF laser operating at 76\,MHz
\parencite{Leonard1994}.
Excimer lasers were also employed
\parencite{Ziegler1983}.

Continuously tunable UV radiation was obtained by frequency-doubled pulsed dye
lasers, firstly pumped by a flash lamp
\parencite{%
	Asher1977,%
	Blazej1977%
},
which enables to obtain wavelengths above 265\,nm.
Later on frequency doubled synchronously pumped dye lasers providing high
repetition ps pulses were used for an excitation with relatively low average
power tunable down to 285\,nm
\parencite{%
	Bushaw1978,%
	Samanta1982,%
	Benson1992%
}.
An excitation source tunable in a large range of 217–750 nm was designed using
a frequency doubled or tripled 20\,Hz Nd:YAG laser to pump a dye laser
\parencite{Asher1983}.
UV wavelengths were obtained by a nonlinear mixing or frequency doubling.
An excimer-pumped dye-laser system with subsequent frequency doubling provided
wavelengths were continuously adjustable down to 205\,nm
\parencite{Gfrorer1993a}.

At the beginning of nineties, the quantity of so far obtained experimental data
has allowed for general conclusions concerning UV RRS spectra of NA obtained by
using pulse lasers.
Although the strong resonance enhancement led to high signal-to-noise (S/N),
the typical S/N ratios remained far below the theoretical limit for high-pulse
energy excitation.
Moreover, disaccords of Raman intensity data, in particular the excitation RRS
profiles, obtained in different laboratories were presented (e.g.
\cite{Bushaw1980}).
The optical excitation sampling limitations, dielectric breakdown and other
nonlinear optical phenomena
\parencite{Teraoka1990},
and Raman saturation phenomena
\parencite{%
	Harmon1990,%
	Johnson1986,%
	Ludwig1988a,%
	Song1991a,%
	Sweeney1990,%
	Teraoka1990%
}
were specified as the reasons.

In the same time, the cw Ar\textsuperscript{+} ion laser with intracavity
doubling has appeared as a new UV source providing 257, 248, 244, 238 and
228.9\,nm excitations
\parencite{%
	Asher1993b,%
	Russell1995%
}.
Due to the cw regime, its use lowered nonlinear sample photochemistry, sample
thermal degradation, and ground-state saturation.
The beam could be focused to a small sample volume, which could be efficiently
collected; this capability resulted in very high spectral S/N ratios.
