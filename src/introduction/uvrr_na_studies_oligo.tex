\subsection{Oligonucleotides and natural NA}

The published UV RRS measurements of natural NA are relatively rare.
In eighties, two measurements on calf thymus DNA (the most common natural NA)
were carried out and Raman spectra assigned by a comparison with nucleotide
spectra
\parencite{%
	Laigle1982a,%
	Fodor1986a%
}.
In the latter work, poly(dAdT), poly(dGdC) were also measured as references and
the spectral shifts caused by the duplex formation were identified.
UV RRS (248, 257, and 264\,nm) of calf-thymus DNA was later also measured to
determine the damage due to ionizing radiation
\parencite{Shaw2009}.

UV RRS of filamentous viruses Pf1 and fd were measured with various excitation
wavelengths (257, 244, 238, and 229\,nm)
\parencite{Wen1999}.
The results indicated markedly different modes of organization of ssDNA in Pf1
and fd virions, despite similar environments for coat protein tyrosines.

UV RRS was also used to observe effect of supercoiling on DNA plasmid pBR322
\parencite{Neugebauer2007}.
In comparison with relaxed form changes were seen in the \ch{C=O} and exocyclic
\ch{NH2} vibration above 1600\,\icm{}, CN stretching mode of adenosine and
guanosine at 1487\,\icm{}, and in the spectral region 1320 -- 1360\,\icm{}
sensitive to deoxyribose conformation.

Oligonucleotides as NA segments with a customer specified nucleobase sequences
can serve as realistic molecular models of important NA structural elements.
The published UV RRS studies concern mainly duplex structures, hairpins and
guanine quadruplexes.

The effect of cytosine and guanosine methylation on the thermal stability of
DNA duplexes was studied via UV RRS of DNA hexamers with alternating C/G
sequences
\parencite{%
	Laigle1986,%
	Chinsky1987%
}.
\textcite{Mukerji2002}
The premelting transition of A/T tracts in DNA duplexes was studied
via UV RRS (and ECD) of DNA 12-mers with palindrome sequences containing
central A/T 6-membered tract consisting of alternating A and T or two triads of
A and T
\parencite{Mukerji2002}.
The premelting effect was observed in the latter case only being pronounced by
frequency shifts of dA exocyclic \ch{NH2} and dT \ch{"C4" \bond{db}O}
vibrational modes.
This suggested formation of three-centered hydrogen bonds at low temperatures.
UV RRS (210, 240, 260\,nm) was also used to monitor A to B transition in three
DNA duplexes --
	\ch{d(CGCGAATTCGCG)2},
	\ch{d(CGCGAATTGCGC) + "complementary~strand"},
	and \ch{d(CGCAAATTTCGC) + "complementary~strand"}
\parencite{Knee2008}.
The A form existed at low water activity for solutions in trifluoroethanol.
The effect of pH (6.4 and 3.45) in combination with possible presence of
\ch{Mn^{2+}} and \ch{Ca^{2+}} ions on UV RRS (275\,nm) spectra of LacDNA 22-mer
duplex
	d(TAATGTGAGTT\allowbreak{}AGCTCACTCAT)%
	$\cdot$%
	d(ATGAGTGAGCT\allowbreak{}AACTCACATTA)
was measured
and compared with that of its individual strands
\parencite{Muntean2013}.

Hairpin studies are represented by UV RRS (257\,nm) spectral measurements of
DNA 12-mer d(5'TTCTCGCGAAGC3'), which forms a stable hairpin at its
5'GCGAAGC3' part, and d(5'TTCTCCGCAAGC3') as a reference without the hairpin
formation
\parencite{Refregiers1997}.
The difference spectrum showed marker peaks of the hairpin formation.
The oligonucleotides were measured with and without the fluorescence labeling
used in previously published FRET study.
The results proved that the labeling did not prevent the hairpin formation.

As for the guanine quadruplexes, UV RRS (240 and 250\,nm) spectral features
related to non-canonical structural arrangement of \ch{d(AG)10} were
determined at different temperatures
\parencite{%
	Mukerji1995,%
	Mukerji1996%
}.
The observed changes of guanine modes exhibited behavior corresponding to
melting transition but the changes of adenine vibrational modes were linear.
The results support the model, in which two hairpins interact to form a helical
structure with G-tetrads and intercalated dA residues.
UV RRS (257\,nm) of several DNA oligonucleotides with telometric sequences (12,
16, and 22-mers containing dT and dG) have confirmed formation of guanine
quadruplexes connected by thymidine loops
\parencite{Krafft2002}.
