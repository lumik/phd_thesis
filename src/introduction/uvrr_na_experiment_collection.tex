\subsection{Scattering geometry, optical paths and spectral analysis}

Both 90-degree and backscattering geometries were employed. An exception is a
grazing incidence geometry used in the case of the sample cuvette with a side
opening
\parencite{Jolles1984}
and a 120\textdegree geometry used for measurement from a jet stream
\parencite{Fodor1985}.
The excitation beam was directed by means of planar mirrors and focused by a
quartz lens.
More diverse were the arrangements of the collecting paths.
Besides the quartz lenses, mirrors (concave mirrors
\cite{Blazej1977},
or later Cassegrain reflective objectives were
\cite{%
	Toyama1991,%
	Russell1995%
})
were used to avoid the chromatic aberration.

Spectral analysis was first performed by using double or triple spectrographs
\parencite{%
	Harada1975,%
	Gfrorer1993a,%
	Toyama1993%
},
from the beginning of nineties equipped with a cooled CCD detector
\parencite{%
	Gfrorer1993a,%
	Toyama1993%
}.
It was later demonstrated that a single-stage spectrograph (which is of higher
throughput) is sufficient to reject the elastically scattered radiation when it
is placed behind a simple prefiltering element.
The premonochromators, used for this purpose
\parencite{%
	Hashimoto1993,%
	Russell1995%
},
were after replaced by proper optical filters
\parencite{%
	Munro1997,%
	Bykov2013%
}.
