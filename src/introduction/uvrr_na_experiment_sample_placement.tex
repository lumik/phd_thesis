\subsection{Sample placement}
\label{introduction_sample_placement}

Resonance Raman spectroscopy uses excitation light
with frequency inside electronic absorption band of samples. It means that the
investigated molecules accept significant part of the power from incoming
laser beam and this excesive energy can destroy the samples. It also locally
increases temperature and causes thermal lens effect which distorts laser
focus. Over the time, resonance Raman spectroscopists invented methods to
minimize these effects.

Except some cases when UV RRS spectra of NA were measured in stationary quartz
capillaries or tubes
\parencite{%
	Blazej1977,%
	Asher1983%
},
three main concepts has been designed to reduce the NA photodamage during the
experiment.
All these concepts are used up to now without any clear evaluation, which of
them should be preferred.

Quartz rotating cell enables to enlarge the sample volume exposed by the
radiation.
The rotating cells were usually constructed for sample volumes of units of mL
\parencite{%
	Kiefer1971,%
	Kiefer1971a,%
	Pezolet1973,%
	Nishimura1977,%
	Laigle1982a,%
	Asher1983,%
	Benson1992,%
	Toyama1993,%
	Russell1995%
},
but a low-volume (down to 2\g{m}L) construction was also published
\parencite{Gfrorer1993a}.

Another possibility is a flow system with a reservoir, where the sample is
circulating.
UV RRS is measured from a jet stream
\parencite{%
	Ziegler1981,%
	Asher1983,%
	Fodor1985,%
	Russell1995,%
	Toyama1993%
}
or a properly designed flow cell
\parencite{%
	Blazej1980,%
	Gfrorer1989%
}.
The sample volume depending on the particular design was from 2 to 50\,mL.

Relatively simple but acceptable variant is a standard cuvette of about 1\,mL
or somewhat more volume with constantly stirring.
In this case, the backscattering geometry is usually used (except for the
mentioned above the grazing incidence
\parencite{Jolles1984}).
The advantage of this approach is a relatively easy temperature control of the
sample
\parencite{Mukerji1995}.

It is worth to mention that a combination of the rotating cuvette and stirring
has also been published
\parencite{Rodgers1992}.
The authors used a Suprasil NMR tube (0.5\,mL of sample) spun at approximately
10\,Hz around a stationary helical stirring wire.
