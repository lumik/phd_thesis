\subsection{Asset of UV RRS to interpretation of NA vibrational modes}

The resonance enhancement concerns almost exclusively in-plane vibrational
modes of nucleobases.
UV RRS can thus serve for an elementary verification of the proposed
interpretation of vibrational transitions observed in non-resonant RS or IR
absorption spectra.
On the other hand, frequencies of some NA lines active in RRS are sensitive to
conformational changes, which indicates their coupling to the sugar vibrational
modes.
For instance, the shifts of 1310\,\icm{} line to 1303\,\icm{} and that at
730\,\icm{} to 725\,\icm{} were found to correlate with anti versus syn
adenosine conformation.
Similarly, the frequency shifts of guanosine lines from 1319 to 1325\,\icm{}
and from 1360 to 1365\,\icm{} correspond to the conformational change from
C4'-exo-syn to C4'-exo-anti
\parencite{Nishimura1987}.
Frequency shifts were observed also for guanine modes at lower frequencies for
B to Z transition of poly(dGdC), when 682\,\icm{} band was shifted to
624\,\icm{} and 862 to 842\,\icm{}
\parencite{Fodor1986a}.
These spectral changes were similar to those previously observed with visible
excitation but with 218-nm excitation were seen in details.

The obtained RRE can also be employed for confirmation of the vibrational mode
interpretation provided the characteristics of the respective electron
transition are known or at least realistically estimated.
So changes of RRS intensities with excitation wavelength enabled assignment of
dUMP \ch{"C4" \bond{db}O} and \ch{"C2" \bond{db}O} stretches, to be at 1674 and
1686\,\icm{}
\parencite{Fodor1985}.
The same study revealed that the strengths of lines at 1230 (dUMP) and
1244\,\icm{} (dTMP) were anomalous and thus a need for normal mode revisions
was suggested.
Use of UV RRS additionally to non-resonant RS and IR absorption enabled
confirmation of the planar modes of thymine and tracking the effect of
deuteration.
Novel assignment of spectral bands was then proposed for several vibrational
modes of the methylene group: \ch{"C2'" H2} antisymmetric stretching
(2995\,\icm{}), symmetric stretching (2956\,\icm{}), scissoring (1404\,\icm{}),
and wagging (1174\,\icm{})
\parencite{Tsuboi1997}.

A great advantage of the resonance RS over the classical one is that it is
relatively simple (lower number of active fundamental transitions).
Moreover, the use of several excitation wavelengths allows for separation of
overlapping bands.
This makes it possible to monitor accurately the frequency changes caused by
varied molecular state or to identify individual bands in complex NA samples
containing all types of nucleobases.
In this way, the sensitivity of guanine vibrational modes to both
donor/acceptor type hydrogen bonds was investigated by means of UV RRS spectra
recorded in various solvents
\parencite{Toyama1996}.
Adenine protonated form was identified via UV RRS in non-canonical \ch{d(AG)10}
single helix
\parencite{Mukerji1995}.
UV RRS was also successfully employed to analyze premelting conformational
changes in poly(dA)$\cdot$poly(dT) and poly(dAdT).
The observed changes of the thymine \ch{"C4" \bond{db}O} stretching and the
adenine N6 scissoring mode indicated cross-strand bifurcated hydrogen bonds
between consecutive A:T pairs at low temperatures
\parencite{Chan1997}.
UV RRS spectrum of guanosine was compared with those of its seven
isotope-substituted analogs and the observed isotope shifts used to assign the
Raman bands to vibrations of the peripheral sites
	(\ch{"N1" \bond{sb}H},
	\ch{"C2" \bond{sb}NH2}
	and \ch{"C6" \bond{db}O}),
the pyrimidine ring and/or the imidazole ring
\parencite{Toyama1999}.
The suitability of UV RRS to study complex NA samples was demonstrated by the
study of tandem repeats of telomeric DNA, when numerous markers of guanine
quadruplexes were established
\parencite{Krafft2002},
or by monitoring A to B transition in DNA dodecamers
\parencite{Knee2008}.
