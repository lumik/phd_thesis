\subsection{Spinning cell design}
\label{introduction_spinning_cell}

Spinning cells for sample placement have been used in resonance Raman spectropy
for long time.
\textcite{Kiefer1971}
reported a spinning cell that eliminated the difficulties connected with
excitation laser interaction with samples and increased the observed Raman
intensity by a factor of about 10. A schematic diagram of the cell is given in
\figref{introduction_spinning_cell:1971Kiefer_rotating_cell}.
They used a cylindrical quartz cell with an outer diameter of 60\,mm and a
height of 25\,mm glued to a circular piece of brass.
The rotation speed could vary from 0 to 3000 rpm.
The cell could contain $\sim 65$\,mL of liquid maximally, but only
$\sim 15$\,mL were necessary for measurement because the liquid is driven to
the cell's walls by centrifugal force.

\begin{figure}
	\centering
	\ig{.3}{introduction/assets/1971Kiefer_rotating_cell}
	\caption[%
		Cell for resonance Raman measurements in liquids.%
	]{%
		\captiontitle{%
			Cell for resonance Raman measurements in liquids proposed by
			\textcite{Kiefer1971}.%
		}
		The picture was taken from the original paper.
	}
	\label{\figlabel{introduction_spinning_cell:1971Kiefer_rotating_cell}}
\end{figure}

\textcite{Shriver1974} improved the design of the spinning cell holder.
They used pyrex tubes which could be directly inserted into the holder.
The tube was secured inside the holder by a split nylon cone compressed by
knurled aluminum nut attached to the aluminum body of chuck (see
\figref{introduction_spinning_cell:1974Shriver_spinning_cell}).
The cell could contain 1\,mL of samples or less for measurement, in which case
the spectra were obtained from a thin film of the liquid on the sides of the
cell.

\begin{figure}
	\centering
	\ig{.35}{introduction/assets/1974Shriver_spinning_cell}
	\caption[%
		Detail of sample spinner, for 180\textdegree{} or oblique illumination.%
	]{%
		\captiontitle{%
			Detail of sample spinner, for 180\textdegree{} or oblique illumination.%
		}
		A: complete system with evacuated Pyrex jacket, J,
			surrounding the sample tube, S.
			Cold (or warm) gas such as \ch{N2} is passed through J to control the
			sample temperature.
			L$_1$ is the focusing cylindrical lens,
			L$_2$ is the collecting lens,
			M is the small front surface mirror.
			C is the sample chuck.
		B: detail of the sample chuck:
			O, split nylon cone;
			P, knurled aluminum nut attached to the aluminum body of chuck;
			Q, spinner shaft.
		The spinner should be constructed to minimize the wobble of the sample
		tube, which decreases the Raman signal at high absorber concentrations.
		Adapted from \textcite{Shriver1974}.%
	}
	\label{\figlabel{introduction_spinning_cell:1971Kiefer_rotating_cell}}
\end{figure}

Similar cells using the above-mentioned design approaches were then regularly
used in resonance Raman studies, see
\cref{introduction_sample_placement}.
