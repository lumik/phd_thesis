\subsubsection{Calibration}

During the first period when the main aim was to determine the UV RRS
excitation profiles of various NA components, great attention was paid to an
intensity calibration.
The calibration procedures served mainly for intensity normalization of Raman
spectra obtained for different excitation wavelengths.
Usually Raman lines (mostly only the most intensive ones) of pure simple
compounds or their relatively concentrated solutions, which did not absorb in
the near UV, were taken as standards.
The dependence of their UV RS intensity on the excitation wavelength was
omitted or simply approximated.
The list of employed standards include cyclohexane
\parencite{Hamaguchi1974},
water (3400\,\icm{} band)
\parencite{%
	Chinsky1978,%
	Fodor1985%
},
cacodylate ion (608\,\icm{} line)
\parencite{%
	Blazej1977,%
	Samanta1982%
},
phosphate ion (994\,\icm{} line)
\parencite{Kubasek1985},
sulphate ion (981\,\icm{} line)
\parencite{%
	Perno1989,%
	Benson1992,%
	Wen1998%
},
and perchlorate (934\,\icm{} line)
\parencite{Rodgers1992}.
Some authors employed the Raman lines naturally occurring in the measured
spectrum but without resonance enhancement, e.g. 980\,\icm{} line of the
\ch{PO3^-}
vibration and the water band at 1640\,\icm{} in the spectrum of nucleotide
solution
\parencite{Tsuboi1974}.

The only work was devoted to an intensity correction through the measured
UV RRS spectrum
\parencite{Chinsky1983a}.
The authors aimed to correct the spectral dependence of the monochromator
throughput in the region of 2000 -- 5000\,\icm.
They used UV radiation from deuterium lamp passing through a double
monochromator or only through its first stage, assuming the same transmission
for the second stage.

Surprisingly only a few published UV RRS studies of NA specified how the
wavenumber scale was calibrated.
Usually the calibrations were based on known positions of Raman lines of
simple compounds.
Carbon tetrachloride was used by
\textcite{Hamaguchi1974}.
Later on, mixtures or sets of separately measured chemicals were employed,
namely ethyl acetate/dioxane
\parencite{Toyama1991},
ethanol and n-pentane
\parencite{Mukerji1995},
dioxane, carbon tetrachloride and acetonitrile
\parencite{Russell1995}.
Spectral calibration with the lines of a low-pressure Hg lamp was presented by
\textcite{Efremov1991}.
