\subsection{Sample cell design}
\label{sample_cell}

Resonance Raman spectroscopy uses excitation light
with frequency inside electronic absorption band. It means that the
investigated molecules accept significant part of the power from incoming
laser beam and this excesive energy can destroy the samples. It also locally
increases temperature and causes thermal lens effect which distorts laser
focus. Over the time, resonance Raman spectroscopists invented methods to
minimize these effects.

\textcite{Kiefer1971} reported spinning cell for resonance Raman spectroscopy
which elimited the above mentioned difficulties and increased the observed
Raman intensity by a factor of about 10. A schematic diagram of the cell is
given in
\figref{sample_cell:1971Kiefer_rotating_cell}.
They used cylindrical quartz cell with outer diameter of 60\,mm and a height
of 25\,mm glued to circular piece of brass. The rotation speed could vary from
0 to 3000 rpm. The cell could contain maximally $\sim 65$\,mL of liquid, but
only $\sim 15$ mL were necessary for measurement, because the liquid is driven
to the walls of the cell by centrifugal force.

\insertfigure{%
	Cell for resonance Raman measurements in liquids proposed by
	\textcite(Kiefer1971). The picture was taken from the original paper.%
}{introduction/assets/1971Kiefer_rotating_cell}%
{.3}%
{sample_cell:1971Kiefer_rotating_cell}%
{t}

\textcite{Shriver1974} improved the design of the spinning cell holder. They
used pyrex tubes which could be directly inserted into the holder. The tube
was secured inside holder by split nylon cone compressed by knurled aluminium
nut attached to aluminium body of chuck (see
\figref{sample_cell:1974Shriver_spinning_cell}).
The cell could contain 1mL samples or less for measurement in which case the
spectra were obtained from thin film of the liquid on the sides of the cell.

\insertfigure{%
	Detail of sample spinner, for 180\textdegree{} or oblique illumination. A:
	complete system with evacuated Pyrex jacket, J, surrounding the sample tube,
	S. Cold (or warm) gas such as \ch{N2} is passed through J to control the
	sample temperature. L$_1$ is the focusing cylindrical lens, L$_2$ is the
	collecting lens, M is the small front surface mirror. C is the sample chuck.
	B: detail of the sample chuck: O, split nylon cone; P, knurled aluminium nut
	attached to aluminum body of chuck; Q, spinner shaft. The spinner should be
	constructed to minimize wobble of the sample tube, which decreases the Raman
	signal at high absorber concentrations. Adapted from \textcite{Shriver1974}.%
}{introduction/assets/1974Shriver_spinning_cell}%
{.35}%
{sample_cell:1974Shriver_spinning_cell}
{t}

The other approach to solve these problems was proposed by
\textcite{Ziegler1981} who used bubble-free flowing jet stream of sample with
about 1\,mm in diameter recirculated by a micropump. \textcite{Asher1983}
created improved sample jet stream apparatus for continuous circulation of the
sample. The apparatus used jet nozzle from a dye laser with 0.2-mm wide
optically uniform stream and teflon stainless-steel centrifugal pump which
maintained the continuous flow of the sample from reservoir. The schema can be
seen in \figref{sample_cell:1983Asher_free_jet}. The advantage of this
approach is that the sample potentially damaged by the excitation beam don't
need to be reused because it can be continually removed from the measurement.
The problems connected with absorption of excitation beam and selfabsorption
of the scattered light can be also minimized by using sufficiently thin stream.
The major disadvantages of jest stream devices are that cleaning of the
apparatus is much more difficult (which can be dangerous for samples sensitive
to contamination like RNAs), usually backscattering geometries, where it is
more difficult to fill input slit of spectrograph by the scattered light and
that large volumes of the samples are usually needed. The needed volumes of
samples are not usually reported in the resonance Raman studies, but it
is usually tens of mL -- 50\,mL \parencite{Ziegler1981},
2 -- 10\,mL \parencite{Fodor1985}.

\insertfigure{%
	Sample jet stream assembly. Adopted from \textcite{Asher1983}.
}{introduction/assets/1983Asher_free_jet}%
{.3}%
{sample_cell:1983Asher_free_jet}
{t}
