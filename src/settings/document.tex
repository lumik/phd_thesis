%%% The settings file. It contains definitions of basic parameters.

%% Settings for single-side (simplex) printing
% Margins: left 40mm, right 25mm, top and bottom 25mm
% (but beware, LaTeX adds 1in implicitly)
\documentclass[12pt,a4paper,twoside,openright,american]{report}
%\documentclass[11pt,a4paper,twoside,openright,english]{report}

%%% Settings for single-side (simplex) printing
%% Margins: left 40mm, right 25mm, top and bottom 25mm
%% (but beware, LaTeX adds 1in implicitly)
%\setlength\textwidth{145mm}
%\setlength\textheight{247mm}
%\setlength\oddsidemargin{15mm}
%\setlength\evensidemargin{15mm}
%\setlength\topmargin{0mm}
%\setlength\headsep{0mm}
%\setlength\headheight{0mm}
%\let\openright=\clearpage % for one sided printing

% Settings for two-sided (duplex) printing
\setlength\textwidth{145mm}
\setlength\textheight{247mm}
\setlength\oddsidemargin{14.2mm}
\setlength\evensidemargin{0mm}
\setlength\topmargin{0mm}
\setlength\headsep{0mm}
\setlength\headheight{0mm}
% \openright makes the following text appear on a right-hand page
\let\openright=\cleardoublepage

%%% Parameters of two-sided page
%%% Size of a4 page is 210 x 297 mm
%\setlength\hoffset{-25.4mm}  % 25.4 mm + \hoffset = left horizontal margin
%\setlength\voffset{-25.4mm}  % 25.4 mm + \voffset = top vertical margin
%\setlength\oddsidemargin{20mm}  % left margin on odd pages
%\setlength\evensidemargin{20mm}  % left margin on even page => margin from the
% center is: 
%% page width - \hoffset - 25.4mm - \textwidth - oddsidemargin
%
%%% header
%\setlength\topmargin{0mm} % margin from top position defined by \voffset
%\setlength\headsep{10mm} % margin of text from header
%\setlength\headheight{0mm} % header height
%
%%% footer
%\setlength\footskip{15mm} % margin of bottom border of footer from main text
%
%%% side notes
%\setlength\marginparwidth{5em} % side note width
%\setlength\marginparsep{5pt} % side note margin from main text
%
%%% size of text area
%\setlength\textwidth{170mm} % width of the text
%\setlength\textheight{267mm} % height of the text

% \openright next chapter on the right side of the book
%\let\openright=\cleardoublepage % for double sided printing

%% Generate PDF/A-2u
\usepackage[a-2u]{pdfx}

\usepackage[utf8]{inputenc}

%% Prefer Latin Modern fonts
\usepackage{lmodern}  % vector fonts Latin Modern, successor of original
% Knuth's Computern Modern fonts
\usepackage[main=american,greek,czech]{babel}
% Greek has only 10pt fonts MikTeXu in default. Therefore it is necessary to
% install cbgreek-complete font package for nice other sizes and not pixelated. 
% Command initexmf --admin --edit-config-file updmap opens text editor. Add Map
% cbgreek-full.map to the end and save and run
% initexmf --admin --verbose --mkmaps.
% Still doesn't work, some error during font instalation.
% Workaround, replace original cbgreek.map by the file cbgreek-full.map (i have
% to rename it) and issue
% initexmf --admin --verbose --mkmaps
% initexmf --verbose --mkmaps

\usepackage{cmap} 		% pdf created by `pdflatexem' is fully searchable and
% copyable
\usepackage[T1]{fontenc}  % font coding
\usepackage{csquotes}  % automatic quotes conversion according to language

%\usepackage[nohyperlinks]{acronym} % acronym package for the acronyms and
% symbol typing
% nohyperlinks - Don't create hypertext links to the list of acronyms

%\usepackage{pdfpages} % enables inserting whole pdf pages
% necessary for the inserting of title pages from IS

\usepackage[pdftex]{graphicx}
%\usepackage{fixltx2e}  % enables \textsubscript
\usepackage{calc}  % enables mathematical operations in \setlength as - and +
\usepackage{datetime2}  % enables \currenttiem

\usepackage{float}  % enables new float creation

\usepackage{epsf}

\usepackage[
    backend=biber,
		style=iso-authoryear,
		shortnumeration=true,
		sortlocale=en_US,
		uniquelist=false,
		autolang=other,
		bibencoding=UTF8,
		hyperref=true,
    backref=true
]{biblatex}

\addbibresource{misc/bibliography.bib}

\usepackage[nottoc]{tocbibind}  % makes sure that bibliography and the lists
			    % of figures/tables are included in the table
			    % of contents
\usepackage{amsmath}  %  Better looking eqnarray. Use environment align instead
% of eqnarray to use bold greek letters
\usepackage{amsfonts}  % math fonts
\usepackage{sansmath}  % enables switching to serif fonts also in math env

\usepackage{comment}  % multiline comments
\usepackage{url}  % hypertext urls in combination with hyperref hyperref
% \url{http://address} \url{URL}{text}
\usepackage{multirow}  % vertical merging of cells in tables (
% \multirow{number of rows}{width}{text}, width can be *. It denotes native
% line width)
\usepackage{dcolumn}  % improved alignment of table columns
\usepackage{tabularx}  % tables with fixed width
\usepackage{booktabs}  % better looking tables, more space after horizontal
% rule (use \midrule) and possibility of bold line usage (\toprule and
% \bottomrule)
\usepackage{paralist}  % improved enumerate and itemize
\usepackage{xcolor}  % typesetting in color

\usepackage{microtype}  % switches on microtypography - (protutrusion=true -
% shift of letter on the edge of the line, expansion=true) - small extension
% and compression of letters
\usepackage{caption}

%%% tikz
\usepackage{tikz}  % nice images in LaTeX
\usetikzlibrary{arrows,shapes,trees,intersections,decorations.markings}

%% The hyperref package for clickable links in PDF and also for storing
%% metadata to PDF (including the table of contents).
%% Most settings are pre-set by the pdfx package.
\hypersetup{colorlinks=false}
\hypersetup{pdfpagemode=UseOutlines}
\hypersetup{pdfdisplaydoctitle=true}
\hypersetup{raiselinks=false}
\hypersetup{pdfhighlight=/N}
\hypersetup{pdfborder={0 0 0}}
\hypersetup{pdflang=en}
\hypersetup{pdfstartview=}
\hypersetup{unicode}
\hypersetup{breaklinks=true}
\hypersetup{pdfencoding=auto,psdextra}

\usepackage[noabbrev]{cleveref}  % References, must be included after hyperref

\pagestyle{plain}
%\floatstyle{plain}
\clubpenalty=100000
\widowpenalty=100000
\brokenpenalty100000
\raggedbottom
