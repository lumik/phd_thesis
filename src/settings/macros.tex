%%% customization of titles

% These macros employ a little dirty trick to convince LaTeX to typeset
% chapter headings sanely, without lots of empty space above them.
% Feel free to ignore.
\makeatletter
\def\@makechapterhead#1{
  {\parindent \z@ \raggedright \normalfont
   \Huge\bfseries \thechapter. #1
   \par\nobreak
   \vskip 20\p@
}}
\def\@makeschapterhead#1{
  {\parindent \z@ \raggedright \normalfont
   \Huge\bfseries #1
   \par\nobreak
   \vskip 20\p@
}}
\makeatother

%%% upgright greek letters
\newcommand{\g}[1]{\foreignlanguage{greek}{#1}}
% if we want to use greek sigma which is not written at the end of the word as
% a standalone letter we have to use s\noboundary to not write it as a terminal
% sigma

%%% Symbols and math

% symbols
\chardef\promile=141
\newcommand{\mat}[1]{\mathbold{#1}}

% functions
\newcommand{\func}[1]{\operatorname{#1}}

% tensor
\newcommand{\tens}[1]{\overset{\scriptscriptstyle\leftrightarrow}{#1}}

% physical quantities and units
\newcommand{\wn}{\tilde{\nu}}  % wavenumber
\newcommand{\icm}{cm$^{-1}$}  % reciprocal centimeters

% quantum mechanic helpers
\newcommand{\bra}[1]{\ensuremath{\left\langle{#1}\right|}}
\newcommand{\ket}[1]{\ensuremath{\left|{#1}\right\rangle}}
\newcommand{\braket}[2]{\ensuremath{\left\langle{#1}\middle|{#2}\right\rangle}}
\newcommand{\bramidket}[3]{%
\ensuremath{\left\langle{#1}\middle|{#2}\middle|{#3}\right\rangle}}

%%% floats

% unified labels
\newcommand{\figlabel}[1]{fig:#1}
\newcommand{\tablabel}[1]{tab:#1}
\newcommand{\eqnlabel}[1]{eqn:#1}

% referemces to floats
%\newcommand{\corb}{\kern-.12em}  % correction for italics before its usage,
% for example inside parens
%\newcommand{\cora}{\kern.12em}  % correction for italics after its usage
%\newcommand{\fig}[1]{\corb\textit{Fig.~\ref{fig:#1}\cora}}
%\newcommand{\tab}[1]{\corb\textit{Tab.~\ref{tab:#1}\cora}}
\crefformat{figure}{\textit{#2fig.~#1#3}}
\Crefformat{figure}{\textit{#2Fig.~#1#3}}
\crefmultiformat{figure}{\textit{figs.~#2#1#3}}{ and~\textit{#2#1#3}}%
	{, \textit{#2#1#3}}{ and~\textit{#2#1#3}}
\Crefmultiformat{figure}{\textit{Figs.~#2#1#3}}{ and~\textit{#2#1#3}}%
	{, \textit{#2#1#3}}{ and~\textit{#2#1#3}}
\crefrangeformat{figure}{\textit{figs.~#3#1#4} to~\textit{#5#2#6}}
\Crefrangeformat{figure}{\textit{Figs.~#3#1#4} to~\textit{#5#2#6}}
\crefrangemultiformat{figure}{\textit{figs.~#3#1#4} to~\textit{#5#2#6}}%
	{ and~\textit{#3#1#4} to~\textit{#5#2#6}}%
	{, \textit{#3#1#4} to~\textit{#5#2#6}}%
	{ and~\textit{#3#1#4} to~\textit{#5#2#6}}%
\Crefrangemultiformat{figure}{\textit{Figs.~#3#1#4} to~\textit{#5#2#6}}%
	{ and~\textit{#3#1#4} to~\textit{#5#2#6}}%
	{, \textit{#3#1#4} to~\textit{#5#2#6}}%
	{ and~\textit{#3#1#4} to~\textit{#5#2#6}}%
\crefformat{table}{\textit{#2tab.~#1#3}}
\Crefformat{table}{\textit{#2Tab.~#1#3}}
\crefrangeformat{table}{\textit{tabs.~#3#1#4} to~\textit{#5#2#6}}
\Crefrangeformat{table}{\textit{Tabs.~#3#1#4} to~\textit{#5#2#6}}
\crefformat{equation}{\textit{#2eq.~#1#3}}
\Crefformat{equation}{\textit{#2Eq.~#1#3}}
\crefmultiformat{equation}{\textit{eqs.~#2#1#3}}{ and~\textit{#2#1#3}}%
	{, \textit{#2#1#3}}{ and~\textit{#2#1#3}}
\Crefmultiformat{equation}{\textit{Eqs.~#2#1#3}}{ and~\textit{#2#1#3}}%
	{, \textit{#2#1#3}}{ and~\textit{#2#1#3}}
\crefrangeformat{equation}{\textit{eqs.~#3#1#4} to~\textit{#5#2#6}}
\Crefrangeformat{equation}{\textit{Eqs.~#3#1#4} to~\textit{#5#2#6}}
\crefrangemultiformat{equation}{\textit{eqs.~#3#1#4} to~\textit{#5#2#6}}%
	{ and~\textit{#3#1#4} to~\textit{#5#2#6}}%
	{, \textit{#3#1#4} to~\textit{#5#2#6}}%
	{ and~\textit{#3#1#4} to~\textit{#5#2#6}}%
\Crefrangemultiformat{equation}{\textit{Eqs.~#3#1#4} to~\textit{#5#2#6}}%
	{ and~\textit{#3#1#4} to~\textit{#5#2#6}}%
	{, \textit{#3#1#4} to~\textit{#5#2#6}}%
	{ and~\textit{#3#1#4} to~\textit{#5#2#6}}%
\labelcrefformat{figure}{\textit{#2#1#3}}
\labelcrefformat{table}{\textit{#2#1#3}}
\labelcrefformat{equation}{\textit{#2#1#3}}
\newcommand{\figref}[1]{\cref{\figlabel{#1}}}
\newcommand{\Figref}[1]{\Cref{\figlabel{#1}}}
\newcommand{\bfigref}[1]{(\textit{\figref{#1}})}
\newcommand{\tabref}[1]{\cref{\tablabel{#1}}}
\newcommand{\Tabref}[1]{\Cref{\tablabel{#1}}}
\newcommand{\btabref}[1]{(\textit{\tabref{#1}})}
\newcommand{\eqnref}[1]{\cref{\eqnlabel{#1}}}
\newcommand{\Eqnref}[1]{\Cref{\eqnlabel{#1}}}
\newcommand{\beqnref}[1]{(\textit{\eqnref{#1}})}

% figures
\newcommand{\ig}[2]{\includegraphics[width=#1\columnwidth]{#2}}
% width in units of textwidth, filename

\newcommand{\insertfigure}[5]{%
	\begin{figure}[#5]
		\centering
		\ig{#3}{#2}
		\caption{#1}
		\label{\figlabel{#4}}
	\end{figure}%
}

\newcommand{\captiontitle}[1]{%
	\textbf{#1}%
}

%%% Enviroments without excessive vertical space

% \compress command prevents insertion of vertical space before the environment
\makeatletter
\newcommand*{\compress}{\@minipagetrue}
\makeatother

\newenvironment{compdesc}[1][\quad]
{\compress\begin{description}%
  \setlength{\itemsep}{0pt}%
  \setlength{\parskip}{0pt}%
  \setlength{\parsep}{0pt}%
	\setlength{\topsep}{0pt}%
	\setlength{\leftmargin}{29.3747pt}%
	\setlength{\itemindent}{\itemindent+5.87494pt}%
	\setlength{\partopsep}{0pt}%
}
{\compress\end{description}\ignorespacesafterend}

\newenvironment{compenum}[1][\quad]
{\compress\begin{enumerate}%
  \setlength{\itemsep}{0pt}%
  \setlength{\parskip}{0pt}%
  \setlength{\parsep}{0pt}%
	\setlength{\topsep}{0pt}%
	\setlength{\leftmargin}{29.3747pt}%
	\setlength{\itemindent}{\itemindent+5.87494pt}%
	\setlength{\partopsep}{0pt}%
}
{\compress\end{enumerate}\ignorespacesafterend}

\newenvironment{compitemize}[1][\quad]
{\compress\begin{itemize}%
  \setlength{\itemsep}{0pt}%
  \setlength{\parskip}{0pt}%
  \setlength{\parsep}{0pt}%
	\setlength{\topsep}{0pt}%
	\setlength{\leftmargin}{29.3747pt}%
	\setlength{\itemindent}{\itemindent+5.87494pt}%
	\setlength{\partopsep}{0pt}%
}
{\compress\end{itemize}\ignorespacesafterend}

% aliases for documentwide usage
\let\docdesc\compdesc
\let\enddocdesc\endcompdesc
\let\docenum\compenum
\let\enddocenum\endcompenum
\let\docitemize\compitemize
\let\enddocitemize\endcompitemize

%%% Other

%\DeclareUrlCommand\url{\def\UrlLeft{<}\def\UrlRight{>} \urlstyle{tt}}

% This macro defines a chapter, which is not numbered, but is included
% in the table of contents.
\def\chapwithtoc#1{
\chapter*{#1}
\addcontentsline{toc}{chapter}{#1}
}

% Draw black "slugs" whenever a line overflows, so that we can spot it easily.
\overfullrule=1mm

% mark missing citations
\newcommand{\CITATION}{%
		\PackageWarning{citations}{Missing citation!!!}%
		\textcolor{red}{CITATION}}
% mark missing references
\newcommand{\REFERENCE}{%
		\PackageWarning{references}{Missing reference!!!}%
		\textcolor{red}{REFERENCE}}
% mark place with inconsistency
\newcommand{\INCONSISTENCY}{%
		\PackageWarning{inconsistencies}{Possible inconsistency!!!}%
		\textcolor{red}{INCONSISTENCY}}
% mark missing citations
\newcommand{\MISSING}{%
		\PackageWarning{missing}{Missing discussion!!!}%
		\textcolor{red}{MISSING}}
% low quality image
\newcommand{\LOWQUALITY}{%
		\PackageWarning{quality}{Low quality image!!!}}

% TikZ macros
\newcommand{\mypgfextractangle}[3]{%
	\pgfmathanglebetweenpoints{%
		\pgfpointanchor{#2}{center}}{%
		\pgfpointanchor{#3}{center}}
	\global\let#1\pgfmathresult
}
